% \iffalse meta-comment
%
% Copyright (C) 2004-2011 by Morten Hoegholm
% Copyright (C) 2012-2019 by Lars Madsen
% Copyright (C) 2020      by Lars Madsen, The LaTeX3 team
%
% This work may be distributed and/or modified under the
% conditions of the LaTeX Project Public License, either
% version 1.3c of this license or (at your option) any later
% version. The latest version of this license is in
%    http://www.latex-project.org/lppl.txt
% and version 1.3c or later is part of all distributions of
% LaTeX version 2008/05/05 or later.
%
% This work has the LPPL maintenance status "maintained".
%
% This Current Maintainer of this work is
% Lars Madsen
%
% This work consists of the main source file mathtools.dtx
% and the derived files
%    mathtools.sty, mathtools.pdf, mathtools.ins, mathtools.drv.
%
% Distribution:
%    CTAN:macros/latex/contrib/mh/mathtools.dtx
%    CTAN:macros/latex/contrib/mh/mathtools.pdf
%
% Unpacking:
%    (a) If mathtools.ins is present:
%           tex mathtools.ins
%    (b) Without mathtools.ins:
%           tex mathtools.dtx
%    (c) If you insist on using LaTeX
%           latex \let\install=y% \iffalse meta-comment
%
% Copyright (C) 2004-2011 by Morten Hoegholm
% Copyright (C) 2012      by Lars Madsen
%
% This work may be distributed and/or modified under the
% conditions of the LaTeX Project Public License, either
% version 1.3 of this license or (at your option) any later
% version. The latest version of this license is in
%    http://www.latex-project.org/lppl.txt
% and version 1.3 or later is part of all distributions of
% LaTeX version 2005/12/01 or later.
%
% This work has the LPPL maintenance status "maintained".
%
% This Current Maintainer of this work is
% Lars Madsen, Will Robertson and Joseph Wright
%
% This work consists of the main source file mathtools.dtx
% and the derived files
%    mathtools.sty, mathtools.pdf, mathtools.ins, mathtools.drv.
%
% Distribution:
%    CTAN:macros/latex/contrib/mh/mathtools.dtx
%    CTAN:macros/latex/contrib/mh/mathtools.pdf
%
% Unpacking:
%    (a) If mathtools.ins is present:
%           tex mathtools.ins
%    (b) Without mathtools.ins:
%           tex mathtools.dtx
%    (c) If you insist on using LaTeX
%           latex \let\install=y% \iffalse meta-comment
%
% Copyright (C) 2004-2011 by Morten Hoegholm
% Copyright (C) 2012      by Lars Madsen
%
% This work may be distributed and/or modified under the
% conditions of the LaTeX Project Public License, either
% version 1.3 of this license or (at your option) any later
% version. The latest version of this license is in
%    http://www.latex-project.org/lppl.txt
% and version 1.3 or later is part of all distributions of
% LaTeX version 2005/12/01 or later.
%
% This work has the LPPL maintenance status "maintained".
%
% This Current Maintainer of this work is
% Lars Madsen, Will Robertson and Joseph Wright
%
% This work consists of the main source file mathtools.dtx
% and the derived files
%    mathtools.sty, mathtools.pdf, mathtools.ins, mathtools.drv.
%
% Distribution:
%    CTAN:macros/latex/contrib/mh/mathtools.dtx
%    CTAN:macros/latex/contrib/mh/mathtools.pdf
%
% Unpacking:
%    (a) If mathtools.ins is present:
%           tex mathtools.ins
%    (b) Without mathtools.ins:
%           tex mathtools.dtx
%    (c) If you insist on using LaTeX
%           latex \let\install=y% \iffalse meta-comment
%
% Copyright (C) 2004-2011 by Morten Hoegholm
% Copyright (C) 2012      by Lars Madsen
%
% This work may be distributed and/or modified under the
% conditions of the LaTeX Project Public License, either
% version 1.3 of this license or (at your option) any later
% version. The latest version of this license is in
%    http://www.latex-project.org/lppl.txt
% and version 1.3 or later is part of all distributions of
% LaTeX version 2005/12/01 or later.
%
% This work has the LPPL maintenance status "maintained".
%
% This Current Maintainer of this work is
% Lars Madsen, Will Robertson and Joseph Wright
%
% This work consists of the main source file mathtools.dtx
% and the derived files
%    mathtools.sty, mathtools.pdf, mathtools.ins, mathtools.drv.
%
% Distribution:
%    CTAN:macros/latex/contrib/mh/mathtools.dtx
%    CTAN:macros/latex/contrib/mh/mathtools.pdf
%
% Unpacking:
%    (a) If mathtools.ins is present:
%           tex mathtools.ins
%    (b) Without mathtools.ins:
%           tex mathtools.dtx
%    (c) If you insist on using LaTeX
%           latex \let\install=y\input{mathtools.dtx}
%        (quote the arguments according to the demands of your shell)
%
% Documentation:
%    (a) If mathtools.drv is present:
%           latex mathtools.drv
%    (b) Without mathtools.drv:
%           latex mathtools.dtx; ...
%    The class ltxdoc loads the configuration file ltxdoc.cfg
%    if available. Here you can specify further options, e.g.
%    use A4 as paper format:
%       \PassOptionsToClass{a4paper}{article}
%
%    Programm calls to get the documentation (example):
%       pdflatex mathtools.dtx
%       makeindex -s gind.ist mathtools.idx
%       pdflatex mathtools.dtx
%       makeindex -s gind.ist mathtools.idx
%       pdflatex mathtools.dtx
%
% Installation:
%    TDS:tex/latex/mh/mathtools.sty
%    TDS:doc/latex/mh/mathtools.pdf
%    TDS:source/latex/mh/mathtools.dtx
%
%<*ignore>
\begingroup
  \def\x{LaTeX2e}
\expandafter\endgroup
\ifcase 0\ifx\install y1\fi\expandafter
         \ifx\csname processbatchFile\endcsname\relax\else1\fi
         \ifx\fmtname\x\else 1\fi\relax
\else\csname fi\endcsname
%</ignore>
%<*install>
\input docstrip.tex
\Msg{************************************************************************}
\Msg{* Installation}
\Msg{* Package: mathtools 2012/05/10 v1.12}
\Msg{************************************************************************}

\keepsilent
\askforoverwritefalse

\preamble

This is a generated file.

Copyright (C) 2002-2011 by Morten Hoegholm

This work may be distributed and/or modified under the
conditions of the LaTeX Project Public License, either
version 1.3 of this license or (at your option) any later
version. The latest version of this license is in
   http://www.latex-project.org/lppl.txt
and version 1.3 or later is part of all distributions of
LaTeX version 2005/12/01 or later.

This work has the LPPL maintenance status "maintained".

This Current Maintainer of this work is  
Lars Madsen, Will Robertson and Joseph Wright.

This work consists of the main source file mathtools.dtx
and the derived files
   mathtools.sty, mathtools.pdf, mathtools.ins, mathtools.drv.

\endpreamble

\generate{%
  \file{mathtools.ins}{\from{mathtools.dtx}{install}}%
  \file{mathtools.drv}{\from{mathtools.dtx}{driver}}%
  \usedir{tex/latex/mh}%
  \file{mathtools.sty}{\from{mathtools.dtx}{package}}%
}

\obeyspaces
\Msg{************************************************************************}
\Msg{*}
\Msg{* To finish the installation you have to move the following}
\Msg{* file into a directory searched by TeX:}
\Msg{*}
\Msg{*     mathtools.sty}
\Msg{*}
\Msg{* To produce the documentation run the file `mathtools.drv'}
\Msg{* through LaTeX.}
\Msg{*}
\Msg{* Happy TeXing!}
\Msg{*}
\Msg{************************************************************************}

\endbatchfile
%</install>
%<*ignore>
\fi
%</ignore>
%<*driver>
\NeedsTeXFormat{LaTeX2e}
\ProvidesFile{mathtools.drv}%
  [2012/05/10 v1.12 mathematical typesetting tools]
\documentclass{ltxdoc}
\IfFileExists{fourier.sty}{\usepackage{fourier}}{}
\addtolength\marginparwidth{-25pt}
\usepackage{mathtools}

\setcounter{IndexColumns}{2}

\providecommand*\pkg[1]{\textsf{#1}}
\providecommand*\env[1]{\texttt{#1}}
\providecommand*\email[1]{\href{mailto:#1}{\texttt{#1}}}
\providecommand*\mode[1]{\texttt{[#1]}}
\providecommand*\file[1]{\texttt{#1}}
\usepackage{xcolor,varioref,amssymb}
\makeatletter
\newcommand*\thinfbox[2][black]{\fboxsep0pt\textcolor{#1}{\rulebox{{\normalcolor#2}}}}
\newcommand*\thinboxed[2][black]{\thinfbox[#1]{\ensuremath{\displaystyle#2}}}
\newcommand*\rulebox[1]{%
  \sbox\z@{\ensuremath{\displaystyle#1}}%
  \@tempdima\dp\z@
  \hbox{%
    \lower\@tempdima\hbox{%
      \vbox{\hrule height\fboxrule\box\z@\hrule height\fboxrule}%
    }%
  }%
}

\newenvironment{codesyntax}
    {\par\small\addvspace{4.5ex plus 1ex}%
     \vskip -\parskip
     \noindent
     \begin{tabular}{|l|}\hline\ignorespaces}%
    {\\\hline\end{tabular}\nobreak\par\nobreak
     \vspace{2.3ex}\vskip -\parskip\noindent\ignorespacesafterend}
\makeatletter

\newcommand*\FeatureRequest[2]{%
  \hskip1sp
  \marginpar{%
    \parbox[b]{\marginparwidth}{\small\sffamily\raggedright
      \strut Feature request by\\#1\\#2%
    }
  }%
}


\newcommand*\ProvidedBy[2]{%
  \hskip1sp
  \marginpar{%
    \parbox[b]{\marginparwidth}{\small\sffamily\raggedright
      \strut Feature provided by\\#1\\#2%
    }
  }%
}

\newcommand*\cttPosting[2]{%
  \hskip1sp
  \marginpar{%
    \parbox[b]{\marginparwidth}{\small\sffamily\raggedright
     \strut Posted on \texttt{comp.text.tex} \\#1\\#2%
    }%
  }%
}

\newcommand*\tsxPosting[2]{%
  \hskip1sp
  \marginpar{%
    \parbox[b]{\marginparwidth}{\small\sffamily\raggedright
     \strut Posted on \texttt{\small tex.stackexchange.com} \\#1\\#2%
    }%
  }%
}


\expandafter\def\expandafter\MakePrivateLetters\expandafter{%
  \MakePrivateLetters  \catcode`\_=11\relax
}

\providecommand*\SpecialOptIndex[1]{%
  \@bsphack
  \index{#1\actualchar{\protect\ttfamily #1}
          (option)\encapchar usage}%
      \index{options:\levelchar#1\actualchar{\protect\ttfamily #1}\encapchar
            usage}\@esphack}
\providecommand*\opt[1]{\texttt{#1}}

\providecommand*\SpecialKeyIndex[1]{%
  \@bsphack
  \index{#1\actualchar{\protect\ttfamily #1}
          (key)\encapchar usage}%
      \index{keys:\levelchar#1\actualchar{\protect\ttfamily #1}\encapchar
            usage}\@esphack}
\providecommand*\key[1]{\textsf{#1}}

\providecommand*\eTeX{$\m@th\varepsilon$-\TeX}

\def\MTmeta#1{%
     \ensuremath\langle
     \ifmmode \expandafter \nfss@text \fi
     {%
      \meta@font@select
      \edef\meta@hyphen@restore
        {\hyphenchar\the\font\the\hyphenchar\font}%
      \hyphenchar\font\m@ne
      \language\l@nohyphenation
      #1\/%
      \meta@hyphen@restore
     }\ensuremath\rangle
     \endgroup
}
\makeatother
\DeclareRobustCommand\meta{\begingroup\MakePrivateLetters\MTmeta}%
\def\MToarg#1{{\ttfamily[}\meta{#1}{\ttfamily]}\endgroup}
\DeclareRobustCommand\oarg{\begingroup\MakePrivateLetters\MToarg}%
\def\MHmarg#1{{\ttfamily\char`\{}\meta{#1}{\ttfamily\char`\}}\endgroup}
\DeclareRobustCommand\marg{\begingroup\MakePrivateLetters\MHmarg}%
\def\MHarg#1{{\ttfamily\char`\{#1\ttfamily\char`\}}\endgroup}
\DeclareRobustCommand\arg{\begingroup\MakePrivateLetters\MHarg}%
\def\MHcs#1{\texttt{\char`\\#1}\endgroup}
\DeclareRobustCommand\cs{\begingroup\MakePrivateLetters\MHcs}

\def\endverbatim{\if@newlist
\leavevmode\fi\endtrivlist\vspace{-\baselineskip}}
\expandafter\let\csname endverbatim*\endcsname =\endverbatim

\let\MTtheindex\theindex
\def\theindex{\MTtheindex\MakePrivateLetters}

\usepackage[final,hyperindex=false]{hyperref}
\renewcommand*\usage[1]{\textit{\hyperpage{#1}}}

\OnlyDescription
\begin{document}
  \DocInput{mathtools.dtx}
\end{document}
%</driver>
%  \fi
%
%  \changes{v1.0}{2004/07/26}{Initial release}
%
%  \GetFileInfo{mathtools.drv}
%
%  \CheckSum{2836}
%
%  \title{The \pkg{mathtools} package\thanks{This file has version number
%  \fileversion, last revised \filedate.}}
%
%  \author{Lars Madsen, Will Robertson and Joseph
%  Wright\\ (maintainers)\thanks{The maintainers would like to thank
%  Morten H\o gholm for his contributions to this package}}
%  \date{\filedate}
%
%  \maketitle
%
%  \begin{abstract}
%    The \pkg{mathtools} package is an extension package to
%    \pkg{amsmath}. There are two things on \pkg{mathtools}' agenda:
%    1)~correct various bugs/defeciencies in \pkg{amsmath} until
%    these are fixed by the \AmS{} and 2)~provide useful tools
%    for mathematical typesetting, be it a small macro for
%    typesetting a prescript or an underbracket, or entirely new
%    display math constructs such as a \env{multlined} environment.
%  \end{abstract}
%
%  \tableofcontents
%
%  \section{Introduction}
%
%  Although \pkg{amsmath} provides many handy tools for mathematical
%  typesetting, it is nonetheless a static package. This is not a bad
%  thing, because what it does, it mostly does quite well and having
%  a stable math typesetting package is ``a good thing.'' However,
%  \pkg{amsmath} does not fulfill all the needs of the mathematical
%  part of the \LaTeX{} community, resulting in many authors writing
%  small snippets of code for tweaking the mathematical layout. Some
%  of these snippets has also been posted to newsgroups and mailing
%  lists over the years, although more often than not without being
%  released as stand-alone packages.
%
%
%  The \pkg{mathtools} package is exactly what its name implies: tools
%  for mathematical typesetting. It is a collection of many of these
%  often needed small tweaks---with some big tweaks added as well. It
%  can only do so by having harvesting newsgroups for code and/or you
%  writing the maintainers with wishes for code to be included, so if
%  you have any good macros or just macros that help you when writing
%  mathematics, then don't hesitate to report them to us. We can be
%  reached at
%  \begin{quote}\email{mh.ctan@gmail.com}\end{quote}
%  This is of course also the address to use in case of bug reports.
%
%  \section{Package loading}
%
%
%  The \pkg{mathtools} package requires \pkg{amsmath} but is able to
%  pass options to it as well. Thus a line like
%  \begin{verbatim}
%    \usepackage[fleqn,tbtags]{mathtools}
%  \end{verbatim}
%  is equivalent to
%  \begin{verbatim}
%    \usepackage[fleqn,tbtags]{amsmath}
%    \usepackage{mathtools}
%  \end{verbatim}
%
%
%  \subsection{Special \pkg{mathtools} options}
%
%  \begin{codesyntax}
%  \SpecialOptIndex{fixamsmath}\opt{fixamsmath}\texttt{~~~~}
%  \SpecialOptIndex{donotfixamsmathbugs}\opt{donotfixamsmathbugs}
%  \end{codesyntax}
%  The option \opt{fixamsmath} (default) fixes two bugs in
%  \pkg{amsmath}.\footnote{See the online \LaTeX{} bugs database
%  \url{http://www.latex-project.org/cgi-bin/ltxbugs2html} under
%  \AmS\LaTeX{} problem reports 3591 and 3614.} Should you for some
%  reason not want to fix these bugs then just add the option
%  \opt{donotfixamsmathbugs} (if you can do it without typos). The
%  reason for this extremely long name is that I really don't see why
%  you wouldn't want these bugs to be fixed, so I've made it slightly
%  difficult not to fix them.
%
%  \begin{codesyntax}
%  \SpecialOptIndex{allowspaces}\opt{allowspaces}\texttt{~~~~}
%  \SpecialOptIndex{disallowspaces}\opt{disallowspaces}
%  \end{codesyntax}
%  Sometimes \pkg{amsmath} gives you nasty surprises, as here where
%  things look seemingly innocent:
%  \begin{verbatim}
%  \[
%      \begin{gathered}
%        [p] = 100 \\
%        [v] = 200
%      \end{gathered}
%  \]
%  \end{verbatim}
%  Without \pkg{mathtools} this will result in this output:
%  \[
%      \begin{gathered}[c]
%        = 100 \\
%        [v] = 200
%      \end{gathered}
%  \]
%  Yes, the \texttt{[p]} has been gobbled without any warning
%  whatsoever.\footnote{\pkg{amsmath} thought the \texttt[p] was an
%  optional argument, checked if it was \texttt{t} or \texttt{b} and
%  when both tests failed, assumed it was a \texttt{c}.} This is
%  hardly what you'd expect as an end user, as the desired output was
%  probably something like this instead:
%  \[
%      \begin{gathered}[c]
%        [p] = 100 \\
%        [v] = 200
%      \end{gathered}
%  \]
%  With the option \opt{disallowspaces} (default) \pkg{mathtools}
%  disallows spaces in front of optional arguments where it could
%  possibly cause problems just as \pkg{amsmath} does with |\\|
%  inside the display environments. This includes the environments
%  \env{gathered} (and also those shown in \S
%  \vref{subsec:gathered}), \env{aligned}, \env{multlined}, and the
%  extended \env{matrix}-environments (\S \vref{subsubsec:matrices}).
%  If you however want to preserve the more dangerous standard
%  optional spaces, simply choose the option \opt{allowspaces}.
%
%
%  \section{Tools for mathematical typesetting}
%
%  \begin{codesyntax}
%    \SpecialUsageIndex{\mathtoolsset}\cs{mathtoolsset}\marg{key val list}
%  \end{codesyntax}
%  Many of the tools shown in this manual can be turned on and off by
%  setting a switch to either true or false. In all cases it is done
%  with the command \cs{mathtoolsset}. A typical use could be something like
%  \begin{verbatim}
%    \mathtoolsset{
%      showonlyrefs,
%      mathic % or mathic = true
%    }
%  \end{verbatim}
%  More information on the keys later on.
%
%  \subsection{Fine-tuning mathematical layout}
%
%  Sometimes you need to tweak the layout of formulas a little to get
%  the best result and this part of the manual describes the various
%  macros \pkg{mathtools} provides for this.
%
%  \subsubsection{A complement to \texttt{\textbackslash smash},
%  \texttt{\textbackslash llap}, and \texttt{\textbackslash rlap}}
%
%  \begin{codesyntax}
%    \SpecialUsageIndex{\mathllap}
%    \cs{mathllap}\oarg{mathstyle}\marg{math}\texttt{~~}
%    \SpecialUsageIndex{\mathclap}
%    \cs{mathclap}\oarg{mathstyle}\marg{math}\\
%    \SpecialUsageIndex{\mathrlap}
%    \cs{mathrlap}\oarg{mathstyle}\marg{math}\texttt{~~}
%    \SpecialUsageIndex{\clap}
%    \cs{clap}\marg{text}\\
%    \SpecialUsageIndex{\mathmbox}
%    \cs{mathmbox}\marg{math}\phantom{\meta{mathstyle}}\texttt{~~~~}
%    \SpecialUsageIndex{\mathmakebox}
%    \cs{mathmakebox}\oarg{width}\oarg{pos}\marg{math}
%  \end{codesyntax}
%  In \cite{Perlis01}, Alexander R.~Perlis describes some simple yet
%  useful macros for use in math displays. For example the display
%  \begin{verbatim}
%    \[
%      X = \sum_{1\le i\le j\le n} X_{ij}
%    \]
%  \end{verbatim}
%  \[
%    X = \sum_{1\le i\le j\le n} X_{ij}
%  \]
%  contains a lot of excessive white space.  The idea that comes to
%  mind is to fake the width of the subscript. The command
%  \cs{mathclap} puts its argument in a zero width box and centers
%  it, so it could possibly be of use here.
%  \begin{verbatim}
%    \[
%      X = \sum_{\mathclap{1\le i\le j\le n}} X_{ij}
%    \]
%  \end{verbatim}
%  \[
%    X = \sum_{\mathclap{1\le i\le j\le n}} X_{ij}
%  \]
%  For an in-depth discussion of
%  these macros I find it better to read the article; an online
%  version can be found at
%  \begin{quote}
%    \url{http://www.tug.org/TUGboat/Articles/tb22-4/tb72perlS.pdf}
%  \end{quote}
%  Note that the definitions shown in the article do not exactly
%  match the definitions in \pkg{mathtools}. Besides providing an
%  optional argument for specifying the desired math style, these
%  versions also work around a most unfortunate \TeX{}
%  ``feature.''\footnote{The faulty reboxing procedure.} The
%  \cs{smash} macro is fixed too.
%
%
%  \subsubsection{Forcing a cramped style}
%
%  \begin{codesyntax}
%    \SpecialUsageIndex{\cramped}
%    \cs{cramped}\oarg{mathstyle}\marg{math}
%  \end{codesyntax}
%  \cttPosting{Michael Herschorn}{1992/07/21}
%  Let's look at another example where we have used \cs{mathclap}:
%  \begin{verbatim}
%    \begin{equation}\label{eq:mathclap}
%      \sum_{\mathclap{a^2<b^2<c}}\qquad
%      \sum_{a^2<b^2<c}
%    \end{equation}
%  \end{verbatim}
%  \begin{equation}\label{eq:mathclap}
%    \sum_{\mathclap{a^2<b^2<c}}\qquad
%    \sum_{a^2<b^2<c}
%  \end{equation}
%  Do you see the difference? Maybe if I zoomed in a bit:
%  \begingroup \fontsize{24}{\baselineskip}\selectfont
%  \[
%    \sum_{\mathclap{a^2<b^2<c}}\qquad
%    \sum_{a^2<b^2<c}
%  \]
%  \endgroup
%  Notice how the limit of the right summation sign is typeset in a
%  more compact style than the left. It is because \TeX{} sets the
%  limits of operators in a \emph{cramped} style. For each of \TeX'
%  four math styles (\cs{displaystyle}, \cs{textstyle},
%  \cs{scriptstyle}, and \cs{scriptscriptstyle}), there also exists a
%  cramped style that doesn't raise exponents as much. Besides in the
%  limits of operators, \TeX{} also automatically uses these cramped
%  styles in radicals such as \cs{sqrt} and in the denominators of
%  fractions, but unfortunately there are no primitive commands that
%  allows you to detect crampedness or switch to it.
%
%  \pkg{mathtools} offers the command \cs{cramped} which forces a
%  cramped style in normal un-cramped math. Additionally you can
%  choose which of the four styles you want it in as well by
%  specifying it as the optional argument:
%  \begin{verbatim}
%    \[
%      \cramped{x^2}               \leftrightarrow x^2    \quad
%      \cramped[\scriptstyle]{x^2} \leftrightarrow {\scriptstyle x^2}
%    \]
%  \end{verbatim}
%  \[
%    \cramped{x^2}               \leftrightarrow x^2    \quad
%    \cramped[\scriptstyle]{x^2} \leftrightarrow {\scriptstyle x^2}
%  \]
%  You may be surprised how often the cramped style can be
%  beneficial yo your output. Take a look at this example:
%  \begin{verbatim}
%    \begin{quote}
%      The 2005 Euro\TeX{} conference is held in Abbaye des
%      Pr\'emontr\'es, France, marking the 16th ($2^{2^2}$) anniversary
%      of both Dante and GUTenberg (the German and French \TeX{} users
%      group resp.).
%    \end{quote}
%  \end{verbatim}
%  \begin{quote}
%    The 2005 Euro\TeX{} conference is held in Abbaye des
%    Pr\'emontr\'es, France, marking the 16th ($2^{2^2}$) anniversary
%    of both Dante and GUTenberg (the German and French \TeX{} users
%    group resp.).
%  \end{quote}
%  Typesetting on a grid is generally considered quite desirable, but
%  as the second line of the example shows, the exponents of $2$
%  causes the line to be too tall for the normal value of
%  \cs{baselineskip}, so \TeX{} inserts a \cs{lineskip} (normal value
%  is \the\lineskip). In order to circumvent the problem, we can
%  force a cramped style so that the exponents aren't raised as much:
%  \begin{verbatim}
%    \begin{quote}
%      The 2005 Euro\TeX{} ... 16th ($\cramped{2^{2^2}}$) ...
%    \end{quote}
%  \end{verbatim}
%  \begin{quote}
%    The 2005 Euro\TeX{} conference is held in Abbaye des
%    Pr\'emontr\'es, France, marking the 16th ($\cramped{2^{2^2}}$)
%    anniversary of both Dante and GUTenberg (the German and French
%    \TeX{} users group resp.).
%  \end{quote}
%
%  \begin{codesyntax}
%    \SpecialUsageIndex{\crampedllap}
%    \cs{crampedllap}\oarg{mathstyle}\marg{math}\texttt{~~}
%    \SpecialUsageIndex{\crampedclap}
%    \cs{crampedclap}\oarg{mathstyle}\marg{math}\\
%    \SpecialUsageIndex{\crampedrlap}
%    \cs{crampedrlap}\oarg{mathstyle}\marg{math}
%  \end{codesyntax}
%  The commands \cs{crampedllap}, \cs{crampedclap}, and
%  \cs{crampedrlap} are identical to the three \cs{mathXlap} commands
%  described earlier except the argument is typeset in cramped style.
%  You need this in order to typeset \eqref{eq:mathclap} correctly
%  while still faking the width of the limit.
%  \begin{verbatim}
%    \begin{equation*}\label{eq:mathclap-b}
%      \sum_{\crampedclap{a^2<b^2<c}}
%      \tag{\ref{eq:mathclap}*}
%    \end{equation*}
%  \end{verbatim}
%  \begin{equation*}\label{eq:mathclap-b}
%    \sum_{\crampedclap{a^2<b^2<c}}
%    \tag{\ref{eq:mathclap}*}
%  \end{equation*}
%  Of course you could just type
%  \begin{verbatim}
%    \sum_{\mathclap{\cramped{a^2<b^2<c}}}
%  \end{verbatim}
%  but it has one major disadvantage: In order for \cs{mathXlap} and
%  \cs{cramped} to get the right size, \TeX{} has to process them
%  four times, meaning that nesting them as shown above will cause
%  \TeX{} to typeset $4^2$ instances before choosing the right one.
%  In this situation however, we will of course need the same style
%  for both commands so it makes sense to combine the commands in
%  one, thus letting \TeX{} make the choice only once rather than
%  twice.
%
%
%
%  \subsubsection{Smashing an operator}
%
%
%
%  \begin{codesyntax}
%    \SpecialUsageIndex{\smashoperator}
%    \cs{smashoperator}\oarg{pos}\marg{operator with limits}
%  \end{codesyntax}
%  \FeatureRequest{Lars Madsen}{2004/05/04}
%  Above we shoved how to get \LaTeX{} to ignore the width of the
%  subscript of an operator. However this approach takes a lot of
%  extra typing, especially if you have a wide superscript, meaning
%  you have to put in \cs{crampedclap} in both sub- and superscript.
%  To make things easier, \pkg{mathtools} provides a
%  \cs{smashoperator} command, which simply ignores the width of the
%  sub- and superscript. It also takes an optional argument,
%  \texttt{l}, \texttt{r}, or \texttt{lr} (default), denoting which
%  side of the operator should be ignored (smashed).
%  \begin{verbatim}
%    \[
%      V = \sum_{1\le i\le j\le n}^{\infty} V_{ij}                  \quad
%      X = \smashoperator{\sum_{1\le i\le j\le n}^{3456}} X_{ij}    \quad
%      Y = \smashoperator[r]{\sum\limits_{1\le i\le j\le n}} Y_{ij} \quad
%      Z = \smashoperator[l]{\mathop{T}_{1\le i\le j\le n}} Z_{ij}
%    \]
%  \end{verbatim}
%    \[
%      V = \sum_{1\le i\le j\le n}^{\infty} V_{ij}                  \quad
%      X = \smashoperator{\sum_{1\le i\le j\le n}^{3456}} X_{ij}    \quad
%      Y = \smashoperator[r]{\sum\limits_{1\le i\le j\le n}} Y_{ij} \quad
%      Z = \smashoperator[l]{\mathop{T}_{1\le i\le j\le n}} Z_{ij}
%    \]
%  Note that \cs{smashoperator} always sets its argument in display
%  style and with limits even if you have used the \opt{nosumlimits}
%  option of \pkg{amsmath}. If you wish, you can use shorthands for
%  \texttt{\_} and \texttt{\textasciicircum} such as \cs{sb} and
%  \cs{sp}.
%
%
%  \subsubsection{Adjusting limits of operators}
%
%  \begin{codesyntax}
%    \SpecialUsageIndex{\adjustlimits}
%    \cs{adjustlimits}\marg{operator$\sb1$}\texttt{\_}\marg{limit$\sb1$}
%                     \marg{operator$\sb2$}\texttt{\_}\marg{limit$\sb2$}
%  \end{codesyntax}
%  \FeatureRequest{Lars Madsen}{2004/07/09}
%  When typesetting two consecutive operators with limits one often
%  wishes the limits of the operators were better aligned. Look
%  closely at these examples:
%  \begin{verbatim}
%    \[
%      \text{a)} \lim_{n\to\infty} \max_{p\ge n} \quad
%      \text{b)} \lim_{n\to\infty} \max_{p^2\ge n} \quad
%      \text{c)} \lim_{n\to\infty} \sup_{p^2\ge nK} \quad
%      \text{d)} \limsup_{n\to\infty} \max_{p\ge n}
%    \]
%  \end{verbatim}
%  \[
%    \text{a)} \lim_{n\to\infty} \max_{p\ge n} \quad
%    \text{b)} \lim_{n\to\infty} \max_{p^2\ge n} \quad
%    \text{c)} \lim_{n\to\infty} \sup_{p^2\ge nK} \quad
%    \text{d)} \limsup_{n\to\infty} \max_{p\ge n}
%  \]
%  a) looks okay, but b) is not quite as good because the second
%  limit ($\cramped{p^2\ge n}$) is significantly taller than the
%  first ($n\to\infty$). With c)~things begin to look really bad,
%  because the second operator has a descender while the first
%  doesn't, and finally we have d)~which looks just as bad as~c). The
%  command \cs{adjustlimits} is useful in these cases, as you can
%  just put it in front of these consecutive operators and it'll make
%  the limits line up.
%  \medskip\par\noindent
%  \begin{minipage}{\textwidth}
%  \begin{verbatim}
%    \[
%      \text{a)} \adjustlimits\lim_{n\to\infty} \max_{p\ge n} \quad
%      \text{b)} \adjustlimits\lim_{n\to\infty} \max_{p^2\ge n} \quad
%      \text{c)} \adjustlimits\lim_{n\to\infty} \sup_{p^2\ge nK} \quad
%      \text{d)} \adjustlimits\limsup_{n\to\infty} \max_{p\ge n}
%    \]
%  \end{verbatim}
%  \end{minipage}
%  \[
%      \text{a)} \adjustlimits\lim_{n\to\infty} \max_{p\ge n} \quad
%      \text{b)} \adjustlimits\lim_{n\to\infty} \max_{p^2\ge n} \quad
%      \text{c)} \adjustlimits\lim_{n\to\infty} \sup_{p^2\ge nK} \quad
%      \text{d)} \adjustlimits\limsup_{n\to\infty} \max_{p\ge n}
%    \]
%  The use of \cs{sb} instead of \texttt{\_} is allowed.
%
%
%  \subsubsection{Swapping space above \texorpdfstring{\AmS}{AMS} display math environments }
%  \label{sec:swapping}
%
%  One feature that the plain old \env{equation} environment has that
%  the \AmS\ environments does not (because of thechnical reasons), is
%  the feature of using less space above the equation if the situation
%  presents itself. The \AmS\ environments cannot do this, but one can
%  manually, using  
%  \begin{codesyntax}
%    \SpecialUsageIndex{\SwapAboveDisplaySkip}
%    \cs{SwapAboveDisplaySkip}
%  \end{codesyntax}
%  as the very first content within an \AmS\ display math
%  environment. It will then issue an \cs{abovedisplayshortskip}
%  instead of the normal \cs{abovedisplayskip}.
%
%  Note it will not work with the \env{equation} or \env{multline} environments.
%  
%  Here is an example of the effect
% \begin{verbatim}
%  \noindent\rule\textwidth{1pt}
%  \begin{align*}  A &= B \end{align*}
%  \noindent\rule\textwidth{1pt}
%  \begin{align*}  
%  \SwapAboveDisplaySkip
%  A &= B 
%  \end{align*}
% \end{verbatim}
%  \noindent\rule\textwidth{1pt}
%  \begin{align*}  A &= B \end{align*}
%  \noindent\rule\textwidth{1pt}
%  \begin{align*}  
%  \SwapAboveDisplaySkip
%  A &= B 
%  \end{align*}
%  
%
%  \subsection{Controlling tags}
%
%  In this section various tools for altering the appearance of tags
%  are shown. All of the tools here can be used at any point in the
%  document but they should probably be affect the whole document, so
%  the preamble is the best place to issue them.
%
%  \subsubsection{The appearance of tags}
%  \begin{codesyntax}
%    \SpecialUsageIndex{\newtagform}
%    \cs{newtagform}\marg{name}\oarg{inner_format}\marg{left}\marg{right}\\
%    \SpecialUsageIndex{\renewtagform}
%    \cs{renewtagform}\marg{name}\oarg{inner_format}\marg{left}\marg{right}\\
%    \SpecialUsageIndex{\usetagform}
%    \cs{usetagform}\marg{name}
%  \end{codesyntax}
%  Altering the layout of equation numbers in \pkg{amsmath} is not
%  very user friendly (it involves a macro with three \texttt{@}'s in
%  its name), so \pkg{mathtools} provides an interface somewhat
%  reminiscent of the page style concept. This way you can define
%  several different tag forms and then choose the one you prefer.
%
%  As an example let's try to define a tag form which puts the
%  equation number in square brackets. First we define a brand new tag
%  form:
%  \begin{verbatim}
%    \newtagform{brackets}{[}{]}
%  \end{verbatim}
%  Then we activate it:
%  \begin{verbatim}
%    \usetagform{brackets}
%  \end{verbatim}
%  The result is then
%  \newtagform{brackets}{[}{]}
%  \usetagform{brackets}
%   \begin{equation}
%     E \neq m c^3
%   \end{equation}
%
%  Similarly you could define a second version of the brackets that
%  prints the equation number in bold face instead
%  \begin{verbatim}
%    \newtagform{brackets2}[\textbf]{[}{]}
%    \usetagform{brackets2}
%    \begin{equation}
%      E \neq m c^3
%    \end{equation}
%  \end{verbatim}
%  \newtagform{brackets2}[\textbf]{[}{]}
%  \usetagform{brackets2}
%  \begin{equation}
%    E \neq m c^3
%  \end{equation}
%  When you reference an equation with \cs{eqref}, the tag form in
%  effect at the time of referencing controls the formatting, so be
%  careful if you use different tag forms throughout your document.
%
%  If you want to renew a tag form, then use the command
%  \cs{renewtagform}. Should you want to
%  return to the standard setting then choose\usetagform{default}
%  \begin{verbatim}
%    \usetagform{default}
%  \end{verbatim}
%
%  \changes{v1.12}{2012/05/09}{Added caveat}
%  \noindent\textbf{Caveat regarding \pkg{ntheorem}}: If you like to
%  change the appearence of the tags \emph{and} you are also using the
%  \pkg{ntheorem} package, then please postpone the change of
%  appearance until \emph{after} loading \pkg{ntheorem}. (In order to
%  do its thing, \pkg{ntheorem} has to mess with the tags\dots)
%
%  \subsubsection{Showing only referenced tags}
%
%  \begin{codesyntax}
%    \SpecialKeyIndex{showonlyrefs}$\key{showonlyrefs}=\texttt{true}\vert\texttt{false}$\\
%    \SpecialKeyIndex{showmanualtags}$\key{showmanualtags}=\texttt{true}\vert\texttt{false}$\\
%    \SpecialUsageIndex{\refeq}\cs{refeq}\marg{label}
%  \end{codesyntax}
%  An equation where the tag is produced with a manual \cs{tag*}
%  shouldn't be referenced with the normal \cs{eqref} because that
%  would format it according to the current tag format. Using just
%  \cs{ref} on the other hand may not be a good solution either as
%  the argument of \cs{tag*} is always set in upright shape in the
%  equation and you may be referencing it in italic text. In the
%  example below, the command \cs{refeq} is used to avoid what could
%  possibly lead to confusion in cases where the tag font has very
%  different form in upright and italic shape (here we switch to
%  Palatino in the example):
%    \begin{verbatim}
%    \begin{quote}\renewcommand*\rmdefault{ppl}\normalfont\itshape
%    \begin{equation*}
%      a=b \label{eq:example}\tag*{Q\&A}
%    \end{equation*}
%    See \ref{eq:example} or is it better with \refeq{eq:example}?
%    \end{quote}
%  \end{verbatim}
%    \begin{quote}\renewcommand*\rmdefault{ppl}\normalfont\itshape
%    \begin{equation*}
%      a=b \label{eq:example}\tag*{Q\&A}
%    \end{equation*}
%    See \ref{eq:example} or is it better with \refeq{eq:example}?
%  \end{quote}
%
%
%  Another problem sometimes faced is the need for showing the
%  equation numbers for only those equations actually referenced. In
%  \pkg{mathtools} this can be done by setting the key
%  \key{showonlyrefs} to either true or false by using
%  \cs{mathtoolsset}. You can also choose whether or not to show the
%  manual tags specified with \cs{tag} or \cs{tag*} by setting the
%  option \key{showmanualtags} to true or false.\footnote{I recommend
%  setting \key{showmanualtags} to true, else the whole idea of using
%  \cs{tag} doesn't really make sense, does it?} For both keys just
%  typing the name of it chooses true as shown in the following
%  example.
%
%  \begin{verbatim}
%  \mathtoolsset{showonlyrefs,showmanualtags}
%  \usetagform{brackets}
%  \begin{gather}
%    a=a \label{eq:a} \\
%    b=b \label{eq:b} \tag{**}
%  \end{gather}
%  This should refer to the equation containing $a=a$: \eqref{eq:a}.
%  Then a switch of tag forms.
%  \usetagform{default}
%  \begin{align}
%    c&=c \label{eq:c} \\
%    d&=d \label{eq:d}
%  \end{align}
%  This should refer to the equation containing $d=d$: \eqref{eq:d}.
%  \begin{equation}
%    e=e
%  \end{equation}
%  Back to normal.\mathtoolsset{showonlyrefs=false}
%  \begin{equation}
%    f=f
%  \end{equation}
%  \end{verbatim}
%  \mathtoolsset{showonlyrefs,showmanualtags}
%  \usetagform{brackets}
%  \begin{gather}
%    a=a \label{eq:a} \\
%    b=b \label{eq:b} \tag{**}
%  \end{gather}
%  This should refer to the equation containing $a=a$: \eqref{eq:a}.
%  Then a switch of tag forms.
%  \usetagform{default}
%  \begin{align}
%    c&=c \label{eq:c} \\
%    d&=d \label{eq:d}
%  \end{align}
%  This should refer to the equation containing $d=d$: \eqref{eq:d}.
%  \begin{equation}
%    e=e
%  \end{equation}
%  Back to normal.\mathtoolsset{showonlyrefs=false}
%  \begin{equation}
%    f=f
%  \end{equation}
%
%  Note that this feature only works if you use \cs{eqref} or
%  \cs{refeq} to reference your equations.
%
%  When using \key{showonlyrefs} it might be useful to be able to
%  actually add a few equation numbers without directly referring to
%  them.
%  \begin{codesyntax}
%    \SpecialUsageIndex{\noeqref}\cs{noeqref}\marg{label,label,\dots}
%  \end{codesyntax}
%  \FeatureRequest{Rasmus Villemoes}{2008/03/26}
%  The syntax is somewhat similar to \cs{nocite}. If a label in the
%  list is undefined we will throw a warning in the same manner as
%  \cs{ref}. 
%
%  \medskip\noindent\textbf{BUG 1:} Unfortunately the use of the
%  \key{showonlyref} introduce a bug within amsmath's typesetting
%  of formula versus equation number. This bug manifest itself by
%  allowing formulas to be typeset close to or over the equation
%  number.  Currently no general fix is known, other than making sure
%  that one's formulas are not long enough to touch the equation
%  number.
%
%  To make a long story stort, amsmath typesets its math environments
%  twice, one time for measuring and one time for the actual
%  typesetting. In the measuring part, the width of the equation
%  number is recorded such that the formula or the equation number can
%  be moved (if necessary) in the typesetting part. When
%  \key{showonlyref} is enabled, the width of the equation number
%  depend on whether or not this number is referred~to. To determine
%  this, we need to know the current label. But the current label is
%  \emph{not} known in the measuring phase. Thus the measured width is
%  always zero (because no label equals not referred to) and therefore
%  the typesetting phase does not take the equation number into
%  account.
%
% \medskip\noindent\textbf{BUG 2:} Currently there is a bug between
% \key{showonlyrefs} and the \pkg{ntheorem} package, when the
% \pkg{ntheorem} option \key{thmmarks} is active. The shown equation
% numbers may come out wrong (seems to be multiplied by 2). The
% easiest fix is to add the following line
% \begin{verbatim}
% \usepackage[overload,ntheorem]{empheq}
% \end{verbatim}
% before loading \pkg{ntheorem}. The \pkg{empheq} package fixes some
% problems with \pkg{ntheorem} and lets \pkg{mathtools} get correct
% access to the equation numbers again.
% 
%  \subsection{Extensible symbols}
%
%  The number of horizontally extensible symbols in standard \LaTeX{}
%  and \pkg{amsmath} is somewhat low. This part of the manual
%  describes what \pkg{mathtools} does to help this situation.
%
%  \subsubsection{Arrow-like symbols}
%
%
%  \begin{codesyntax}
%    \SpecialUsageIndex{\xleftrightarrow}
%    \cs{xleftrightarrow}\oarg{sub}\marg{sup}\texttt{~~~~~~~~~}
%    \SpecialUsageIndex{\xRightarrow}
%    \cs{xRightarrow}\oarg{sub}\marg{sup}\\
%    \SpecialUsageIndex{\xLeftarrow}
%    \cs{xLeftarrow}\oarg{sub}\marg{sup}\texttt{~~~~~~~~~~~~~~}
%    \SpecialUsageIndex{\xLeftrightarrow}
%    \cs{xLeftrightarrow}\oarg{sub}\marg{sup}\\
%    \SpecialUsageIndex{\xhookleftarrow}
%    \cs{xhookleftarrow}\oarg{sub}\marg{sup}\texttt{~~~~~~~~~~}
%    \SpecialUsageIndex{\xhookrightarrow}
%    \cs{xhookrightarrow}\oarg{sub}\marg{sup}\\
%    \SpecialUsageIndex{\xmapsto}
%    \cs{xmapsto}\oarg{sub}\marg{sup}
%  \end{codesyntax}
%  Extensible arrows are part of \pkg{amsmath} in the form of the
%  commands
%  \begin{quote}
%    \cs{xrightarrow}\oarg{subscript}\marg{superscript}\quad and\\
%    \cs{xleftarrow}\oarg{subscript}\marg{superscript}
%  \end{quote}
%  But what about extensible versions of say, \cs{leftrightarrow} or
%  \cs{Longleftarrow}? It turns out that the above mentioned
%  extensible arrows are the only two of their kind defined by
%  \pkg{amsmath}, but luckily \pkg{mathtools} helps with that. The
%  extensible arrow-like symbols in \pkg{mathtools} follow the same
%  naming scheme as the one's in \pkg{amsmath} so to get an extensible
%  \cs{Leftarrow} you simply do a
%  \begin{verbatim}
%    \[
%      A \xLeftarrow[under]{over} B
%    \]
%  \end{verbatim}
%    \[
%      A \xLeftarrow[under]{over} B
%    \]
%  \begin{codesyntax}
%    \SpecialUsageIndex{\xrightharpoondown}
%    \cs{xrightharpoondown}\oarg{sub}\marg{sup}\texttt{~~~~}
%    \SpecialUsageIndex{\xrightharpoonup}
%    \cs{xrightharpoonup}\oarg{sub}\marg{sup}\\
%    \SpecialUsageIndex{\xleftharpoondown}
%    \cs{xleftharpoondown}\oarg{sub}\marg{sup}\texttt{~~~~~}
%    \SpecialUsageIndex{\xleftharpoonup}
%    \cs{xleftharpoonup}\oarg{sub}\marg{sup}\\
%    \SpecialUsageIndex{\xrightleftharpoons}
%    \cs{xrightleftharpoons}\oarg{sub}\marg{sup}\texttt{~~~}
%    \SpecialUsageIndex{\xleftrightharpoons}
%    \cs{xleftrightharpoons}\oarg{sub}\marg{sup}
%  \end{codesyntax}
%  \pkg{mathtools} also provides the extensible harpoons shown above.
%  They're taken from~\cite{Voss:2004}.
%
%  \subsubsection{Braces and brackets}
%
%  \LaTeX{} defines other kinds of extensible symbols like
%  \cs{overbrace} and \cs{underbrace}, but sometimes you may want
%  another symbol, say, a bracket.
%  \begin{codesyntax}
%    \SpecialUsageIndex{\underbracket}\cs{underbracket}\oarg{rule thickness}
%      \oarg{bracket height}\marg{arg}\\
%    \SpecialUsageIndex{\overbracket}\cs{overbracket}\oarg{rule thickness}
%      \oarg{bracket height}\marg{arg}
%  \end{codesyntax}
%  The commands \cs{underbracket} and \cs{overbracket} are inspired
%  by \cite{Voss:2004}, although the implementation here is slightly
%  different.
%  Used without the optional arguments the bracket commands produce this:
%  \begin{quote}
%   |$\underbracket {foo\ bar}_{baz}$|\quad  $\underbracket {foo\ bar}_{baz}$ \\
%   |$\overbracket {foo\ bar}^{baz}$ |\quad  $\overbracket {foo\ bar}^{baz}$
%  \end{quote}
%  The default rule thickness is equal to that of \cs{underbrace}
%  (app.~$5/18$\,ex) while the default bracket height is equal to
%  app.~$0.7$\,ex. These values give really pleasing results in all
%  font sizes, but feel free to use the optional arguments. That way
%  you may get ``beauties'' like
%  \begin{verbatim}
%    \[
%      \underbracket[3pt]{xxx\  yyy}_{zzz} \quad \text{and} \quad
%      \underbracket[1pt][7pt]{xxx\  yyy}_{zzz}
%    \]
%  \end{verbatim}
%    \[
%      \underbracket[3pt]{xxx\  yyy}_{zzz} \quad \text{and} \quad
%      \underbracket[1pt][7pt]{xxx\  yyy}_{zzz}
%    \]
%  \begin{codesyntax}
%    \SpecialUsageIndex{\underbrace}\cs{underbrace}\marg{arg}\texttt{~~}
%    \SpecialUsageIndex{\LaTeXunderbrace}\cs{LaTeXunderbrace}\marg{arg}\\
%    \SpecialUsageIndex{\overbrace}\cs{overbrace}\marg{arg}\texttt{~~~}
%    \SpecialUsageIndex{\LaTeXoverbrace}\cs{LaTeXoverbrace}\marg{arg}
%  \end{codesyntax}
%  The standard implementation of the math operators \cs{underbrace}
%  and \cs{overbrace} in \LaTeX{} has some deficiencies. For example,
%  all lengths used internally are \emph{fixed} and optimized for
%  10\,pt typesetting. As a direct consequence thereof, using font
%  sizes other than 10 will produce less than optimal results.
%  Another unfortunate feature is the size of the braces. In the
%  example below, notice how the math operator \cs{sum} places its
%  limit compared to \cs{underbrace}.
%  \[
%    \mathop{\thinboxed[blue]{\sum}}_{n}
%    \mathop{\thinboxed[blue]{\LaTeXunderbrace{\thinboxed[green]{foof}}}}_{zzz}
%  \]
%  The blue lines indicate the dimensions of the math operator and
%  the green lines the dimensions of $foof$. As you can see, there
%  seems to be too much space between the brace and the $zzz$ whereas
%  the space between brace and $foof$ is okay. Let's see what happens
%  when we use a bigger font size:\par\Huge\vskip-\baselineskip
%  \[
%    \mathop{\thinboxed[blue]{\sum}}_{n}
%    \mathop{\thinboxed[blue]{\LaTeXunderbrace{\thinboxed[green]{foof}}}}_{zzz}
%  \]
%  \normalsize Now there's too little space between the brace and the
%  $zzz$ and also too little space between the brace and the $foof$.
%  If you use Computer Modern you'll actually see that the $f$
%  overlaps with the brace! Let's try in \cs{footnotesize}:
%  \par\footnotesize
%  \[
%    \mathop{\thinboxed[blue]{\sum}}_{n}
%    \mathop{\thinboxed[blue]{\LaTeXunderbrace{\thinboxed[green]{foof}}}}_{zzz}
%  \]\normalsize
%  Here the spacing above and below the brace is quite excessive.
%
%  As \cs{overbrace} has the exact same problems, there are good
%  reasons for \pkg{mathtools} to make redefinitions of
%  \cs{underbrace} and \cs{overbrace}. These new versions work
%  equally well in all font sizes and fixes the spacing issues and
%  apart from working with the default Computer Modern fonts, they
%  also work with the packages \pkg{mathpazo}, \pkg{pamath},
%  \pkg{fourier}, \pkg{eulervm}, \pkg{cmbright}, and \pkg{mathptmx}.
%  If you use the \pkg{ccfonts} to get the full Concrete fonts, the
%  original version saved under the names \cs{LaTeXunderbrace} and
%  \cs{LaTeXoverbrace} are better, due to of the special design of
%  the Concrete extensible braces. In that case you should probably
%  just add the lines
%  \begin{verbatim}
%    \let\underbrace\LaTeXunderbrace
%    \let\overbrace\LaTeXoverbrace
%  \end{verbatim}
%  to your preamble after loading \pkg{mathtools} which will restore
%  the original definitions of \cs{overbrace} and \cs{underbrace}.
%
%
%  
%
%  \subsection{New mathematical building blocks}
%
%  In this part of the manual, various mathematical environments are
%  described.
%
%  \subsubsection{Matrices}\label{subsubsec:matrices}
%
%  \begin{codesyntax}
%    \SpecialEnvIndex{matrix*}\cs{begin}\arg{matrix*}\texttt{ }\oarg{col}
%        \meta{contents} \cs{end}\arg{matrix*}\\
%    \SpecialEnvIndex{pmatrix*}\cs{begin}\arg{pmatrix*}\oarg{col}
%        \meta{contents} \cs{end}\arg{pmatrix*}\\
%    \SpecialEnvIndex{bmatrix*}\cs{begin}\arg{bmatrix*}\oarg{col}
%        \meta{contents} \cs{end}\arg{bmatrix*}\\
%    \SpecialEnvIndex{Bmatrix*}\cs{begin}\arg{Bmatrix*}\oarg{col}
%        \meta{contents} \cs{end}\arg{Bmatrix*}\\
%    \SpecialEnvIndex{vmatrix*}\cs{begin}\arg{vmatrix*}\oarg{col}
%        \meta{contents} \cs{end}\arg{vmatrix*}\\
%    \SpecialEnvIndex{Vmatrix*}\cs{begin}\arg{Vmatrix*}\oarg{col}
%        \meta{contents} \cs{end}\arg{Vmatrix*}
%  \end{codesyntax}
%  \FeatureRequest{Lars Madsen}{2004/04/05}
%  All of the \pkg{amsmath} \env{matrix} environments center the
%  columns by default, which is not always what you want. Thus
%  \pkg{mathtools} provides a starred version for each of the original
%  environments. These starred environments take an optional argument
%  specifying the alignment of the columns, so that
%  \begin{verbatim}
%    \[
%      \begin{pmatrix*}[r]
%        -1 & 3 \\
%        2  & -4
%      \end{pmatrix*}
%    \]
%  \end{verbatim}
%  yields
%    \[
%      \begin{pmatrix*}[r]
%        -1 & 3 \\
%        2  & -4
%      \end{pmatrix*}
%    \]
%  The optional argument (default is \texttt{[c]}) can be any column
%  type valid in the usual \env{array} environment.
%
%  While we are at it, we also provide fenced versions of the
%  \env{smallmatrix} environment, To keep up with the naming of the
%  large matrix environments, we provide both a starred and a
%  non-starred version. Since \env{smallmatrix} is defined in a
%  different manner than the \env{matrix} environment, the option to
%  say \env{smallmatrix*} \emph{has} to be either \texttt{c},
%  \texttt{l} \emph{or}~\texttt{r}. The default is \texttt{c}, which
%  can be changed globally using the \key{smallmatrix-align}=\meta{c,l
%    or r}.
%  \begin{codesyntax}
%    \SpecialEnvIndex{smallmatrix*}\cs{begin}\arg{smallmatrix*}\texttt{ }\oarg{col}
%        \meta{contents} \cs{end}\arg{smallmatrix*}\\
%    \SpecialEnvIndex{psmallmatrix}\cs{begin}\arg{psmallmatrix}
%        \meta{contents} \cs{end}\arg{psmallmatrix}\\
%    \SpecialEnvIndex{psmallmatrix*}\cs{begin}\arg{psmallmatrix*}\oarg{col}
%        \meta{contents} \cs{end}\arg{psmallmatrix*}\\
%    \SpecialEnvIndex{bsmallmatrix}\cs{begin}\arg{bsmallmatrix}
%        \meta{contents} \cs{end}\arg{bsmallmatrix}\\
%    \SpecialEnvIndex{bsmallmatrix*}\cs{begin}\arg{bsmallmatrix*}\oarg{col}
%        \meta{contents} \cs{end}\arg{bsmallmatrix*}\\
%    \SpecialEnvIndex{Bsmallmatrix}\cs{begin}\arg{Bsmallmatrix}
%        \meta{contents} \cs{end}\arg{Bsmallmatrix}\\
%    \SpecialEnvIndex{Bsmallmatrix*}\cs{begin}\arg{Bsmallmatrix*}\oarg{col}
%        \meta{contents} \cs{end}\arg{Bsmallmatrix*}\\
%    \SpecialEnvIndex{vsmallmatrix}\cs{begin}\arg{vsmallmatrix}
%        \meta{contents} \cs{end}\arg{vsmallmatrix}\\
%    \SpecialEnvIndex{vsmallmatrix*}\cs{begin}\arg{vsmallmatrix*}\oarg{col}
%        \meta{contents} \cs{end}\arg{vsmallmatrix*}\\
%    \SpecialEnvIndex{Vsmallmatrix}\cs{begin}\arg{Vsmallmatrix}
%        \meta{contents} \cs{end}\arg{Vsmallmatrix}\\
%    \SpecialEnvIndex{Vsmallmatrix*}\cs{begin}\arg{Vsmallmatrix*}\oarg{col}
%        \meta{contents} \cs{end}\arg{Vsmallmatrix*}\\
%    \SpecialKeyIndex{smallmatrix-align}\makebox{$\key{smallmatrix-align}=\meta{c,l or r}$}\\
%    \SpecialKeyIndex{smallmatrix-inner-space}\makebox{$\key{smallmatrix-inner-space}=\cs{,}$}
%  \end{codesyntax}
%  \ProvidedBy{Rasmus Villemoes}{2011/01/17}
% \begin{verbatim}
% \[
% \begin{bsmallmatrix}    a & -b \\ -c & d \end{bsmallmatrix}
% \begin{bsmallmatrix*}[r] a & -b \\ -c & d \end{bsmallmatrix*}
% \]
% \end{verbatim}
%  yields
% \[
% \begin{bsmallmatrix} a & -b \\ -c & d \end{bsmallmatrix}
% \begin{bsmallmatrix*}[r] a & -b \\ -c & d \end{bsmallmatrix*}
% \]
% Inside the \verb?Xsmallmatrix? construction a small space is
% inserted between the fences and the contents, the size of it can be
% changed using \key{smallmatrix-align}=\meta{some spacing command},
% the default is \cs{,}.
%
% As an extra trick the fences will behave as open and closing fences
% in constract to their auto-scaling nature.\footnote{\cs{left} and
%   \cs{right} do \emph{not} produce open and closing fences, thus
%   the space before or after may be too large. Inside this
%   construction they behave.}
% 
%  \subsubsection{The \env{multlined} environment}
%
%  \begin{codesyntax}
%    \SpecialEnvIndex{multlined}\cs{begin}\arg{multlined}\oarg{pos}\oarg{width}
%        \meta{contents} \cs{end}\arg{multlined}\\
%    \SpecialUsageIndex{\shoveleft}\cs{shoveleft}\oarg{dimen}\marg{arg}\texttt{~~}
%    \SpecialUsageIndex{\shoveright}\cs{shoveright}\oarg{dimen}\marg{arg}\\
%    \makeatletter\settowidth\@tempdimc{\cs{shoveleft}\oarg{dimen}\marg{arg}}\global\@tempdimc\@tempdimc
%    \SpecialKeyIndex{firstline-afterskip}\makebox[\@tempdimc][l]{$\key{firstline-afterskip}=\meta{dimen}$}\texttt{~~}
%    \SpecialKeyIndex{lastline-preskip}$\key{lastline-preskip}=\meta{dimen}$\\
%    \makeatletter\SpecialKeyIndex{multlined-width}\makebox[\@tempdimc][l]{$\key{multlined-width}=\meta{dimen}$}\texttt{~~}
%    \SpecialKeyIndex{multlined-pos}$\key{multlined-pos}=\texttt{c}\vert\texttt{b}\vert\texttt{t}$
%  \end{codesyntax}
%  Some of the \pkg{amsmath} environments exist in two forms: an
%  outer and an inner environment. One example is the pair
%  \env{gather} \& \env{gathered}. There is one important omission on
%  this list however, as there is no inner \env{multlined}
%  environment, so this is where \pkg{mathtools} steps in.
%
%  One might wonder what the sensible behavior should be. We want it
%  to be an inner environment so that it is not wider than necessary,
%  but on the other hand we would like to be able to control the
%  width. The current implementation of \env{multlined} handles both
%  cases. The idea is this: Set the first line flush left and add a
%  hard space after it; this space is governed by the
%  \key{firstline-afterskip} key. The last line should be set flush
%  right and preceded by a hard space of size \key{lastline-preskip}.
%  Both these hard spaces have a default value of \cs{multlinegap}.
%  Here we use a `t' in the first optional argument denoting a
%  top-aligned building block (the default is `c').
%  \begin{verbatim}
%    \[
%      A = \begin{multlined}[t]
%            \framebox[4cm]{first} \\
%            \framebox[4cm]{last}
%          \end{multlined} B
%    \]
%  \end{verbatim}
%    \[
%      A = \begin{multlined}[t]
%            \framebox[4cm]{first} \\
%            \framebox[4cm]{last}
%          \end{multlined} B
%    \]
%  Note also that \env{multlined} gives you access to an extended
%  syntax for \cs{shoveleft} and \cs{shoveright} as shown in the
%  example below.
%  \begin{verbatim}
%    \[
%      \begin{multlined}
%        \framebox[.65\columnwidth]{First line}        \\
%        \framebox[.5\columnwidth]{Second line}        \\
%        \shoveleft{L+E+F+T}                           \\
%        \shoveright{R+I+G+H+T}                        \\
%        \shoveleft[1cm]{L+E+F+T}                      \\
%        \shoveright[\widthof{$R+I+G+H+T$}]{R+I+G+H+T} \\
%        \framebox[.65\columnwidth]{Last line}
%      \end{multlined}
%    \]
%  \end{verbatim}
%  \[
%    \begin{multlined}
%      \framebox[.65\columnwidth]{First line} \\
%      \framebox[.5\columnwidth]{Second line} \\
%      \shoveleft{L+E+F+T}         \\
%      \shoveright{R+I+G+H+T}         \\
%      \shoveleft[1cm]{L+E+F+T}         \\
%      \shoveright[\widthof{$R+I+G+H+T$}]{R+I+G+H+T}         \\
%      \framebox[.65\columnwidth]{Last line}
%    \end{multlined}
%  \]
%
%  You can also choose the width yourself by specifying it as an
%  optional argument:
%  \begin{verbatim}
%    \[
%      \begin{multlined}[b][7cm]
%        \framebox[4cm]{first} \\
%        \framebox[4cm]{last}
%      \end{multlined} = B
%    \]
%  \end{verbatim}
%    \[
%       \begin{multlined}[b][7cm]
%            \framebox[4cm]{first} \\
%            \framebox[4cm]{last}
%          \end{multlined} = B
%    \]
%  There can be two optional arguments (position and width) and
%  they're interchangeable.
%
%  \subsubsection{More \env{cases}-like environments}
%
%  \begin{codesyntax}
%    \SpecialEnvIndex{dcases}
%    \cs{begin}\arg{dcases}\texttt{~}  \meta{math_column} |&| \meta{math_column}
%    \cs{end}\arg{dcases}\\
%    \SpecialEnvIndex{dcases*}
%    \cs{begin}\arg{dcases*}  \meta{math_column} |&| \makebox[\widthof{\meta{math\_column}}][l]{\meta{text\_column}}
%    \cs{end}\arg{dcases*}\\
%    \SpecialEnvIndex{rcases}
%    \cs{begin}\arg{rcases}\texttt{~~}  \meta{math_column} |&| \makebox[\widthof{\meta{math\_column}}][l]{\meta{math\_column}}
%    \cs{end}\arg{rcases}\\
%    \SpecialEnvIndex{rcases*}
%    \cs{begin}\arg{rcases*}\texttt{~}  \meta{math_column} |&| \makebox[\widthof{\meta{math\_column}}][l]{\meta{text\_column}}
%    \cs{end}\arg{rcases*}\\
%    \SpecialEnvIndex{drcases}
%    \cs{begin}\arg{drcases}\texttt{~}  \meta{math_column} |&| \makebox[\widthof{\meta{math\_column}}][l]{\meta{math\_column}}
%    \cs{end}\arg{drcases}\\
%    \SpecialEnvIndex{drcases*}
%    \cs{begin}\arg{drcases*}  \meta{math_column} |&| \makebox[\widthof{\meta{math\_column}}][l]{\meta{text\_column}}
%    \cs{end}\arg{drcases*}\\
%    \SpecialEnvIndex{cases*}
%    \cs{begin}\arg{cases*}\texttt{~}  \meta{math_column} |&| \makebox[\widthof{\meta{math\_column}}][l]{\meta{text\_column}}
%    \cs{end}\arg{cases*}
%  \end{codesyntax}
%  \FeatureRequest{Lars Madsen}{2004/07/01}
%  Anyone who have tried to use an integral in the regular
%  \env{cases} environment from \pkg{amsmath} will have noticed that
%  it is set as
%  \[
%    a=\begin{cases}
%      E = m c^2     & \text{Nothing to see here} \\
%      \int x-3\, dx & \text{Integral is text style}
%    \end{cases}
%  \]
%  \pkg{mathtools} provides two environments similar to \env{cases}.
%  Using the \env{dcases} environment you get the same output as with
%  \env{cases} except that the rows are set in display style.
%  \begin{verbatim}
%  \[
%    \begin{dcases}
%      E = m c^2     & c \approx 3.00\times 10^{8}\,\mathrm{m}/\mathrm{s} \\
%      \int x-3\, dx & \text{Integral is display style}
%    \end{dcases}
%  \]
%  \end{verbatim}
%  \[
%    \begin{dcases}
%      E = m c^2     & c \approx 3.00\times 10^{8}\,\mathrm{m}/\mathrm{s} \\
%      \int x-3\, dx & \text{Integral is display style}
%    \end{dcases}
%  \]
%  Additionally the environment \env{dcases*} acts just the same, but
%  the second column is set in the normal roman font of the
%  document.\footnote{Or rather: it inherits the font characteristics
%  active just before the \env{dcases*} environment.}
%  \begin{verbatim}
%  \[
%    a= \begin{dcases*}
%      E = m c^2     & Nothing to see here \\
%      \int x-3\, dx & Integral is display style
%    \end{dcases*}
%  \]
%  \end{verbatim}
%  \[
%    a= \begin{dcases*}
%      E = m c^2     & Nothing to see here \\
%      \int x-3\, dx & Integral is display style
%    \end{dcases*}
%  \]
%  The environments \env{rcases}, \env{rcases*}, \env{drcases} and
%  \env{drcases*} are equivalent to \env{cases} and \env{dcases}, but
%  here the brace is placed on the right instead of on the left.
% \begin{verbatim}
% \[
% \begin{rcases*}
%   x^2 & for $x>0$\\ 
%   x^3 & else
% \end{rcases*} \quad \Rightarrow \cdots
% \]
% \end{verbatim}
% \[
% \begin{rcases*}
%   x^2 & for $x>0$\\ 
%   x^3 & else
% \end{rcases*} \quad \Rightarrow\cdots
% \]
%
%
%  \subsubsection{Emulating indented lines in alignments}
%  \begin{codesyntax}
%    \SpecialEnvIndex{\MoveEqLeft}\cs{MoveEqLeft}\oarg{number}
%  \end{codesyntax}
%  \ProvidedBy{Lars Madsen}{2008/06/05} In \cite{Swanson}, Ellen
%  Swanson recommends that when ever one has a long displayed formula,
%  spanning several lines, and it is unfeasible to align against a
%  relation within the first line, then all lines in the display
%  should be aligned at the left most edge of the first line, and all
%  subsequent lines should be indented by 2\,em (or if needed by a
%  smaller amount). That is we are talking about displayes that end up
%  looking like this
%  \begin{align*}
%    \MoveEqLeft \framebox[10cm][c]{Long first line}\\
%    & = \framebox[6cm][c]{ \hphantom{g} 2nd line}\\ 
%    & \leq \dots
%  \end{align*}
%  Traditionally one could do this by starting subsequent lines by
%  \verb+&\qquad ...+, but that is tedious. Instead the example above
%  was made using \cs{MoveEqLeft}:
%  \begin{verbatim}
%  \begin{align*}
%    \MoveEqLeft \framebox[10cm][c]{Long first line}\\
%    & = \framebox[6cm][c]{ \hphantom{g} 2nd line}\\ 
%    & \leq \dots
%  \end{align*}
%  \end{verbatim}
%  \cs{MoveEqLeft} is placed instead of the \verb+&+ on the first
%  line, and will effectively \emph{move} the entire first line
%  \oarg{number} of ems to the left (default is 2). If you choose to
%  align to the right of the relation, use \cs{MoveEqLeft}\verb+[3]+
%  to accommodate the extra distance to the alignment point:
%  \begin{verbatim}
%  \begin{align*}
%    \MoveEqLeft[3] \framebox[10cm][c]{Part 1}\\
%     = {} & \framebox[8cm][c]{2nd line}\\ 
%          & + \framebox[4cm][c]{ last part}
%  \end{align*}
%  \end{verbatim}
%  \begin{align*}
%    \MoveEqLeft[3] \framebox[10cm][c]{Long first line}\\
%     = {} & \framebox[6cm][c]{  2nd line}\\ 
%          & + \framebox[4cm][c]{ last part}
%  \end{align*}
%
%  \subsubsection{Boxing a single line in an alignment}
%  
%  The \texttt{amsmath} package provie the \cs{boxed} macro to box
%  material in math mode. But this of course will not work if the box
%  should cross an alignment point. We provide a macro that
%  can.\footnote{Note that internally \cs{Aboxed} does use \cs{boxed}.}
%    \hskip1sp
%   \marginpar{%
%    \parbox[b]{\marginparwidth}{\small\sffamily\raggedright
%      \strut Evolved from a request by\\Merciadri Luca\\
%       2010/06/28\\on comp.text.tex%
%    }\strut
%  }%
%   \marginpar{\strut\\%
%    \parbox[b]{\marginparwidth}{\small\sffamily\raggedright
%      \strut Reimplemented by\\Florent Chervet (GL) \\
%       2011/06/11\\on comp.text.tex%
%    }\strut
%  }%
%  \begin{codesyntax}
%    \SpecialEnvIndex{\Aboxed}\cs{Aboxed}\marg{left hand side 
%     \quad\texttt{\textnormal{\&}}\quad right hand side}
%  \end{codesyntax}
%  Example
% \begin{verbatim}
% \begin{align*}
%   \Aboxed{ f(x) & = \int h(x)\, dx} \\
%                 & = g(x)
% \end{align*}
% \end{verbatim}
% Resulting in:
% \begin{align*}
%   \Aboxed{ f(x) & = \int h(x)\, dx} \\
%                 & = g(x)
% \end{align*}
% One can have multiple boxes on each line, and the
% >>\texttt{\textnormal{\&}}\quad right hand side<< can even be
% missing. Here is an example of how the padding in the box can be changed
% \begin{verbatim}
% \begin{align*}
%   \setlength\fboxsep{1em}
%   \Aboxed{ f(x) &= 0 } & \Aboxed{ g(x) &= b} \\
%   \Aboxed{ h(x) }      & \Aboxed{ i(x) }   
% \end{align*}
% \end{verbatim}
% \begin{align*}
%   \setlength\fboxsep{1em}
%   \Aboxed{ f(x) &= 0 } & \Aboxed{ g(x) &= b} \\
%   \Aboxed{ h(x) }      & \Aboxed{ i(x) }   
% \end{align*}
% Note how the \cs{fboxsep} change only affect the box coming
% immediately after it.  
%
%  \subsubsection{Adding arrows between lines in an alignment}
%
%  This first macro is a bit misleading, it is only intended to be
%  used in combination with the \env{alignat(*)} environment.
%  \begin{codesyntax}
%    \SpecialEnvIndex{\ArrowBetweenLines}\cs{ArrowBetweenLines}\oarg{symbol}\\
%    \SpecialEnvIndex{\ArrowBetweenLines*}\cs{ArrowBetweenLines*}\oarg{symbol}
%  \end{codesyntax}
%    \hskip1sp
%   \marginpar{%
%    \parbox[b]{\marginparwidth}{\small\sffamily\raggedright
%      \strut Evolved from a request by\\Christian
%      Bohr-Halling\\2004/03/31\\on dk.edb.tekst%
%    }
%  }%
%  To add, say $\Updownarrow$ between two lines in an alignment use
%  \cs{ArrowBetweenLines} and the \env{alignat} environment (note the
%  extra pair of  \texttt{\&}'s in front):
%  \begin{verbatim}
%  \begin{alignat}{2}
%    && \framebox[1.5cm]{} &= \framebox[3cm]{}\\
%    \ArrowBetweenLines % \Updownarrow is the default
%    && \framebox[1.5cm]{} &= \framebox[2cm]{}
%  \end{alignat}
%  \end{verbatim}
%  resulting in
%  \begin{alignat}{2}
%    && \framebox[1.5cm]{} &= \framebox[3cm]{}\\
%    \ArrowBetweenLines 
%    && \framebox[1.5cm]{} &= \framebox[2cm]{}
%  \end{alignat}
%  Note the use of \verb+&&+ starting each \emph{regular} line of
%  math. For adding the arrow on the right, use
%  \cs{ArrowBetweenLines*}\oarg{symbol}, and end each line of math
%  with \verb+&&+.
%  \begin{verbatim}
%  \begin{alignat*}{2}
%    \framebox[1.5cm]{} &= \framebox[3cm]{}  &&\\
%    \ArrowBetweenLines*[\Downarrow] 
%    \framebox[1.5cm]{} &= \framebox[2cm]{}  &&
%  \end{alignat*}
%  \end{verbatim}
%  resulting in
%  \begin{alignat*}{2}
%    \framebox[1.5cm]{} &= \framebox[3cm]{}  &&\\
%    \ArrowBetweenLines*[\Downarrow] 
%    \framebox[1.5cm]{} &= \framebox[2cm]{}  &&
%  \end{alignat*}
%
%
% \subsubsection{Centered \cs{vdots}}
%
%  If one want to mark a vertical continuation, there is
%  the \verb?\vdots? command, but combine this with an alignment and
%  we get something rather suboptimal
% \FeatureRequest{Bruno Le Floch \\(and many others)}{2011/01/25}
%  \begin{align*}
%    \framebox[1.5cm]{} &= \framebox[3cm]{}\\
%                       & \vdots\\
%                       &= \framebox[3cm]{} 
%  \end{align*}
%  It would be nice to have (1) a \verb?\vdots? centered within the
%  width of another symbol, and (2) a construction similar to
%  \verb?\ArrowBetweenLines? that does not take up so much space. 
%  We provide both.
%  \begin{codesyntax}
%    \SpecialUsageIndex{\vdotswithin}\cs{vdotswithin}\marg{symbol}\\
%    \SpecialUsageIndex{\shortvdotswithin}\cs{shortvdotswithin}\marg{symbol}\\
%    \SpecialUsageIndex{\shortvdotswithin*}\cs{shortvdotswithin*}\marg{symbol}\\
%    \SpecialUsageIndex{\MTFlushSpaceAbove}\cs{MTFlushSpaceAbove}\\
%    \SpecialUsageIndex{\MTFlushSpaceBelow}\cs{MTFlushSpaceBelow}\\
%    \SpecialKeyIndex{shortvdotsadjustabove}\makebox{$\key{shortvdotsadjustabove}=\meta{length}$}\\
%    \SpecialKeyIndex{shortvdotsadjustbelow}\makebox{$\key{shortvdotsadjustbelow}=\meta{length}$}
%  \end{codesyntax}
%  Two examples in one
% \begin{verbatim}
% \begin{align*}
%   a &= b              \\
%     & \vdotswithin{=} \\
%     & = c             \\
%     \shortvdotswithin{=}
%     & = d
% \end{align*}
% \end{verbatim}
% yielding
% \begin{align*}
%   a &= b              \\
%     & \vdotswithin{=} \\
%     & = c             \\
%     \shortvdotswithin{=}
%     & = d
% \end{align*}
% Thus \verb?\vdotswithin{=}? create a box corersponding to
% \verb?{}={}? and typeset a >>$\vdots$<< centered inside it. When doing
% this as a normal line in an alignment leaves us with excessive space
% which \verb?\shortvdotswithin{=}? takes care with for us.
%
% \verb?\shortvdotswithin{=}? corresponds to
% \begin{verbatim}
% \MTFlushSpaceAbove
% & \vdotswithin{=} \\
% \MTFlushSpaceBelow
% \end{verbatim}
% whereas \verb?\shortvdotswithin*{=}? is the case with 
% \verb?\vdotswithin{=} & \\?. This also means one cannot write more
% on the line when using \verb?\shortvdotswithin? or the starred
% version. But one can de-construct the macro and arrive at
% \begin{verbatim}
% \begin{alignat*}{3}
%   A&+ B &&= C &&+ D \\
%   \MTFlushSpaceAbove
%   &\vdotswithin{+} &&&& \vdotswithin{+}
%   \MTFlushSpaceBelow
%   C &+ D &&= Y &&+K
% \end{alignat*}
% \end{verbatim}
% yielding
% \begin{alignat*}{3}
%   A&+ B &&= C &&+ D \\
%   \MTFlushSpaceAbove
%   &\vdotswithin{+} &&&& \vdotswithin{+}
%   \MTFlushSpaceBelow
%   C &+ D &&= Y &&+K
% \end{alignat*}
% If one has the need for such a construction.
%
% The de-spaced version does support the \env{spreadlines}
% environment. The actual amount of space being \emph{flushed} above
% and below can be controlled by the user using the two options
% indicated. Their original values are \verb?2.15\origjot? and
% \verb?\origjot? respectively (\verb?\origjot? is usually 3pt). 
%
%  \subsection{Intertext and short intertext}
%
%
%  \begin{codesyntax}
%    \SpecialUsageIndex{\shortintertext}\cs{shortintertext}\marg{text}
%  \end{codesyntax}
%  \cttPosting{Gabriel Zachmann and Donald Arseneau}{2000/05/12--13}
%  \pkg{amsmath} provides the command \cs{intertext} for interrupting
%  a multiline display while still maintaining the alignment points.
%  However the spacing often seems quite excessive as seen below.
%  \begin{verbatim}
%    \begin{align}
%      a&=b \intertext{Some text}
%      c&=d
%    \end{align}
%  \end{verbatim}
%    \begin{align}
%      a&=b \intertext{Some text}
%      c&=d
%    \end{align}
%
%  Using the command \cs{shortintertext} alleviates this situation
%  somewhat:
%  \begin{verbatim}
%    \begin{align}
%      a&=b \shortintertext{Some text}
%      c&=d
%    \end{align}
%  \end{verbatim}
%  \begin{align}
%    a&=b \shortintertext{Some text}
%    c&=d
%  \end{align}
%
%  \noindent
%  It turns out that both \cs{shortintertext} and the original
%  \cs{intertext} from \pkg{amsmath} has a slight problem. If we use
%  the \env{spreadlines} (see section~\ref{sec:spread}) to open up
%  the equations in a multiline calculation, then this opening up
%  value also applies to the spacing above and below the original
%  \cs{shortintertext} and \cs{intertext}.  \tsxPosting{Tobias Weh
%    \\(referring to a suggestion by Chung-chieh Shan)}{2011/05/29}
% It can be illustrated using the following example, an interested
% reader, can apply it with and with out the original \cs{intertext}
% and \cs{shortintertext}.
% \begin{verbatim}
% % the original \intertext and \shortintertext
% \mathtoolsset{original-intertext,original-shortintertext}
% \newcommand\myline{\par\noindent\rule{\textwidth}{1mm}} 
% \myline
% \begin{spreadlines}{1em}
%   \begin{align*}
%     AA\\  BB\\  \intertext{\myline}
%     AA\\  BB\\  \shortintertext{\myline}
%     AA\\  BB
%   \end{align*}
% \end{spreadlines}
% \myline
% \end{verbatim}
%
%  We now fix this internaly for both \cs{intertext} and
%  \cs{shortintertext}, plus we add the posibility to fine tune
%  spacing around these constructions. The original versions can be
%  brought back using the \texttt{original-x} keys below.
%  \begin{codesyntax}
%    \SpecialUsageIndex{\intertext}\cs{intertext}\marg{text}\\
%    \SpecialUsageIndex{\shortintertext}\cs{shortintertext}\marg{text}\\
%    \SpecialKeyIndex{original-intertext}$\key{original-intertext}=\texttt{true}\vert\texttt{false}$ \quad(default: \texttt{false})\\
%    \SpecialKeyIndex{original-shortintertext}$\key{original-shortintertext}=\texttt{true}\vert\texttt{false}$ 
%    \quad(default: \texttt{false})\\ 
%    \SpecialKeyIndex{above-intertext-sep}$\key{above-intertext-sep}=\meta{dimen}$ \quad(default: 0pt)\\
%    \SpecialKeyIndex{below-intertext-sep}$\key{below-intertext-sep}=\meta{dimen}$ \quad(default: 0pt)\\
%    \SpecialKeyIndex{above-shortintertext-sep}$\key{above-shortintertext-sep}=\meta{dimen}$ \quad(default: 3pt)\\
%    \SpecialKeyIndex{below-shortintertext-sep}$\key{below-shortintertext-sep}=\meta{dimen}$ \quad(default: 3pt)
%  \end{codesyntax}
%  The updated \cs{shortintertext} will look like the original version
%  unless for areas with an enlarged \cs{jot} value (see for example
%  the \env{spreadlines}, section~\ref{sec:spread}). Whereas \cs{intertext}
%  will have a slightly smaller value above and below (corresponding
%  to about 3pt less space above and below), the spacing around
%  \cs{intertext} should now match the normal spacing going into and
%  out of an \env{align}.
%
% \textbf{Tip:} \cs{intertext} and \cs{shortintertext} also works
% within \env{gather}.
%
%  \subsection{Paired delimiters}
%
%
%  \begin{codesyntax}
%    \SpecialUsageIndex{\DeclarePairedDelimiter}
%    \cs{DeclarePairedDelimiter}\marg{cmd}\marg{left_delim}\marg{right_delim}
%  \end{codesyntax}
%  \FeatureRequest{Lars Madsen}{2004/06/25}
%  In the \pkg{amsmath} documentation it is shown how to define a few
%  commands for typesetting the absolute value and norm. These
%  definitions are:
%  \begin{verbatim}
%    \newcommand*\abs[1]{\lvert#1\rvert}
%    \newcommand*\norm[1]{\lVert#1\rVert}
%  \end{verbatim}
%  \DeclarePairedDelimiter\abs\lvert\rvert
%  While they produce correct horizontal spacing you have to be
%  careful about the vertical spacing if the argument is just a
%  little taller than usual as in
%  \[
%    \abs{\frac{a}{b}}
%  \]
%  Here it won't give a nice result, so you have to manually put in
%  either \cs{left}--\cs{right} pair or a \cs{bigl}--\cs{bigr} pair.
%  Both methods mean that you have to delete your \cs{abs} command,
%  which may not sound like an ideal solution.
%
%  With the command \cs{DeclarePairedDelimiter} you can combine all
%  these features in one easy to use command. Let's show an example:
%  \begin{verbatim}
%    \DeclarePairedDelimiter\abs{\lvert}{\rvert}
%  \end{verbatim}
%  This defines the command \cs{abs} just like in the
%  \pkg{amsmath} documentation but with a few additions:
%  \begin{itemize}
%    \item A starred variant: \cs{abs*} produces delimiters that are preceded
%  by \cs{left} and \cs{right} resp.:
%  \begin{verbatim}
%  \[
%    \abs*{\frac{a}{b}}
%  \]
%  \end{verbatim}
%      \[
%        \abs*{\frac{a}{b}}
%      \]
%  \item A variant with an optional argument:
%  \cs{abs}\oarg{size_cmd}, where
%  \meta{size_cmd} is either \cs{big}, \cs{Big}, \cs{bigg}, or
%  \cs{Bigg} (if you have any bigggger versions you can use them
%  too).
%  \begin{verbatim}
%  \[
%    \abs[\Bigg]{\frac{a}{b}}
%  \]
%  \end{verbatim}
%      \[
%        \abs[\Bigg]{\frac{a}{b}}
%      \]
%  \end{itemize}
%
%  \begin{codesyntax}
%    \SpecialUsageIndex{\DeclarePairedDelimiterX}
%    \cs{DeclarePairedDelimiterX}\marg{cmd}\oarg{num args}\marg{left_delim}\marg{right_delim}\marg{code}\\
%     \cs{delimsize}
%  \end{codesyntax}
%  \ProvidedBy{Lars Madsen}{2010/06/15} 
%  Sometimes \cs{DeclarePairedDelimiter} just is not enough.  One
%  might want to have the capabilities of \cs{DeclarePairedDelimiter},
%  but also want a macro the takes more than one argument.
%
%  \cs{DeclarePairedDelimiterX} extends the features of
%  \cs{DeclarePairedDelimiter} such that the user will get a macro
%  which is fenced off at either end, plus the capability to provide
%  the code for what ever the macro should do within these fences. 
%
%  Inside the \meta{code} part, the macro \cs{delimsize} refer to the
%  size of the outer fences. It can then be used inside \meta{code} to
%  scale any inner fences.
%
%  In this setting
%  \cs{DeclarePairedDelimiter}\marg{cmd}\marg{left_delim}\marg{right_delim} is the same thing as 
%  \begin{center}
%    \cs{DeclarePairedDelimiterX}\marg{cmd}\verb|[1]|\marg{left_delim}\marg{right_delim}\verb|{#1}|
%  \end{center}
%  %
%  Let us do some examples. First we want to prepare a macro for inner
%  products, with two arguments such that we can hide the character
%  separating the arguments (a journal style might require a
%  semi-colon, so we will save a lot of hand editing). This can be
%  done via
% \begin{verbatim}
% \DeclarePairedDelimiterX\innerp[2]{\langle}{\rangle}{#1,#2}
% \end{verbatim}
% More interestingly we can refer to the size inside the
% \meta{code}. Here we do a weird three argument `braket'
% \begin{verbatim}
% \DeclarePairedDelimiterX\braket[3]{\langle}{\rangle}%
% {#1\,\delimsize\vert\,#2\,\delimsize\vert\,#3}
% \end{verbatim}
% Then we can get
% \DeclarePairedDelimiterX\innerp[2]{\langle}{\rangle}{#1,#2}
% \DeclarePairedDelimiterX\braket[3]{\langle}{\rangle}%
% {#1\,\delimsize\vert\,#2\,\delimsize\vert\,#3}
% \begin{verbatim}
% \[
% \innerp*{A}{ \frac{1}{2} } \quad
% \braket[\Big]{B}{\sum_{k} f_k}{C}
% \]
% \end{verbatim}
% \[
% \innerp*{A}{ \frac{1}{2} } \quad
% \braket[\Big]{B}{\sum_{k}}{C}
% \]
% \iffalse
% \bigskip
% 
% \noindent
% \textbf{\textit{Side note:}} We have changed the internal code of
% \cs{DeclarePairedDelimiter} and \cs{DeclarePairedDelimiterX} such
% that the starred version does no longer give the odd spacings that
% \cs{left}\dots\cs{right} sometimes give. Compare this
% \begin{verbatim}
% \[
% 2\innerp{A}{B} \quad  2\innerp*{A}{B} \quad 2\left\langle A,B \right\rangle
% \]
% \end{verbatim}
% \[
% \sin\innerp{A}{B} \quad
% \sin\innerp*{A}{B} \quad
% \sin\left\langle A,B \right\rangle
% \]
% The spacing in the last one does not behave like other fences (this
% is a feature of the \cs{left}\dots\cs{right} construction).
% \fi
%
%
%  \subsubsection{Expert use}
%
%  Within the starred version of \cs{DeclarePairedDelimiter} and
%  \cs{DeclarePairedDelimiterX} we make a few changes such that the
%  auto scaled \cs{left} and \cs{right} fences behave as opening and
%  closing fences, i.e.\ $\sin(x)$ vs. $\sin\left(x\right)$ (the later
%  made via \verb|$\sin\left(x\right)$|), notice the gab between
%  '$\sin$' and '('.  In some special cases it may be useful to be
%  able to tinker with the behavior.
%  \begin{codesyntax}
%    \SpecialUsageIndex{\reDeclarePairedDelimiterInnerWrapper}\cs{reDeclarePairedDelimiterInnerWrapper}\marg{macro name}\marg{\textnormal{\texttt{star}} or \textnormal{\texttt{nostar}}}\marg{code}
%  \end{codesyntax}
%  Internally several macros are created, including two call backs
%  that take care of the formatting. There is one internal macro for
%  the starred version, labeled \texttt{star}, the other one is
%  labeled \texttt{nostar}. Within \meta{code}, \texttt{\#1} will be
%  replaced by the scaled left fence, \texttt{\#3} the corresponding
%  scaled right fence, and \texttt{\#2} the stuff in between. For
%  example, here is how one might turn the content into \cs{mathinner}:
% \begin{verbatim}
% \DeclarePairedDelimiter\abs\lvert\rvert
% \reDeclarePairedDelimiterInnerWrapper\abs{star}{#1#2#3}
% \reDeclarePairedDelimiterInnerWrapper\abs{nostar}{\mathinner{#1#2#3}}
% \end{verbatim}
%  The default values for the call backs corresponds to
% \begin{verbatim}
% star:   \mathopen{}\mathclose\bgroup #1#2\aftergroup\egroup #3
% nostar: \mathopen{#1}#2\mathclose{#3}
% \end{verbatim}
%
%
%  \subsection{Special symbols}
%
%  This part of the manual is about special symbols. So far only one
%  technique is covered, but more will come.
%
%  \subsubsection{Left and right parentheses}
%
%  \begin{codesyntax}
%    \SpecialUsageIndex{\lparen}\cs{lparen}\texttt{~~}
%    \SpecialUsageIndex{\rparen}\cs{rparen}
%  \end{codesyntax}
%  When you want a big parenthesis or bracket in a math display you
%  usually just type
%  \begin{quote}
%    |\left( ... \right)|\quad  or\quad |\left[ ... \right]|
%  \end{quote}
%  \LaTeX{} also defines the macro names \cs{lbrack} and \cs{rbrack}
%  to be shorthands for the left and right square bracket resp., but
%  doesn't provide similar definitions for the parentheses. Some
%  packages need command names to work with\footnote{The \pkg{empheq}
%  package needs command names for delimiters in order to make
%  auto-scaling versions.} so \pkg{mathtools} defines the commands
%  \cs{lparen} and \cs{rparen} to represent the left and right
%  parenthesis resp.
%
%
%  \subsubsection{Vertically centered colon}
%
%  \begin{codesyntax}
%    \SpecialKeyIndex{centercolon}$\key{centercolon}=\texttt{true}\vert\texttt{false}$\\
%    \SpecialUsageIndex{\vcentcolon}\cs{vcentcolon}\texttt{~~}
%    \SpecialUsageIndex{\ordinarycolon}\cs{ordinarycolon}
%  \end{codesyntax}
%  \cttPosting{Donald Arseneau}{2000/12/07}
%  When trying to show assignment operations as in $ a := b $, one
%  quickly notices that the colon is not centered on the math axis as
%  the equal sign, leading to an odd-looking output. The command
%  \cs{vcentcolon} is a shorthand for such a vertically centered
%  colon, and can be used as in |$a \vcentcolon= b$| and results in
%  the desired output:  $a \vcentcolon= b$. % for now
%
%  Typing \cs{vcentcolon} every time is quite tedious, so one can use
%  the key \key{centercolon} to make the colon active instead.
%  \begin{verbatim}
%  \mathtoolsset{centercolon}
%  \[
%    a := b
%  \]
%  \mathtoolsset{centercolon=false}
%  \end{verbatim}
%  \[\mathtoolsset{centercolon}
%    a := b
%  \]
%  In this case the command \cs{ordinarycolon} typesets an~\ldots\
%  ordinary colon (what a surprise).
%
%  \medskip
%  \noindent\textbf{Warning:} \texttt{centercolon} \emph{does not}
%  work with languages that make use of an active colon, most notably
%  \emph{French}. Sadly the \texttt{babel} package does not distinguish
%  between text and math when it comes to active characters. Nor does
%  it provide any hooks to deal with math. So currently no general
%  solution exists for this problem.
%
%  \begin{codesyntax}
%    \SpecialUsageIndex{\coloneqq}\cs{coloneqq}\texttt{~~~~~}
%    \SpecialUsageIndex{\Coloneqq}\cs{Coloneqq}\texttt{~~~~~}
%    \SpecialUsageIndex{\coloneq}\cs{coloneq}\texttt{~~~}
%    \SpecialUsageIndex{\Coloneq}\cs{Coloneqq}\\
%    \SpecialUsageIndex{\eqqcolon}\cs{eqqcolon}\texttt{~~~~~}
%    \SpecialUsageIndex{\Eqqcolon}\cs{Eqqcolon}\texttt{~~~~~}
%    \SpecialUsageIndex{\eqcolon}\cs{eqcolon}\texttt{~~~}
%    \SpecialUsageIndex{\Eqcolon}\cs{Eqcolon}\\
%    \SpecialUsageIndex{\colonapprox}\cs{colonapprox}\texttt{~~}
%    \SpecialUsageIndex{\Colonapprox}\cs{Colonapprox}\texttt{~~}
%    \SpecialUsageIndex{\colonsim}\cs{colonsim}\texttt{~~}
%    \SpecialUsageIndex{\Colonsim}\cs{Colonsim}\\
%    \SpecialUsageIndex{\dblcolon}\cs{dblcolon}
%  \end{codesyntax}
%  The font packages \pkg{txfonts} and \pkg{pxfonts} provides various
%  symbols that include a vertically centered colon but with tighter
%  spacing. For example, the combination |:=| exists as the symbol
%  \cs{coloneqq} which typesets as $\coloneqq$ instead of
%  $\vcentcolon=$. The primary disadvantage of using these fonts are
%  the support packages' lack of support for \pkg{amsmath} (and thus
%  \pkg{mathtools}) and worse yet, the side-bearings are way too
%  tight; see~\cite{A-W:MG04} for examples. If you're not using these
%  fonts, \pkg{mathtools} provides the symbols for you. Here are a few
%  examples:
%  \begin{verbatim}
%  \[
%    a \coloneqq b \quad c \Colonapprox d \quad e \dblcolon f
%  \]
%  \end{verbatim}
%  \[
%    a \coloneqq b \quad c \Colonapprox d \quad e \dblcolon f
%  \]
%
%
%
%  \subsubsection{A few missing symbols}
%
%  Most provided math font sets are missing the symbols \cs{nuparrow}
%  and \cs{ndownarrow} (i.e.\ negated up- and downarrow) plus a `big'
%  version of \cs{times}. Therefore we will provide constructed
%  versions of these whenever they are not already available.
%  \begin{codesyntax}
%    \SpecialUsageIndex{\nuparrow}\cs{nuparrow}\\
%    \SpecialUsageIndex{\ndownarrow}\cs{ndownarrow}\\
%    \SpecialUsageIndex{\bigtimes}\cs{bigtimes}
%  \end{codesyntax}
%
%  \noindent
%  \textbf{Note:} that these symbols are constructed via
%  features from the \pkg{graphicx} package, and thus may not display
%  correctly in most DVI previewers. Also note that \cs{nuparrow} and
%  \cs{ndownarrow} are constructed via \cs{nrightarrow} and
%  \cs{nleftarrrow} respectively, so these needs to be
%  present. Usually this is done via \pkg{amssymb}, but some packages
%  may be incompatible with \pkg{amssymb} so the user will have to
%  load \pkg{amssymb} or a similar package, that provides
%  \cs{nrightarrow} and \cs{nleftarrow}, themselves. 
%
%  With those requirements in place, we have
%  \begin{verbatim}
%    \[
%       \lim_{a\ndownarrow 0} f(a) \neq \bigtimes_n X_n \qquad
%       \frac{ \bigtimes_{k=1}^7 B_k \nuparrow \Omega }{2}     
%    \]
%  \end{verbatim}
%    \[
%       \lim_{a\ndownarrow 0} f(a) \neq \bigtimes_n X_n \qquad
%       \frac{ \bigtimes_{k=1}^7 B_k \nuparrow \Omega }{2}     
%    \]
%
%
%
%
%  \section{A tribute to Michael J.~Downes}
%
%  Michael J.~Downes (1958--2003) was one of the major architects
%  behind \pkg{amsmath} and member of the \LaTeX{} Team. He made many
%  great contributions to the \TeX{} community; not only by the means
%  of widely spread macro packages such as \pkg{amsmath} but also in
%  the form of actively giving advice on newsgroups. Some of
%  Michael's macro solutions on the newsgroups never made it into
%  publicly available macro packages although they certainly deserved
%  it, so \pkg{mathtools} tries to rectify this matter. The macros
%  described in this section are either straight copies or heavily
%  inspired by his original postings.
%
%  \subsection{Mathematics within italic text}
%
%  \begin{codesyntax}
%    \SpecialKeyIndex{mathic}$\key{mathic}=\texttt{true}\vert\texttt{false}$
%  \end{codesyntax}
%  \cttPosting{Michael J.~Downes}{1998/05/14}
%  \TeX{} usually takes care of italic corrections in text, but fails
%  when it comes to math. If you use the \LaTeX{} inline math
%  commands \cs{(} and \cs{)} you can however work around it by
%  setting the key \key{mathic} to true as shown below.
%  \begin{verbatim}
%    \begin{quote}\itshape
%    Compare these lines: \par
%    \mathtoolsset{mathic} % or \mathtoolsset{mathic=true}
%    Subset of \(V\) and subset of \(A\). \par
%    \mathtoolsset{mathic=false}
%    Subset of \(V\) and subset of \(A\).
%    \par
%    \end{quote}
%  \end{verbatim}
%  \begin{quote}\itshape
%  Compare these lines: \par
%  \mathtoolsset{mathic}
%  Subset of \(V\) and subset of \(A\). \par
%  \mathtoolsset{mathic=false}
%  Subset of \(V\) and subset of \(A\).
%  \par
%  \end{quote}
%
%  \noindent
%  It is recommended to load the \pkg{fixltx2e} package \emph{after}
%  \pkg{mathtools} and the \verb|\mathtoolsset{mathic=true}| option,
%  as it will make \verb|\(...\)| robust, while maintaining the added
%  italic correction.
%
%  \subsection{Left sub/superscripts}
%
%  \begin{codesyntax}
%    \SpecialUsageIndex{\prescript}
%        \cs{prescript}\marg{sup}\marg{sub}\marg{arg}\texttt{~~}
%    \SpecialKeyIndex{prescript-sup-format}
%        $\key{prescript-sup-format}=\meta{cmd}$\\
%    \SpecialKeyIndex{prescript-sub-format}
%        $\key{prescript-sub-format}=\meta{cmd}$\hfill
%    \SpecialKeyIndex{prescript-arg-format}
%        \rlap{$\key{prescript-arg-format}=\meta{cmd}$}^^A
%        \phantom{$\key{prescript-sup-format}=\meta{cmd}$}
%  \end{codesyntax}
%  \cttPosting{Michael J.~Downes}{2000/12/20}
%  Sometimes one wants to put a sub- or superscript on the left of
%  the argument. The \cs{prescript} command does just that:
%  \begin{verbatim}
%    \[
%      {}^{4}_{12}\mathbf{C}^{5+}_{2}          \quad
%      \prescript{14}{2}{\mathbf{C}}^{5+}_{2}  \quad
%      \prescript{4}{12}{\mathbf{C}}^{5+}_{2}  \quad
%      \prescript{14}{}{\mathbf{C}}^{5+}_{2}   \quad
%      \prescript{}{2}{\mathbf{C}}^{5+}_{2}
%    \]
%  \end{verbatim}
%  \[
%    {}^{4}_{12}\mathbf{C}^{5+}_{2}          \quad
%    \prescript{14}{2}{\mathbf{C}}^{5+}_{2}  \quad
%    \prescript{4}{12}{\mathbf{C}}^{5+}_{2}  \quad
%    \prescript{14}{}{\mathbf{C}}^{5+}_{2}   \quad
%    \prescript{}{2}{\mathbf{C}}^{5+}_{2}
%  \]
%
%  The formatting of the arguments is controlled by three keys. This
%  silly example shows you how to use them:
%  \begin{verbatim}
%  \newcommand*\myisotope[3]{%
%    \begingroup % to keep changes local. We cannot use a brace group
%                % as it affects spacing!
%      \mathtoolsset{
%        prescript-sup-format=\mathit,
%        prescript-sub-format=\mathbf,
%        prescript-arg-format=\mathrm,
%      }%
%    \prescript{#1}{#2}{#3}%
%    \endgroup
%  }
%  \[
%    \myisotope{A}{Z}{X}\to \myisotope{A-4}{Z-2}{Y}+
%    \myisotope{4}{2}{\alpha}
%  \]
%  \end{verbatim}
%  \newcommand*\myisotope[3]{%
%    \begingroup
%      \mathtoolsset{
%        prescript-sup-format=\mathit,
%        prescript-sub-format=\mathbf,
%        prescript-arg-format=\mathrm,
%      }%
%    \prescript{#1}{#2}{#3}%
%    \endgroup
%  }
%  \[
%    \myisotope{A}{Z}{X}\to \myisotope{A-4}{Z-2}{Y}+
%    \myisotope{4}{2}{\alpha}
%  \]
% (Though a package like \pkg{mhchem} might be more suitable for this
% type of material.)
%
%  \subsection{Declaring math sizes}
%
%  \begin{codesyntax}
%    \SpecialUsageIndex{\DeclareMathSizes}
%    \cs{DeclareMathSizes}\marg{dimen}\marg{dimen}\marg{dimen}\marg{dimen}
%  \end{codesyntax}
%  \cttPosting{Michael J.~Downes}{2002/10/17}
%  If you don't know about \cs{DeclareMathSizes}, then skip the rest
%  of this text. If you do know, then all that is needed to say is
%  that with \pkg{mathtools} it is patched so that all regular
%  dimension suffixes are now valid in the last three arguments. Thus
%  a declaration such as
%  \begin{verbatim}
%    \DeclareMathSize{9.5dd}{9.5dd}{7.5dd}{6.5dd}
%  \end{verbatim}
%  will now work (it doesn't in standard \LaTeX). When this bug has
%  been fixed in \LaTeX, this fix will be removed from
%  \pkg{mathtools}.
%
%  \subsection{Spreading equations}\label{sec:spread}
%
%  \begin{codesyntax}
%    \SpecialEnvIndex{spreadlines}
%    \cs{begin}\arg{spreadlines}\marg{dimen} \meta{contents}
%    \cs{end}\arg{spreadlines}
%  \end{codesyntax}
%  \cttPosting{Michael J.~Downes}{2002/10/17}
%  The spacing between lines in a multiline math environment such as
%  \env{gather} is governed by the dimension \cs{jot}. The
%  \env{spreadlines} environment takes one argument denoting the
%  value of \cs{jot} inside the environment:
%  \begin{verbatim}
%    \begin{spreadlines}{20pt}
%    Large spaces between the lines.
%    \begin{gather}
%      a=b\\
%      c=d
%    \end{gather}
%    \end{spreadlines}
%    Back to normal spacing.
%    \begin{gather}
%      a=b\\
%      c=d
%    \end{gather}
%  \end{verbatim}
%    \begin{spreadlines}{20pt}
%    Large spaces between the lines.
%    \begin{gather}
%      a=b\\
%      c=d
%    \end{gather}
%    \end{spreadlines}
%    Back to normal spacing.
%    \begin{gather}
%      a=b\\
%      c=d
%    \end{gather}
%
%
%  \subsection{Gathered environments}\label{subsec:gathered}
%
%  \begin{codesyntax}
%    \SpecialEnvIndex{lgathered}\cs{begin}\arg{lgathered}\oarg{pos}
%    \meta{contents}  \cs{end}\arg{lgathered} \\
%    \SpecialEnvIndex{rgathered}\cs{begin}\arg{rgathered}\oarg{pos}
%    \meta{contents}  \cs{end}\arg{rgathered} \\
%    \SpecialUsageIndex{\newgathered}\cs{newgathered}\marg{name}\marg{pre_line}\marg{post_line}\marg{after}\\
%    \SpecialUsageIndex{\renewgathered}\cs{renewgathered}\marg{name}\marg{pre_line}\marg{post_line}\marg{after}
%  \end{codesyntax}
%  \cttPosting{Michael J.~Downes}{2001/01/17}
%  In a document set in \opt{fleqn}, you might sometimes want an
%  inner \env{gathered} environment that doesn't center its lines but
%  puts them flush left. The \env{lgathered} environment works just
%  like the standard \env{gathered} except that it flushes its
%  contents left:
%  \begin{verbatim}
%    \begin{equation}
%      \begin{lgathered}
%        x=1,\quad x+1=2 \\
%        y=2
%      \end{lgathered}
%    \end{equation}
%  \end{verbatim}
%  \begin{equation}
%    \begin{lgathered}
%        x=1,\quad x+1=2 \\
%        y=2
%    \end{lgathered}
%  \end{equation}
%  Similarly the \env{rgathered} puts it contents flush right.
%
%  More interesting is probably the command \cs{newgathered}. In this
%  example we define a gathered version that centers the lines and
%  also prints a star and a number at the left of each line.
%  \begin{verbatim}
%    \newcounter{steplinecnt}
%    \newcommand\stepline{\stepcounter{steplinecnt}\thesteplinecnt}
%    \newgathered{stargathered}
%                {\llap{\stepline}$*$\quad\hfil}% \hfil for centering
%                {\hfil}%                         \hfil for centering
%                {\setcounter{steplinecnt}{0}}%   reset counter
%  \end{verbatim}
%  \newcounter{steplinecnt}
%  \newcommand\stepline{\stepcounter{steplinecnt}\thesteplinecnt}
%  \newgathered{stargathered}{\llap{\stepline}$*$\quad\hfil}{\hfil}{\setcounter{steplinecnt}{0}}
%  With these definitions we can get something like this:
%  \begin{verbatim}
%    \begin{gather}
%      \begin{stargathered}
%        x=1,\quad x+1=2 \\
%        y=2
%      \end{stargathered}
%    \end{gather}
%  \end{verbatim}
%  \begin{gather}
%    \begin{stargathered}
%      x=1,\quad x+1=2 \\
%      y=2
%    \end{stargathered}
%  \end{gather}
%  \cs{renewgathered} renews a gathered environment of course.
%
%  In all fairness it should be stated that the original concept by
%  Michael has been extended quite a bit in \pkg{mathtools}. Only the
%  end product of \env{lgathered} is the same.
%
%  \subsection{Split fractions}
%
%  \begin{codesyntax}
%    \SpecialUsageIndex{\splitfrac}\cs{splitfrac}\marg{numer}\marg{denom}\texttt{~~}
%    \SpecialUsageIndex{\splitdfrac}\cs{splitdfrac}\marg{numer}\marg{denom}
%  \end{codesyntax}
%  \cttPosting{Michael J.~Downes}{2001/12/06}
%  These commands provide split fractions e.g., multiline fractions:
%  \begin{verbatim}
%    \[
%      a=\frac{
%          \splitfrac{xy + xy + xy + xy + xy}
%                    {+ xy + xy + xy + xy}
%        }
%        {z}
%      =\frac{
%          \splitdfrac{xy + xy + xy + xy + xy}
%                    {+ xy + xy + xy + xy}
%        }
%        {z}
%  \]
%  \end{verbatim}
%  \[
%  a=\frac{
%      \splitfrac{xy + xy + xy + xy + xy}
%                {+ xy + xy + xy + xy}
%    }
%    {z}
%  =\frac{
%      \splitdfrac{xy + xy + xy + xy + xy}
%                {+ xy + xy + xy + xy}
%    }
%    {z}
%  \]
%
%
%
%
%
%
%
%
%
%
%
%
%
%
%
%
%
%
%
%
%
%
%
%
%
%
%
%
%
%
%
%
%
%
%
%
%
%
%
%
%
%
%
%
%
%
%
%
%
%
%
%
%
%
%
%
%
%
%
%
%
%
%
%
%
%
%
%
%
%
%
%
%
%
%
%
%
%
%
%
%
%
%
%
%
% \begin{thebibliography}{9}
%  \bibitem{Perlis01}
%    Alexander R. Perlis,
%    \emph{A complement to \cs{smash}, \cs{llap}, and \cs{rlap}},
%    TUGboat 22(4) (2001).
%  \bibitem{Ams99}
%    American Mathematical Society and Michael Downes,
%    \emph{Technical notes on the \pkg{amsmath} package} Version 2.0,
%    1999/10/29.
%    (Available from CTAN as file \texttt{technote.tex}.)
%  \bibitem{Ams00}
%    Frank Mittelbach, Rainer Sch\"opf, Michael Downes, and David M.~Jones,
%    \emph{The \pkg{amsmath} package} Version 2.13,
%    2000/07/18.
%    (Available from CTAN as file \texttt{amsmath.dtx}.)
%  \bibitem{A-W:MG04}
%    Frank Mittelbach and Michel Goossens.
%     \emph{The {\LaTeX} Companion}.
%    Tools and Techniques for Computer Typesetting. Addison-Wesley,
%    Boston, Massachusetts, 2 edition, 2004.
%    With Johannes Braams, David Carlisle, and Chris Rowley.
%
% \bibitem{Carl99}
%   David Carlisle,
%   \emph{The \pkg{keyval} Package},
%   Version 1.13, 1999/03/16.
%   (Available from CTAN as file \texttt{keyval.dtx}.)
%
%  \bibitem{Voss:2004}
%    Herbert Vo\ss,
%    \emph{Math mode}, Version 1.71,
%    2004/07/06.
%    (Available from CTAN as file \texttt{Voss-Mathmode.pdf}.)
%  
%   \bibitem{Swanson} 
%     Ellen Swanson,
%     \emph{Mathematics into type}.
%     American Mathematical Society, updated edition, 1999.
%     Updated by Arlene O'Sean and Antoinette Schleyer.
%  \end{thebibliography}
%
%
%  \StopEventually{}  
%
%
%  \section{Options and package loading}
%
%
%  Lets start the package.
%    \begin{macrocode}
%<*package>
\ProvidesPackage{mathtools}%
  [2012/04/24 v1.12 mathematical typesetting tools]
%    \end{macrocode}
% \changes{v1.10}{2011/02/12}{Might as well make sure that we need the
% latest version of \texttt{mhsetup}}
%    \begin{macrocode}
\RequirePackage{keyval,calc}
\RequirePackage{mhsetup}[2010/01/21]
\MHInternalSyntaxOn
%    \end{macrocode}
%
%  \begin{macro}{\MT_options_name:}
%  \begin{macro}{\mathtoolsset}
%  The name for the options and a user interface for setting keys.
%    \begin{macrocode}
\def\MT_options_name:{mathtools}
\newcommand*\mathtoolsset[1]{\setkeys{\MT_options_name:}{#1}}
%    \end{macrocode}
%  \end{macro}
%  \end{macro}
%
%  Fix \pkg{amsmath} bugs (strongly recommended!). It requires a
%  great deal of typing to avoid fixing the bugs. He he.
%    \begin{macrocode}
\MH_new_boolean:n {fixamsmath}
\DeclareOption{fixamsmath}{
  \MH_set_boolean_T:n {fixamsmath}
}
\DeclareOption{donotfixamsmathbugs}{
  \MH_set_boolean_F:n {fixamsmath}
}
%    \end{macrocode}
%  Disallow spaces before optional arguments in certain \pkg{amsmath}
%  building blocks.
%    \begin{macrocode}
\DeclareOption{allowspaces}{
  \MH_let:NwN \MaybeMHPrecedingSpacesOff
              \relax
    \MH_let:NwN \MH_maybe_nospace_ifnextchar:Nnn \kernel@ifnextchar
}
\DeclareOption{disallowspaces}{
  \MH_let:NwN \MaybeMHPrecedingSpacesOff
              \MHPrecedingSpacesOff
  \MH_let:NwN \MH_maybe_nospace_ifnextchar:Nnn \MH_nospace_ifnextchar:Nnn
}
%    \end{macrocode}
%  Pass all other options directly to \pkg{amsmath}.
%    \begin{macrocode}
\DeclareOption*{
  \PassOptionsToPackage{\CurrentOption}{amsmath}
}
\ExecuteOptions{fixamsmath,disallowspaces}
\ProcessOptions\relax
%    \end{macrocode}
%  We have to turn off the new syntax when \pkg{amstext} is loaded.
%    \begin{macrocode}
\MHInternalSyntaxOff
\RequirePackage{amsmath}[2000/07/18]
\MHInternalSyntaxOn
\AtEndOfPackage{\MHInternalSyntaxOff}
%    \end{macrocode}
%  \begin{macro}{\MT_true_false_error:}
%  Make sure the user selects either `true' or `false' when asked too.
%    \begin{macrocode}
\def\MT_true_false_error:{
  \PackageError{mathtools}
    {You~ have~ to~ select~ either~ `true'~ or~ `false'}
    {I'll~ assume~ you~ chose~ `false'~ for~ now.}
}
%    \end{macrocode}
%  \end{macro}
%
%  \section{Macros I got ideas for myself}
%
%
%
%  \subsection{Tag forms}
%  This is quite simple, but why isn't it then a part of some widely
%  distributed package? Beats me.
%
%  \begin{macro}{\MT_define_tagform:nwnn}
%  We start out by defining a command that will allow us to define
%  commands similar to \cs{tagform@} only this will give us tag form
%  \emph{types}. The actual code is very similar to the one in
%  \pkg{amsmath}.
%    \begin{macrocode}
\def\MT_define_tagform:nwnn #1[#2]#3#4{
  \@namedef{MT_tagform_#1:n}##1
    {\maketag@@@{#3\ignorespaces#2{##1}\unskip\@@italiccorr#4}}
}
%    \end{macrocode}
%  \end{macro}
%
%  \begin{macro}{\newtagform}
%  Similar to \cs{newcommand}. Check if defined and scan for presence
%  of optional argument. Then call generic command.
%    \begin{macrocode}
\providecommand*\newtagform[1]{%
  \@ifundefined{MT_tagform_#1:n}
  {\@ifnextchar[%
    {\MT_define_tagform:nwnn #1}%
    {\MT_define_tagform:nwnn #1[]}%
  }{\PackageError{mathtools}
  {The~ tag~ form~ `#1'~ is~ already~ defined\MessageBreak
  You~ probably~ want~ to~ look~ up~ \@backslashchar renewtagform~
  instead}
  {I~ will~ just~ ignore~ your~ wish~ for~ now.}}
}
%    \end{macrocode}
%  Provide a default tag form which---surprise, surprise---is
%  identical to the standard definition.
%    \begin{macrocode}
\newtagform{default}{(}{)}
%    \end{macrocode}
%  \end{macro}
%  \begin{macro}{\renewtagform}
%  Similar to \cs{renewcommand}.
%    \begin{macrocode}
\providecommand*\renewtagform[1]{%
  \@ifundefined{MT_tagform_#1:n}
  {\PackageError{mathtools}
  {The~ tag~ form~ `#1'~ is~ not~ defined\MessageBreak
  You~ probably~ want~ to~ look~ up~ \@backslashchar newtagform~ instead}
  {I~ will~ just~ ignore~ your~ wish~ for~ now.}}
  {\@ifnextchar[%
    {\MT_define_tagform:nwnn #1}%
    {\MT_define_tagform:nwnn #1[]}%
  }
}
%    \end{macrocode}
%  \end{macro}
%  \begin{macro}{\usetagform}
%  Then the activator. Test if the tag form is defined and then
%  activate it by redefining \cs{tagform@}.
%    \begin{macrocode}
\providecommand*\usetagform[1]{%
  \@ifundefined{MT_tagform_#1:n}
    {
      \PackageError{mathtools}{%
        You~ have~ chosen~ the~ tag~ form~ `#1'\MessageBreak
        but~ it~ appears~ to~ be~ undefined}
        {I~ will~ use~ the~ default~ tag~ form~ instead.}%
        \@namedef{tagform@}{\@nameuse{MT_tagform_default:n}}
      }
  { \@namedef{tagform@}{\@nameuse{MT_tagform_#1:n}} }
%    \end{macrocode}
%  Here we patch if we're using the special ``show only referenced
%  equations'' feature.
%    \begin{macrocode}
  \MH_if_boolean:nT {show_only_refs}{
    \MH_let:NwN \MT_prev_tagform:n \tagform@
    \def\tagform@##1{\MT_extended_tagform:n {##1}}
  }
}
%    \end{macrocode}
%  \end{macro}
%
%  \subsubsection{Showing only referenced tags}
%  A little more interesting is the way to print only the equation
%  numbers that are actually referenced.
%
%  A few booleans to help determine which situations we're in.
%    \begin{macrocode}
\MH_new_boolean:n {manual_tag}
\MH_new_boolean:n {raw_maketag}
%    \end{macrocode}
%  \begin{macro}{\MT_AmS_tag_in_align:}
%  \begin{macro}{\tag@in@align}
%  \begin{macro}{\tag@in@display}
%  We'll need to know when the user has put in a manual tag, and since
%  \cs{tag} is \cs{let} to all sorts of things inside the \pkg{amsmath}
%  code it is safer to provide a small hack to the functions it is copied
%  from. Note that we can't use \cs{iftag@}.
%    \begin{macrocode}
\MH_let:NwN \MT_AmS_tag_in_align: \tag@in@align
\def\tag@in@align{
  \global\MH_set_boolean_T:n {manual_tag}
  \MT_AmS_tag_in_align:
}
\def\tag@in@display#1#{
  \relax
  \global\MH_set_boolean_T:n {manual_tag}
  \tag@in@display@a{#1}
}
%    \end{macrocode}
%  \end{macro}
%  \end{macro}
%  \end{macro}
%
%  \begin{macro}{\MT_extended_tagform:n}
%  \changes{v1.01}{2004/08/03}{Simplified quite a bit}
%  The extended version of \cs{tagform@}.
%    \begin{macrocode}
\def\MT_extended_tagform:n #1{
  \MH_set_boolean_F:n {raw_maketag}
%    \end{macrocode}
% We test if the equation was labelled. We already know if it was
% tagged manually. Have to watch out for \TeX\ inserting a blank line
% so do not let the tag have width zero. Rememeber
% \cs{@safe@activestrue/false} in order to handle active chars in labels.
% \changes{v1.12}{2012/04/24}{Added \cs{@safe@activestrue/false}}
%    \begin{macrocode}
  \if_meaning:NN \df@label\@empty
    \MH_if_boolean:nTF {manual_tag}% this was \MH_if_boolean:nT before
    { \MH_if_boolean:nTF {show_manual_tags}
      { \MT_prev_tagform:n {#1} }
      { \stepcounter{equation}  }
    }{\kern1sp}% this last {\kern1sp} is new.
  \else:
    \MH_if_boolean:nTF {manual_tag}
      { \MH_if_boolean:nTF {show_manual_tags}
          { \MT_prev_tagform:n {#1} }
          { \@safe@activestrue
            \@ifundefined{MT_r_\df@label}
%    \end{macrocode}
% Next we need to remember to deactivate the manual tags switch. This
% is usually done using \verb|\MT_extended_maketag:n|, but this is not
% the case if the show manual tags is false and the manual tag is not
% referred to. 
% \changes{v1.12}{2011/06/08}{Added the falsification of manual tag
% when show manual tags is off and maual tag is not referred to}
%    \begin{macrocode}
              { \global\MH_set_boolean_F:n {manual_tag} }
              { \MT_prev_tagform:n {#1} }
              \@safe@activesfalse
          }
      }
      { 
        \@safe@activestrue
        \@ifundefined{MT_r_\df@label}
          { }
          { \refstepcounter{equation}\MT_prev_tagform:n {#1} }
        \@safe@activesfalse
      }
  \fi:
  \global\MH_set_boolean_T:n {raw_maketag}
}
%    \end{macrocode}
%  \end{macro}
%  \begin{macro}{\MT_extended_maketag:n}
%  The extended version of \cs{maketag@@@}.
% \changes{v1.12}{2012/04/24}{Added \cs{@safe@activestrue/false}}
%    \begin{macrocode}
\def\MT_extended_maketag:n #1{
  \ifx\df@label\@empty
    \MT_maketag:n {#1}
  \else:
    \MH_if_boolean:nTF {raw_maketag}
      {
        \MH_if_boolean:nTF {show_manual_tags}
          { \MT_maketag:n {#1} }
          { \@safe@activestrue
            \@ifundefined{MT_r_\df@label}
              { }
              { \MT_maketag:n {#1}     }
            \@safe@activesfalse
          }
      }
      { \MT_maketag:n {#1} }
  \fi:
%    \end{macrocode}
%  As this function is always called we let it set the marker for a manual
%  tag false when exiting (well actually not true, see above).
%    \begin{macrocode}
  \global\MH_set_boolean_F:n {manual_tag}
}
%    \end{macrocode}
%  \end{macro}
%  \begin{macro}{\MT_extended_eqref:n}
%  \changes{v1.01}{2004/08/03}{Make it robust}
%  We let \cs{eqref} write the label to the \file{aux} file, which is
%  read at the beginning of the next run. Then we print the equation
%  number as usual.
%    \begin{macrocode}
\def\MT_extended_eqref:n #1{
  \protected@write\@auxout{}
    {\string\MT@newlabel{#1}}
  \textup{\MT_prev_tagform:n {\ref{#1}}}
}
%    \end{macrocode}
%  \end{macro}
%
%  \begin{macro}{\refeq}
%  \begin{macro}{\MT_extended_refeq:n}
%  Similar to \cs{eqref} and \cs{MT_extended_eqref:n}.
%    \begin{macrocode}
\newcommand*\refeq[1]{
  \textup{\ref{#1}}
}
\def\MT_extended_refeq:n #1{
  \protected@write\@auxout{}
    {\string\MT@newlabel{#1}}
  \textup{\ref{#1}}
}
%    \end{macrocode}
%  \end{macro}
%  \end{macro}
%
%  \begin{macro}{\MT@newlabel}
%  We can't use |:| or |_| in the command name (yet). We define the
%  special labels for the equations that have been referenced in the
%  previous run.
%    \begin{macrocode}
\newcommand*\MT@newlabel[1]{  \global\@namedef{MT_r_#1}{}  }
%    \end{macrocode}
%  \end{macro}
%    \begin{macrocode}
\MH_new_boolean:n {show_only_refs}
\MH_new_boolean:n {show_manual_tags}
\define@key{\MT_options_name:}{showmanualtags}[true]{
  \@ifundefined{boolean_show_manual_tags_#1:}
    { \MT_true_false_error:
      \@nameuse{boolean_show_manual_tags_false:}
    }
    { \@nameuse{boolean_show_manual_tags_#1:} }
}
%    \end{macrocode}
%  \begin{macro}{\MT_showonlyrefs_true:}
%  The implementation is based on the idea that \cs{tagform@} can be
%  called in two circumstances: when the tag is being printed in the
%  equation and when it is being printed during a reference.
%    \begin{macrocode}
\newcommand*\MT_showonlyrefs_true:{
  \MH_if_boolean:nF {show_only_refs}{
    \MH_set_boolean_T:n {show_only_refs}
%    \end{macrocode}
%  Save the definitions of the original commands.
%    \begin{macrocode}
    \MH_let:NwN \MT_incr_eqnum: \incr@eqnum
    \MH_let:NwN \incr@eqnum \@empty
    \MH_let:NwN \MT_array_parbox_restore: \@arrayparboxrestore
    \@xp\def\@xp\@arrayparboxrestore\@xp{\@arrayparboxrestore
      \MH_let:NwN \incr@eqnum \@empty
    }
    \MH_let:NwN \MT_prev_tagform:n \tagform@
    \MH_let:NwN \MT_eqref:n \eqref
    \MH_let:NwN \MT_refeq:n \refeq
    \MH_let:NwN \MT_maketag:n \maketag@@@
    \MH_let:NwN \maketag@@@ \MT_extended_maketag:n
%    \end{macrocode}
%  We redefine \cs{tagform@}.
%    \begin{macrocode}
    \def\tagform@##1{\MT_extended_tagform:n {##1}}
%    \end{macrocode}
%  Then \cs{eqref}:
%    \begin{macrocode}
    \MH_let:NwN \eqref \MT_extended_eqref:n
    \MH_let:NwN \refeq \MT_extended_refeq:n
  }
}
%    \end{macrocode}
%  \end{macro}
%  \begin{macro}{\MT_showonlyrefs_false:}
%  This macro reverts the settings.
%    \begin{macrocode}
\def\MT_showonlyrefs_false: {
  \MH_if_boolean:nT {show_only_refs}{
    \MH_set_boolean_F:n {show_only_refs}
    \MH_let:NwN \tagform@  \MT_prev_tagform:n
    \MH_let:NwN \eqref \MT_eqref:n
    \MH_let:NwN \refeq \MT_refeq:n
    \MH_let:NwN \maketag@@@ \MT_maketag:n
    \MH_let:NwN \incr@eqnum \MT_incr_eqnum:
    \MH_let:NwN \@arrayparboxrestore \MT_array_parbox_restore:
  }
}
\define@key{\MT_options_name:}{showonlyrefs}[true]{
  \@nameuse{MT_showonlyrefs_#1:}
}
%    \end{macrocode}
%  \end{macro}
%
%
%  \begin{macro}{\nonumber}
%  \changes{v1.01}{2004/08/03}{Fixed using \cs{notag} or \cs{nonumber}
%  with the \key{showonlyrefs} feature}
%  We have to redefine \cs{nonumber} else it will subtract one from the
%  equation number where we don't want it. This is probably not needed
%  since \cs{nonumber} is unnecessary when \key{showonlyrefs} is in
%  effect, but now you can use it with old documents as well.
%    \begin{macrocode}
\renewcommand\nonumber{
  \if@eqnsw
    \if_meaning:NN \incr@eqnum\@empty
%    \end{macrocode}
%  Only subtract the number if |show_only_refs| is false.
%    \begin{macrocode}
      \MH_if_boolean:nF {show_only_refs}
        {\addtocounter{equation}\m@ne}
    \fi:
  \fi:
  \MH_let:NwN \print@eqnum\@empty \MH_let:NwN \incr@eqnum\@empty
  \global\@eqnswfalse
}
%    \end{macrocode}
%  \end{macro}
%
%  \begin{macro}{\noeqref}
%   \changes{v1.04}{2008/03/26}{Added \cs{noeqref} (daleif)}
%   \changes{v1.12}{2012/04/20}{Labels containing active chars (babel)
%   are now allowed}
%   \changes{v1.12}{2012/04/24}{\cs{noeqref} will now make a reference
%   warning if users use undefined labels in \cs{noeqref}, requested
%   by Tue Christensen}
%   Macro for adding numbers to non-referred equations. Syntax similar
%   to \cs{nocite}.
%    \begin{macrocode}
\MHInternalSyntaxOff
\newcommand\noeqref[1]{\@bsphack
  \@for\@tempa:=#1\do{%
    \@safe@activestrue%
    \edef\@tempa{\expandafter\@firstofone\@tempa}%
    \@ifundefined{r@\@tempa}{%
      \protect\G@refundefinedtrue%
      \@latex@warning{Reference `\@tempa' on page \thepage \space
        undefined (\string\noeqref)}%
    }{}%
    \if@filesw\protected@write\@auxout{}%
    {\string\MT@newlabel{\@tempa}}\fi%
  \@safe@activesfalse}
  \@esphack}

%    \end{macrocode}
%  \end{macro}
%    
% \begin{macro}{\@safe@activestrue}
% \begin{macro}{\@safe@activesfalse}
%   These macros are provided by babel. We \emph{provide} them here,
%   just to make sure they exist.
%    \begin{macrocode}
\providecommand\@safe@activestrue{}%
\providecommand\@safe@activesfalse{}%

\MHInternalSyntaxOn
%    \end{macrocode}
%   
% \end{macro}
% \end{macro}
%
%  \subsection{Extensible arrows etc.}
%
%  \begin{macro}{\xleftrightarrow}
%  \begin{macro}{\MT_leftrightarrow_fill:}
%  \begin{macro}{\xLeftarrow}
%  \begin{macro}{\xRightarrow}
%  \begin{macro}{\xLeftrightarrow}
%
%  These are straight adaptions from \pkg{amsmath}.
%    \begin{macrocode}
\providecommand*\xleftrightarrow[2][]{%
  \ext@arrow 3095\MT_leftrightarrow_fill:{#1}{#2}}
\def\MT_leftrightarrow_fill:{%
  \arrowfill@\leftarrow\relbar\rightarrow}
\providecommand*\xLeftarrow[2][]{%
  \ext@arrow 0055{\Leftarrowfill@}{#1}{#2}}
\providecommand*\xRightarrow[2][]{%
  \ext@arrow 0055{\Rightarrowfill@}{#1}{#2}}
\providecommand*\xLeftrightarrow[2][]{%
  \ext@arrow 0055{\Leftrightarrowfill@}{#1}{#2}}
%    \end{macrocode}
%  \end{macro}
%  \end{macro}
%  \end{macro}
%  \end{macro}
%  \end{macro}
%  \begin{macro}{\MT_rightharpoondown_fill:}
%  \begin{macro}{\MT_rightharpoonup_fill:}
%  \begin{macro}{\MT_leftharpoondown_fill:}
%  \begin{macro}{\MT_leftharpoonup_fill:}
%  \begin{macro}{\xrightharpoondown}
%  \begin{macro}{\xrightharpoonup}
%  \begin{macro}{\xleftharpoondown}
%  \begin{macro}{\xleftharpoonup}
%  \begin{macro}{\xleftrightharpoons}
%  \begin{macro}{\xrightleftharpoons}
%  The harpoons.
%    \begin{macrocode}
\def\MT_rightharpoondown_fill:{%
  \arrowfill@\relbar\relbar\rightharpoondown}
\def\MT_rightharpoonup_fill:{%
  \arrowfill@\relbar\relbar\rightharpoonup}
\def\MT_leftharpoondown_fill:{%
  \arrowfill@\leftharpoondown\relbar\relbar}
\def\MT_leftharpoonup_fill:{%
  \arrowfill@\leftharpoonup\relbar\relbar}
\providecommand*\xrightharpoondown[2][]{%
  \ext@arrow 0359\MT_rightharpoondown_fill:{#1}{#2}}
\providecommand*\xrightharpoonup[2][]{%
  \ext@arrow 0359\MT_rightharpoonup_fill:{#1}{#2}}
\providecommand*\xleftharpoondown[2][]{%
  \ext@arrow 3095\MT_leftharpoondown_fill:{#1}{#2}}
\providecommand*\xleftharpoonup[2][]{%
  \ext@arrow 3095\MT_leftharpoonup_fill:{#1}{#2}}
\providecommand*\xleftrightharpoons[2][]{\mathrel{%
  \raise.22ex\hbox{%
    $\ext@arrow 3095\MT_leftharpoonup_fill:{\phantom{#1}}{#2}$}%
  \setbox0=\hbox{%
    $\ext@arrow 0359\MT_rightharpoondown_fill:{#1}{\phantom{#2}}$}%
  \kern-\wd0 \lower.22ex\box0}}
\providecommand*\xrightleftharpoons[2][]{\mathrel{%
  \raise.22ex\hbox{%
    $\ext@arrow 0359\MT_rightharpoonup_fill:{\phantom{#1}}{#2}$}%
  \setbox0=\hbox{%
    $\ext@arrow 3095\MT_leftharpoondown_fill:{#1}{\phantom{#2}}$}%
  \kern-\wd0 \lower.22ex\box0}}
%    \end{macrocode}
%  \end{macro}
%  \end{macro}
%  \end{macro}
%  \end{macro}
%  \end{macro}
%  \end{macro}
%  \end{macro}
%  \end{macro}
%  \end{macro}
%  \end{macro}
%  \begin{macro}{\xhookleftarrow}
%  \begin{macro}{\xhookrightarrow}
%  \begin{macro}{\MT_hookright_fill:}
%  The hooks.
%    \begin{macrocode}
\providecommand*\xhookleftarrow[2][]{%
  \ext@arrow 3095\MT_hookleft_fill:{#1}{#2}}
\def\MT_hookleft_fill:{%
  \arrowfill@\leftarrow\relbar{\relbar\joinrel\rhook}}
\providecommand*\xhookrightarrow[2][]{%
  \ext@arrow 3095\MT_hookright_fill:{#1}{#2}}
\def\MT_hookright_fill:{%
  \arrowfill@{\lhook\joinrel\relbar}\relbar\rightarrow}
%    \end{macrocode}
%  \end{macro}
%  \end{macro}
%  \end{macro}
%  \begin{macro}{\xmapsto}
%  \begin{macro}{\MT_mapsto_fill:}
%  The maps-to arrow.
%    \begin{macrocode}
\providecommand*\xmapsto[2][]{%
  \ext@arrow 0395\MT_mapsto_fill:{#1}{#2}}
\def\MT_mapsto_fill:{%
  \arrowfill@{\mapstochar\relbar}\relbar\rightarrow}
%    \end{macrocode}
%  \end{macro}
%  \end{macro}
%  \subsection{Underbrackets etc.}
%  \begin{macro}{\underbracket}
%  \begin{macro}{\MT_underbracket_I:w}
%  \begin{macro}{\MT_underbracket_II:w}
%  \begin{macro}{\upbracketfill}
%  \begin{macro}{\upbracketend}
%  The \cs{underbracket} macro. Scan for two optional arguments. When
%  \pkg{xparse} becomes the standard this will be so much easier.
%    \begin{macrocode}
\providecommand*\underbracket{
  \@ifnextchar[
    {\MT_underbracket_I:w}
    {\MT_underbracket_I:w[\l_MT_bracketheight_fdim]}}
\def\MT_underbracket_I:w[#1]{
  \@ifnextchar[
    {\MT_underbracket_II:w[#1]}
    {\MT_underbracket_II:w[#1][.7\fontdimen5\textfont2]}}
\def\MT_underbracket_II:w[#1][#2]#3{%
  \mathop{\vtop{\m@th\ialign{##
    \crcr
      $\hfil\displaystyle{#3}\hfil$%
    \crcr
      \noalign{\kern .2\fontdimen5\textfont2 \nointerlineskip}%
      \upbracketfill {#1}{#2}%
    \crcr}}}
  \limits}
\def\upbracketfill#1#2{%
  \sbox\z@{$\braceld$}
  \edef\l_MT_bracketheight_fdim{\the\ht\z@}%
  \upbracketend{#1}{#2}
  \leaders \vrule \@height \z@ \@depth #1 \hfill
  \upbracketend{#1}{#2}%
}
\def\upbracketend#1#2{\vrule \@height #2 \@width #1\relax}
%    \end{macrocode}
%  \end{macro}
%  \end{macro}
%  \end{macro}
%  \end{macro}
%  \end{macro}
%  \begin{macro}{\overbracket}
%  \begin{macro}{\MT_overbracket_I:w}
%  \begin{macro}{\MT_overbracket_II:w}
%  \begin{macro}{\downbracketfill}
%  \begin{macro}{\downbracketend}
%  The overbracket is quite similar.
%    \begin{macrocode}
\providecommand*\overbracket{
  \@ifnextchar[
    {\MT_overbracket_I:w}
    {\MT_overbracket_I:w[\l_MT_bracketheight_fdim]}}
\def\MT_overbracket_I:w[#1]{
  \@ifnextchar[
    {\MT_overbracket_II:w[#1]}
    {\MT_overbracket_II:w[#1][.7\fontdimen5\textfont2]}}
\def\MT_overbracket_II:w[#1][#2]#3{%
  \mathop{\vbox{\m@th\ialign{##
        \crcr
          \downbracketfill{#1}{#2}%
        \crcr
          \noalign{\kern .2\fontdimen5\textfont2 \nointerlineskip}%
          $\hfil\displaystyle{#3}\hfil$
        \crcr}}}%
  \limits}
\def\downbracketfill#1#2{%
  \sbox\z@{$\braceld$}\edef\l_MT_bracketheight_fdim{\the\ht\z@}
  \downbracketend{#1}{#2}
  \leaders \vrule \@height #1 \@depth \z@ \hfill
  \downbracketend{#1}{#2}%
}
\def\downbracketend#1#2{\vrule \@width #1\@depth #2\relax}
%    \end{macrocode}
%  \end{macro}
%  \end{macro}
%  \end{macro}
%  \end{macro}
%  \end{macro}
%  \begin{macro}{\LaTeXunderbrace}
%  \begin{macro}{\underbrace}
%  Redefinition of \cs{underbrace} and \cs{overbrace}.
%    \begin{macrocode}
\MH_let:NwN \LaTeXunderbrace \underbrace
\def\underbrace#1{\mathop{\vtop{\m@th\ialign{##\crcr
   $\hfil\displaystyle{#1}\hfil$\crcr
   \noalign{\kern.7\fontdimen5\textfont2\nointerlineskip}%
%    \end{macrocode}
%  |.5\fontdimen5\textfont2| is the height of the tip of the brace.
%  the remaining |.2\fontdimen5\textfont2| is for space between
%    \begin{macrocode}
   \upbracefill\crcr\noalign{\kern.5\fontdimen5\textfont2}}}}\limits}
%    \end{macrocode}
%  \end{macro}
%  \end{macro}
%  \begin{macro}{\LaTeXoverbrace}
%  \begin{macro}{\overbrace}
%  Same technique for \cs{overbrace}.
%    \begin{macrocode}
\MH_let:NwN \LaTeXoverbrace \overbrace
\def\overbrace#1{\mathop{\vbox{\m@th\ialign{##\crcr
  \noalign{\kern.5\fontdimen5\textfont2}%
%    \end{macrocode}
%  Adjust for tip height
%    \begin{macrocode}
  \downbracefill\crcr
  \noalign{\kern.7\fontdimen5\textfont2\nointerlineskip}%
%    \end{macrocode}
%  |.5\fontdimen5\textfont2| is the height of the tip of the brace.
%  The remaining |.2\fontdimen5\textfont2| is for space between
%    \begin{macrocode}
  $\hfil\displaystyle{#1}\hfil$\crcr}}}\limits}
%    \end{macrocode}
%  \end{macro}
%  \end{macro}
%
%
%
%
%
%  \subsection{Special symbols}
%
%  \subsubsection{Command names for parentheses}
%  \begin{macro}{\lparen}
%  \begin{macro}{\rparen}
%  Just an addition to the \LaTeXe\ kernel.
%    \begin{macrocode}
\providecommand*\lparen{(}
\providecommand*\rparen{)}
%    \end{macrocode}
%  \end{macro}
%  \end{macro}

%  \subsubsection{Vertically centered colon}
%
%  \begin{macro}{\vcentcolon}
%  \begin{macro}{\ordinarycolon}
%  \begin{macro}{\MT_active_colon_true:}
%  \begin{macro}{\MT_active_colon_false:}
%  This is from the hands of Donald Arseneau. Somehow it is not
%  distributed, so I include it here. Here's the original text by
%  Donald:
%  \begin{verbatim}
%  centercolon.sty                 Dec 7, 2000
%  Donald Arseneau                 asnd@triumf.ca
%  Public domain.
%  Vertically center colon characters (:) in math mode.
%  Particularly useful for $ a:=b$, and still correct for
%  $f : x\to y$.  May be used in any TeX.
%  \end{verbatim}
%
% Slight change: the colon meaning is given only if \verb|centercolon|
% is explicitly requested (before it was always assigned even if : remained
% catcode 12). This allows better interaction with packages like babel
% that also make colon active.
%    \begin{macrocode}
\def\vcentcolon{\mathrel{\mathop\ordinarycolon}}
\providecommand\ordinarycolon{:}
\begingroup
  \catcode`\:=\active
  \lowercase{\endgroup
\def\MT_activate_colon{%
    \ifnum\mathcode`\:=32768\relax
      \let\ordinarycolon= :%
    \else
      \mathchardef\ordinarycolon\mathcode`\: %
    \fi 
    \let :\vcentcolon
  }
}
%    \end{macrocode}
% Option processing.
% The `false' branch can only be requested if the option has previously been set `true'.
% (By default neither are set.)
%    \begin{macrocode}
\MH_new_boolean:n {center_colon}
\define@key{\MT_options_name:}{centercolon}[true]{
  \@ifundefined{MT_active_colon_#1:}
    { \MT_true_false_error:n
      \@nameuse{MT_active_colon_false:}
    }
    { \@nameuse{MT_active_colon_#1:} }
}
\def\MT_active_colon_true: {
  \MT_activate_colon
  \MH_if_boolean:nF {center_colon}{
    \MH_set_boolean_T:n {center_colon}
    \edef\MT_active_colon_false:
      {\mathcode`\noexpand\:=\the\mathcode`\:\relax}
    \mathcode`\:=32768
  }
}
%    \end{macrocode}
%  \end{macro}
%  \end{macro}
%  \end{macro}
%  \end{macro}
%  \begin{macro}{\dblcolon}
%  \begin{macro}{\coloneqq}
%  \begin{macro}{\Coloneqq}
%  \begin{macro}{\coloneq}
%  \begin{macro}{\Coloneq}
%  \begin{macro}{\eqqcolon}
%  \begin{macro}{\Eqqcolon}
%  \begin{macro}{\eqcolon}
%  \begin{macro}{\Eqcolon}
%  \begin{macro}{\colonapprox}
%  \begin{macro}{\Colonapprox}
%  \begin{macro}{\colonsim}
%  \begin{macro}{\Colonsim}
%  This is just to simulate all the \cs{..colon..} symbols from
%  \pkg{txfonts} and \pkg{pxfonts}.
% \changes{v1.08c}{2010/11/17}{Enclosed all in \cs{mathrel}}
%    \begin{macrocode}
\AtBeginDocument{
  \providecommand*\dblcolon{\mathrel{\vcentcolon\mkern-.9mu\vcentcolon}}
  \providecommand*\coloneqq{\mathrel{\vcentcolon\mkern-1.2mu=}}
  \providecommand*\Coloneqq{\mathrel{\dblcolon\mkern-1.2mu=}}
  \providecommand*\coloneq{\mathrel{\vcentcolon\mkern-1.2mu\mathrel{-}}}
  \providecommand*\Coloneq{\mathrel{\dblcolon\mkern-1.2mu\mathrel{-}}}
  \providecommand*\eqqcolon{\mathrel{=\mkern-1.2mu\vcentcolon}}
  \providecommand*\Eqqcolon{\mathrel{=\mkern-1.2mu\dblcolon}}
  \providecommand*\eqcolon{\mathrel{\mathrel{-}\mkern-1.2mu\vcentcolon}}
  \providecommand*\Eqcolon{\mathrel{\mathrel{-}\mkern-1.2mu\dblcolon}}
  \providecommand*\colonapprox{\mathrel{\vcentcolon\mkern-1.2mu\approx}}
  \providecommand*\Colonapprox{\mathrel{\dblcolon\mkern-1.2mu\approx}}
  \providecommand*\colonsim{\mathrel{\vcentcolon\mkern-1.2mu\sim}}
  \providecommand*\Colonsim{\mathrel{\dblcolon\mkern-1.2mu\sim}}
}
%    \end{macrocode}
%  \end{macro}
%  \end{macro}
%  \end{macro}
%  \end{macro}
%  \end{macro}
%  \end{macro}
%  \end{macro}
%  \end{macro}
%  \end{macro}
%  \end{macro}
%  \end{macro}
%  \end{macro}
%  \end{macro}
%
%
%  \subsection{Multlined}
%
%  \begin{macro}{\g_MT_multlinerow_int}
%  \begin{macro}{\l_MT_multwidth_dim}
%  Helpers.
%    \begin{macrocode}
\let \AMS@math@cr@@ \math@cr@@
\MH_new_boolean:n {mult_firstline}
\MH_new_boolean:n {outer_mult}
\newcount\g_MT_multlinerow_int
\newdimen\l_MT_multwidth_dim
%    \end{macrocode}
%  \end{macro}
%  \end{macro}
%  \begin{macro}{\MT_test_for_tcb_other:nnnnn}
%  This tests if the token(s) is/are equal to either t, c, or~b, or
%  something entirely different.
%    \begin{macrocode}
\newcommand*\MT_test_for_tcb_other:nnnnn [1]{
  \if:w t#1\relax
    \expandafter\MH_use_choice_i:nnnn
  \else:
    \if:w c#1\relax
      \expandafter\expandafter\expandafter\MH_use_choice_ii:nnnn
    \else:
      \if:w b#1\relax
        \expandafter\expandafter\expandafter
        \expandafter\expandafter\expandafter\expandafter
        \MH_use_choice_iii:nnnn
      \else:
        \expandafter\expandafter\expandafter
        \expandafter\expandafter\expandafter\expandafter
        \MH_use_choice_iv:nnnn
      \fi:
    \fi:
  \fi:
}
%    \end{macrocode}
%  \end{macro}
%  \begin{macro}{\MT_mult_invisible_line:}
%  An invisible line.
%    \begin{macrocode}
\def\MT_mult_invisible_line: {
  \crcr
  \global\MH_set_boolean_F:n {mult_firstline}
  \hbox to \l_MT_multwidth_dim{}\crcr
  \noalign{\vskip-\baselineskip \vskip-\normallineskip}
}
%    \end{macrocode}
%  \end{macro}
%  \begin{macro}{\MT_mult_mathcr_atat:w}
%  The normal \cs{math@cr@@} with our hooks.
%    \begin{macrocode}
\def\MT_mult_mathcr_atat:w [#1]{%
  \if_num:w 0=`{\fi: \iffalse}\fi:
  \MH_if_boolean:nT {mult_firstline}{
    \kern\l_MT_mult_left_fdim
    \MT_mult_invisible_line:
  }
  \crcr
  \noalign{\vskip#1\relax}
  \global\advance\g_MT_multlinerow_int\@ne
  \if_num:w \g_MT_multlinerow_int=\l_MT_multline_lastline_fint
    \MH_let:NwN \math@cr@@\MT_mult_last_mathcr:w
  \fi:
}
%    \end{macrocode}
%  \end{macro}
%  \begin{macro}{\MT_mult_firstandlast_mathcr:w}
%  The special case where there is a two-line \env{multlined}. We
%  insert the first kern, then the invisible line of the desired
%  width, the optional vertical space and then the last kern.
%    \begin{macrocode}
\def\MT_mult_firstandlast_mathcr:w [#1]{%
  \if_num:w 0=`{\fi: \iffalse}\fi:
  \kern\l_MT_mult_left_fdim
  \MT_mult_invisible_line:
  \noalign{\vskip#1\relax}
  \kern\l_MT_mult_right_fdim
}
%    \end{macrocode}
%  \end{macro}
%  \begin{macro}{\MT_mult_last_mathcr:w}
%  The normal last \cs{math@cr@@} which inserts the last kern.
%    \begin{macrocode}
\def\MT_mult_last_mathcr:w [#1]{
  \if_num:w 0=`{\fi: \iffalse}\fi:\math@cr@@@
  \noalign{\vskip#1\relax}
  \kern\l_MT_mult_right_fdim}
%    \end{macrocode}
%  \end{macro}
%  \begin{macro}{\MT_start_mult:N}
%  Setup for \env{multlined}. Finds the position.
%    \begin{macrocode}
\newcommand\MT_start_mult:N [1]{
  \MT_test_for_tcb_other:nnnnn {#1}
    { \MH_let:NwN \MT_next:\vtop }
    { \MH_let:NwN \MT_next:\vcenter }
    { \MH_let:NwN \MT_next:\vbox }
    {
      \PackageError{mathtools}
        {Invalid~ position~ specifier.~ I'll~ try~ to~ recover~ with~
        `c'}\@ehc
    }
  \collect@body\MT_mult_internal:n
}
%    \end{macrocode}
%  \end{macro}
%  \begin{macro}{\MT_shoveright:wn}
%  \begin{macro}{\MT_shoveleft:wn}
%  Extended versions of \cs{shoveleft} and \cs{shoveright}.
%    \begin{macrocode}
\newcommand*\MT_shoveright:wn [2][0pt]{%
  #2\hfilneg
  \setlength\@tempdima{#1}
  \kern\@tempdima
}
\newcommand*\MT_shoveleft:wn [2][0pt]{%
  \hfilneg
  \setlength\@tempdima{#1}
  \kern\@tempdima
  #2
}
%    \end{macrocode}
%  \end{macro}
%  \end{macro}
%  \begin{macro}{\MT_mult_internal:n}
%  \changes{v1.01a}{2004/10/10}{Added Ord atom to beginning of each line}
%  The real internal \env{multlined}.
%    \begin{macrocode}
\newcommand*\MT_mult_internal:n [1]{
 \MH_if_boolean:nF {outer_mult}{\null\,}
  \MT_next:
  \bgroup
%    \end{macrocode}
%  Restore the meaning of \cmd{\\} inside \env{multlined}, else it
%  wouldn't work in the \env{equation} environment. Set the fake row
%  counter to zero.
%    \begin{macrocode}
    \Let@
    \def\l_MT_multline_lastline_fint{0 }
    \chardef\dspbrk@context\@ne \restore@math@cr
%    \end{macrocode}
%  Use private versions.
%    \begin{macrocode}
    \MH_let:NwN \math@cr@@\MT_mult_mathcr_atat:w
    \MH_let:NwN \shoveleft\MT_shoveleft:wn
    \MH_let:NwN \shoveright\MT_shoveright:wn
    \spread@equation
    \MH_set_boolean_F:n {mult_firstline}
%    \end{macrocode}
%  Do some measuring.
%    \begin{macrocode}
    \MT_measure_mult:n {#1}
%    \end{macrocode}
%  Make sure the box is wide enough.
%    \begin{macrocode}
    \if_dim:w \l_MT_multwidth_dim<\l_MT_multline_measure_fdim
      \MH_setlength:dn \l_MT_multwidth_dim{\l_MT_multline_measure_fdim}
    \fi
    \MH_set_boolean_T:n {mult_firstline}
%    \end{macrocode}
%  Tricky bit: If we only encountered one \cmd{\\} then use a very
%  special \cs{math@cr@@} that inserts everything needed.
%    \begin{macrocode}
    \if_num:w \l_MT_multline_lastline_fint=\@ne
      \MH_let:NwN \math@cr@@ \MT_mult_firstandlast_mathcr:w
    \fi:
%    \end{macrocode}
%  Do the typesetting.
%    \begin{macrocode}
    \ialign\bgroup
      \hfil\strut@$\m@th\displaystyle{}##$\hfil
      \crcr
      \hfilneg
      #1
}
%    \end{macrocode}
%  \end{macro}
%  \begin{macro}{\MT_measure_mult:n}
%  \changes{v1.01a}{2004/10/10}{Added Ord atom to beginning of each line}
%  Measuring. Disable all labelling and check the number of lines.
%    \begin{macrocode}
\newcommand\MT_measure_mult:n [1]{
  \begingroup
    \g_MT_multlinerow_int\@ne
    \MH_let:NwN \label\MT_gobblelabel:w
    \MH_let:NwN \tag\gobble@tag
    \setbox\z@\vbox{
      \ialign{\strut@$\m@th\displaystyle{}##$
        \crcr
        #1
        \crcr
      }
    }
    \xdef\l_MT_multline_measure_fdim{\the\wdz@}
    \advance\g_MT_multlinerow_int\m@ne
    \xdef\l_MT_multline_lastline_fint{\number\g_MT_multlinerow_int}
  \endgroup
  \g_MT_multlinerow_int\@ne
}
%    \end{macrocode}
%  \end{macro}
%  \begin{macro}{\MT_multlined_second_arg:w}
%  Scan for a second optional argument.
%    \begin{macrocode}
\MaybeMHPrecedingSpacesOff
\newcommand*\MT_multlined_second_arg:w [1][\@empty]{
  \MT_test_for_tcb_other:nnnnn {#1}
    {\def\MT_mult_default_pos:{#1}}
    {\def\MT_mult_default_pos:{#1}}
    {\def\MT_mult_default_pos:{#1}}
    {
      \if_meaning:NN \@empty#1\@empty
      \else:
        \setlength \l_MT_multwidth_dim{#1}
      \fi:
    }
  \MT_start_mult:N \MT_mult_default_pos:
}
%    \end{macrocode}
%  \end{macro}
%  \begin{environment}{multlined}
%  The user environment. Scan for an optional argument.
%    \begin{macrocode}
\newenvironment{multlined}[1][]
  {\MH_group_align_safe_begin:
  \MT_test_for_tcb_other:nnnnn {#1}
    {\def\MT_mult_default_pos:{#1}}
    {\def\MT_mult_default_pos:{#1}}
    {\def\MT_mult_default_pos:{#1}}
    {
      \if_meaning:NN \@empty#1\@empty
      \else:
        \setlength \l_MT_multwidth_dim{#1}
      \fi:
    }
    \MT_multlined_second_arg:w
  }
  {
    \hfilneg  \endaligned \MH_group_align_safe_end:
  }
\MHPrecedingSpacesOn
%    \end{macrocode}
%  \end{environment}
%  The keys needed.
%    \begin{macrocode}
\define@key{\MT_options_name:}
  {firstline-afterskip}{\def\l_MT_mult_left_fdim{#1}}
\define@key{\MT_options_name:}
  {lastline-preskip}{\def\l_MT_mult_right_fdim{#1}}
\define@key{\MT_options_name:}
  {multlined-width}{\setlength \l_MT_multwidth_dim{#1}}
\define@key{\MT_options_name:}
  {multlined-pos}{\def\MT_mult_default_pos:{#1}}
\setkeys{\MT_options_name:}{
  firstline-afterskip=\multlinegap,
  lastline-preskip=\multlinegap,
  multlined-width=0pt,
  multlined-pos=c,
}
%    \end{macrocode}
%  \begin{macro}{\MT_gobblelabel:w}
%  Better than to assume that \cs{label} has exactly one mandatory
%  argument, hence the \texttt{w} specifier.
%    \begin{macrocode}
\def\MT_gobblelabel:w #1{}
%    \end{macrocode}
%  \end{macro}
%
%
%
%
%  \section{Macros suggested/requested by Lars Madsen}
%
%  The macros in this section are all requests made by Lars Madsen.
%
%  \subsection{Paired delimiters}
%
%  \changes{v1.13}{2012/05/10}{Extended \cs{DeclarePairedDelimiter(X)}}
%  \begin{macro}{\MT_delim_default_inner_wrappers:n}
%  In some cases users may want to control the internals a bit more. We
%  therefore create two call back macros each time the
%  \cs{DeclarePaired...} macro is issued. The default value of these
%  call backs are provided by |\MT_delim_default_inner_wrappers:n|
%    \begin{macrocode}
\newcommand\MT_delim_default_inner_wrappers:n [1]{
   \@namedef{MT_delim_\MH_cs_to_str:N #1 _star_wrapper:nnn}##1##2##3{
      \mathopen{}\mathclose\bgroup ##1 ##2 \aftergroup\egroup ##3
    }
    \@namedef{MT_delim_\MH_cs_to_str:N #1 _nostar_wrapper:nnn}##1##2##3{
      \mathopen{##1}##2\mathclose{##3}
    }
  }

%    \end{macrocode}
% \begin{macro}{\reDeclarePairedDelimiterInnerWrapper}
%   Macro enabling the user to alter an existing call back. Note that
%   currently no checks are performed. First argument is the name of
%   the macro we are altering (as defined via \cs{DeclarePaired...}),
%   the second is \texttt{star} or \texttt{nostar}. In the last
%   argument \texttt{\#1} and \texttt{\#3} respectively refer to the
%   scaled fences and \texttt{\#3} refer to whatever come between.
%    \begin{macrocode}
\newcommand\reDeclarePairedDelimiterInnerWrapper[3]{
  \@namedef{MT_delim_\MH_cs_to_str:N #1 _ #2 _wrapper:nnn}##1##2##3{
    #3
  }
}

%    \end{macrocode}
%   
% \end{macro}
%  \end{macro}
%  \begin{macro}{\DeclarePairedDelimiter}
%  \changes{v1.06}{2008/08/01}{Made user command robust}
%  This macro defines |#1| to be a control sequence that takes either
%  a star or an optional argument.
%    \begin{macrocode}
\newcommand*\DeclarePairedDelimiter[3]{%
  \@ifdefinable{#1}{
%    \end{macrocode}
%  Define the starred command to just put \cs{left} and \cs{right}
%  before the delimiters.
%  \changes{1.08e}{2010/09/02}{`Fixed' \cs{left}\dots\cs{right} bad spacing}
%  \changes{1.08e}{2010/09/14}{redid the \cs{left}\dots\cs{right} fix,
%  see \cs{DeclarePairedDelimiterX} for details.}
%  \changes{v1.13}{2012/05/10}{Using call back instead}
%    \begin{macrocode}
    \MT_delim_default_inner_wrappers:n{#1}
    \@namedef{MT_delim_\MH_cs_to_str:N #1 _star:}##1
      %{\mathopen{}\mathclose\bgroup\left#2 ##1 \aftergroup\egroup\right #3}%
      { \@nameuse{MT_delim_\MH_cs_to_str:N #1 _star_wrapper:nnn}%
           {\left#2}{##1}{\right#3} }%
%    \end{macrocode}
%  The command with optional argument. It should be \cs{bigg} or
%  alike.
%    \begin{macrocode}
    \@xp\@xp\@xp
      \newcommand
        \@xp\csname MT_delim_\MH_cs_to_str:N #1 _nostar:\endcsname
        [2][\\@gobble]
        { 
%    \end{macrocode}
%  With the default optional argument we wind up with \cs{relax},
%  else we get \cs{biggr} and \cs{biggl} etc.
%  \changes{v1.13}{2012/05/10}{Using call back instead}
%    \begin{macrocode}
          %\mathopen{\@nameuse {\MH_cs_to_str:N ##1 l} #2} ##2 
          %\mathclose{\@nameuse {\MH_cs_to_str:N ##1 r} #3}}
          \@nameuse{MT_delim_\MH_cs_to_str:N #1 _nostar_wrapper:nnn}%
             {\@nameuse {\MH_cs_to_str:N ##1 l} #2}
             {##2}
             {\@nameuse {\MH_cs_to_str:N ##1 r} #3}
        }
%    \end{macrocode}
%  The user command comes here. Just check for the star and choose
%  the right internal command.
%    \begin{macrocode}
    \DeclareRobustCommand{#1}{
      \@ifstar
        {\@nameuse{MT_delim_\MH_cs_to_str:N #1 _star:}}
        {\@nameuse{MT_delim_\MH_cs_to_str:N #1 _nostar:}}
    }
  }
}
%    \end{macrocode}
%  \end{macro}
%
% \begin{macro}{\DeclarePairedDelimiterX}
%  \changes{v1.08}{2010/06/10}{Added \cs{DeclarePairedDelimiterX}}
%  It has turned out that it is convenient to have a more general
%  version of \cs{DeclarePairedDelimiter}. In this version the user
%  can specify the number of arguments the created macro has, and they
%  can specify the code for the inner part of the macro. Other than
%  that the code is fairly similar to \cs{DeclarePairedDelimiter}
%  \changes{v1.08e}{2010/09/02}{Provided better implementation of 
%  \cs{DeclarePairedDelimiterX}}
%    \begin{macrocode}
\def\MHempty{}
\def\DeclarePairedDelimiterX#1[#2]#3#4#5{%
  \@ifdefinable{#1}{
%    \end{macrocode}
% The constructor takes five arguments, the name of the macro, the
% number of arguments (1-9), the left and right delimiter, the inner
% code for the two macros. First we verify that the number of arguments fit.
%    \begin{macrocode}
    \ifnum#2>9\relax
      \PackageError{mathtools}{No~ more~ than~ 9~ arguments}{}
    \else
      \ifnum#2<1\relax
        \PackageError{mathtools}{Macro~ need~ 1~ or~ more~ arguments}{}
      \fi
    \fi
%    \end{macrocode}
%  \changes{v1.13}{2012/05/10}{Using call back instead}
%  Initiate the default call backs.
%    \begin{macrocode}
    \MT_delim_default_inner_wrappers:n{#1}
%    \end{macrocode}
% We make sure to store the delimiter size in the local variable
% \cs{delimsize}. Then users can refer to the size in the fifth
% argument. In the starred version it will refer to \cs{middle} and in
% the normal version it will hold the provided optional argument.
%    \begin{macrocode}
    \@xp\@xp\@xp
      \newcommand
        \@xp\csname MT_delim_\MH_cs_to_str:N #1 _star:\endcsname
        [#2]
        {
          \begingroup
            \def\delimsize{\middle}
%    \end{macrocode}
% This is slightly controversial, \cs{left}\dots\cs{right} are known
% to produce an inner atom, thus may cause different spacing than
% normal delimiters. We `fix' this by introducing \cs{mathopen} and
% \cs{mathclose}. This change is now factored out into call backs. 
% \changes{v1.08e}{2010/09/14}{redid the left/right fix, inspired by
% ctt thread named `spacing after \cs{right}) and before \cs{left})'
% started 2010-08-12.}
% \changes{v1.13}{2012/05/10}{Using call back instead}
%    \begin{macrocode}
            %\mathopen{}\mathclose\bgroup\left#3 #5 \aftergroup\egroup\right#4
            \@nameuse{MT_delim_\MH_cs_to_str:N #1 _star_wrapper:nnn}
              {\left#3}{#5}{\right#4}
          \endgroup
        }
%    \end{macrocode}
% In order for the starred and non-starred version to have the same
% arguments, we need to introduce an extra macro to catch the optional
% argument (this means that the non-starred version can actually
% support ten arguments!).
% Here we do things a little differently than with
% \cs{DeclarePairedDelimiter}. The optional argument have \cs{MHempty}
% as the default. This is locally changed when we scale the
% delimiters, such that it can eat the l/r if needed.
%    \begin{macrocode}
    \@xp\@xp\@xp
      \newcommand
        \@xp\csname MT_delim_\MH_cs_to_str:N #1 _nostar:\endcsname
        [1][\MHempty]
      {
%    \end{macrocode}
% We need to introduce a local group in order to support nesting. It
% is ended inside \verb|\MT_delim_\MH_cs_to_str:N #1 _nostar_inner:|
%    \begin{macrocode}
        \begingroup
        \def\delimsize{##1}
        \@nameuse{MT_delim_\MH_cs_to_str:N #1 _nostar_inner:}
      } 
%    \end{macrocode}
% Next we provide the inner workhorse. We need a bit of expansion
% magic to get \cs{delimsize} to work.
% \changes{v1.13}{2012/05/10}{Using call back instead}
%    \begin{macrocode}
    \@xp\@xp\@xp
      \newcommand
        \@xp\csname MT_delim_\MH_cs_to_str:N #1 _nostar_inner:\endcsname
        [#2]
        {
          %\mathopen{%
          %  \let\MHempty\@gobble
          %  \@xp\@xp\@xp\csname\@xp\MH_cs_to_str:N \delimsize l\endcsname #3} 
          %#5
          %\mathclose{%
          %  \let\MHempty\@gobble
          %  \@xp\@xp\@xp\csname\@xp\MH_cs_to_str:N \delimsize r\endcsname #4}
          \@nameuse{MT_delim_\MH_cs_to_str:N #1 _nostar_wrapper:nnn}
          {
            \let\MHempty\@gobble
            \@xp\@xp\@xp\csname\@xp\MH_cs_to_str:N \delimsize l\endcsname #3
          }
          {#5}
          {
            \let\MHempty\@gobble
            \@xp\@xp\@xp\csname\@xp\MH_cs_to_str:N \delimsize r\endcsname #4
          }
          \endgroup
        }
    \DeclareRobustCommand{#1}{
      \@ifstar
        {\@nameuse{MT_delim_\MH_cs_to_str:N #1 _star:}}
        {\@nameuse{MT_delim_\MH_cs_to_str:N #1 _nostar:}}
    }
  }
}
%    \end{macrocode}
% \end{macro}
%
%  \subsection{A \texttt{\textbackslash displaystyle} \env{cases} environment}
%
%  \begin{macro}{\MT_start_cases:nnn}
%  We define a single command that does all the hard work.
%  \changes{v1.08}{2010/06/10}{made \cs{MT_start_cases:nnnn} more general}
%    \begin{macrocode}
\def\MT_start_cases:nnnn #1#2#3#4{ % #1=sep,#2=lpreamble,#3=rpreamble,#4=delim
 \RIfM@\else
   \nonmatherr@{\begin{\@currenvir}}
 \fi
 \MH_group_align_safe_begin:
 \left#4
 \vcenter \bgroup
     \Let@ \chardef\dspbrk@context\@ne \restore@math@cr
     \let  \math@cr@@\AMS@math@cr@@
     \spread@equation
     \ialign\bgroup
%    \end{macrocode}
%  Set the first column flush left in \cs{displaystyle} math and the
%  second as specified by the second argument. The first argument is
%  the separation between the columns. It could be a \cs{quad} or
%  something entirely different.
%    \begin{macrocode}
       \strut@#2 &#1\strut@
       #3
       \crcr
}
%    \end{macrocode}
%  \end{macro}
% \begin{macro}{\MH_end_cases:}
%    \begin{macrocode}
\def\MH_end_cases:{\crcr\egroup
 \restorecolumn@
 \egroup
 \MH_group_align_safe_end:
}
%    \end{macrocode}
% \end{macro}
%  \begin{macro}{\newcases}
%  \begin{macro}{\renewcases}
%  Easy creation of new \env{cases}-like environments.
%  \changes{v1.08}{2010/06/10}{changed to match the change in \cs{MT_start_cases:nnnn}}
%    \begin{macrocode}
\newcommand*\newcases[6]{% #1=name, #2=sep, #3=preamble, #4=left, #5=right
 \newenvironment{#1}
   {\MT_start_cases:nnnn {#2}{#3}{#4}{#5}}
   {\MH_end_cases:\right#6}
}
\newcommand*\renewcases[6]{
 \renewenvironment{#1}
   {\MT_start_cases:nnnn {#2}{#3}{#4}{#5}}
   {\MH_end_cases:\right#6}
}
%    \end{macrocode}
%  \begin{environment}{dcases}
%  \begin{environment}{dcases*}
%  \begin{environment}{rcases}
%  \begin{environment}{rcases*}
%  \begin{environment}{drcases}
%  \begin{environment}{drcases*}
%  \begin{environment}{cases*}
%  \env{dcases} is a traditional cases with display style math in
%  both columns, while \env{dcases*} has text in the second column.
%  \changes{v1.08}{2010/06/10}{changed to match the change in
%  \cs{newcases} plus added rcases and drcases}
%    \begin{macrocode}
\newcases{dcases}{\quad}{%
  $\m@th\displaystyle{##}$\hfil}{$\m@th\displaystyle{##}$\hfil}{\lbrace}{.}
\newcases{dcases*}{\quad}{%
  $\m@th\displaystyle{##}$\hfil}{{##}\hfil}{\lbrace}{.}
\newcases{rcases}{\quad}{%
  $\m@th{##}$\hfil}{$\m@th{##}$\hfil}{.}{\rbrace}
\newcases{rcases*}{\quad}{%
  $\m@th{##}$\hfil}{{##}\hfil}{.}{\rbrace}
\newcases{drcases}{\quad}{%
  $\m@th\displaystyle{##}$\hfil}{$\m@th\displaystyle{##}$\hfil}{.}{\rbrace}
\newcases{drcases*}{\quad}{%
  $\m@th\displaystyle{##}$\hfil}{{##}\hfil}{.}{\rbrace}
\newcases{cases*}{\quad}{%
  $\m@th{##}$\hfil}{{##}\hfil}{\lbrace}{.}
%    \end{macrocode}
%  \end{environment}
%  \end{environment}
%  \end{environment}
%  \end{environment}
%  \end{environment}
%  \end{environment}
%  \end{environment}
%  \end{macro}
%  \end{macro}
%
%  \subsection{New matrix environments}
%  \begin{macro}{\MT_matrix_begin:N}
%  \begin{macro}{\MT_matrix_end:}
%  Here are a few helpers for the matrices. \cs{MT_matrix_begin:N}
%  takes one argument specifying the column type for the array inside
%  the matrix. and \cs{MT_matrix_end:} inserts the correct ending.
%    \begin{macrocode}
\def\MT_matrix_begin:N #1{%
  \hskip -\arraycolsep
  \MH_let:NwN \@ifnextchar \MH_nospace_ifnextchar:Nnn
  \array{*\c@MaxMatrixCols #1}}
\def\MT_matrix_end:{\endarray \hskip -\arraycolsep}
%    \end{macrocode}
%  \end{macro}
%  \end{macro}
%  Before we define the environments we better make sure that spaces
%  before the optional argument is disallowed. Else a user who types
%  \begin{verbatim}
%  \[
%    \begin{pmatrix*}
%      [c] & a \\
%       b  & d
%    \end{pmatrix*}
%  \]
%  \end{verbatim}
%  will lose the \texttt{[c]}!
%    \begin{macrocode}
\MaybeMHPrecedingSpacesOff
%    \end{macrocode}
%  \begin{environment}{matrix*}
%  This environment is just like \env{matrix} only it takes an
%  optional argument specifying the column type.
%    \begin{macrocode}
\newenvironment{matrix*}[1][c]
  {\MT_matrix_begin:N #1}
  {\MT_matrix_end:}
%    \end{macrocode}
%  \end{environment}
%  \begin{environment}{pmatrix*}
%  \begin{environment}{bmatrix*}
%  \begin{environment}{Bmatrix*}
%  \begin{environment}{vmatrix*}
%  \begin{environment}{Vmatrix*}
%  Then starred versions of the other \AmS{} matrices.
%    \begin{macrocode}
\newenvironment{pmatrix*}[1][c]
  {\left(\MT_matrix_begin:N #1}
  {\MT_matrix_end:\right)}
\newenvironment{bmatrix*}[1][c]
  {\left[\MT_matrix_begin:N #1}
  {\MT_matrix_end:\right]}
\newenvironment{Bmatrix*}[1][c]
  {\left\lbrace\MT_matrix_begin:N #1}
  {\MT_matrix_end:\right\rbrace}
\newenvironment{vmatrix*}[1][c]
  {\left\lvert\MT_matrix_begin:N #1}
  {\MT_matrix_end:\right\rvert}
\newenvironment{Vmatrix*}[1][c]
  {\left\lVert\MT_matrix_begin:N #1}
  {\MT_matrix_end:\right\lVert}
%    \end{macrocode}
%  \end{environment}
%  \end{environment}
%  \end{environment}
%  \end{environment}
%  \end{environment}
%
% \changes{v1.10}{2011/02/12}{Added the code below, courtesy of Rasmus Villemoes}
% Now we are at it why not provide fenced versions of the
% \env{smallmatrix} construction as well. We will only provide a
% version that can be adjusted as \env{matrix*} above, thus we keep
% the * in the name. The implementation is courtesy of Rasmus
% Villemoes. Rasmus also suggested making the default alignment in
% these environments globally adjustable, so we did
% (\texttt{smallmatrix-align=c} by default). It \emph{is} possible to
% do something similar with the large matrix environments, but that
% might cause problems with the \texttt{array} package, thus for now
% we lease that feature alone.
%
% The base code is a variation over the original \env{smallmatrix}
% environmetn fround in \texttt{amsmath}, thus we will not comment it further.
% 
% TODO: make the code check that the optional argument is either
% \texttt{c}, \texttt{l} or \texttt{r}.
%    \begin{macrocode}
\def\MT_smallmatrix_begin:N #1{%
  \Let@\restore@math@cr\default@tag
  \baselineskip6\ex@ \lineskip1.5\ex@ \lineskiplimit\lineskip
  \csname MT_smallmatrix_#1_begin:\endcsname
}
\def\MT_smallmatrix_end:{\crcr\egroup\egroup\MT_smallmatrix_inner_space:}
\def\MT_smallmatrix_l_begin:{\null\MT_smallmatrix_inner_space:\vcenter\bgroup
  \ialign\bgroup$\m@th\scriptstyle##$\hfil&&\thickspace
  $\m@th\scriptstyle##$\hfil\crcr
}
\def\MT_smallmatrix_c_begin:{\null\MT_smallmatrix_inner_space:\vcenter\bgroup
  \ialign\bgroup\hfil$\m@th\scriptstyle##$\hfil&&\thickspace\hfil
  $\m@th\scriptstyle##$\hfil\crcr
}
\def\MT_smallmatrix_r_begin:{\null\MT_smallmatrix_inner_space:\vcenter\bgroup
  \ialign\bgroup\hfil$\m@th\scriptstyle##$&&\thickspace\hfil
  $\m@th\scriptstyle##$\crcr
}
\newenvironment{smallmatrix*}[1][\MT_smallmatrix_default_align:]
  {\MT_smallmatrix_begin:N #1}
  {\MT_smallmatrix_end:}
%    \end{macrocode}
% We would like to keep to the tradition of the \verb?Xmatrix? and
% \verb?Xmatrix*? macros we added earlier, since most code is similar
% we define them using a constructor macro. We also apply the trick
% used within \verb?\DeclarePairedDelimiter(X)? such that \verb?\left?
% \verb?\right?  constructions produce spacings corresponding to
% \verb?\mathopen? and \verb?\mathclose?.
%    \begin{macrocode}
\def\MT_fenced_sm_generator:nnn #1#2#3{%
  \@ifundefined{#1}{%
    \newenvironment{#1}
    {\@nameuse{#1hook}\mathopen{}\mathclose\bgroup\left#2\MT_smallmatrix_begin:N c}%
      {\MT_smallmatrix_end:\aftergroup\egroup\right#3}%
  }{}%
  \@ifundefined{#1*}{%
    \newenvironment{#1*}[1][\MT_smallmatrix_default_align:]%
    {\@nameuse{#1hook}\mathopen{}\mathclose\bgroup\left#2\MT_smallmatrix_begin:N ##1}%
      {\MT_smallmatrix_end:\aftergroup\egroup\right#3}%
  }{}%
}
\MT_fenced_sm_generator:nnn{psmallmatrix}()
\MT_fenced_sm_generator:nnn{bsmallmatrix}[]
\MT_fenced_sm_generator:nnn{Bsmallmatrix}\lbrace\rbrace
\MT_fenced_sm_generator:nnn{vsmallmatrix}\lvert\rvert
\MT_fenced_sm_generator:nnn{Vsmallmatrix}\lVert\rVert
%    \end{macrocode}
% 
% The options associated with this.
%    \begin{macrocode}
\define@key{\MT_options_name:}
  {smallmatrix-align}{\def\MT_smallmatrix_default_align:{#1}}
\define@key{\MT_options_name:}
  {smallmatrix-inner-space}{\def\MT_smallmatrix_inner_space:{#1}}
\setkeys{\MT_options_name:}{
  smallmatrix-align=c,
  smallmatrix-inner-space=\,
}

%    \end{macrocode}
%  Restore the usual spacing behavior.
%    \begin{macrocode}
\MHPrecedingSpacesOn
%    \end{macrocode}
%
%  \subsection{Smashing an operator with limits}
%
%  \begin{macro}{\smashoperator}
%  The user command. Define \cs{MT_smop_use:NNNNN} to be one of the
%  specialized commands \cs{MT_smop_smash_l:NNNNN},
%  \cs{MT_smop_smash_r:NNNNN}, or the default
%  \cs{MT_smop_smash_lr:NNNNN}.
%    \begin{macrocode}
\newcommand*\smashoperator[2][lr]{
  \def\MT_smop_use:NNNNN {\@nameuse{MT_smop_smash_#1:NNNNN}}
  \toks@{#2}
  \expandafter\MT_smop_get_args:wwwNnNn
    \the\toks@\@nil\@nil\@nil\@nil\@nil\@nil\@@nil
}
%    \end{macrocode}
%  \end{macro}
%  \begin{macro}{\MT_smop_remove_nil_vi:N}
%  \begin{macro}{\MT_smop_mathop:n}
%  \begin{macro}{\MT_smop_limits:}
%  Some helper functions.
%    \begin{macrocode}
\def\MT_smop_remove_nil_vi:N #1\@nil\@nil\@nil\@nil\@nil\@nil{#1}
\def\MT_smop_mathop:n {\mathop}
\def\MT_smop_limits: {\limits}
%    \end{macrocode}
%  \end{macro}
%  \end{macro}
%  \end{macro}
%  Some conditionals.
%    \begin{macrocode}
\MH_new_boolean:n {smop_one}
\MH_new_boolean:n {smop_two}
%    \end{macrocode}
%  \begin{macro}{\MT_smop_get_args:wwwNnNn}
%  The argument stripping. There are three different valid types of
%  input:
%  \begin{enumerate}
%    \item An operator with neither subscript nor superscript.
%    \item An operator with one subscript or superscript.
%    \item An operator with both subscript and superscript.
%  \end{enumerate}
%  Additionally an operator can be either a single macro as in
%  \cs{sum} or in \cs{mathop}\arg{A} and people might be tempted to
%  put a \cs{limits} after the operator, even though it's not
%  necessary. Thus the input with most tokens would be something like
%  \begin{verbatim}
%    \mathop{TTT}\limits_{sub}^{sup}
%  \end{verbatim}
%  Therefore we have to scan for seven arguments, but there might
%  only be one actually. So let's list the possible situations:
%  \begin{enumerate}
%    \item \verb|\mathop{TTT}\limits_{subsub}^{supsup}|
%    \item \verb|\mathop{TTT}_{subsub}^{supsup}|
%    \item \verb|\sum\limits_{subsub}^{supsup}|
%    \item \verb|\sum_{subsub}^{supsup}|
%  \end{enumerate}
%  Furthermore the |_{subsub}^{supsup}| part can also just be
%  |_{subsub}| or empty.
%    \begin{macrocode}
\def\MT_smop_get_args:wwwNnNn #1#2#3#4#5#6#7\@@nil{%
  \begingroup
    \def\MT_smop_arg_A: {#1} \def\MT_smop_arg_B: {#2}
    \def\MT_smop_arg_C: {#3} \def\MT_smop_arg_D: {#4}
    \def\MT_smop_arg_E: {#5} \def\MT_smop_arg_F: {#6}
    \def\MT_smop_arg_G: {#7}
%    \end{macrocode}
%  Check if A is \cs{mathop}. If it is, we know that B is the argument
%  of the \cs{mathop}.
%    \begin{macrocode}
    \if_meaning:NN \MT_smop_arg_A: \MT_smop_mathop:n
%    \end{macrocode}
%  If A was \cs{mathop} we check if C is \cs{limits}
%    \begin{macrocode}
      \if_meaning:NN \MT_smop_arg_C:\MT_smop_limits:
        \def\MT_smop_final_arg_A:{#1{#2}}%
%    \end{macrocode}
%  Now we have something like \verb|\mathop{TTT}\limits|. Then check
%  if D is \cs{@nil}.
%    \begin{macrocode}
        \if_meaning:NN \MT_smop_arg_D: \@nnil
        \else:
          \MH_set_boolean_T:n {smop_one}
          \MH_let:NwN \MT_smop_final_arg_B: \MT_smop_arg_D:
          \MH_let:NwN \MT_smop_final_arg_C: \MT_smop_arg_E:
          \if_meaning:NN \MT_smop_arg_F: \@nnil
          \else:
            \MH_set_boolean_T:n {smop_two}
            \MH_let:NwN \MT_smop_final_arg_D: \MT_smop_arg_F:
            \edef\MT_smop_final_arg_E:
              {\expandafter\MT_smop_remove_nil_vi:N \MT_smop_arg_G: }
          \fi:
        \fi:
      \else:
%    \end{macrocode}
%  Here we have something like \verb|\mathop{TTT}|. Still check
%  if D is \cs{@nil}.
%    \begin{macrocode}
        \def\MT_smop_final_arg_A:{#1{#2}}%
        \if_meaning:NN \MT_smop_arg_D: \@nnil
        \else:
          \MH_set_boolean_T:n {smop_one}
          \MH_let:NwN \MT_smop_final_arg_B: \MT_smop_arg_C:
          \MH_let:NwN \MT_smop_final_arg_C: \MT_smop_arg_D:
          \if_meaning:NN \MT_smop_arg_F: \@nnil
          \else:
            \MH_set_boolean_T:n {smop_two}
            \MH_let:NwN \MT_smop_final_arg_D: \MT_smop_arg_E:
            \MH_let:NwN \MT_smop_final_arg_E: \MT_smop_arg_F:
          \fi:
        \fi:
      \fi:
%    \end{macrocode}
%  If A was not \cs{mathop}, it is an operator in itself, so we check
%  if B is \cs{limits}
%    \begin{macrocode}
    \else:
      \if_meaning:NN \MT_smop_arg_B:\MT_smop_limits:
        \def\MT_smop_final_arg_A:{#1}%
        \if_meaning:NN \MT_smop_arg_D: \@nnil
        \else:
          \MH_set_boolean_T:n {smop_one}
          \MH_let:NwN \MT_smop_final_arg_B: \MT_smop_arg_C:
          \MH_let:NwN \MT_smop_final_arg_C: \MT_smop_arg_D:
          \if_meaning:NN \MT_smop_arg_F: \@nnil
          \else:
            \MH_set_boolean_T:n {smop_two}
            \MH_let:NwN \MT_smop_final_arg_D: \MT_smop_arg_E:
            \MH_let:NwN \MT_smop_final_arg_E: \MT_smop_arg_F:
          \fi:
        \fi:
      \else:
%    \end{macrocode}
%  No \cs{limits} was found, so we already have the right input. Just
%  forget about the last two arguments.
%    \begin{macrocode}
        \def\MT_smop_final_arg_A:{#1}%
        \if_meaning:NN \MT_smop_arg_C: \@nnil
        \else:
          \MH_set_boolean_T:n {smop_one}
          \MH_let:NwN \MT_smop_final_arg_B: \MT_smop_arg_B:
          \MH_let:NwN \MT_smop_final_arg_C: \MT_smop_arg_C:
          \if_meaning:NN \MT_smop_arg_D: \@nnil
          \else:
            \MH_set_boolean_T:n {smop_two}
            \MH_let:NwN \MT_smop_final_arg_D: \MT_smop_arg_D:
            \MH_let:NwN \MT_smop_final_arg_E: \MT_smop_arg_E:
          \fi:
        \fi:
      \fi:
    \fi:
%    \end{macrocode}
%  No reason to measure if there's no sub or sup.
%    \begin{macrocode}
    \MH_if_boolean:nT {smop_one}{
      \MT_smop_measure:NNNNN
      \MT_smop_final_arg_A: \MT_smop_final_arg_B: \MT_smop_final_arg_C:
      \MT_smop_final_arg_D: \MT_smop_final_arg_E:
    }
    \MT_smop_use:NNNNN
      \MT_smop_final_arg_A: \MT_smop_final_arg_B: \MT_smop_final_arg_C:
      \MT_smop_final_arg_D: \MT_smop_final_arg_E:
  \endgroup
}
%    \end{macrocode}
%  \end{macro}
%  Typeset what is necessary and ignore width of sub and sup:
%    \begin{macrocode}
\def\MT_smop_needed_args:NNNNN #1#2#3#4#5{%
  \displaystyle #1
  \MH_if_boolean:nT {smop_one}{
%    \end{macrocode}
%  Let's use the internal versions of \cs{crampedclap} now that we now
%  it is set in \cs{scriptstyle}.
%    \begin{macrocode}
    \limits#2{\MT_cramped_clap_internal:Nn \scriptstyle{#3}}
    \MH_if_boolean:nT {smop_two}{
      #4{\MT_cramped_clap_internal:Nn \scriptstyle{#5}}
    }
  }
}
%    \end{macrocode}
%  Measure the natural width. \cs{@tempdima} holds the dimen we need to
%  adjust it all with.
%    \begin{macrocode}
\def\MT_smop_measure:NNNNN #1#2#3#4#5{%
  \MH_let:NwN \MT_saved_mathclap:Nn \MT_cramped_clap_internal:Nn
  \MH_let:NwN \MT_cramped_clap_internal:Nn \@secondoftwo
  \sbox\z@{$\m@th\MT_smop_needed_args:NNNNN #1#2#3#4#5$}
  \MH_let:NwN \MT_cramped_clap_internal:Nn \MT_saved_mathclap:Nn
  \sbox\tw@{$\m@th\displaystyle#1$}
  \@tempdima=.5\wd0
  \advance\@tempdima-.5\wd2
}
%    \end{macrocode}
%  The `l' variant
%    \begin{macrocode}
\def\MT_smop_smash_l:NNNNN #1#2#3#4#5{
  \MT_smop_needed_args:NNNNN #1#2#3#4#5\kern\@tempdima
}
%    \end{macrocode}
%  The `r' variant
%    \begin{macrocode}
\def\MT_smop_smash_r:NNNNN #1#2#3#4#5{
  \kern\@tempdima\MT_smop_needed_args:NNNNN #1#2#3#4#5
}
%    \end{macrocode}
%  The `lr' variant
%    \begin{macrocode}
\def\MT_smop_smash_lr:NNNNN #1#2#3#4#5{
  \MT_smop_needed_args:NNNNN #1#2#3#4#5
}
%    \end{macrocode}
%
%
%  \subsection{Adjusting limits}
%
%
%  \begin{macro}{\MT_vphantom:Nn}
%  \begin{macro}{\MT_hphantom:Nn}
%  \begin{macro}{\MT_phantom:Nn}
%  \begin{macro}{\MT_internal_phantom:N}
%  The main advantage of \cs{phantom} et al., is the ability to
%  choose the right size automatically, but it requires the input to
%  be typeset four times. Since we will need to have a \cs{cramped}
%  inside a \cs{vphantom} it is much, much faster to choose the style
%  ourselves (we already know it). These macros make it possible.
%    \begin{macrocode}
\def\MT_vphantom:Nn {\v@true\h@false\MT_internal_phantom:N}
\def\MT_hphantom:Nn {\v@false\h@true\MT_internal_phantom:N}
\def\MT_phantom:Nn {\v@true\h@true\MT_internal_phantom:N}
\def\MT_internal_phantom:N #1{
  \ifmmode
    \expandafter\mathph@nt\expandafter#1
  \else
    \expandafter\makeph@nt
  \fi
}
%    \end{macrocode}
%  \end{macro}
%  \end{macro}
%  \end{macro}
%  \end{macro}
%
%  \begin{macro}{\adjustlimits}
%  This is for making sure limits line up on two consecutive
%  operators.
%    \begin{macrocode}
\newcommand*\adjustlimits[6]{
%    \end{macrocode}
%  We measure the two operators and save the difference of their
%  depths.
%    \begin{macrocode}
  \sbox\z@{$\m@th \displaystyle #1$}
  \sbox\tw@{$\m@th \displaystyle #4$}
  \@tempdima=\dp\z@ \advance\@tempdima-\dp\tw@
%    \end{macrocode}
%  We force \cs{displaystyle} for the operator and \cs{scripstyle}
%  for the limit. If we make use of the regular \cs{smash},
%  \cs{vphantom}, and \cs{cramped} macros, and let \TeX{} choose the
%  right style for each one of them, we get a lot of redundant code
%  as we have no need for the combination
%  $(\cs{displaystyle},\cs{textstyle})$ etc. Only
%  $(\cs{scriptstyle},\cs{scriptstyle})$ is useful.
%    \begin{macrocode}
  \if_dim:w \@tempdima>\z@
    \mathop{#1}\limits#2{#3}
  \else:
    \mathop{#1\MT_vphantom:Nn \displaystyle{#4}}\limits
    #2{
        \def\finsm@sh{\ht\z@\z@ \box\z@}
        \mathsm@sh\scriptstyle{\MT_cramped_internal:Nn \scriptstyle{#3}}
        \MT_vphantom:Nn \scriptstyle
          {\MT_cramped_internal:Nn \scriptstyle{#6}}
    }
  \fi:
  \if_dim:w \@tempdima>\z@
    \mathop{#4\MT_vphantom:Nn \displaystyle{#1}}\limits
    #5
    {
      \MT_vphantom:Nn \scriptstyle
        {\MT_cramped_internal:Nn \scriptstyle{#3}}
      \def\finsm@sh{\ht\z@\z@ \box\z@}
      \mathsm@sh\scriptstyle{\MT_cramped_internal:Nn \scriptstyle{#6}}
    }
  \else:
    \mathop{#4}\limits#5{#6}
  \fi:
}
%    \end{macrocode}
%  \end{macro}
% 
%  \subsection{Swapping above display skip}
%
% \begin{macro}{\SwapAboveDisplaySkip}
%   This macro is intended to be used at the start of \AmS\
%   environments, in order to force it to use
%   \cs{abovedisplayshortskip} instead of \cs{abovedisplayskip} above
%   the displayed math. Because of the use of \cs{noalign} it will not
%   work inside \env{equation} or \env{multline}.
%    \begin{macrocode}
\newcommand\SwapAboveDisplaySkip{%
  \noalign{\vskip-\abovedisplayskip\vskip\abovedisplayshortskip}
}

%    \end{macrocode}
%   
% \end{macro}
%
%  \subsection{An aid to alignment}
%
% \begin{macro}{\MoveEqLeft}
%   \changes{v1.05}{2008/06/05}{Added \cs{MoveEqLeft} (daleif)}
%   \changes{v1.05b}{2008/06/18}{We don't need \cs{setlength} here
%     (daleif), after discussion about \cs{global} and \cs{setlength}
%     on ctt} 
%   \changes{v1.12}{2011/06/12}{We don't even need lengths. GL
%   suggested on ctt to just apply them directly.}
%   This is a very simple macro, we `move' a line in an
%   alignment backwards in order to simulate that all subsequent lines
%   have been indented. Note that simply using \verb+\kern-2m+ after
%   the \verb+&+ is not enough, then the alignemnt environment never
%   detects that there is anything (though simulated) in the cell
%   before the \verb+&+.
%    \begin{macrocode}
\newcommand\MoveEqLeft[1][2]{\kern #1em  &   \kern -#1em}
%    \end{macrocode}
% \end{macro}
%
% \begin{macro}{\Aboxed}
% \changes{v1.08}{2010/06/29}{Added \cs{Aboxed}}
% The idea from \cs{MoveEqLeft} can be used for other things. Here we
% create a macro that will allow a user to box an equation inside an
% alignment.
% \changes{v1.12}{2011/08/17}{\cs{Aboxed} reimplemented, cudos to GL}
%    \begin{macrocode}
\newcommand\Aboxed[1]{\let\bgroup{\romannumeral-`}\@Aboxed#1&&\ENDDNE}
%    \end{macrocode}
% The macro has been reimplemented courtesy of Florent Chervet out of
% a posting on ctt, \url{https://groups.google.com/group/comp.text.tex/browse_thread/thread/5d66395f2a1b5134/93fd9661484bd8d8?#93fd9661484bd8d8}
%    \begin{macrocode}
\def\@Aboxed#1&#2&#3\ENDDNE{%
  \ifnum0=`{}\fi \setbox \z@
    \hbox{$\displaystyle#1{}\m@th$\kern\fboxsep \kern\fboxrule }%
    \edef\@tempa {\kern  \wd\z@ &\kern -\the\wd\z@ \fboxsep
        \the\fboxsep \fboxrule \the\fboxrule }\@tempa \boxed {#1#2}%
} 
%    \end{macrocode}
% \end{macro}
%
% \begin{macro}{\ArrowBetweenLines}
%   \changes{v1.05}{2008/06/05}{Added \cs{ArrowBetweenLines} as it
%   belongs here and not just in my \LaTeX book (daleif)}
%   ????Implementation notes are needed????
%    \begin{macrocode}
\MHInternalSyntaxOff
\def\ArrowBetweenLines{\relax
  \iffalse{\fi\ifnum0=`}\fi
  \@ifstar{\ArrowBetweenLines@auxI{00}}{\ArrowBetweenLines@auxI{01}}}
\def\ArrowBetweenLines@auxI#1{%
  \@ifnextchar[%
  {\ArrowBetweenLines@auxII{#1}}%
  {\ArrowBetweenLines@auxII{#1}[\Updownarrow]}}
\def\ArrowBetweenLines@auxII#1[#2]{%
  \ifnum0=`{\fi \iffalse}\fi
%    \end{macrocode}
% It turns out that for some reason the \cs{crcr} (next) removes the
% automatic equation number replacement. The replacement hack seems to
% the trick, though I have no idea why \cs{crcr} broke things (/daleif).
% \changes{v1.08}{2010/06/15}{fixed eq num replacement bug}
%    \begin{macrocode}
%  \crcr
    \expandafter\in@\expandafter{\@currenvir}%
      {alignedat,aligned,gathered}%
      \ifin@ \else
      \notag
      \fi%
   \\
  \noalign{\nobreak\vskip-\baselineskip\vskip-\lineskip}%
  \noalign{\expandafter\in@\expandafter{\@currenvir}%
      {alignedat,aligned,gathered}%
      \ifin@ \else\notag\fi%
  }%
  \if#1 &&\quad #2\else #2\quad\fi
  \\\noalign{\nobreak\vskip-\lineskip}}

\MHInternalSyntaxOn
%    \end{macrocode}
% \end{macro}
%
%
% \subsection{Centered vertical dots}
% 
% Doing a \verb?\vdots? centered within a different sized box, is
% rather easy with the tools available. Note that it does \emph{not}
% check for the style we are running in, thus do not expect this to
% work well within \verb?\scriptstyle? and smaller. Basically we
% create a box of a width corresponding to \verb?{}#1{}? and center
% the \verb?\vdots? within it.
%    \begin{macrocode}
\newcommand\vdotswithin[1]{%
  {\mathmakebox[\widthof{\ensuremath{{}#1{}}}][c]{{\vdots}}}}
%    \end{macrocode}
% Next we are inspired by \verb?\ArrowBetweenLines? and provide a
% costruction to be used within alignments with much less vertical
% space above and below. 
%
% First in order to support \env{spreadlines} we need to store the
% original value of \verb?\jot? (and hope the user does not mess with it).
%    \begin{macrocode}
\newlength\origjot
\setlength\origjot{\jot}
%    \end{macrocode}
% Next define how much we spacing we flush out, and make this user adjustable.
%    \begin{macrocode}
\newdimen\l_MT_shortvdotswithinadjustabove_dim
\newdimen\l_MT_shortvdotswithinadjustbelow_dim
\define@key{\MT_options_name:}
  {shortvdotsadjustabove}{\setlength\l_MT_shortvdotswithinadjustabove_dim{#1}}
\define@key{\MT_options_name:}
  {shortvdotsadjustbelow}{\setlength\l_MT_shortvdotswithinadjustbelow_dim{#1}}
%    \end{macrocode}
% The actual defaults we found by trail and error.
%    \begin{macrocode}
\setkeys{\MT_options_name:}{
  shortvdotsadjustabove=2.15\origjot,
  shortvdotsadjustbelow=\origjot
}
%    \end{macrocode}
% The user macro comes in two versions, starred version corresponding
% to alignment \emph{before} and \verb?&? and a non-starred version
% with alignment \emph{after} \verb?&?.
%    \begin{macrocode}
\def\shortvdotswithin{\relax
  \@ifstar{\MT_svwi_aux:nn{00}}{\MT_svwi_aux:nn{01}}}
\def\MT_svwi_aux:nn #1#2{
  \MTFlushSpaceAbove
  \if#1 \vdotswithin{#2}& \else &\vdotswithin{#2}  \fi
  \MTFlushSpaceBelow
}
%    \end{macrocode}
% We will need a way to remove any tags (eq. numbers) on the
% \verb?\vdots? line. We cannot use the method used by
% \verb?\ArrowBetweenLines? so we use inspiration from
% \texttt{etoolbox}.  
%    \begin{macrocode}
\def\MT_remove_tag_unless_inner:n #1{%
  \begingroup
  \def\etb@tempa##1|#1|##2\MT@END{\endgroup
    \ifx\@empty##2\@empty\notag\fi}%
  \expandafter\etb@tempa\expandafter|alignedat|aligned|split|#1|\MT@END}
%    \end{macrocode}
% These macros take care of removing the space above or below. Since
% these may be useful for the user in very special cases, we provide
% them as separate macros.
%    \begin{macrocode}
\newcommand\MTFlushSpaceAbove{  
  \expandafter\MT_remove_tag_unless_inner:n\expandafter{\@currenvir}
  \\
  \noalign{%
    \nobreak\vskip-\baselineskip\vskip-\lineskip%
      \vskip-\l_MT_shortvdotswithinadjustabove_dim 
      \vskip-\origjot
      \vskip\jot
  }%
  \noalign{
    \expandafter\MT_remove_tag_unless_inner:n\expandafter{\@currenvir}
  }
}
\newcommand\MTFlushSpaceBelow{
  \\\noalign{%
    \nobreak\vskip-\lineskip
    \vskip-\l_MT_shortvdotswithinadjustbelow_dim
    \vskip-\origjot
    \vskip\jot
  }
}

%    \end{macrocode}
%
%
%  \section{A few extra symbols}
%
%  Most math font sets are missing three symbols: \cs{nuparrow},
%  \cs{ndownarrow} and \cs{bigtimes}. We provide \emph{simulated}
%  versions of these symbols in case they are missing. 
%
%  \subsection{Negated up- and down arrows} 
%  
%  Note that the \cs{nuparrow} and the \cs{ndownarrow} are made from
%  \cs{nrightarrow} and \cs{nleftarrow}, so these have to be
%  present. If they are not, we throw an error at use. The
%  implementation details are due to Enrico Gregorio
%  (\url{http://groups.google.com/group/comp.text.tex/msg/689cc8bd604fdb51}),
%  the basic idea is to reflect and rotate existing negated
%  arrows. Note that the reflection and rotation will not show up i
%  most DVI previewers.
% \begin{macro}{\MH_nrotarrow:NN}
% \changes{v1.07}{2008/08/11}{Added support for \cs{nuparrow} and \cs{ndownaddow}}
%  First a common construction macro.
%    \begin{macrocode}
\def\MH_nrotarrow:NN #1#2{%
  \setbox0=\hbox{$\m@th#1\uparrow$}\dimen0=\dp0
  \setbox0=\hbox{%
    \reflectbox{\rotatebox[origin=c]{90}{$\m@th#1\mkern2.22mu #2$}}}%
  \dp0=\dimen0 \box0 \mkern2.3965mu
}
%    \end{macrocode}
% \end{macro}
% The negated arrows are then made using this macro on respectively
% \cs{nrightarrow} and \cs{nleftarrow}
% \begin{macro}{\MH_nuparrow:}
% \begin{macro}{\MH_ndownarrow:}
%    \begin{macrocode}
\def\MH_nuparrow: {%
  \mathrel{\mathpalette\MH_nrotarrow:NN\nrightarrow} }
\def\MH_ndownarrow: {%
  \mathrel{\mathpalette\MH_nrotarrow:NN\nleftarrow} }
%    \end{macrocode}
% \end{macro}
% \end{macro}
% \begin{macro}{\nuparrow}
% \begin{macro}{\ndownarrow}
%   Next we provide \cs{nuparrow} and \cs{ndownarrow} at begin
%   document. Since they depend on \cs{nrightarrow} and
%   \cs{nleftarrow} we test for these and let the macros throw an
%   error if they are missing.
% \changes{v1.08b}{2010/07/21}{Moved graphicx loading down here such
% that we do not get into option clash problems}
%    \begin{macrocode}
\AtBeginDocument{%
  \RequirePackage{graphicx}%
  \@ifundefined{nrightarrow}{%
    \providecommand\nuparrow{%
      \PackageError{mathtools}{\string\nuparrow\space~ is~
        constructed~ from~ \string\nrightarrow,~ which~ is~ not~
        provided.~ Please~ load~ the~ amssymb~ package~ or~ similar}{}
    }}{ \providecommand\nuparrow{\MH_nuparrow:}}
  \@ifundefined{nleftarrow}{%
    \providecommand\ndownarrow{%
      \PackageError{mathtools}{\string\ndownarrow\space~ is~
        constructed~ from~ \string\nleftarrow,~ which~ is~ not~
        provided.~ Please~ load~ the~ amssymb~ package~ or~ similar}{}
    }}{ \providecommand\ndownarrow{\MH_ndownarrow:}} }
%    \end{macrocode}
% \end{macro}
% \end{macro}
% 
%
%  \subsection{Providing bigtimes}
%
%  The idea is to use the original \cs{times} and then scale it
%  accordingly. Again the implementation details have been improved by
%  Enrico Gregorio
%  (\url{http://groups.google.com/group/comp.text.tex/msg/9685c9405df2ff94}). 
%
% \begin{macro}{\MH_bigtimes_scaler:N}
% \begin{macro}{\MH_bigtimes_inner:}
% \begin{macro}{\MH_csym_bigtimes:}
% \changes{v1.07}{2008/08/11}{Added support for \cs{bigtimes}}
%    \begin{macrocode}
\def\MH_bigtimes_scaler:N #1{%
  \vcenter{\hbox{#1$\m@th\mkern-2mu\times\mkern-2mu$}}}
%    \end{macrocode}
%  This is then combined with \cs{mathchoice} to form the inner parts
%  of the macro
%    \begin{macrocode}
\def\MH_bigtimes_inner: {
  \mathchoice{\MH_bigtimes_scaler:N \huge}         % display style
             {\MH_bigtimes_scaler:N \LARGE}        % text style
             {\MH_bigtimes_scaler:N {}}            % script style
             {\MH_bigtimes_scaler:N \footnotesize} % script script style
}
%    \end{macrocode}
%  And thus the internal prepresentaion of the \cs{bigtimes} macro.
%    \begin{macrocode}
\def\MH_csym_bigtimes: {\mathop{\MH_bigtimes_inner:}\displaylimits}
%    \end{macrocode}
% \end{macro}
% \end{macro}
% \end{macro}
% \begin{macro}{\bigtimes}
%   In the end we provide \cs{bigtimes} if otherwise not defined.
%    \begin{macrocode}
\AtBeginDocument{
  \providecommand\bigtimes{\MH_csym_bigtimes:}
}
%    \end{macrocode}
% \end{macro}
%
%  \section{Macros by other people}
%
%  \subsection{Intertext and short intertext}
%
%  It turns out that \cs{intertext} use a bit too much
%  space. Especially noticable if combined with the \env{spreadlines}
%  environment, the extra space is also applied above and below
%  \cs{intertext}, which ends up looking unproffesional.  Chung-chieh
%  Shan
%  (\url{http://conway.rutgers.edu/~ccshan/wiki/blog/posts/Beyond_amsmath/})
%  via Tobias Weh suggested a fix. We apply it here, but also keep the
%  original \cs{intertext} in case a user would rather want it.
% \begin{macro}{\MT_orig_intertext:}
% \begin{macro}{\MT_intertext:}
% \changes{v1.13}{2012/08/19}{\cs{l_MT_X_intertext_dim} renamed to 
% \cs{l_MT_X_intertext_sep}}
% \begin{macro}{\l_MT_above_intertext_sep}
% \begin{macro}{\l_MT_below_intertext_sep}
%  First store the originam Ams version.
%    \begin{macrocode}
\MH_let:NwN \MT_orig_intertext: \intertext@
%    \end{macrocode}
% And then for some reconfiguration. First a few lengths
% \changes{v1.13}{2012/08/19}{Fixed typos and changed the names}
%    \begin{macrocode}
\newdimen\l_MT_above_intertext_sep
\newdimen\l_MT_below_intertext_sep
\define@key{\MT_options_name:}
  {aboveintertextdim}{\setlength\l_MT_above_intertext_sep{#1}}
\define@key{\MT_options_name:}
  {belowintertextdim}{\setlength\l_MT_below_intertext_sep{#1}}
\define@key{\MT_options_name:}
  {above-intertext-dim}{\setlength\l_MT_above_intertext_sep{#1}}
\define@key{\MT_options_name:}
  {below-intertext-dim}{\setlength\l_MT_below_intertext_sep{#1}}
\define@key{\MT_options_name:}
  {above-intertext-sep}{\setlength\l_MT_above_intertext_sep{#1}}
\define@key{\MT_options_name:}
  {below-intertext-sep}{\setlength\l_MT_below_intertext_sep{#1}}
%    \end{macrocode}
% Their default values are zero. Now for our extended version of
% CCShan's solution.  
%    \begin{macrocode}
\def\MT_intertext: {%
  \def\intertext##1{%
    \ifvmode\else\\\@empty\fi
    \noalign{%
      \penalty\postdisplaypenalty\vskip\belowdisplayskip
      \vskip-\lineskiplimit      % CCS
      \vskip\normallineskiplimit % CCS
      \vskip\l_MT_above_intertext_sep
       \vbox{\normalbaselines
        \ifdim\linewidth=\columnwidth
        \else \parshape\@ne \@totalleftmargin \linewidth
        \fi
        \noindent##1\par}%
      \penalty\predisplaypenalty\vskip\abovedisplayskip%
      \vskip-\lineskiplimit      % CCS
      \vskip\normallineskiplimit % CCS
      \vskip\l_MT_above_intertext_sep
   }%
}}
%    \end{macrocode}
% And provide a key to switch
%    \begin{macrocode}
\def\MT_orig_intertext_true:  { \MH_let:NwN \intertext@ \MT_orig_intertext: }
\def\MT_orig_intertext_false: { \MH_let:NwN \intertext@ \MT_intertext: }
\define@key{\MT_options_name:}{original-intertext}[true]{
  \@nameuse{MT_orig_intertext_#1:}
}
%    \end{macrocode}
% And use the new version as default.
%    \begin{macrocode}
\setkeys{\MT_options_name:}{
  original-intertext=false
}
%    \end{macrocode}
% 
% \end{macro}
% \end{macro}
% \end{macro}
% \end{macro}
%
%  Gabriel Zachmann, Donald Arseneau on comp.text.tex 2000/05/12-13
%  \begin{macro}{\shortintertext}
%  \begin{macro}{\MT_orig_shortintertext}
%  \begin{macro}{\MT_shortintertext}
%  \begin{macro}{\l_above_shortintertext_sep}
%  \begin{macro}{\l_below_shortintertext_sep}
%  This is like \cs{intertext} but uses shorter skips between the
%  math. Again this turned out to have the same problem as
%  \cs{intertext}, so we provide two versions.
%    \begin{macrocode}
\def\MT_orig_shortintertext:n #1{%
  \ifvmode\else\\\@empty\fi
  \noalign{%
    \penalty\postdisplaypenalty\vskip\abovedisplayshortskip
    \vbox{\normalbaselines
      \if_dim:w \linewidth=\columnwidth
      \else:
        \parshape\@ne \@totalleftmargin \linewidth
      \fi:
      \noindent#1\par}%
    \penalty\predisplaypenalty\vskip\abovedisplayshortskip%
  }%
}
%    \end{macrocode}
% Lengths like above
% \changes{v1.13}{2012/08/19}{The option was named differently in the
% manual. Also renamed to use the postfix \emph{sep} instead. Though
% the old names remain for compatibility.}
%    \begin{macrocode}
\newdimen\l_MT_above_shortintertext_sep
\newdimen\l_MT_below_shortintertext_sep
\define@key{\MT_options_name:}
  {aboveshortintertextdim}{\setlength \l_MT_above_shortintertext_sep{#1}}
\define@key{\MT_options_name:}
  {belowshortintertextdim}{\setlength \l_MT_below_shortintertext_sep{#1}}
\define@key{\MT_options_name:}
  {above-short-intertext-dim}{\setlength \l_MT_above_shortintertext_sep{#1}}
\define@key{\MT_options_name:}
  {below-short-intertext-dim}{\setlength \l_MT_below_shortintertext_sep{#1}}
\define@key{\MT_options_name:}
  {above-short-intertext-sep}{\setlength \l_MT_above_shortintertext_sep{#1}}
\define@key{\MT_options_name:}
  {below-short-intertext-sep}{\setlength \l_MT_below_shortintertext_sep{#1}}
%    \end{macrocode}
% Looks best with the `old' values of the original \cs{jot}
% setting. So we set them to 3pt each.
%    \begin{macrocode}
\setkeys{\MT_options_name:}{
  aboveshortintertextdim=3pt,
  belowshortintertextdim=3pt
}
%    \end{macrocode}
%  Next, just add the same as we did for \cs{intertext}
%    \begin{macrocode}
\def\MT_shortintertext:n #1{%
  \ifvmode\else\\\@empty\fi
  \noalign{%
    \penalty\postdisplaypenalty\vskip\abovedisplayshortskip
    \vskip-\lineskiplimit      
    \vskip\normallineskiplimit 
    \vskip\l_MT_above_shortintertext_sep
    \vbox{\normalbaselines
      \if_dim:w \linewidth=\columnwidth
      \else:
        \parshape\@ne \@totalleftmargin \linewidth
      \fi:
      \noindent#1\par}%
    \penalty\predisplaypenalty\vskip\abovedisplayshortskip%
    \vskip-\lineskiplimit      
    \vskip\normallineskiplimit 
    \vskip\l_MT_below_shortintertext_sep
  }%
}
%    \end{macrocode}
% Next we need to be able to switch.
%    \begin{macrocode}
\def\MT_orig_shortintertext_true:  { \MH_let:NwN \shortintertext \MT_orig_shortintertext:n }
\def\MT_orig_shortintertext_false: { \MH_let:NwN \shortintertext \MT_shortintertext:n }
\define@key{\MT_options_name:}{original-shortintertext}[true]{
  \@nameuse{MT_orig_shortintertext_#1:}
}
%    \end{macrocode}
% With the updated one as the default.
%    \begin{macrocode}
\setkeys{\MT_options_name:}{
  original-shortintertext=false
}
%    \end{macrocode}
%  \end{macro}
%  \end{macro}
%  \end{macro}
%  \end{macro}
%  \end{macro}
%
%  \subsection{Fine-tuning mathematical layout}
%
%  \subsubsection{A complement to \texttt{\textbackslash smash},
%  \texttt{\textbackslash llap}, and \texttt{\textbackslash rlap}}
%  \begin{macro}{\clap}
%  \begin{macro}{\mathllap}
%  \begin{macro}{\mathrlap}
%  \begin{macro}{\mathclap}
%  \begin{macro}{\MT_mathllap:Nn}
%  \begin{macro}{\MT_mathrlap:Nn}
%  \begin{macro}{\MT_mathclap:Nn}
%  First we'll \cs{provide} those macros (they are so simple that I
%  think other packages might define them as well).
%    \begin{macrocode}
\providecommand*\clap[1]{\hb@xt@\z@{\hss#1\hss}}
\providecommand*\mathllap[1][\@empty]{
  \ifx\@empty#1\@empty
    \expandafter \mathpalette \expandafter \MT_mathllap:Nn
  \else
    \expandafter \MT_mathllap:Nn \expandafter #1
  \fi
}
\providecommand*\mathrlap[1][\@empty]{
  \ifx\@empty#1\@empty
    \expandafter \mathpalette \expandafter \MT_mathrlap:Nn
  \else
    \expandafter \MT_mathrlap:Nn \expandafter #1
  \fi
}
\providecommand*\mathclap[1][\@empty]{
  \ifx\@empty#1\@empty
    \expandafter \mathpalette \expandafter \MT_mathclap:Nn
  \else
    \expandafter \MT_mathclap:Nn \expandafter #1
  \fi
}
%    \end{macrocode}
%  We have to insert |{}| because we otherwise risk triggering a
%  ``feature'' in \TeX.
%    \begin{macrocode}
\def\MT_mathllap:Nn #1#2{{}\llap{$\m@th#1{#2}$}}
\def\MT_mathrlap:Nn #1#2{{}\rlap{$\m@th#1{#2}$}}
\def\MT_mathclap:Nn #1#2{{}\clap{$\m@th#1{#2}$}}
%    \end{macrocode}
%  \end{macro}
%  \end{macro}
%  \end{macro}
%  \end{macro}
%  \end{macro}
%  \end{macro}
%  \end{macro}
%  \begin{macro}{\mathmbox}
%  \begin{macro}{\MT_mathmbox:nn}
%  \begin{macro}{\mathmakebox}
%  \begin{macro}{\MT_mathmakebox_I:w}
%  \begin{macro}{\MT_mathmakebox_II:w}
%  \begin{macro}{\MT_mathmakebox_III:w}
%  Then the \cs{mathmbox}\marg{arg} and
%  \cs{mathmakebox}\oarg{width}\oarg{pos}\marg{arg} macros which are
%  very similar to \cs{mbox} and \cs{makebox}. The differences are:
%  \begin{itemize}
%    \item \meta{arg} is set in math mode of course.
%    \item No need for \cs{leavevmode} as we're in math mode.
%    \item No need to make them \cs{long} (we're still in math mode).
%    \item No need to support a picture version.
%  \end{itemize}
%  The first is easy.
%    \begin{macrocode}
\providecommand*\mathmbox{\mathpalette\MT_mathmbox:nn}
\def\MT_mathmbox:nn #1#2{\mbox{$\m@th#1#2$}}
%    \end{macrocode}
%  We scan for the optional arguments first.
%    \begin{macrocode}
\providecommand*\mathmakebox{
  \@ifnextchar[  \MT_mathmakebox_I:w
                 \mathmbox}
\def\MT_mathmakebox_I:w[#1]{%
  \@ifnextchar[  {\MT_mathmakebox_II:w[#1]}
                 {\MT_mathmakebox_II:w[#1][c]}}
%    \end{macrocode}
%  We had to get the optional arguments out of the way before calling
%  upon the powers of \cs{mathpalette}.
%    \begin{macrocode}
\def\MT_mathmakebox_II:w[#1][#2]{
  \mathpalette{\MT_mathmakebox_III:w[#1][#2]}}
\def\MT_mathmakebox_III:w[#1][#2]#3#4{%
  \@begin@tempboxa\hbox{$\m@th#3#4$}%
    \setlength\@tempdima{#1}%
    \hbox{\hb@xt@\@tempdima{\csname bm@#2\endcsname}}%
  \@end@tempboxa}
%    \end{macrocode}
%  \end{macro}
%  \end{macro}
%  \end{macro}
%  \end{macro}
%  \end{macro}
%  \end{macro}
%  \begin{macro}{\mathsm@sh}
%  Fix \cs{smash}.
%    \begin{macrocode}
\def\mathsm@sh#1#2{%
  \setbox\z@\hbox{$\m@th#1{#2}$}{}\finsm@sh}
%    \end{macrocode}
%  \end{macro}
%
%
%  \subsubsection{A cramped style}
%
%  comp.text.tex on 1992/07/21 by Michael Herschorn.
%  With speed-ups by the Grand Wizard himself as shown on
%  \begin{quote}\rightskip-\leftmargini
%  \url{http://www.tug.org/tex-archive/digests/tex-implementors/042}
%  \end{quote}
%  The (better) user interface by the author.
%
%  \begin{macro}{\cramped}
%  Make sure the expansion is timed correctly.
%    \begin{macrocode}
\providecommand*\cramped[1][\@empty]{
  \ifx\@empty#1\@empty
    \expandafter \mathpalette \expandafter \MT_cramped_internal:Nn
  \else
    \expandafter \MT_cramped_internal:Nn \expandafter #1
  \fi
}
%    \end{macrocode}
%  \end{macro}
%  \begin{macro}{\MT_cramped_internal:Nn}
%  The internal command.
%    \begin{macrocode}
\def\MT_cramped_internal:Nn #1#2{
%    \end{macrocode}
%  Create a box containing the math and force a cramped style by
%  issuing a non-existing radical.
%    \begin{macrocode}
  \sbox\z@{$\m@th#1\nulldelimiterspace=\z@\radical\z@{#2}$}
%    \end{macrocode}
%  Then make sure the height is correct.
%    \begin{macrocode}
    \ifx#1\displaystyle
      \dimen@=\fontdimen8\textfont3
      \advance\dimen@ .25\fontdimen5\textfont2
    \else
      \dimen@=1.25\fontdimen8
      \ifx#1\textstyle\textfont
      \else
        \ifx#1\scriptstyle
          \scriptfont
        \else
          \scriptscriptfont
        \fi
      \fi
      3
    \fi
    \advance\dimen@-\ht\z@ \ht\z@=-\dimen@
    \box\z@
}
%    \end{macrocode}
%  \end{macro}
%
%  \subsubsection{Cramped versions of \texttt{\textbackslash
%  mathllap}, \texttt{\textbackslash mathclap}, and
%  \texttt{\textbackslash mathrlap}}
%  \begin{macro}{\crampedllap}
%  \begin{macro}{\MT_cramped_llap_internal:Nn}
%  \begin{macro}{\crampedclap}
%  \begin{macro}{\MT_cramped_clap_internal:Nn}
%  \begin{macro}{\crampedrlap}
%  \begin{macro}{\MT_cramped_rlap_internal:Nn}
%  Cramped versions of \cs{mathXlap} (for speed). Made by the author.
%    \begin{macrocode}
\providecommand*\crampedllap[1][\@empty]{
  \ifx\@empty#1\@empty
    \expandafter \mathpalette \expandafter \MT_cramped_llap_internal:Nn
  \else
    \expandafter \MT_cramped_llap_internal:Nn \expandafter #1
  \fi
}
\def\MT_cramped_llap_internal:Nn #1#2{
  {}\llap{\MT_cramped_internal:Nn #1{#2}}
}
\providecommand*\crampedclap[1][\@empty]{
  \ifx\@empty#1\@empty
    \expandafter \mathpalette \expandafter \MT_cramped_clap_internal:Nn
  \else
    \expandafter \MT_cramped_clap_internal:Nn \expandafter #1
  \fi
}
\def\MT_cramped_clap_internal:Nn #1#2{
  {}\clap{\MT_cramped_internal:Nn #1{#2}}
}
\providecommand*\crampedrlap[1][\@empty]{
  \ifx\@empty#1\@empty
    \expandafter \mathpalette \expandafter \MT_cramped_rlap_internal:Nn
  \else
    \expandafter \MT_cramped_rlap_internal:Nn \expandafter #1
  \fi
}
\def\MT_cramped_rlap_internal:Nn #1#2{
  {}\rlap{\MT_cramped_internal:Nn #1{#2}}
}
%    \end{macrocode}
%  \end{macro}
%  \end{macro}
%  \end{macro}
%  \end{macro}
%  \end{macro}
%  \end{macro}
%
%
%  \section{Macros by Michael J.~Downes}
%
%  The macros in this section are all by Michael J.~Downes. Either
%  they are straight copies of his original macros or inspired and
%  extended here.
%
%
%  \subsection{Prescript}
%  \begin{macro}{\prescript}
%  This command is taken from a posting to comp.text.tex on
%  December~20th 2000 by Michael J.~Downes. The comments are his. I
%  have added some formatting options to the arguments so that a user
%  can emulate the \pkg{isotope} package.
%
% \changes{v.1.12}{2012/04/19}{Extended \cs{prescript} to change style
% if used in say S context. Requestd by Oliver Buerschaper.}
%  Update 2012: One drawback from MJD's original implementation, is
%  that the math style is hardwired, such that if used in say
%  \cs{scriptstyle} context, then the style/size of the prescript
%  remain the same size. A slightly expensive fix, is to use the
%  \cs{mathchoice} construction. First exdent MJD's code a little
%  (keeping his comments)
% \begin{macro}{\MT_prescript_inner:}
% We make the style an extra forth argument
%    \begin{macrocode}
\newcommand{\MT_prescript_inner:}[4]{
%    \end{macrocode}
%  Put the sup in box 0 and the sub in box 2.
%    \begin{macrocode}
  \@mathmeasure\z@#4{\MT_prescript_sup:{#1}}
  \@mathmeasure\tw@#4{\MT_prescript_sub:{#2}}
  \if_dim:w \wd\tw@>\wd\z@
    \setbox\z@\hbox to\wd\tw@{\hfil\unhbox\z@}
  \else:
    \setbox\tw@\hbox to\wd\z@{\hfil\unhbox\tw@}
  \fi:
%    \end{macrocode}
%  Do not let a preceding mathord symbol approach without any
%  intervening space.
%    \begin{macrocode}
  \mathop{}
%    \end{macrocode}
%  Use \cs{mathopen} to suppress space between the prescripts and the
%  base object even when the latter is not of type ord.
%    \begin{macrocode}
  \mathopen{\vphantom{\MT_prescript_arg:{#3}}}^{\box\z@}\sb{\box\tw@}
  \MT_prescript_arg:{#3}
}
%    \end{macrocode}
% \end{macro}
% Next create \cs{prescript} using \cs{mathchoice} 
%    \begin{macrocode}
\DeclareRobustCommand{\prescript}[3]{
  \mathchoice
%    \end{macrocode}
%  In D and T style, we use MJD's default:
%    \begin{macrocode}
    {\MT_prescript_inner:{#1}{#2}{#3}{\scriptstyle}}
    {\MT_prescript_inner:{#1}{#2}{#3}{\scriptstyle}}
%    \end{macrocode}
%  In the others we step one style down. Of couse in SS style, using
%  \cs{scriptscript} may seem wrong, but there is no lower style.
%    \begin{macrocode}
    {\MT_prescript_inner:{#1}{#2}{#3}{\scriptscriptstyle}}
    {\MT_prescript_inner:{#1}{#2}{#3}{\scriptscriptstyle}}
}
%    \end{macrocode}
%  \end{macro}
%  Then the named arguments. Can you see I'm preparing for templates?
%    \begin{macrocode}
\define@key{\MT_options_name:}
  {prescript-sup-format}{\def\MT_prescript_sup:{#1}}
\define@key{\MT_options_name:}
  {prescript-sub-format}{\def\MT_prescript_sub:{#1}}
\define@key{\MT_options_name:}
  {prescript-arg-format}{\def\MT_prescript_arg:{#1}}
\setkeys{\MT_options_name:}{
  prescript-sup-format={},
  prescript-sub-format={},
  prescript-arg-format={},
}
%    \end{macrocode}
%
%  \subsection{Math sizes}
%  \begin{macro}{\@DeclareMathSizes}
%  This command is taken from a posting to comp.text.tex on
%  October~17th 2002 by Michael J.~Downes. The purpose is to be able
%  to put dimensions on the last three arguments of
%  \cs{DeclareMathSizes}.
%    \begin{macrocode}
\def\@DeclareMathSizes #1#2#3#4#5{%
  \@defaultunits\dimen@ #2pt\relax\@nnil
  \if:w $#3$%
    \MH_let:cN {S@\strip@pt\dimen@}\math@fontsfalse
  \else:
    \@defaultunits\dimen@ii #3pt\relax\@nnil
    \@defaultunits\@tempdima #4pt\relax\@nnil
    \@defaultunits\@tempdimb #5pt\relax\@nnil
    \toks@{#1}%
    \expandafter\xdef\csname S@\strip@pt\dimen@\endcsname{%
      \gdef\noexpand\tf@size{\strip@pt\dimen@ii}%
      \gdef\noexpand\sf@size{\strip@pt\@tempdima}%
      \gdef\noexpand\ssf@size{\strip@pt\@tempdimb}%
      \the\toks@
    }%
  \fi:
}
%    \end{macrocode}
%  \end{macro}
%
%  \subsection{Mathematics within italic text}
%  mathic: Michael J.~Downes on comp.text.tex, 1998/05/14.
%  \begin{macro}{\MT_mathic_true:}
%  \begin{macro}{\MT_mathic_false:}
%  Renew \cs{(} so that it detects the slant of the font and inserts
%  an italic correction.
%    \begin{macrocode}
\def\MT_mathic_true: {
  \MH_if_boolean:nF {math_italic_corr}{
    \MH_set_boolean_T:n {math_italic_corr}
%    \end{macrocode}
%  Save the original meaning if you need to go back.
%    \begin{macrocode}
    \MH_let:NwN \MT_begin_inlinemath: \(
    \renewcommand*\({\relax\ifmmode\@badmath\else
      \ifhmode
        \if_dim:w \fontdimen\@ne\font>\z@
          \if_dim:w \lastskip>\z@
            \skip@\lastskip\unskip
            \@@italiccorr
            \hskip\skip@
          \else:
            \@@italiccorr
          \fi:
        \fi:
      \fi:
      $\fi:
    }
  }
}
%$ for emacs coloring ;-)
%    \end{macrocode}
%  Just for restoring the old behavior.
%    \begin{macrocode}
\def\MT_mathic_false: {
  \MH_if_boolean:nT {math_italic_corr}{
    \MH_set_boolean_F:n {math_italic_corr}
    \MH_let:NwN \( \MT_begin_inlinemath:
  }
}
\MH_new_boolean:n {math_italic_corr}
\define@key{\MT_options_name:}{mathic}[true]{
  \@ifundefined{MT_mathic_#1:}
    { \MT_true_false_error:
      \@nameuse{MT_mathic_false:}
    }
    { \@nameuse{MT_mathic_#1:} }
}
%    \end{macrocode}
%  \end{macro}
%  \end{macro}
%
%  \subsection{Spreading equations}
%
%  Michael J.~Downes on comp.text.tex 1999/08/25
%  \begin{environment}{spreadlines}
%  This is meant to be used outside math, just like
%  \env{subequations}.
%    \begin{macrocode}
\newenvironment{spreadlines}[1]{
  \setlength{\jot}{#1}
  \ignorespaces
}{ \ignorespacesafterend }
%    \end{macrocode}
%  \end{environment}
%
%  \subsection{Gathered}
%
%  Inspired by Michael J.~Downes on comp.text.tex 2002/01/17.
%  \begin{environment}{MT_gathered_env}
%  Just like the normal \env{gathered}, only here we're allowed to
%  specify actions before and after each line.
%    \begin{macrocode}
\MaybeMHPrecedingSpacesOff
\newenvironment{MT_gathered_env}[1][c]{%
    \RIfM@\else
        \nonmatherr@{\begin{\@currenvir}}%
    \fi
    \null\,%
    \if #1t\vtop \else \if#1b\vbox \else \vcenter \fi\fi \bgroup
        \Let@ \chardef\dspbrk@context\@ne \restore@math@cr
        \spread@equation
        \ialign\bgroup
            \MT_gathered_pre:
            \strut@$\m@th\displaystyle##$
            \MT_gathered_post:
            \crcr
}{%
  \endaligned
  \MT_gathered_env_end:
}
\MHPrecedingSpacesOn
%    \end{macrocode}
%  \end{environment}
%  \begin{macro}{\newgathered}
%  \begin{macro}{\renewgathered}
%  \begin{environment}{lgathered}
%  \begin{environment}{rgathered}
%  \begin{environment}{gathered}
%  An easier interface.
%    \begin{macrocode}
\newcommand*\newgathered[4]{
  \newenvironment{#1}
    { \def\MT_gathered_pre:{#2}
      \def\MT_gathered_post:{#3}
      \def\MT_gathered_env_end:{#4}
      \MT_gathered_env
    }{\endMT_gathered_env}
}
\newcommand*\renewgathered[4]{
  \renewenvironment{#1}
    { \def\MT_gathered_pre:{#2}
      \def\MT_gathered_post:{#3}
      \def\MT_gathered_env_end:{#4}
      \MT_gathered_env
    }{\endMT_gathered_env}
}
\newgathered{lgathered}{}{\hfil}{}
\newgathered{rgathered}{\hfil}{}{}
\renewgathered{gathered}{\hfil}{\hfil}{}
%    \end{macrocode}
%  \end{environment}
%  \end{environment}
%  \end{environment}
%  \end{macro}
%  \end{macro}
%
%  \subsection{Split fractions}
%
%  Michael J.~Downes on comp.text.tex 2001/12/06.
%  \begin{macro}{\splitfrac}
%  \begin{macro}{\splitdfrac}
%  These commands use \cs{genfrac} to typeset a split fraction. The
%  thickness of the fraction rule is simply set to zero.
%    \begin{macrocode}
\newcommand*\splitfrac[2]{%
  \genfrac{}{}{0pt}{1}%
    {\textstyle#1\quad\hfill}%
    {\textstyle\hfill\quad\mathstrut#2}%
}
\newcommand*\splitdfrac[2]{%
  \genfrac{}{}{0pt}{0}{#1\quad\hfill}{\hfill\quad\mathstrut #2}%
}
%    \end{macrocode}
%  \end{macro}
%  \end{macro}
%
%
%  \section{Bug fixes for \pkg{amsmath}}
%  The following fixes some bugs in \pkg{amsmath}, but only if the
%  switch is true.
%    \begin{macrocode}
\MH_if_boolean:nT {fixamsmath}{
%    \end{macrocode}
%  \begin{macro}{\place@tag}
%  This corrects a bug in \pkg{amsmath} affecting tag placement in
%  \env{flalign}.\footnote{See
%  \url{http://www.latex-project.org/cgi-bin/ltxbugs2html?pr=amslatex/3591}}
%    \begin{macrocode}
\def\place@tag{%
  \iftagsleft@
    \kern-\tagshift@
%    \end{macrocode}
%  The addition. If we're in \env{flalign} (meaning
%  $\cs{xatlevel@}=\cs{tw@}$) we skip back by an amount of
%  \cs{@mathmargin}. This test is also true for the \env{xxalignat}
%  environment, but it doesn't matter because a)~it's not
%  supported/described in the documentation anymore so new users
%  won't know about it and b)~it forbids the use of \cs{tag}
%  anyway.
%    \begin{macrocode}
    \if@fleqn
      \if_num:w \xatlevel@=\tw@
        \kern-\@mathmargin
      \fi:
    \fi:
%    \end{macrocode}
%  End of additions.
%    \begin{macrocode}
    \if:w 1\shift@tag\row@\relax
      \rlap{\vbox{%
        \normalbaselines
        \boxz@
        \vbox to\lineht@{}%
        \raise@tag
      }}%
    \else:
      \rlap{\boxz@}%
    \fi:
    \kern\displaywidth@
  \else:
    \kern-\tagshift@
    \if:w 1\shift@tag\row@\relax
      \llap{\vtop{%
        \raise@tag
        \normalbaselines
        \setbox\@ne\null
        \dp\@ne\lineht@
        \box\@ne
        \boxz@
      }}%
    \else:
      \llap{\boxz@}%
    \fi:
  \fi:
}
%    \end{macrocode}
%  \end{macro}
%
%  \begin{macro}{\x@calc@shift@lf}
%  This corrects a bug\footnote{See
%  \url{http://www.latex-project.org/cgi-bin/ltxbugs2html?pr=amslatex/3614}}
%  in \pkg{amsmath} that could cause a non-positive value of the dimension
%  \cs{@mathmargin} to cause an
%  \begin{verbatim}
%  ! Arithmetic overflow.
%  <recently read> \@tempcntb
%  \end{verbatim}
%  when in \mode{fleqn,leqno} mode. Not very comprehensible for the user.
%    \begin{macrocode}
\def\x@calc@shift@lf{%
  \if_dim:w \eqnshift@=\z@
    \global\eqnshift@\@mathmargin\relax
      \alignsep@\displaywidth
      \advance\alignsep@-\totwidth@
%    \end{macrocode}
%  The addition: If \cs{@tempcntb} is zero we avoid division.
%    \begin{macrocode}
      \if_num:w \@tempcntb=0
      \else:
        \global\divide\alignsep@\@tempcntb % original line
      \fi:
%    \end{macrocode}
%  Addition end.
%    \begin{macrocode}
      \if_dim:w \alignsep@<\minalignsep\relax
        \global\alignsep@\minalignsep\relax
      \fi:
  \fi:
  \if_dim:w \tag@width\row@>\@tempdima
    \saveshift@1%
  \else:
    \saveshift@0%
  \fi:}%
%    \end{macrocode}
%  \end{macro}
%    \begin{macrocode}
}
%    \end{macrocode}
%  End of bug fixing.
%
%  \subsection{Making environments safer}
%
%  \begin{macro}{\aligned@a}
%  Here we make the \pkg{amsmath} inner environments disallow spaces
%  before their optional positioning specifier.
%    \begin{macrocode}
\MaybeMHPrecedingSpacesOff
\renewcommand\aligned@a[1][c]{\start@aligned{#1}\m@ne}
\MHPrecedingSpacesOn
%    \end{macrocode}
%  \end{macro}
%
%  This is the end of the \pkg{mathtools} package.
%    \begin{macrocode}
%</package>
%    \end{macrocode}
%
%  \Finale
\endinput

%        (quote the arguments according to the demands of your shell)
%
% Documentation:
%    (a) If mathtools.drv is present:
%           latex mathtools.drv
%    (b) Without mathtools.drv:
%           latex mathtools.dtx; ...
%    The class ltxdoc loads the configuration file ltxdoc.cfg
%    if available. Here you can specify further options, e.g.
%    use A4 as paper format:
%       \PassOptionsToClass{a4paper}{article}
%
%    Programm calls to get the documentation (example):
%       pdflatex mathtools.dtx
%       makeindex -s gind.ist mathtools.idx
%       pdflatex mathtools.dtx
%       makeindex -s gind.ist mathtools.idx
%       pdflatex mathtools.dtx
%
% Installation:
%    TDS:tex/latex/mh/mathtools.sty
%    TDS:doc/latex/mh/mathtools.pdf
%    TDS:source/latex/mh/mathtools.dtx
%
%<*ignore>
\begingroup
  \def\x{LaTeX2e}
\expandafter\endgroup
\ifcase 0\ifx\install y1\fi\expandafter
         \ifx\csname processbatchFile\endcsname\relax\else1\fi
         \ifx\fmtname\x\else 1\fi\relax
\else\csname fi\endcsname
%</ignore>
%<*install>
\input docstrip.tex
\Msg{************************************************************************}
\Msg{* Installation}
\Msg{* Package: mathtools 2012/05/10 v1.12}
\Msg{************************************************************************}

\keepsilent
\askforoverwritefalse

\preamble

This is a generated file.

Copyright (C) 2002-2011 by Morten Hoegholm

This work may be distributed and/or modified under the
conditions of the LaTeX Project Public License, either
version 1.3 of this license or (at your option) any later
version. The latest version of this license is in
   http://www.latex-project.org/lppl.txt
and version 1.3 or later is part of all distributions of
LaTeX version 2005/12/01 or later.

This work has the LPPL maintenance status "maintained".

This Current Maintainer of this work is  
Lars Madsen, Will Robertson and Joseph Wright.

This work consists of the main source file mathtools.dtx
and the derived files
   mathtools.sty, mathtools.pdf, mathtools.ins, mathtools.drv.

\endpreamble

\generate{%
  \file{mathtools.ins}{\from{mathtools.dtx}{install}}%
  \file{mathtools.drv}{\from{mathtools.dtx}{driver}}%
  \usedir{tex/latex/mh}%
  \file{mathtools.sty}{\from{mathtools.dtx}{package}}%
}

\obeyspaces
\Msg{************************************************************************}
\Msg{*}
\Msg{* To finish the installation you have to move the following}
\Msg{* file into a directory searched by TeX:}
\Msg{*}
\Msg{*     mathtools.sty}
\Msg{*}
\Msg{* To produce the documentation run the file `mathtools.drv'}
\Msg{* through LaTeX.}
\Msg{*}
\Msg{* Happy TeXing!}
\Msg{*}
\Msg{************************************************************************}

\endbatchfile
%</install>
%<*ignore>
\fi
%</ignore>
%<*driver>
\NeedsTeXFormat{LaTeX2e}
\ProvidesFile{mathtools.drv}%
  [2012/05/10 v1.12 mathematical typesetting tools]
\documentclass{ltxdoc}
\IfFileExists{fourier.sty}{\usepackage{fourier}}{}
\addtolength\marginparwidth{-25pt}
\usepackage{mathtools}

\setcounter{IndexColumns}{2}

\providecommand*\pkg[1]{\textsf{#1}}
\providecommand*\env[1]{\texttt{#1}}
\providecommand*\email[1]{\href{mailto:#1}{\texttt{#1}}}
\providecommand*\mode[1]{\texttt{[#1]}}
\providecommand*\file[1]{\texttt{#1}}
\usepackage{xcolor,varioref,amssymb}
\makeatletter
\newcommand*\thinfbox[2][black]{\fboxsep0pt\textcolor{#1}{\rulebox{{\normalcolor#2}}}}
\newcommand*\thinboxed[2][black]{\thinfbox[#1]{\ensuremath{\displaystyle#2}}}
\newcommand*\rulebox[1]{%
  \sbox\z@{\ensuremath{\displaystyle#1}}%
  \@tempdima\dp\z@
  \hbox{%
    \lower\@tempdima\hbox{%
      \vbox{\hrule height\fboxrule\box\z@\hrule height\fboxrule}%
    }%
  }%
}

\newenvironment{codesyntax}
    {\par\small\addvspace{4.5ex plus 1ex}%
     \vskip -\parskip
     \noindent
     \begin{tabular}{|l|}\hline\ignorespaces}%
    {\\\hline\end{tabular}\nobreak\par\nobreak
     \vspace{2.3ex}\vskip -\parskip\noindent\ignorespacesafterend}
\makeatletter

\newcommand*\FeatureRequest[2]{%
  \hskip1sp
  \marginpar{%
    \parbox[b]{\marginparwidth}{\small\sffamily\raggedright
      \strut Feature request by\\#1\\#2%
    }
  }%
}


\newcommand*\ProvidedBy[2]{%
  \hskip1sp
  \marginpar{%
    \parbox[b]{\marginparwidth}{\small\sffamily\raggedright
      \strut Feature provided by\\#1\\#2%
    }
  }%
}

\newcommand*\cttPosting[2]{%
  \hskip1sp
  \marginpar{%
    \parbox[b]{\marginparwidth}{\small\sffamily\raggedright
     \strut Posted on \texttt{comp.text.tex} \\#1\\#2%
    }%
  }%
}

\newcommand*\tsxPosting[2]{%
  \hskip1sp
  \marginpar{%
    \parbox[b]{\marginparwidth}{\small\sffamily\raggedright
     \strut Posted on \texttt{\small tex.stackexchange.com} \\#1\\#2%
    }%
  }%
}


\expandafter\def\expandafter\MakePrivateLetters\expandafter{%
  \MakePrivateLetters  \catcode`\_=11\relax
}

\providecommand*\SpecialOptIndex[1]{%
  \@bsphack
  \index{#1\actualchar{\protect\ttfamily #1}
          (option)\encapchar usage}%
      \index{options:\levelchar#1\actualchar{\protect\ttfamily #1}\encapchar
            usage}\@esphack}
\providecommand*\opt[1]{\texttt{#1}}

\providecommand*\SpecialKeyIndex[1]{%
  \@bsphack
  \index{#1\actualchar{\protect\ttfamily #1}
          (key)\encapchar usage}%
      \index{keys:\levelchar#1\actualchar{\protect\ttfamily #1}\encapchar
            usage}\@esphack}
\providecommand*\key[1]{\textsf{#1}}

\providecommand*\eTeX{$\m@th\varepsilon$-\TeX}

\def\MTmeta#1{%
     \ensuremath\langle
     \ifmmode \expandafter \nfss@text \fi
     {%
      \meta@font@select
      \edef\meta@hyphen@restore
        {\hyphenchar\the\font\the\hyphenchar\font}%
      \hyphenchar\font\m@ne
      \language\l@nohyphenation
      #1\/%
      \meta@hyphen@restore
     }\ensuremath\rangle
     \endgroup
}
\makeatother
\DeclareRobustCommand\meta{\begingroup\MakePrivateLetters\MTmeta}%
\def\MToarg#1{{\ttfamily[}\meta{#1}{\ttfamily]}\endgroup}
\DeclareRobustCommand\oarg{\begingroup\MakePrivateLetters\MToarg}%
\def\MHmarg#1{{\ttfamily\char`\{}\meta{#1}{\ttfamily\char`\}}\endgroup}
\DeclareRobustCommand\marg{\begingroup\MakePrivateLetters\MHmarg}%
\def\MHarg#1{{\ttfamily\char`\{#1\ttfamily\char`\}}\endgroup}
\DeclareRobustCommand\arg{\begingroup\MakePrivateLetters\MHarg}%
\def\MHcs#1{\texttt{\char`\\#1}\endgroup}
\DeclareRobustCommand\cs{\begingroup\MakePrivateLetters\MHcs}

\def\endverbatim{\if@newlist
\leavevmode\fi\endtrivlist\vspace{-\baselineskip}}
\expandafter\let\csname endverbatim*\endcsname =\endverbatim

\let\MTtheindex\theindex
\def\theindex{\MTtheindex\MakePrivateLetters}

\usepackage[final,hyperindex=false]{hyperref}
\renewcommand*\usage[1]{\textit{\hyperpage{#1}}}

\OnlyDescription
\begin{document}
  \DocInput{mathtools.dtx}
\end{document}
%</driver>
%  \fi
%
%  \changes{v1.0}{2004/07/26}{Initial release}
%
%  \GetFileInfo{mathtools.drv}
%
%  \CheckSum{2836}
%
%  \title{The \pkg{mathtools} package\thanks{This file has version number
%  \fileversion, last revised \filedate.}}
%
%  \author{Lars Madsen, Will Robertson and Joseph
%  Wright\\ (maintainers)\thanks{The maintainers would like to thank
%  Morten H\o gholm for his contributions to this package}}
%  \date{\filedate}
%
%  \maketitle
%
%  \begin{abstract}
%    The \pkg{mathtools} package is an extension package to
%    \pkg{amsmath}. There are two things on \pkg{mathtools}' agenda:
%    1)~correct various bugs/defeciencies in \pkg{amsmath} until
%    these are fixed by the \AmS{} and 2)~provide useful tools
%    for mathematical typesetting, be it a small macro for
%    typesetting a prescript or an underbracket, or entirely new
%    display math constructs such as a \env{multlined} environment.
%  \end{abstract}
%
%  \tableofcontents
%
%  \section{Introduction}
%
%  Although \pkg{amsmath} provides many handy tools for mathematical
%  typesetting, it is nonetheless a static package. This is not a bad
%  thing, because what it does, it mostly does quite well and having
%  a stable math typesetting package is ``a good thing.'' However,
%  \pkg{amsmath} does not fulfill all the needs of the mathematical
%  part of the \LaTeX{} community, resulting in many authors writing
%  small snippets of code for tweaking the mathematical layout. Some
%  of these snippets has also been posted to newsgroups and mailing
%  lists over the years, although more often than not without being
%  released as stand-alone packages.
%
%
%  The \pkg{mathtools} package is exactly what its name implies: tools
%  for mathematical typesetting. It is a collection of many of these
%  often needed small tweaks---with some big tweaks added as well. It
%  can only do so by having harvesting newsgroups for code and/or you
%  writing the maintainers with wishes for code to be included, so if
%  you have any good macros or just macros that help you when writing
%  mathematics, then don't hesitate to report them to us. We can be
%  reached at
%  \begin{quote}\email{mh.ctan@gmail.com}\end{quote}
%  This is of course also the address to use in case of bug reports.
%
%  \section{Package loading}
%
%
%  The \pkg{mathtools} package requires \pkg{amsmath} but is able to
%  pass options to it as well. Thus a line like
%  \begin{verbatim}
%    \usepackage[fleqn,tbtags]{mathtools}
%  \end{verbatim}
%  is equivalent to
%  \begin{verbatim}
%    \usepackage[fleqn,tbtags]{amsmath}
%    \usepackage{mathtools}
%  \end{verbatim}
%
%
%  \subsection{Special \pkg{mathtools} options}
%
%  \begin{codesyntax}
%  \SpecialOptIndex{fixamsmath}\opt{fixamsmath}\texttt{~~~~}
%  \SpecialOptIndex{donotfixamsmathbugs}\opt{donotfixamsmathbugs}
%  \end{codesyntax}
%  The option \opt{fixamsmath} (default) fixes two bugs in
%  \pkg{amsmath}.\footnote{See the online \LaTeX{} bugs database
%  \url{http://www.latex-project.org/cgi-bin/ltxbugs2html} under
%  \AmS\LaTeX{} problem reports 3591 and 3614.} Should you for some
%  reason not want to fix these bugs then just add the option
%  \opt{donotfixamsmathbugs} (if you can do it without typos). The
%  reason for this extremely long name is that I really don't see why
%  you wouldn't want these bugs to be fixed, so I've made it slightly
%  difficult not to fix them.
%
%  \begin{codesyntax}
%  \SpecialOptIndex{allowspaces}\opt{allowspaces}\texttt{~~~~}
%  \SpecialOptIndex{disallowspaces}\opt{disallowspaces}
%  \end{codesyntax}
%  Sometimes \pkg{amsmath} gives you nasty surprises, as here where
%  things look seemingly innocent:
%  \begin{verbatim}
%  \[
%      \begin{gathered}
%        [p] = 100 \\
%        [v] = 200
%      \end{gathered}
%  \]
%  \end{verbatim}
%  Without \pkg{mathtools} this will result in this output:
%  \[
%      \begin{gathered}[c]
%        = 100 \\
%        [v] = 200
%      \end{gathered}
%  \]
%  Yes, the \texttt{[p]} has been gobbled without any warning
%  whatsoever.\footnote{\pkg{amsmath} thought the \texttt[p] was an
%  optional argument, checked if it was \texttt{t} or \texttt{b} and
%  when both tests failed, assumed it was a \texttt{c}.} This is
%  hardly what you'd expect as an end user, as the desired output was
%  probably something like this instead:
%  \[
%      \begin{gathered}[c]
%        [p] = 100 \\
%        [v] = 200
%      \end{gathered}
%  \]
%  With the option \opt{disallowspaces} (default) \pkg{mathtools}
%  disallows spaces in front of optional arguments where it could
%  possibly cause problems just as \pkg{amsmath} does with |\\|
%  inside the display environments. This includes the environments
%  \env{gathered} (and also those shown in \S
%  \vref{subsec:gathered}), \env{aligned}, \env{multlined}, and the
%  extended \env{matrix}-environments (\S \vref{subsubsec:matrices}).
%  If you however want to preserve the more dangerous standard
%  optional spaces, simply choose the option \opt{allowspaces}.
%
%
%  \section{Tools for mathematical typesetting}
%
%  \begin{codesyntax}
%    \SpecialUsageIndex{\mathtoolsset}\cs{mathtoolsset}\marg{key val list}
%  \end{codesyntax}
%  Many of the tools shown in this manual can be turned on and off by
%  setting a switch to either true or false. In all cases it is done
%  with the command \cs{mathtoolsset}. A typical use could be something like
%  \begin{verbatim}
%    \mathtoolsset{
%      showonlyrefs,
%      mathic % or mathic = true
%    }
%  \end{verbatim}
%  More information on the keys later on.
%
%  \subsection{Fine-tuning mathematical layout}
%
%  Sometimes you need to tweak the layout of formulas a little to get
%  the best result and this part of the manual describes the various
%  macros \pkg{mathtools} provides for this.
%
%  \subsubsection{A complement to \texttt{\textbackslash smash},
%  \texttt{\textbackslash llap}, and \texttt{\textbackslash rlap}}
%
%  \begin{codesyntax}
%    \SpecialUsageIndex{\mathllap}
%    \cs{mathllap}\oarg{mathstyle}\marg{math}\texttt{~~}
%    \SpecialUsageIndex{\mathclap}
%    \cs{mathclap}\oarg{mathstyle}\marg{math}\\
%    \SpecialUsageIndex{\mathrlap}
%    \cs{mathrlap}\oarg{mathstyle}\marg{math}\texttt{~~}
%    \SpecialUsageIndex{\clap}
%    \cs{clap}\marg{text}\\
%    \SpecialUsageIndex{\mathmbox}
%    \cs{mathmbox}\marg{math}\phantom{\meta{mathstyle}}\texttt{~~~~}
%    \SpecialUsageIndex{\mathmakebox}
%    \cs{mathmakebox}\oarg{width}\oarg{pos}\marg{math}
%  \end{codesyntax}
%  In \cite{Perlis01}, Alexander R.~Perlis describes some simple yet
%  useful macros for use in math displays. For example the display
%  \begin{verbatim}
%    \[
%      X = \sum_{1\le i\le j\le n} X_{ij}
%    \]
%  \end{verbatim}
%  \[
%    X = \sum_{1\le i\le j\le n} X_{ij}
%  \]
%  contains a lot of excessive white space.  The idea that comes to
%  mind is to fake the width of the subscript. The command
%  \cs{mathclap} puts its argument in a zero width box and centers
%  it, so it could possibly be of use here.
%  \begin{verbatim}
%    \[
%      X = \sum_{\mathclap{1\le i\le j\le n}} X_{ij}
%    \]
%  \end{verbatim}
%  \[
%    X = \sum_{\mathclap{1\le i\le j\le n}} X_{ij}
%  \]
%  For an in-depth discussion of
%  these macros I find it better to read the article; an online
%  version can be found at
%  \begin{quote}
%    \url{http://www.tug.org/TUGboat/Articles/tb22-4/tb72perlS.pdf}
%  \end{quote}
%  Note that the definitions shown in the article do not exactly
%  match the definitions in \pkg{mathtools}. Besides providing an
%  optional argument for specifying the desired math style, these
%  versions also work around a most unfortunate \TeX{}
%  ``feature.''\footnote{The faulty reboxing procedure.} The
%  \cs{smash} macro is fixed too.
%
%
%  \subsubsection{Forcing a cramped style}
%
%  \begin{codesyntax}
%    \SpecialUsageIndex{\cramped}
%    \cs{cramped}\oarg{mathstyle}\marg{math}
%  \end{codesyntax}
%  \cttPosting{Michael Herschorn}{1992/07/21}
%  Let's look at another example where we have used \cs{mathclap}:
%  \begin{verbatim}
%    \begin{equation}\label{eq:mathclap}
%      \sum_{\mathclap{a^2<b^2<c}}\qquad
%      \sum_{a^2<b^2<c}
%    \end{equation}
%  \end{verbatim}
%  \begin{equation}\label{eq:mathclap}
%    \sum_{\mathclap{a^2<b^2<c}}\qquad
%    \sum_{a^2<b^2<c}
%  \end{equation}
%  Do you see the difference? Maybe if I zoomed in a bit:
%  \begingroup \fontsize{24}{\baselineskip}\selectfont
%  \[
%    \sum_{\mathclap{a^2<b^2<c}}\qquad
%    \sum_{a^2<b^2<c}
%  \]
%  \endgroup
%  Notice how the limit of the right summation sign is typeset in a
%  more compact style than the left. It is because \TeX{} sets the
%  limits of operators in a \emph{cramped} style. For each of \TeX'
%  four math styles (\cs{displaystyle}, \cs{textstyle},
%  \cs{scriptstyle}, and \cs{scriptscriptstyle}), there also exists a
%  cramped style that doesn't raise exponents as much. Besides in the
%  limits of operators, \TeX{} also automatically uses these cramped
%  styles in radicals such as \cs{sqrt} and in the denominators of
%  fractions, but unfortunately there are no primitive commands that
%  allows you to detect crampedness or switch to it.
%
%  \pkg{mathtools} offers the command \cs{cramped} which forces a
%  cramped style in normal un-cramped math. Additionally you can
%  choose which of the four styles you want it in as well by
%  specifying it as the optional argument:
%  \begin{verbatim}
%    \[
%      \cramped{x^2}               \leftrightarrow x^2    \quad
%      \cramped[\scriptstyle]{x^2} \leftrightarrow {\scriptstyle x^2}
%    \]
%  \end{verbatim}
%  \[
%    \cramped{x^2}               \leftrightarrow x^2    \quad
%    \cramped[\scriptstyle]{x^2} \leftrightarrow {\scriptstyle x^2}
%  \]
%  You may be surprised how often the cramped style can be
%  beneficial yo your output. Take a look at this example:
%  \begin{verbatim}
%    \begin{quote}
%      The 2005 Euro\TeX{} conference is held in Abbaye des
%      Pr\'emontr\'es, France, marking the 16th ($2^{2^2}$) anniversary
%      of both Dante and GUTenberg (the German and French \TeX{} users
%      group resp.).
%    \end{quote}
%  \end{verbatim}
%  \begin{quote}
%    The 2005 Euro\TeX{} conference is held in Abbaye des
%    Pr\'emontr\'es, France, marking the 16th ($2^{2^2}$) anniversary
%    of both Dante and GUTenberg (the German and French \TeX{} users
%    group resp.).
%  \end{quote}
%  Typesetting on a grid is generally considered quite desirable, but
%  as the second line of the example shows, the exponents of $2$
%  causes the line to be too tall for the normal value of
%  \cs{baselineskip}, so \TeX{} inserts a \cs{lineskip} (normal value
%  is \the\lineskip). In order to circumvent the problem, we can
%  force a cramped style so that the exponents aren't raised as much:
%  \begin{verbatim}
%    \begin{quote}
%      The 2005 Euro\TeX{} ... 16th ($\cramped{2^{2^2}}$) ...
%    \end{quote}
%  \end{verbatim}
%  \begin{quote}
%    The 2005 Euro\TeX{} conference is held in Abbaye des
%    Pr\'emontr\'es, France, marking the 16th ($\cramped{2^{2^2}}$)
%    anniversary of both Dante and GUTenberg (the German and French
%    \TeX{} users group resp.).
%  \end{quote}
%
%  \begin{codesyntax}
%    \SpecialUsageIndex{\crampedllap}
%    \cs{crampedllap}\oarg{mathstyle}\marg{math}\texttt{~~}
%    \SpecialUsageIndex{\crampedclap}
%    \cs{crampedclap}\oarg{mathstyle}\marg{math}\\
%    \SpecialUsageIndex{\crampedrlap}
%    \cs{crampedrlap}\oarg{mathstyle}\marg{math}
%  \end{codesyntax}
%  The commands \cs{crampedllap}, \cs{crampedclap}, and
%  \cs{crampedrlap} are identical to the three \cs{mathXlap} commands
%  described earlier except the argument is typeset in cramped style.
%  You need this in order to typeset \eqref{eq:mathclap} correctly
%  while still faking the width of the limit.
%  \begin{verbatim}
%    \begin{equation*}\label{eq:mathclap-b}
%      \sum_{\crampedclap{a^2<b^2<c}}
%      \tag{\ref{eq:mathclap}*}
%    \end{equation*}
%  \end{verbatim}
%  \begin{equation*}\label{eq:mathclap-b}
%    \sum_{\crampedclap{a^2<b^2<c}}
%    \tag{\ref{eq:mathclap}*}
%  \end{equation*}
%  Of course you could just type
%  \begin{verbatim}
%    \sum_{\mathclap{\cramped{a^2<b^2<c}}}
%  \end{verbatim}
%  but it has one major disadvantage: In order for \cs{mathXlap} and
%  \cs{cramped} to get the right size, \TeX{} has to process them
%  four times, meaning that nesting them as shown above will cause
%  \TeX{} to typeset $4^2$ instances before choosing the right one.
%  In this situation however, we will of course need the same style
%  for both commands so it makes sense to combine the commands in
%  one, thus letting \TeX{} make the choice only once rather than
%  twice.
%
%
%
%  \subsubsection{Smashing an operator}
%
%
%
%  \begin{codesyntax}
%    \SpecialUsageIndex{\smashoperator}
%    \cs{smashoperator}\oarg{pos}\marg{operator with limits}
%  \end{codesyntax}
%  \FeatureRequest{Lars Madsen}{2004/05/04}
%  Above we shoved how to get \LaTeX{} to ignore the width of the
%  subscript of an operator. However this approach takes a lot of
%  extra typing, especially if you have a wide superscript, meaning
%  you have to put in \cs{crampedclap} in both sub- and superscript.
%  To make things easier, \pkg{mathtools} provides a
%  \cs{smashoperator} command, which simply ignores the width of the
%  sub- and superscript. It also takes an optional argument,
%  \texttt{l}, \texttt{r}, or \texttt{lr} (default), denoting which
%  side of the operator should be ignored (smashed).
%  \begin{verbatim}
%    \[
%      V = \sum_{1\le i\le j\le n}^{\infty} V_{ij}                  \quad
%      X = \smashoperator{\sum_{1\le i\le j\le n}^{3456}} X_{ij}    \quad
%      Y = \smashoperator[r]{\sum\limits_{1\le i\le j\le n}} Y_{ij} \quad
%      Z = \smashoperator[l]{\mathop{T}_{1\le i\le j\le n}} Z_{ij}
%    \]
%  \end{verbatim}
%    \[
%      V = \sum_{1\le i\le j\le n}^{\infty} V_{ij}                  \quad
%      X = \smashoperator{\sum_{1\le i\le j\le n}^{3456}} X_{ij}    \quad
%      Y = \smashoperator[r]{\sum\limits_{1\le i\le j\le n}} Y_{ij} \quad
%      Z = \smashoperator[l]{\mathop{T}_{1\le i\le j\le n}} Z_{ij}
%    \]
%  Note that \cs{smashoperator} always sets its argument in display
%  style and with limits even if you have used the \opt{nosumlimits}
%  option of \pkg{amsmath}. If you wish, you can use shorthands for
%  \texttt{\_} and \texttt{\textasciicircum} such as \cs{sb} and
%  \cs{sp}.
%
%
%  \subsubsection{Adjusting limits of operators}
%
%  \begin{codesyntax}
%    \SpecialUsageIndex{\adjustlimits}
%    \cs{adjustlimits}\marg{operator$\sb1$}\texttt{\_}\marg{limit$\sb1$}
%                     \marg{operator$\sb2$}\texttt{\_}\marg{limit$\sb2$}
%  \end{codesyntax}
%  \FeatureRequest{Lars Madsen}{2004/07/09}
%  When typesetting two consecutive operators with limits one often
%  wishes the limits of the operators were better aligned. Look
%  closely at these examples:
%  \begin{verbatim}
%    \[
%      \text{a)} \lim_{n\to\infty} \max_{p\ge n} \quad
%      \text{b)} \lim_{n\to\infty} \max_{p^2\ge n} \quad
%      \text{c)} \lim_{n\to\infty} \sup_{p^2\ge nK} \quad
%      \text{d)} \limsup_{n\to\infty} \max_{p\ge n}
%    \]
%  \end{verbatim}
%  \[
%    \text{a)} \lim_{n\to\infty} \max_{p\ge n} \quad
%    \text{b)} \lim_{n\to\infty} \max_{p^2\ge n} \quad
%    \text{c)} \lim_{n\to\infty} \sup_{p^2\ge nK} \quad
%    \text{d)} \limsup_{n\to\infty} \max_{p\ge n}
%  \]
%  a) looks okay, but b) is not quite as good because the second
%  limit ($\cramped{p^2\ge n}$) is significantly taller than the
%  first ($n\to\infty$). With c)~things begin to look really bad,
%  because the second operator has a descender while the first
%  doesn't, and finally we have d)~which looks just as bad as~c). The
%  command \cs{adjustlimits} is useful in these cases, as you can
%  just put it in front of these consecutive operators and it'll make
%  the limits line up.
%  \medskip\par\noindent
%  \begin{minipage}{\textwidth}
%  \begin{verbatim}
%    \[
%      \text{a)} \adjustlimits\lim_{n\to\infty} \max_{p\ge n} \quad
%      \text{b)} \adjustlimits\lim_{n\to\infty} \max_{p^2\ge n} \quad
%      \text{c)} \adjustlimits\lim_{n\to\infty} \sup_{p^2\ge nK} \quad
%      \text{d)} \adjustlimits\limsup_{n\to\infty} \max_{p\ge n}
%    \]
%  \end{verbatim}
%  \end{minipage}
%  \[
%      \text{a)} \adjustlimits\lim_{n\to\infty} \max_{p\ge n} \quad
%      \text{b)} \adjustlimits\lim_{n\to\infty} \max_{p^2\ge n} \quad
%      \text{c)} \adjustlimits\lim_{n\to\infty} \sup_{p^2\ge nK} \quad
%      \text{d)} \adjustlimits\limsup_{n\to\infty} \max_{p\ge n}
%    \]
%  The use of \cs{sb} instead of \texttt{\_} is allowed.
%
%
%  \subsubsection{Swapping space above \texorpdfstring{\AmS}{AMS} display math environments }
%  \label{sec:swapping}
%
%  One feature that the plain old \env{equation} environment has that
%  the \AmS\ environments does not (because of thechnical reasons), is
%  the feature of using less space above the equation if the situation
%  presents itself. The \AmS\ environments cannot do this, but one can
%  manually, using  
%  \begin{codesyntax}
%    \SpecialUsageIndex{\SwapAboveDisplaySkip}
%    \cs{SwapAboveDisplaySkip}
%  \end{codesyntax}
%  as the very first content within an \AmS\ display math
%  environment. It will then issue an \cs{abovedisplayshortskip}
%  instead of the normal \cs{abovedisplayskip}.
%
%  Note it will not work with the \env{equation} or \env{multline} environments.
%  
%  Here is an example of the effect
% \begin{verbatim}
%  \noindent\rule\textwidth{1pt}
%  \begin{align*}  A &= B \end{align*}
%  \noindent\rule\textwidth{1pt}
%  \begin{align*}  
%  \SwapAboveDisplaySkip
%  A &= B 
%  \end{align*}
% \end{verbatim}
%  \noindent\rule\textwidth{1pt}
%  \begin{align*}  A &= B \end{align*}
%  \noindent\rule\textwidth{1pt}
%  \begin{align*}  
%  \SwapAboveDisplaySkip
%  A &= B 
%  \end{align*}
%  
%
%  \subsection{Controlling tags}
%
%  In this section various tools for altering the appearance of tags
%  are shown. All of the tools here can be used at any point in the
%  document but they should probably be affect the whole document, so
%  the preamble is the best place to issue them.
%
%  \subsubsection{The appearance of tags}
%  \begin{codesyntax}
%    \SpecialUsageIndex{\newtagform}
%    \cs{newtagform}\marg{name}\oarg{inner_format}\marg{left}\marg{right}\\
%    \SpecialUsageIndex{\renewtagform}
%    \cs{renewtagform}\marg{name}\oarg{inner_format}\marg{left}\marg{right}\\
%    \SpecialUsageIndex{\usetagform}
%    \cs{usetagform}\marg{name}
%  \end{codesyntax}
%  Altering the layout of equation numbers in \pkg{amsmath} is not
%  very user friendly (it involves a macro with three \texttt{@}'s in
%  its name), so \pkg{mathtools} provides an interface somewhat
%  reminiscent of the page style concept. This way you can define
%  several different tag forms and then choose the one you prefer.
%
%  As an example let's try to define a tag form which puts the
%  equation number in square brackets. First we define a brand new tag
%  form:
%  \begin{verbatim}
%    \newtagform{brackets}{[}{]}
%  \end{verbatim}
%  Then we activate it:
%  \begin{verbatim}
%    \usetagform{brackets}
%  \end{verbatim}
%  The result is then
%  \newtagform{brackets}{[}{]}
%  \usetagform{brackets}
%   \begin{equation}
%     E \neq m c^3
%   \end{equation}
%
%  Similarly you could define a second version of the brackets that
%  prints the equation number in bold face instead
%  \begin{verbatim}
%    \newtagform{brackets2}[\textbf]{[}{]}
%    \usetagform{brackets2}
%    \begin{equation}
%      E \neq m c^3
%    \end{equation}
%  \end{verbatim}
%  \newtagform{brackets2}[\textbf]{[}{]}
%  \usetagform{brackets2}
%  \begin{equation}
%    E \neq m c^3
%  \end{equation}
%  When you reference an equation with \cs{eqref}, the tag form in
%  effect at the time of referencing controls the formatting, so be
%  careful if you use different tag forms throughout your document.
%
%  If you want to renew a tag form, then use the command
%  \cs{renewtagform}. Should you want to
%  return to the standard setting then choose\usetagform{default}
%  \begin{verbatim}
%    \usetagform{default}
%  \end{verbatim}
%
%  \changes{v1.12}{2012/05/09}{Added caveat}
%  \noindent\textbf{Caveat regarding \pkg{ntheorem}}: If you like to
%  change the appearence of the tags \emph{and} you are also using the
%  \pkg{ntheorem} package, then please postpone the change of
%  appearance until \emph{after} loading \pkg{ntheorem}. (In order to
%  do its thing, \pkg{ntheorem} has to mess with the tags\dots)
%
%  \subsubsection{Showing only referenced tags}
%
%  \begin{codesyntax}
%    \SpecialKeyIndex{showonlyrefs}$\key{showonlyrefs}=\texttt{true}\vert\texttt{false}$\\
%    \SpecialKeyIndex{showmanualtags}$\key{showmanualtags}=\texttt{true}\vert\texttt{false}$\\
%    \SpecialUsageIndex{\refeq}\cs{refeq}\marg{label}
%  \end{codesyntax}
%  An equation where the tag is produced with a manual \cs{tag*}
%  shouldn't be referenced with the normal \cs{eqref} because that
%  would format it according to the current tag format. Using just
%  \cs{ref} on the other hand may not be a good solution either as
%  the argument of \cs{tag*} is always set in upright shape in the
%  equation and you may be referencing it in italic text. In the
%  example below, the command \cs{refeq} is used to avoid what could
%  possibly lead to confusion in cases where the tag font has very
%  different form in upright and italic shape (here we switch to
%  Palatino in the example):
%    \begin{verbatim}
%    \begin{quote}\renewcommand*\rmdefault{ppl}\normalfont\itshape
%    \begin{equation*}
%      a=b \label{eq:example}\tag*{Q\&A}
%    \end{equation*}
%    See \ref{eq:example} or is it better with \refeq{eq:example}?
%    \end{quote}
%  \end{verbatim}
%    \begin{quote}\renewcommand*\rmdefault{ppl}\normalfont\itshape
%    \begin{equation*}
%      a=b \label{eq:example}\tag*{Q\&A}
%    \end{equation*}
%    See \ref{eq:example} or is it better with \refeq{eq:example}?
%  \end{quote}
%
%
%  Another problem sometimes faced is the need for showing the
%  equation numbers for only those equations actually referenced. In
%  \pkg{mathtools} this can be done by setting the key
%  \key{showonlyrefs} to either true or false by using
%  \cs{mathtoolsset}. You can also choose whether or not to show the
%  manual tags specified with \cs{tag} or \cs{tag*} by setting the
%  option \key{showmanualtags} to true or false.\footnote{I recommend
%  setting \key{showmanualtags} to true, else the whole idea of using
%  \cs{tag} doesn't really make sense, does it?} For both keys just
%  typing the name of it chooses true as shown in the following
%  example.
%
%  \begin{verbatim}
%  \mathtoolsset{showonlyrefs,showmanualtags}
%  \usetagform{brackets}
%  \begin{gather}
%    a=a \label{eq:a} \\
%    b=b \label{eq:b} \tag{**}
%  \end{gather}
%  This should refer to the equation containing $a=a$: \eqref{eq:a}.
%  Then a switch of tag forms.
%  \usetagform{default}
%  \begin{align}
%    c&=c \label{eq:c} \\
%    d&=d \label{eq:d}
%  \end{align}
%  This should refer to the equation containing $d=d$: \eqref{eq:d}.
%  \begin{equation}
%    e=e
%  \end{equation}
%  Back to normal.\mathtoolsset{showonlyrefs=false}
%  \begin{equation}
%    f=f
%  \end{equation}
%  \end{verbatim}
%  \mathtoolsset{showonlyrefs,showmanualtags}
%  \usetagform{brackets}
%  \begin{gather}
%    a=a \label{eq:a} \\
%    b=b \label{eq:b} \tag{**}
%  \end{gather}
%  This should refer to the equation containing $a=a$: \eqref{eq:a}.
%  Then a switch of tag forms.
%  \usetagform{default}
%  \begin{align}
%    c&=c \label{eq:c} \\
%    d&=d \label{eq:d}
%  \end{align}
%  This should refer to the equation containing $d=d$: \eqref{eq:d}.
%  \begin{equation}
%    e=e
%  \end{equation}
%  Back to normal.\mathtoolsset{showonlyrefs=false}
%  \begin{equation}
%    f=f
%  \end{equation}
%
%  Note that this feature only works if you use \cs{eqref} or
%  \cs{refeq} to reference your equations.
%
%  When using \key{showonlyrefs} it might be useful to be able to
%  actually add a few equation numbers without directly referring to
%  them.
%  \begin{codesyntax}
%    \SpecialUsageIndex{\noeqref}\cs{noeqref}\marg{label,label,\dots}
%  \end{codesyntax}
%  \FeatureRequest{Rasmus Villemoes}{2008/03/26}
%  The syntax is somewhat similar to \cs{nocite}. If a label in the
%  list is undefined we will throw a warning in the same manner as
%  \cs{ref}. 
%
%  \medskip\noindent\textbf{BUG 1:} Unfortunately the use of the
%  \key{showonlyref} introduce a bug within amsmath's typesetting
%  of formula versus equation number. This bug manifest itself by
%  allowing formulas to be typeset close to or over the equation
%  number.  Currently no general fix is known, other than making sure
%  that one's formulas are not long enough to touch the equation
%  number.
%
%  To make a long story stort, amsmath typesets its math environments
%  twice, one time for measuring and one time for the actual
%  typesetting. In the measuring part, the width of the equation
%  number is recorded such that the formula or the equation number can
%  be moved (if necessary) in the typesetting part. When
%  \key{showonlyref} is enabled, the width of the equation number
%  depend on whether or not this number is referred~to. To determine
%  this, we need to know the current label. But the current label is
%  \emph{not} known in the measuring phase. Thus the measured width is
%  always zero (because no label equals not referred to) and therefore
%  the typesetting phase does not take the equation number into
%  account.
%
% \medskip\noindent\textbf{BUG 2:} Currently there is a bug between
% \key{showonlyrefs} and the \pkg{ntheorem} package, when the
% \pkg{ntheorem} option \key{thmmarks} is active. The shown equation
% numbers may come out wrong (seems to be multiplied by 2). The
% easiest fix is to add the following line
% \begin{verbatim}
% \usepackage[overload,ntheorem]{empheq}
% \end{verbatim}
% before loading \pkg{ntheorem}. The \pkg{empheq} package fixes some
% problems with \pkg{ntheorem} and lets \pkg{mathtools} get correct
% access to the equation numbers again.
% 
%  \subsection{Extensible symbols}
%
%  The number of horizontally extensible symbols in standard \LaTeX{}
%  and \pkg{amsmath} is somewhat low. This part of the manual
%  describes what \pkg{mathtools} does to help this situation.
%
%  \subsubsection{Arrow-like symbols}
%
%
%  \begin{codesyntax}
%    \SpecialUsageIndex{\xleftrightarrow}
%    \cs{xleftrightarrow}\oarg{sub}\marg{sup}\texttt{~~~~~~~~~}
%    \SpecialUsageIndex{\xRightarrow}
%    \cs{xRightarrow}\oarg{sub}\marg{sup}\\
%    \SpecialUsageIndex{\xLeftarrow}
%    \cs{xLeftarrow}\oarg{sub}\marg{sup}\texttt{~~~~~~~~~~~~~~}
%    \SpecialUsageIndex{\xLeftrightarrow}
%    \cs{xLeftrightarrow}\oarg{sub}\marg{sup}\\
%    \SpecialUsageIndex{\xhookleftarrow}
%    \cs{xhookleftarrow}\oarg{sub}\marg{sup}\texttt{~~~~~~~~~~}
%    \SpecialUsageIndex{\xhookrightarrow}
%    \cs{xhookrightarrow}\oarg{sub}\marg{sup}\\
%    \SpecialUsageIndex{\xmapsto}
%    \cs{xmapsto}\oarg{sub}\marg{sup}
%  \end{codesyntax}
%  Extensible arrows are part of \pkg{amsmath} in the form of the
%  commands
%  \begin{quote}
%    \cs{xrightarrow}\oarg{subscript}\marg{superscript}\quad and\\
%    \cs{xleftarrow}\oarg{subscript}\marg{superscript}
%  \end{quote}
%  But what about extensible versions of say, \cs{leftrightarrow} or
%  \cs{Longleftarrow}? It turns out that the above mentioned
%  extensible arrows are the only two of their kind defined by
%  \pkg{amsmath}, but luckily \pkg{mathtools} helps with that. The
%  extensible arrow-like symbols in \pkg{mathtools} follow the same
%  naming scheme as the one's in \pkg{amsmath} so to get an extensible
%  \cs{Leftarrow} you simply do a
%  \begin{verbatim}
%    \[
%      A \xLeftarrow[under]{over} B
%    \]
%  \end{verbatim}
%    \[
%      A \xLeftarrow[under]{over} B
%    \]
%  \begin{codesyntax}
%    \SpecialUsageIndex{\xrightharpoondown}
%    \cs{xrightharpoondown}\oarg{sub}\marg{sup}\texttt{~~~~}
%    \SpecialUsageIndex{\xrightharpoonup}
%    \cs{xrightharpoonup}\oarg{sub}\marg{sup}\\
%    \SpecialUsageIndex{\xleftharpoondown}
%    \cs{xleftharpoondown}\oarg{sub}\marg{sup}\texttt{~~~~~}
%    \SpecialUsageIndex{\xleftharpoonup}
%    \cs{xleftharpoonup}\oarg{sub}\marg{sup}\\
%    \SpecialUsageIndex{\xrightleftharpoons}
%    \cs{xrightleftharpoons}\oarg{sub}\marg{sup}\texttt{~~~}
%    \SpecialUsageIndex{\xleftrightharpoons}
%    \cs{xleftrightharpoons}\oarg{sub}\marg{sup}
%  \end{codesyntax}
%  \pkg{mathtools} also provides the extensible harpoons shown above.
%  They're taken from~\cite{Voss:2004}.
%
%  \subsubsection{Braces and brackets}
%
%  \LaTeX{} defines other kinds of extensible symbols like
%  \cs{overbrace} and \cs{underbrace}, but sometimes you may want
%  another symbol, say, a bracket.
%  \begin{codesyntax}
%    \SpecialUsageIndex{\underbracket}\cs{underbracket}\oarg{rule thickness}
%      \oarg{bracket height}\marg{arg}\\
%    \SpecialUsageIndex{\overbracket}\cs{overbracket}\oarg{rule thickness}
%      \oarg{bracket height}\marg{arg}
%  \end{codesyntax}
%  The commands \cs{underbracket} and \cs{overbracket} are inspired
%  by \cite{Voss:2004}, although the implementation here is slightly
%  different.
%  Used without the optional arguments the bracket commands produce this:
%  \begin{quote}
%   |$\underbracket {foo\ bar}_{baz}$|\quad  $\underbracket {foo\ bar}_{baz}$ \\
%   |$\overbracket {foo\ bar}^{baz}$ |\quad  $\overbracket {foo\ bar}^{baz}$
%  \end{quote}
%  The default rule thickness is equal to that of \cs{underbrace}
%  (app.~$5/18$\,ex) while the default bracket height is equal to
%  app.~$0.7$\,ex. These values give really pleasing results in all
%  font sizes, but feel free to use the optional arguments. That way
%  you may get ``beauties'' like
%  \begin{verbatim}
%    \[
%      \underbracket[3pt]{xxx\  yyy}_{zzz} \quad \text{and} \quad
%      \underbracket[1pt][7pt]{xxx\  yyy}_{zzz}
%    \]
%  \end{verbatim}
%    \[
%      \underbracket[3pt]{xxx\  yyy}_{zzz} \quad \text{and} \quad
%      \underbracket[1pt][7pt]{xxx\  yyy}_{zzz}
%    \]
%  \begin{codesyntax}
%    \SpecialUsageIndex{\underbrace}\cs{underbrace}\marg{arg}\texttt{~~}
%    \SpecialUsageIndex{\LaTeXunderbrace}\cs{LaTeXunderbrace}\marg{arg}\\
%    \SpecialUsageIndex{\overbrace}\cs{overbrace}\marg{arg}\texttt{~~~}
%    \SpecialUsageIndex{\LaTeXoverbrace}\cs{LaTeXoverbrace}\marg{arg}
%  \end{codesyntax}
%  The standard implementation of the math operators \cs{underbrace}
%  and \cs{overbrace} in \LaTeX{} has some deficiencies. For example,
%  all lengths used internally are \emph{fixed} and optimized for
%  10\,pt typesetting. As a direct consequence thereof, using font
%  sizes other than 10 will produce less than optimal results.
%  Another unfortunate feature is the size of the braces. In the
%  example below, notice how the math operator \cs{sum} places its
%  limit compared to \cs{underbrace}.
%  \[
%    \mathop{\thinboxed[blue]{\sum}}_{n}
%    \mathop{\thinboxed[blue]{\LaTeXunderbrace{\thinboxed[green]{foof}}}}_{zzz}
%  \]
%  The blue lines indicate the dimensions of the math operator and
%  the green lines the dimensions of $foof$. As you can see, there
%  seems to be too much space between the brace and the $zzz$ whereas
%  the space between brace and $foof$ is okay. Let's see what happens
%  when we use a bigger font size:\par\Huge\vskip-\baselineskip
%  \[
%    \mathop{\thinboxed[blue]{\sum}}_{n}
%    \mathop{\thinboxed[blue]{\LaTeXunderbrace{\thinboxed[green]{foof}}}}_{zzz}
%  \]
%  \normalsize Now there's too little space between the brace and the
%  $zzz$ and also too little space between the brace and the $foof$.
%  If you use Computer Modern you'll actually see that the $f$
%  overlaps with the brace! Let's try in \cs{footnotesize}:
%  \par\footnotesize
%  \[
%    \mathop{\thinboxed[blue]{\sum}}_{n}
%    \mathop{\thinboxed[blue]{\LaTeXunderbrace{\thinboxed[green]{foof}}}}_{zzz}
%  \]\normalsize
%  Here the spacing above and below the brace is quite excessive.
%
%  As \cs{overbrace} has the exact same problems, there are good
%  reasons for \pkg{mathtools} to make redefinitions of
%  \cs{underbrace} and \cs{overbrace}. These new versions work
%  equally well in all font sizes and fixes the spacing issues and
%  apart from working with the default Computer Modern fonts, they
%  also work with the packages \pkg{mathpazo}, \pkg{pamath},
%  \pkg{fourier}, \pkg{eulervm}, \pkg{cmbright}, and \pkg{mathptmx}.
%  If you use the \pkg{ccfonts} to get the full Concrete fonts, the
%  original version saved under the names \cs{LaTeXunderbrace} and
%  \cs{LaTeXoverbrace} are better, due to of the special design of
%  the Concrete extensible braces. In that case you should probably
%  just add the lines
%  \begin{verbatim}
%    \let\underbrace\LaTeXunderbrace
%    \let\overbrace\LaTeXoverbrace
%  \end{verbatim}
%  to your preamble after loading \pkg{mathtools} which will restore
%  the original definitions of \cs{overbrace} and \cs{underbrace}.
%
%
%  
%
%  \subsection{New mathematical building blocks}
%
%  In this part of the manual, various mathematical environments are
%  described.
%
%  \subsubsection{Matrices}\label{subsubsec:matrices}
%
%  \begin{codesyntax}
%    \SpecialEnvIndex{matrix*}\cs{begin}\arg{matrix*}\texttt{ }\oarg{col}
%        \meta{contents} \cs{end}\arg{matrix*}\\
%    \SpecialEnvIndex{pmatrix*}\cs{begin}\arg{pmatrix*}\oarg{col}
%        \meta{contents} \cs{end}\arg{pmatrix*}\\
%    \SpecialEnvIndex{bmatrix*}\cs{begin}\arg{bmatrix*}\oarg{col}
%        \meta{contents} \cs{end}\arg{bmatrix*}\\
%    \SpecialEnvIndex{Bmatrix*}\cs{begin}\arg{Bmatrix*}\oarg{col}
%        \meta{contents} \cs{end}\arg{Bmatrix*}\\
%    \SpecialEnvIndex{vmatrix*}\cs{begin}\arg{vmatrix*}\oarg{col}
%        \meta{contents} \cs{end}\arg{vmatrix*}\\
%    \SpecialEnvIndex{Vmatrix*}\cs{begin}\arg{Vmatrix*}\oarg{col}
%        \meta{contents} \cs{end}\arg{Vmatrix*}
%  \end{codesyntax}
%  \FeatureRequest{Lars Madsen}{2004/04/05}
%  All of the \pkg{amsmath} \env{matrix} environments center the
%  columns by default, which is not always what you want. Thus
%  \pkg{mathtools} provides a starred version for each of the original
%  environments. These starred environments take an optional argument
%  specifying the alignment of the columns, so that
%  \begin{verbatim}
%    \[
%      \begin{pmatrix*}[r]
%        -1 & 3 \\
%        2  & -4
%      \end{pmatrix*}
%    \]
%  \end{verbatim}
%  yields
%    \[
%      \begin{pmatrix*}[r]
%        -1 & 3 \\
%        2  & -4
%      \end{pmatrix*}
%    \]
%  The optional argument (default is \texttt{[c]}) can be any column
%  type valid in the usual \env{array} environment.
%
%  While we are at it, we also provide fenced versions of the
%  \env{smallmatrix} environment, To keep up with the naming of the
%  large matrix environments, we provide both a starred and a
%  non-starred version. Since \env{smallmatrix} is defined in a
%  different manner than the \env{matrix} environment, the option to
%  say \env{smallmatrix*} \emph{has} to be either \texttt{c},
%  \texttt{l} \emph{or}~\texttt{r}. The default is \texttt{c}, which
%  can be changed globally using the \key{smallmatrix-align}=\meta{c,l
%    or r}.
%  \begin{codesyntax}
%    \SpecialEnvIndex{smallmatrix*}\cs{begin}\arg{smallmatrix*}\texttt{ }\oarg{col}
%        \meta{contents} \cs{end}\arg{smallmatrix*}\\
%    \SpecialEnvIndex{psmallmatrix}\cs{begin}\arg{psmallmatrix}
%        \meta{contents} \cs{end}\arg{psmallmatrix}\\
%    \SpecialEnvIndex{psmallmatrix*}\cs{begin}\arg{psmallmatrix*}\oarg{col}
%        \meta{contents} \cs{end}\arg{psmallmatrix*}\\
%    \SpecialEnvIndex{bsmallmatrix}\cs{begin}\arg{bsmallmatrix}
%        \meta{contents} \cs{end}\arg{bsmallmatrix}\\
%    \SpecialEnvIndex{bsmallmatrix*}\cs{begin}\arg{bsmallmatrix*}\oarg{col}
%        \meta{contents} \cs{end}\arg{bsmallmatrix*}\\
%    \SpecialEnvIndex{Bsmallmatrix}\cs{begin}\arg{Bsmallmatrix}
%        \meta{contents} \cs{end}\arg{Bsmallmatrix}\\
%    \SpecialEnvIndex{Bsmallmatrix*}\cs{begin}\arg{Bsmallmatrix*}\oarg{col}
%        \meta{contents} \cs{end}\arg{Bsmallmatrix*}\\
%    \SpecialEnvIndex{vsmallmatrix}\cs{begin}\arg{vsmallmatrix}
%        \meta{contents} \cs{end}\arg{vsmallmatrix}\\
%    \SpecialEnvIndex{vsmallmatrix*}\cs{begin}\arg{vsmallmatrix*}\oarg{col}
%        \meta{contents} \cs{end}\arg{vsmallmatrix*}\\
%    \SpecialEnvIndex{Vsmallmatrix}\cs{begin}\arg{Vsmallmatrix}
%        \meta{contents} \cs{end}\arg{Vsmallmatrix}\\
%    \SpecialEnvIndex{Vsmallmatrix*}\cs{begin}\arg{Vsmallmatrix*}\oarg{col}
%        \meta{contents} \cs{end}\arg{Vsmallmatrix*}\\
%    \SpecialKeyIndex{smallmatrix-align}\makebox{$\key{smallmatrix-align}=\meta{c,l or r}$}\\
%    \SpecialKeyIndex{smallmatrix-inner-space}\makebox{$\key{smallmatrix-inner-space}=\cs{,}$}
%  \end{codesyntax}
%  \ProvidedBy{Rasmus Villemoes}{2011/01/17}
% \begin{verbatim}
% \[
% \begin{bsmallmatrix}    a & -b \\ -c & d \end{bsmallmatrix}
% \begin{bsmallmatrix*}[r] a & -b \\ -c & d \end{bsmallmatrix*}
% \]
% \end{verbatim}
%  yields
% \[
% \begin{bsmallmatrix} a & -b \\ -c & d \end{bsmallmatrix}
% \begin{bsmallmatrix*}[r] a & -b \\ -c & d \end{bsmallmatrix*}
% \]
% Inside the \verb?Xsmallmatrix? construction a small space is
% inserted between the fences and the contents, the size of it can be
% changed using \key{smallmatrix-align}=\meta{some spacing command},
% the default is \cs{,}.
%
% As an extra trick the fences will behave as open and closing fences
% in constract to their auto-scaling nature.\footnote{\cs{left} and
%   \cs{right} do \emph{not} produce open and closing fences, thus
%   the space before or after may be too large. Inside this
%   construction they behave.}
% 
%  \subsubsection{The \env{multlined} environment}
%
%  \begin{codesyntax}
%    \SpecialEnvIndex{multlined}\cs{begin}\arg{multlined}\oarg{pos}\oarg{width}
%        \meta{contents} \cs{end}\arg{multlined}\\
%    \SpecialUsageIndex{\shoveleft}\cs{shoveleft}\oarg{dimen}\marg{arg}\texttt{~~}
%    \SpecialUsageIndex{\shoveright}\cs{shoveright}\oarg{dimen}\marg{arg}\\
%    \makeatletter\settowidth\@tempdimc{\cs{shoveleft}\oarg{dimen}\marg{arg}}\global\@tempdimc\@tempdimc
%    \SpecialKeyIndex{firstline-afterskip}\makebox[\@tempdimc][l]{$\key{firstline-afterskip}=\meta{dimen}$}\texttt{~~}
%    \SpecialKeyIndex{lastline-preskip}$\key{lastline-preskip}=\meta{dimen}$\\
%    \makeatletter\SpecialKeyIndex{multlined-width}\makebox[\@tempdimc][l]{$\key{multlined-width}=\meta{dimen}$}\texttt{~~}
%    \SpecialKeyIndex{multlined-pos}$\key{multlined-pos}=\texttt{c}\vert\texttt{b}\vert\texttt{t}$
%  \end{codesyntax}
%  Some of the \pkg{amsmath} environments exist in two forms: an
%  outer and an inner environment. One example is the pair
%  \env{gather} \& \env{gathered}. There is one important omission on
%  this list however, as there is no inner \env{multlined}
%  environment, so this is where \pkg{mathtools} steps in.
%
%  One might wonder what the sensible behavior should be. We want it
%  to be an inner environment so that it is not wider than necessary,
%  but on the other hand we would like to be able to control the
%  width. The current implementation of \env{multlined} handles both
%  cases. The idea is this: Set the first line flush left and add a
%  hard space after it; this space is governed by the
%  \key{firstline-afterskip} key. The last line should be set flush
%  right and preceded by a hard space of size \key{lastline-preskip}.
%  Both these hard spaces have a default value of \cs{multlinegap}.
%  Here we use a `t' in the first optional argument denoting a
%  top-aligned building block (the default is `c').
%  \begin{verbatim}
%    \[
%      A = \begin{multlined}[t]
%            \framebox[4cm]{first} \\
%            \framebox[4cm]{last}
%          \end{multlined} B
%    \]
%  \end{verbatim}
%    \[
%      A = \begin{multlined}[t]
%            \framebox[4cm]{first} \\
%            \framebox[4cm]{last}
%          \end{multlined} B
%    \]
%  Note also that \env{multlined} gives you access to an extended
%  syntax for \cs{shoveleft} and \cs{shoveright} as shown in the
%  example below.
%  \begin{verbatim}
%    \[
%      \begin{multlined}
%        \framebox[.65\columnwidth]{First line}        \\
%        \framebox[.5\columnwidth]{Second line}        \\
%        \shoveleft{L+E+F+T}                           \\
%        \shoveright{R+I+G+H+T}                        \\
%        \shoveleft[1cm]{L+E+F+T}                      \\
%        \shoveright[\widthof{$R+I+G+H+T$}]{R+I+G+H+T} \\
%        \framebox[.65\columnwidth]{Last line}
%      \end{multlined}
%    \]
%  \end{verbatim}
%  \[
%    \begin{multlined}
%      \framebox[.65\columnwidth]{First line} \\
%      \framebox[.5\columnwidth]{Second line} \\
%      \shoveleft{L+E+F+T}         \\
%      \shoveright{R+I+G+H+T}         \\
%      \shoveleft[1cm]{L+E+F+T}         \\
%      \shoveright[\widthof{$R+I+G+H+T$}]{R+I+G+H+T}         \\
%      \framebox[.65\columnwidth]{Last line}
%    \end{multlined}
%  \]
%
%  You can also choose the width yourself by specifying it as an
%  optional argument:
%  \begin{verbatim}
%    \[
%      \begin{multlined}[b][7cm]
%        \framebox[4cm]{first} \\
%        \framebox[4cm]{last}
%      \end{multlined} = B
%    \]
%  \end{verbatim}
%    \[
%       \begin{multlined}[b][7cm]
%            \framebox[4cm]{first} \\
%            \framebox[4cm]{last}
%          \end{multlined} = B
%    \]
%  There can be two optional arguments (position and width) and
%  they're interchangeable.
%
%  \subsubsection{More \env{cases}-like environments}
%
%  \begin{codesyntax}
%    \SpecialEnvIndex{dcases}
%    \cs{begin}\arg{dcases}\texttt{~}  \meta{math_column} |&| \meta{math_column}
%    \cs{end}\arg{dcases}\\
%    \SpecialEnvIndex{dcases*}
%    \cs{begin}\arg{dcases*}  \meta{math_column} |&| \makebox[\widthof{\meta{math\_column}}][l]{\meta{text\_column}}
%    \cs{end}\arg{dcases*}\\
%    \SpecialEnvIndex{rcases}
%    \cs{begin}\arg{rcases}\texttt{~~}  \meta{math_column} |&| \makebox[\widthof{\meta{math\_column}}][l]{\meta{math\_column}}
%    \cs{end}\arg{rcases}\\
%    \SpecialEnvIndex{rcases*}
%    \cs{begin}\arg{rcases*}\texttt{~}  \meta{math_column} |&| \makebox[\widthof{\meta{math\_column}}][l]{\meta{text\_column}}
%    \cs{end}\arg{rcases*}\\
%    \SpecialEnvIndex{drcases}
%    \cs{begin}\arg{drcases}\texttt{~}  \meta{math_column} |&| \makebox[\widthof{\meta{math\_column}}][l]{\meta{math\_column}}
%    \cs{end}\arg{drcases}\\
%    \SpecialEnvIndex{drcases*}
%    \cs{begin}\arg{drcases*}  \meta{math_column} |&| \makebox[\widthof{\meta{math\_column}}][l]{\meta{text\_column}}
%    \cs{end}\arg{drcases*}\\
%    \SpecialEnvIndex{cases*}
%    \cs{begin}\arg{cases*}\texttt{~}  \meta{math_column} |&| \makebox[\widthof{\meta{math\_column}}][l]{\meta{text\_column}}
%    \cs{end}\arg{cases*}
%  \end{codesyntax}
%  \FeatureRequest{Lars Madsen}{2004/07/01}
%  Anyone who have tried to use an integral in the regular
%  \env{cases} environment from \pkg{amsmath} will have noticed that
%  it is set as
%  \[
%    a=\begin{cases}
%      E = m c^2     & \text{Nothing to see here} \\
%      \int x-3\, dx & \text{Integral is text style}
%    \end{cases}
%  \]
%  \pkg{mathtools} provides two environments similar to \env{cases}.
%  Using the \env{dcases} environment you get the same output as with
%  \env{cases} except that the rows are set in display style.
%  \begin{verbatim}
%  \[
%    \begin{dcases}
%      E = m c^2     & c \approx 3.00\times 10^{8}\,\mathrm{m}/\mathrm{s} \\
%      \int x-3\, dx & \text{Integral is display style}
%    \end{dcases}
%  \]
%  \end{verbatim}
%  \[
%    \begin{dcases}
%      E = m c^2     & c \approx 3.00\times 10^{8}\,\mathrm{m}/\mathrm{s} \\
%      \int x-3\, dx & \text{Integral is display style}
%    \end{dcases}
%  \]
%  Additionally the environment \env{dcases*} acts just the same, but
%  the second column is set in the normal roman font of the
%  document.\footnote{Or rather: it inherits the font characteristics
%  active just before the \env{dcases*} environment.}
%  \begin{verbatim}
%  \[
%    a= \begin{dcases*}
%      E = m c^2     & Nothing to see here \\
%      \int x-3\, dx & Integral is display style
%    \end{dcases*}
%  \]
%  \end{verbatim}
%  \[
%    a= \begin{dcases*}
%      E = m c^2     & Nothing to see here \\
%      \int x-3\, dx & Integral is display style
%    \end{dcases*}
%  \]
%  The environments \env{rcases}, \env{rcases*}, \env{drcases} and
%  \env{drcases*} are equivalent to \env{cases} and \env{dcases}, but
%  here the brace is placed on the right instead of on the left.
% \begin{verbatim}
% \[
% \begin{rcases*}
%   x^2 & for $x>0$\\ 
%   x^3 & else
% \end{rcases*} \quad \Rightarrow \cdots
% \]
% \end{verbatim}
% \[
% \begin{rcases*}
%   x^2 & for $x>0$\\ 
%   x^3 & else
% \end{rcases*} \quad \Rightarrow\cdots
% \]
%
%
%  \subsubsection{Emulating indented lines in alignments}
%  \begin{codesyntax}
%    \SpecialEnvIndex{\MoveEqLeft}\cs{MoveEqLeft}\oarg{number}
%  \end{codesyntax}
%  \ProvidedBy{Lars Madsen}{2008/06/05} In \cite{Swanson}, Ellen
%  Swanson recommends that when ever one has a long displayed formula,
%  spanning several lines, and it is unfeasible to align against a
%  relation within the first line, then all lines in the display
%  should be aligned at the left most edge of the first line, and all
%  subsequent lines should be indented by 2\,em (or if needed by a
%  smaller amount). That is we are talking about displayes that end up
%  looking like this
%  \begin{align*}
%    \MoveEqLeft \framebox[10cm][c]{Long first line}\\
%    & = \framebox[6cm][c]{ \hphantom{g} 2nd line}\\ 
%    & \leq \dots
%  \end{align*}
%  Traditionally one could do this by starting subsequent lines by
%  \verb+&\qquad ...+, but that is tedious. Instead the example above
%  was made using \cs{MoveEqLeft}:
%  \begin{verbatim}
%  \begin{align*}
%    \MoveEqLeft \framebox[10cm][c]{Long first line}\\
%    & = \framebox[6cm][c]{ \hphantom{g} 2nd line}\\ 
%    & \leq \dots
%  \end{align*}
%  \end{verbatim}
%  \cs{MoveEqLeft} is placed instead of the \verb+&+ on the first
%  line, and will effectively \emph{move} the entire first line
%  \oarg{number} of ems to the left (default is 2). If you choose to
%  align to the right of the relation, use \cs{MoveEqLeft}\verb+[3]+
%  to accommodate the extra distance to the alignment point:
%  \begin{verbatim}
%  \begin{align*}
%    \MoveEqLeft[3] \framebox[10cm][c]{Part 1}\\
%     = {} & \framebox[8cm][c]{2nd line}\\ 
%          & + \framebox[4cm][c]{ last part}
%  \end{align*}
%  \end{verbatim}
%  \begin{align*}
%    \MoveEqLeft[3] \framebox[10cm][c]{Long first line}\\
%     = {} & \framebox[6cm][c]{  2nd line}\\ 
%          & + \framebox[4cm][c]{ last part}
%  \end{align*}
%
%  \subsubsection{Boxing a single line in an alignment}
%  
%  The \texttt{amsmath} package provie the \cs{boxed} macro to box
%  material in math mode. But this of course will not work if the box
%  should cross an alignment point. We provide a macro that
%  can.\footnote{Note that internally \cs{Aboxed} does use \cs{boxed}.}
%    \hskip1sp
%   \marginpar{%
%    \parbox[b]{\marginparwidth}{\small\sffamily\raggedright
%      \strut Evolved from a request by\\Merciadri Luca\\
%       2010/06/28\\on comp.text.tex%
%    }\strut
%  }%
%   \marginpar{\strut\\%
%    \parbox[b]{\marginparwidth}{\small\sffamily\raggedright
%      \strut Reimplemented by\\Florent Chervet (GL) \\
%       2011/06/11\\on comp.text.tex%
%    }\strut
%  }%
%  \begin{codesyntax}
%    \SpecialEnvIndex{\Aboxed}\cs{Aboxed}\marg{left hand side 
%     \quad\texttt{\textnormal{\&}}\quad right hand side}
%  \end{codesyntax}
%  Example
% \begin{verbatim}
% \begin{align*}
%   \Aboxed{ f(x) & = \int h(x)\, dx} \\
%                 & = g(x)
% \end{align*}
% \end{verbatim}
% Resulting in:
% \begin{align*}
%   \Aboxed{ f(x) & = \int h(x)\, dx} \\
%                 & = g(x)
% \end{align*}
% One can have multiple boxes on each line, and the
% >>\texttt{\textnormal{\&}}\quad right hand side<< can even be
% missing. Here is an example of how the padding in the box can be changed
% \begin{verbatim}
% \begin{align*}
%   \setlength\fboxsep{1em}
%   \Aboxed{ f(x) &= 0 } & \Aboxed{ g(x) &= b} \\
%   \Aboxed{ h(x) }      & \Aboxed{ i(x) }   
% \end{align*}
% \end{verbatim}
% \begin{align*}
%   \setlength\fboxsep{1em}
%   \Aboxed{ f(x) &= 0 } & \Aboxed{ g(x) &= b} \\
%   \Aboxed{ h(x) }      & \Aboxed{ i(x) }   
% \end{align*}
% Note how the \cs{fboxsep} change only affect the box coming
% immediately after it.  
%
%  \subsubsection{Adding arrows between lines in an alignment}
%
%  This first macro is a bit misleading, it is only intended to be
%  used in combination with the \env{alignat(*)} environment.
%  \begin{codesyntax}
%    \SpecialEnvIndex{\ArrowBetweenLines}\cs{ArrowBetweenLines}\oarg{symbol}\\
%    \SpecialEnvIndex{\ArrowBetweenLines*}\cs{ArrowBetweenLines*}\oarg{symbol}
%  \end{codesyntax}
%    \hskip1sp
%   \marginpar{%
%    \parbox[b]{\marginparwidth}{\small\sffamily\raggedright
%      \strut Evolved from a request by\\Christian
%      Bohr-Halling\\2004/03/31\\on dk.edb.tekst%
%    }
%  }%
%  To add, say $\Updownarrow$ between two lines in an alignment use
%  \cs{ArrowBetweenLines} and the \env{alignat} environment (note the
%  extra pair of  \texttt{\&}'s in front):
%  \begin{verbatim}
%  \begin{alignat}{2}
%    && \framebox[1.5cm]{} &= \framebox[3cm]{}\\
%    \ArrowBetweenLines % \Updownarrow is the default
%    && \framebox[1.5cm]{} &= \framebox[2cm]{}
%  \end{alignat}
%  \end{verbatim}
%  resulting in
%  \begin{alignat}{2}
%    && \framebox[1.5cm]{} &= \framebox[3cm]{}\\
%    \ArrowBetweenLines 
%    && \framebox[1.5cm]{} &= \framebox[2cm]{}
%  \end{alignat}
%  Note the use of \verb+&&+ starting each \emph{regular} line of
%  math. For adding the arrow on the right, use
%  \cs{ArrowBetweenLines*}\oarg{symbol}, and end each line of math
%  with \verb+&&+.
%  \begin{verbatim}
%  \begin{alignat*}{2}
%    \framebox[1.5cm]{} &= \framebox[3cm]{}  &&\\
%    \ArrowBetweenLines*[\Downarrow] 
%    \framebox[1.5cm]{} &= \framebox[2cm]{}  &&
%  \end{alignat*}
%  \end{verbatim}
%  resulting in
%  \begin{alignat*}{2}
%    \framebox[1.5cm]{} &= \framebox[3cm]{}  &&\\
%    \ArrowBetweenLines*[\Downarrow] 
%    \framebox[1.5cm]{} &= \framebox[2cm]{}  &&
%  \end{alignat*}
%
%
% \subsubsection{Centered \cs{vdots}}
%
%  If one want to mark a vertical continuation, there is
%  the \verb?\vdots? command, but combine this with an alignment and
%  we get something rather suboptimal
% \FeatureRequest{Bruno Le Floch \\(and many others)}{2011/01/25}
%  \begin{align*}
%    \framebox[1.5cm]{} &= \framebox[3cm]{}\\
%                       & \vdots\\
%                       &= \framebox[3cm]{} 
%  \end{align*}
%  It would be nice to have (1) a \verb?\vdots? centered within the
%  width of another symbol, and (2) a construction similar to
%  \verb?\ArrowBetweenLines? that does not take up so much space. 
%  We provide both.
%  \begin{codesyntax}
%    \SpecialUsageIndex{\vdotswithin}\cs{vdotswithin}\marg{symbol}\\
%    \SpecialUsageIndex{\shortvdotswithin}\cs{shortvdotswithin}\marg{symbol}\\
%    \SpecialUsageIndex{\shortvdotswithin*}\cs{shortvdotswithin*}\marg{symbol}\\
%    \SpecialUsageIndex{\MTFlushSpaceAbove}\cs{MTFlushSpaceAbove}\\
%    \SpecialUsageIndex{\MTFlushSpaceBelow}\cs{MTFlushSpaceBelow}\\
%    \SpecialKeyIndex{shortvdotsadjustabove}\makebox{$\key{shortvdotsadjustabove}=\meta{length}$}\\
%    \SpecialKeyIndex{shortvdotsadjustbelow}\makebox{$\key{shortvdotsadjustbelow}=\meta{length}$}
%  \end{codesyntax}
%  Two examples in one
% \begin{verbatim}
% \begin{align*}
%   a &= b              \\
%     & \vdotswithin{=} \\
%     & = c             \\
%     \shortvdotswithin{=}
%     & = d
% \end{align*}
% \end{verbatim}
% yielding
% \begin{align*}
%   a &= b              \\
%     & \vdotswithin{=} \\
%     & = c             \\
%     \shortvdotswithin{=}
%     & = d
% \end{align*}
% Thus \verb?\vdotswithin{=}? create a box corersponding to
% \verb?{}={}? and typeset a >>$\vdots$<< centered inside it. When doing
% this as a normal line in an alignment leaves us with excessive space
% which \verb?\shortvdotswithin{=}? takes care with for us.
%
% \verb?\shortvdotswithin{=}? corresponds to
% \begin{verbatim}
% \MTFlushSpaceAbove
% & \vdotswithin{=} \\
% \MTFlushSpaceBelow
% \end{verbatim}
% whereas \verb?\shortvdotswithin*{=}? is the case with 
% \verb?\vdotswithin{=} & \\?. This also means one cannot write more
% on the line when using \verb?\shortvdotswithin? or the starred
% version. But one can de-construct the macro and arrive at
% \begin{verbatim}
% \begin{alignat*}{3}
%   A&+ B &&= C &&+ D \\
%   \MTFlushSpaceAbove
%   &\vdotswithin{+} &&&& \vdotswithin{+}
%   \MTFlushSpaceBelow
%   C &+ D &&= Y &&+K
% \end{alignat*}
% \end{verbatim}
% yielding
% \begin{alignat*}{3}
%   A&+ B &&= C &&+ D \\
%   \MTFlushSpaceAbove
%   &\vdotswithin{+} &&&& \vdotswithin{+}
%   \MTFlushSpaceBelow
%   C &+ D &&= Y &&+K
% \end{alignat*}
% If one has the need for such a construction.
%
% The de-spaced version does support the \env{spreadlines}
% environment. The actual amount of space being \emph{flushed} above
% and below can be controlled by the user using the two options
% indicated. Their original values are \verb?2.15\origjot? and
% \verb?\origjot? respectively (\verb?\origjot? is usually 3pt). 
%
%  \subsection{Intertext and short intertext}
%
%
%  \begin{codesyntax}
%    \SpecialUsageIndex{\shortintertext}\cs{shortintertext}\marg{text}
%  \end{codesyntax}
%  \cttPosting{Gabriel Zachmann and Donald Arseneau}{2000/05/12--13}
%  \pkg{amsmath} provides the command \cs{intertext} for interrupting
%  a multiline display while still maintaining the alignment points.
%  However the spacing often seems quite excessive as seen below.
%  \begin{verbatim}
%    \begin{align}
%      a&=b \intertext{Some text}
%      c&=d
%    \end{align}
%  \end{verbatim}
%    \begin{align}
%      a&=b \intertext{Some text}
%      c&=d
%    \end{align}
%
%  Using the command \cs{shortintertext} alleviates this situation
%  somewhat:
%  \begin{verbatim}
%    \begin{align}
%      a&=b \shortintertext{Some text}
%      c&=d
%    \end{align}
%  \end{verbatim}
%  \begin{align}
%    a&=b \shortintertext{Some text}
%    c&=d
%  \end{align}
%
%  \noindent
%  It turns out that both \cs{shortintertext} and the original
%  \cs{intertext} from \pkg{amsmath} has a slight problem. If we use
%  the \env{spreadlines} (see section~\ref{sec:spread}) to open up
%  the equations in a multiline calculation, then this opening up
%  value also applies to the spacing above and below the original
%  \cs{shortintertext} and \cs{intertext}.  \tsxPosting{Tobias Weh
%    \\(referring to a suggestion by Chung-chieh Shan)}{2011/05/29}
% It can be illustrated using the following example, an interested
% reader, can apply it with and with out the original \cs{intertext}
% and \cs{shortintertext}.
% \begin{verbatim}
% % the original \intertext and \shortintertext
% \mathtoolsset{original-intertext,original-shortintertext}
% \newcommand\myline{\par\noindent\rule{\textwidth}{1mm}} 
% \myline
% \begin{spreadlines}{1em}
%   \begin{align*}
%     AA\\  BB\\  \intertext{\myline}
%     AA\\  BB\\  \shortintertext{\myline}
%     AA\\  BB
%   \end{align*}
% \end{spreadlines}
% \myline
% \end{verbatim}
%
%  We now fix this internaly for both \cs{intertext} and
%  \cs{shortintertext}, plus we add the posibility to fine tune
%  spacing around these constructions. The original versions can be
%  brought back using the \texttt{original-x} keys below.
%  \begin{codesyntax}
%    \SpecialUsageIndex{\intertext}\cs{intertext}\marg{text}\\
%    \SpecialUsageIndex{\shortintertext}\cs{shortintertext}\marg{text}\\
%    \SpecialKeyIndex{original-intertext}$\key{original-intertext}=\texttt{true}\vert\texttt{false}$ \quad(default: \texttt{false})\\
%    \SpecialKeyIndex{original-shortintertext}$\key{original-shortintertext}=\texttt{true}\vert\texttt{false}$ 
%    \quad(default: \texttt{false})\\ 
%    \SpecialKeyIndex{above-intertext-sep}$\key{above-intertext-sep}=\meta{dimen}$ \quad(default: 0pt)\\
%    \SpecialKeyIndex{below-intertext-sep}$\key{below-intertext-sep}=\meta{dimen}$ \quad(default: 0pt)\\
%    \SpecialKeyIndex{above-shortintertext-sep}$\key{above-shortintertext-sep}=\meta{dimen}$ \quad(default: 3pt)\\
%    \SpecialKeyIndex{below-shortintertext-sep}$\key{below-shortintertext-sep}=\meta{dimen}$ \quad(default: 3pt)
%  \end{codesyntax}
%  The updated \cs{shortintertext} will look like the original version
%  unless for areas with an enlarged \cs{jot} value (see for example
%  the \env{spreadlines}, section~\ref{sec:spread}). Whereas \cs{intertext}
%  will have a slightly smaller value above and below (corresponding
%  to about 3pt less space above and below), the spacing around
%  \cs{intertext} should now match the normal spacing going into and
%  out of an \env{align}.
%
% \textbf{Tip:} \cs{intertext} and \cs{shortintertext} also works
% within \env{gather}.
%
%  \subsection{Paired delimiters}
%
%
%  \begin{codesyntax}
%    \SpecialUsageIndex{\DeclarePairedDelimiter}
%    \cs{DeclarePairedDelimiter}\marg{cmd}\marg{left_delim}\marg{right_delim}
%  \end{codesyntax}
%  \FeatureRequest{Lars Madsen}{2004/06/25}
%  In the \pkg{amsmath} documentation it is shown how to define a few
%  commands for typesetting the absolute value and norm. These
%  definitions are:
%  \begin{verbatim}
%    \newcommand*\abs[1]{\lvert#1\rvert}
%    \newcommand*\norm[1]{\lVert#1\rVert}
%  \end{verbatim}
%  \DeclarePairedDelimiter\abs\lvert\rvert
%  While they produce correct horizontal spacing you have to be
%  careful about the vertical spacing if the argument is just a
%  little taller than usual as in
%  \[
%    \abs{\frac{a}{b}}
%  \]
%  Here it won't give a nice result, so you have to manually put in
%  either \cs{left}--\cs{right} pair or a \cs{bigl}--\cs{bigr} pair.
%  Both methods mean that you have to delete your \cs{abs} command,
%  which may not sound like an ideal solution.
%
%  With the command \cs{DeclarePairedDelimiter} you can combine all
%  these features in one easy to use command. Let's show an example:
%  \begin{verbatim}
%    \DeclarePairedDelimiter\abs{\lvert}{\rvert}
%  \end{verbatim}
%  This defines the command \cs{abs} just like in the
%  \pkg{amsmath} documentation but with a few additions:
%  \begin{itemize}
%    \item A starred variant: \cs{abs*} produces delimiters that are preceded
%  by \cs{left} and \cs{right} resp.:
%  \begin{verbatim}
%  \[
%    \abs*{\frac{a}{b}}
%  \]
%  \end{verbatim}
%      \[
%        \abs*{\frac{a}{b}}
%      \]
%  \item A variant with an optional argument:
%  \cs{abs}\oarg{size_cmd}, where
%  \meta{size_cmd} is either \cs{big}, \cs{Big}, \cs{bigg}, or
%  \cs{Bigg} (if you have any bigggger versions you can use them
%  too).
%  \begin{verbatim}
%  \[
%    \abs[\Bigg]{\frac{a}{b}}
%  \]
%  \end{verbatim}
%      \[
%        \abs[\Bigg]{\frac{a}{b}}
%      \]
%  \end{itemize}
%
%  \begin{codesyntax}
%    \SpecialUsageIndex{\DeclarePairedDelimiterX}
%    \cs{DeclarePairedDelimiterX}\marg{cmd}\oarg{num args}\marg{left_delim}\marg{right_delim}\marg{code}\\
%     \cs{delimsize}
%  \end{codesyntax}
%  \ProvidedBy{Lars Madsen}{2010/06/15} 
%  Sometimes \cs{DeclarePairedDelimiter} just is not enough.  One
%  might want to have the capabilities of \cs{DeclarePairedDelimiter},
%  but also want a macro the takes more than one argument.
%
%  \cs{DeclarePairedDelimiterX} extends the features of
%  \cs{DeclarePairedDelimiter} such that the user will get a macro
%  which is fenced off at either end, plus the capability to provide
%  the code for what ever the macro should do within these fences. 
%
%  Inside the \meta{code} part, the macro \cs{delimsize} refer to the
%  size of the outer fences. It can then be used inside \meta{code} to
%  scale any inner fences.
%
%  In this setting
%  \cs{DeclarePairedDelimiter}\marg{cmd}\marg{left_delim}\marg{right_delim} is the same thing as 
%  \begin{center}
%    \cs{DeclarePairedDelimiterX}\marg{cmd}\verb|[1]|\marg{left_delim}\marg{right_delim}\verb|{#1}|
%  \end{center}
%  %
%  Let us do some examples. First we want to prepare a macro for inner
%  products, with two arguments such that we can hide the character
%  separating the arguments (a journal style might require a
%  semi-colon, so we will save a lot of hand editing). This can be
%  done via
% \begin{verbatim}
% \DeclarePairedDelimiterX\innerp[2]{\langle}{\rangle}{#1,#2}
% \end{verbatim}
% More interestingly we can refer to the size inside the
% \meta{code}. Here we do a weird three argument `braket'
% \begin{verbatim}
% \DeclarePairedDelimiterX\braket[3]{\langle}{\rangle}%
% {#1\,\delimsize\vert\,#2\,\delimsize\vert\,#3}
% \end{verbatim}
% Then we can get
% \DeclarePairedDelimiterX\innerp[2]{\langle}{\rangle}{#1,#2}
% \DeclarePairedDelimiterX\braket[3]{\langle}{\rangle}%
% {#1\,\delimsize\vert\,#2\,\delimsize\vert\,#3}
% \begin{verbatim}
% \[
% \innerp*{A}{ \frac{1}{2} } \quad
% \braket[\Big]{B}{\sum_{k} f_k}{C}
% \]
% \end{verbatim}
% \[
% \innerp*{A}{ \frac{1}{2} } \quad
% \braket[\Big]{B}{\sum_{k}}{C}
% \]
% \iffalse
% \bigskip
% 
% \noindent
% \textbf{\textit{Side note:}} We have changed the internal code of
% \cs{DeclarePairedDelimiter} and \cs{DeclarePairedDelimiterX} such
% that the starred version does no longer give the odd spacings that
% \cs{left}\dots\cs{right} sometimes give. Compare this
% \begin{verbatim}
% \[
% 2\innerp{A}{B} \quad  2\innerp*{A}{B} \quad 2\left\langle A,B \right\rangle
% \]
% \end{verbatim}
% \[
% \sin\innerp{A}{B} \quad
% \sin\innerp*{A}{B} \quad
% \sin\left\langle A,B \right\rangle
% \]
% The spacing in the last one does not behave like other fences (this
% is a feature of the \cs{left}\dots\cs{right} construction).
% \fi
%
%
%  \subsubsection{Expert use}
%
%  Within the starred version of \cs{DeclarePairedDelimiter} and
%  \cs{DeclarePairedDelimiterX} we make a few changes such that the
%  auto scaled \cs{left} and \cs{right} fences behave as opening and
%  closing fences, i.e.\ $\sin(x)$ vs. $\sin\left(x\right)$ (the later
%  made via \verb|$\sin\left(x\right)$|), notice the gab between
%  '$\sin$' and '('.  In some special cases it may be useful to be
%  able to tinker with the behavior.
%  \begin{codesyntax}
%    \SpecialUsageIndex{\reDeclarePairedDelimiterInnerWrapper}\cs{reDeclarePairedDelimiterInnerWrapper}\marg{macro name}\marg{\textnormal{\texttt{star}} or \textnormal{\texttt{nostar}}}\marg{code}
%  \end{codesyntax}
%  Internally several macros are created, including two call backs
%  that take care of the formatting. There is one internal macro for
%  the starred version, labeled \texttt{star}, the other one is
%  labeled \texttt{nostar}. Within \meta{code}, \texttt{\#1} will be
%  replaced by the scaled left fence, \texttt{\#3} the corresponding
%  scaled right fence, and \texttt{\#2} the stuff in between. For
%  example, here is how one might turn the content into \cs{mathinner}:
% \begin{verbatim}
% \DeclarePairedDelimiter\abs\lvert\rvert
% \reDeclarePairedDelimiterInnerWrapper\abs{star}{#1#2#3}
% \reDeclarePairedDelimiterInnerWrapper\abs{nostar}{\mathinner{#1#2#3}}
% \end{verbatim}
%  The default values for the call backs corresponds to
% \begin{verbatim}
% star:   \mathopen{}\mathclose\bgroup #1#2\aftergroup\egroup #3
% nostar: \mathopen{#1}#2\mathclose{#3}
% \end{verbatim}
%
%
%  \subsection{Special symbols}
%
%  This part of the manual is about special symbols. So far only one
%  technique is covered, but more will come.
%
%  \subsubsection{Left and right parentheses}
%
%  \begin{codesyntax}
%    \SpecialUsageIndex{\lparen}\cs{lparen}\texttt{~~}
%    \SpecialUsageIndex{\rparen}\cs{rparen}
%  \end{codesyntax}
%  When you want a big parenthesis or bracket in a math display you
%  usually just type
%  \begin{quote}
%    |\left( ... \right)|\quad  or\quad |\left[ ... \right]|
%  \end{quote}
%  \LaTeX{} also defines the macro names \cs{lbrack} and \cs{rbrack}
%  to be shorthands for the left and right square bracket resp., but
%  doesn't provide similar definitions for the parentheses. Some
%  packages need command names to work with\footnote{The \pkg{empheq}
%  package needs command names for delimiters in order to make
%  auto-scaling versions.} so \pkg{mathtools} defines the commands
%  \cs{lparen} and \cs{rparen} to represent the left and right
%  parenthesis resp.
%
%
%  \subsubsection{Vertically centered colon}
%
%  \begin{codesyntax}
%    \SpecialKeyIndex{centercolon}$\key{centercolon}=\texttt{true}\vert\texttt{false}$\\
%    \SpecialUsageIndex{\vcentcolon}\cs{vcentcolon}\texttt{~~}
%    \SpecialUsageIndex{\ordinarycolon}\cs{ordinarycolon}
%  \end{codesyntax}
%  \cttPosting{Donald Arseneau}{2000/12/07}
%  When trying to show assignment operations as in $ a := b $, one
%  quickly notices that the colon is not centered on the math axis as
%  the equal sign, leading to an odd-looking output. The command
%  \cs{vcentcolon} is a shorthand for such a vertically centered
%  colon, and can be used as in |$a \vcentcolon= b$| and results in
%  the desired output:  $a \vcentcolon= b$. % for now
%
%  Typing \cs{vcentcolon} every time is quite tedious, so one can use
%  the key \key{centercolon} to make the colon active instead.
%  \begin{verbatim}
%  \mathtoolsset{centercolon}
%  \[
%    a := b
%  \]
%  \mathtoolsset{centercolon=false}
%  \end{verbatim}
%  \[\mathtoolsset{centercolon}
%    a := b
%  \]
%  In this case the command \cs{ordinarycolon} typesets an~\ldots\
%  ordinary colon (what a surprise).
%
%  \medskip
%  \noindent\textbf{Warning:} \texttt{centercolon} \emph{does not}
%  work with languages that make use of an active colon, most notably
%  \emph{French}. Sadly the \texttt{babel} package does not distinguish
%  between text and math when it comes to active characters. Nor does
%  it provide any hooks to deal with math. So currently no general
%  solution exists for this problem.
%
%  \begin{codesyntax}
%    \SpecialUsageIndex{\coloneqq}\cs{coloneqq}\texttt{~~~~~}
%    \SpecialUsageIndex{\Coloneqq}\cs{Coloneqq}\texttt{~~~~~}
%    \SpecialUsageIndex{\coloneq}\cs{coloneq}\texttt{~~~}
%    \SpecialUsageIndex{\Coloneq}\cs{Coloneqq}\\
%    \SpecialUsageIndex{\eqqcolon}\cs{eqqcolon}\texttt{~~~~~}
%    \SpecialUsageIndex{\Eqqcolon}\cs{Eqqcolon}\texttt{~~~~~}
%    \SpecialUsageIndex{\eqcolon}\cs{eqcolon}\texttt{~~~}
%    \SpecialUsageIndex{\Eqcolon}\cs{Eqcolon}\\
%    \SpecialUsageIndex{\colonapprox}\cs{colonapprox}\texttt{~~}
%    \SpecialUsageIndex{\Colonapprox}\cs{Colonapprox}\texttt{~~}
%    \SpecialUsageIndex{\colonsim}\cs{colonsim}\texttt{~~}
%    \SpecialUsageIndex{\Colonsim}\cs{Colonsim}\\
%    \SpecialUsageIndex{\dblcolon}\cs{dblcolon}
%  \end{codesyntax}
%  The font packages \pkg{txfonts} and \pkg{pxfonts} provides various
%  symbols that include a vertically centered colon but with tighter
%  spacing. For example, the combination |:=| exists as the symbol
%  \cs{coloneqq} which typesets as $\coloneqq$ instead of
%  $\vcentcolon=$. The primary disadvantage of using these fonts are
%  the support packages' lack of support for \pkg{amsmath} (and thus
%  \pkg{mathtools}) and worse yet, the side-bearings are way too
%  tight; see~\cite{A-W:MG04} for examples. If you're not using these
%  fonts, \pkg{mathtools} provides the symbols for you. Here are a few
%  examples:
%  \begin{verbatim}
%  \[
%    a \coloneqq b \quad c \Colonapprox d \quad e \dblcolon f
%  \]
%  \end{verbatim}
%  \[
%    a \coloneqq b \quad c \Colonapprox d \quad e \dblcolon f
%  \]
%
%
%
%  \subsubsection{A few missing symbols}
%
%  Most provided math font sets are missing the symbols \cs{nuparrow}
%  and \cs{ndownarrow} (i.e.\ negated up- and downarrow) plus a `big'
%  version of \cs{times}. Therefore we will provide constructed
%  versions of these whenever they are not already available.
%  \begin{codesyntax}
%    \SpecialUsageIndex{\nuparrow}\cs{nuparrow}\\
%    \SpecialUsageIndex{\ndownarrow}\cs{ndownarrow}\\
%    \SpecialUsageIndex{\bigtimes}\cs{bigtimes}
%  \end{codesyntax}
%
%  \noindent
%  \textbf{Note:} that these symbols are constructed via
%  features from the \pkg{graphicx} package, and thus may not display
%  correctly in most DVI previewers. Also note that \cs{nuparrow} and
%  \cs{ndownarrow} are constructed via \cs{nrightarrow} and
%  \cs{nleftarrrow} respectively, so these needs to be
%  present. Usually this is done via \pkg{amssymb}, but some packages
%  may be incompatible with \pkg{amssymb} so the user will have to
%  load \pkg{amssymb} or a similar package, that provides
%  \cs{nrightarrow} and \cs{nleftarrow}, themselves. 
%
%  With those requirements in place, we have
%  \begin{verbatim}
%    \[
%       \lim_{a\ndownarrow 0} f(a) \neq \bigtimes_n X_n \qquad
%       \frac{ \bigtimes_{k=1}^7 B_k \nuparrow \Omega }{2}     
%    \]
%  \end{verbatim}
%    \[
%       \lim_{a\ndownarrow 0} f(a) \neq \bigtimes_n X_n \qquad
%       \frac{ \bigtimes_{k=1}^7 B_k \nuparrow \Omega }{2}     
%    \]
%
%
%
%
%  \section{A tribute to Michael J.~Downes}
%
%  Michael J.~Downes (1958--2003) was one of the major architects
%  behind \pkg{amsmath} and member of the \LaTeX{} Team. He made many
%  great contributions to the \TeX{} community; not only by the means
%  of widely spread macro packages such as \pkg{amsmath} but also in
%  the form of actively giving advice on newsgroups. Some of
%  Michael's macro solutions on the newsgroups never made it into
%  publicly available macro packages although they certainly deserved
%  it, so \pkg{mathtools} tries to rectify this matter. The macros
%  described in this section are either straight copies or heavily
%  inspired by his original postings.
%
%  \subsection{Mathematics within italic text}
%
%  \begin{codesyntax}
%    \SpecialKeyIndex{mathic}$\key{mathic}=\texttt{true}\vert\texttt{false}$
%  \end{codesyntax}
%  \cttPosting{Michael J.~Downes}{1998/05/14}
%  \TeX{} usually takes care of italic corrections in text, but fails
%  when it comes to math. If you use the \LaTeX{} inline math
%  commands \cs{(} and \cs{)} you can however work around it by
%  setting the key \key{mathic} to true as shown below.
%  \begin{verbatim}
%    \begin{quote}\itshape
%    Compare these lines: \par
%    \mathtoolsset{mathic} % or \mathtoolsset{mathic=true}
%    Subset of \(V\) and subset of \(A\). \par
%    \mathtoolsset{mathic=false}
%    Subset of \(V\) and subset of \(A\).
%    \par
%    \end{quote}
%  \end{verbatim}
%  \begin{quote}\itshape
%  Compare these lines: \par
%  \mathtoolsset{mathic}
%  Subset of \(V\) and subset of \(A\). \par
%  \mathtoolsset{mathic=false}
%  Subset of \(V\) and subset of \(A\).
%  \par
%  \end{quote}
%
%  \noindent
%  It is recommended to load the \pkg{fixltx2e} package \emph{after}
%  \pkg{mathtools} and the \verb|\mathtoolsset{mathic=true}| option,
%  as it will make \verb|\(...\)| robust, while maintaining the added
%  italic correction.
%
%  \subsection{Left sub/superscripts}
%
%  \begin{codesyntax}
%    \SpecialUsageIndex{\prescript}
%        \cs{prescript}\marg{sup}\marg{sub}\marg{arg}\texttt{~~}
%    \SpecialKeyIndex{prescript-sup-format}
%        $\key{prescript-sup-format}=\meta{cmd}$\\
%    \SpecialKeyIndex{prescript-sub-format}
%        $\key{prescript-sub-format}=\meta{cmd}$\hfill
%    \SpecialKeyIndex{prescript-arg-format}
%        \rlap{$\key{prescript-arg-format}=\meta{cmd}$}^^A
%        \phantom{$\key{prescript-sup-format}=\meta{cmd}$}
%  \end{codesyntax}
%  \cttPosting{Michael J.~Downes}{2000/12/20}
%  Sometimes one wants to put a sub- or superscript on the left of
%  the argument. The \cs{prescript} command does just that:
%  \begin{verbatim}
%    \[
%      {}^{4}_{12}\mathbf{C}^{5+}_{2}          \quad
%      \prescript{14}{2}{\mathbf{C}}^{5+}_{2}  \quad
%      \prescript{4}{12}{\mathbf{C}}^{5+}_{2}  \quad
%      \prescript{14}{}{\mathbf{C}}^{5+}_{2}   \quad
%      \prescript{}{2}{\mathbf{C}}^{5+}_{2}
%    \]
%  \end{verbatim}
%  \[
%    {}^{4}_{12}\mathbf{C}^{5+}_{2}          \quad
%    \prescript{14}{2}{\mathbf{C}}^{5+}_{2}  \quad
%    \prescript{4}{12}{\mathbf{C}}^{5+}_{2}  \quad
%    \prescript{14}{}{\mathbf{C}}^{5+}_{2}   \quad
%    \prescript{}{2}{\mathbf{C}}^{5+}_{2}
%  \]
%
%  The formatting of the arguments is controlled by three keys. This
%  silly example shows you how to use them:
%  \begin{verbatim}
%  \newcommand*\myisotope[3]{%
%    \begingroup % to keep changes local. We cannot use a brace group
%                % as it affects spacing!
%      \mathtoolsset{
%        prescript-sup-format=\mathit,
%        prescript-sub-format=\mathbf,
%        prescript-arg-format=\mathrm,
%      }%
%    \prescript{#1}{#2}{#3}%
%    \endgroup
%  }
%  \[
%    \myisotope{A}{Z}{X}\to \myisotope{A-4}{Z-2}{Y}+
%    \myisotope{4}{2}{\alpha}
%  \]
%  \end{verbatim}
%  \newcommand*\myisotope[3]{%
%    \begingroup
%      \mathtoolsset{
%        prescript-sup-format=\mathit,
%        prescript-sub-format=\mathbf,
%        prescript-arg-format=\mathrm,
%      }%
%    \prescript{#1}{#2}{#3}%
%    \endgroup
%  }
%  \[
%    \myisotope{A}{Z}{X}\to \myisotope{A-4}{Z-2}{Y}+
%    \myisotope{4}{2}{\alpha}
%  \]
% (Though a package like \pkg{mhchem} might be more suitable for this
% type of material.)
%
%  \subsection{Declaring math sizes}
%
%  \begin{codesyntax}
%    \SpecialUsageIndex{\DeclareMathSizes}
%    \cs{DeclareMathSizes}\marg{dimen}\marg{dimen}\marg{dimen}\marg{dimen}
%  \end{codesyntax}
%  \cttPosting{Michael J.~Downes}{2002/10/17}
%  If you don't know about \cs{DeclareMathSizes}, then skip the rest
%  of this text. If you do know, then all that is needed to say is
%  that with \pkg{mathtools} it is patched so that all regular
%  dimension suffixes are now valid in the last three arguments. Thus
%  a declaration such as
%  \begin{verbatim}
%    \DeclareMathSize{9.5dd}{9.5dd}{7.5dd}{6.5dd}
%  \end{verbatim}
%  will now work (it doesn't in standard \LaTeX). When this bug has
%  been fixed in \LaTeX, this fix will be removed from
%  \pkg{mathtools}.
%
%  \subsection{Spreading equations}\label{sec:spread}
%
%  \begin{codesyntax}
%    \SpecialEnvIndex{spreadlines}
%    \cs{begin}\arg{spreadlines}\marg{dimen} \meta{contents}
%    \cs{end}\arg{spreadlines}
%  \end{codesyntax}
%  \cttPosting{Michael J.~Downes}{2002/10/17}
%  The spacing between lines in a multiline math environment such as
%  \env{gather} is governed by the dimension \cs{jot}. The
%  \env{spreadlines} environment takes one argument denoting the
%  value of \cs{jot} inside the environment:
%  \begin{verbatim}
%    \begin{spreadlines}{20pt}
%    Large spaces between the lines.
%    \begin{gather}
%      a=b\\
%      c=d
%    \end{gather}
%    \end{spreadlines}
%    Back to normal spacing.
%    \begin{gather}
%      a=b\\
%      c=d
%    \end{gather}
%  \end{verbatim}
%    \begin{spreadlines}{20pt}
%    Large spaces between the lines.
%    \begin{gather}
%      a=b\\
%      c=d
%    \end{gather}
%    \end{spreadlines}
%    Back to normal spacing.
%    \begin{gather}
%      a=b\\
%      c=d
%    \end{gather}
%
%
%  \subsection{Gathered environments}\label{subsec:gathered}
%
%  \begin{codesyntax}
%    \SpecialEnvIndex{lgathered}\cs{begin}\arg{lgathered}\oarg{pos}
%    \meta{contents}  \cs{end}\arg{lgathered} \\
%    \SpecialEnvIndex{rgathered}\cs{begin}\arg{rgathered}\oarg{pos}
%    \meta{contents}  \cs{end}\arg{rgathered} \\
%    \SpecialUsageIndex{\newgathered}\cs{newgathered}\marg{name}\marg{pre_line}\marg{post_line}\marg{after}\\
%    \SpecialUsageIndex{\renewgathered}\cs{renewgathered}\marg{name}\marg{pre_line}\marg{post_line}\marg{after}
%  \end{codesyntax}
%  \cttPosting{Michael J.~Downes}{2001/01/17}
%  In a document set in \opt{fleqn}, you might sometimes want an
%  inner \env{gathered} environment that doesn't center its lines but
%  puts them flush left. The \env{lgathered} environment works just
%  like the standard \env{gathered} except that it flushes its
%  contents left:
%  \begin{verbatim}
%    \begin{equation}
%      \begin{lgathered}
%        x=1,\quad x+1=2 \\
%        y=2
%      \end{lgathered}
%    \end{equation}
%  \end{verbatim}
%  \begin{equation}
%    \begin{lgathered}
%        x=1,\quad x+1=2 \\
%        y=2
%    \end{lgathered}
%  \end{equation}
%  Similarly the \env{rgathered} puts it contents flush right.
%
%  More interesting is probably the command \cs{newgathered}. In this
%  example we define a gathered version that centers the lines and
%  also prints a star and a number at the left of each line.
%  \begin{verbatim}
%    \newcounter{steplinecnt}
%    \newcommand\stepline{\stepcounter{steplinecnt}\thesteplinecnt}
%    \newgathered{stargathered}
%                {\llap{\stepline}$*$\quad\hfil}% \hfil for centering
%                {\hfil}%                         \hfil for centering
%                {\setcounter{steplinecnt}{0}}%   reset counter
%  \end{verbatim}
%  \newcounter{steplinecnt}
%  \newcommand\stepline{\stepcounter{steplinecnt}\thesteplinecnt}
%  \newgathered{stargathered}{\llap{\stepline}$*$\quad\hfil}{\hfil}{\setcounter{steplinecnt}{0}}
%  With these definitions we can get something like this:
%  \begin{verbatim}
%    \begin{gather}
%      \begin{stargathered}
%        x=1,\quad x+1=2 \\
%        y=2
%      \end{stargathered}
%    \end{gather}
%  \end{verbatim}
%  \begin{gather}
%    \begin{stargathered}
%      x=1,\quad x+1=2 \\
%      y=2
%    \end{stargathered}
%  \end{gather}
%  \cs{renewgathered} renews a gathered environment of course.
%
%  In all fairness it should be stated that the original concept by
%  Michael has been extended quite a bit in \pkg{mathtools}. Only the
%  end product of \env{lgathered} is the same.
%
%  \subsection{Split fractions}
%
%  \begin{codesyntax}
%    \SpecialUsageIndex{\splitfrac}\cs{splitfrac}\marg{numer}\marg{denom}\texttt{~~}
%    \SpecialUsageIndex{\splitdfrac}\cs{splitdfrac}\marg{numer}\marg{denom}
%  \end{codesyntax}
%  \cttPosting{Michael J.~Downes}{2001/12/06}
%  These commands provide split fractions e.g., multiline fractions:
%  \begin{verbatim}
%    \[
%      a=\frac{
%          \splitfrac{xy + xy + xy + xy + xy}
%                    {+ xy + xy + xy + xy}
%        }
%        {z}
%      =\frac{
%          \splitdfrac{xy + xy + xy + xy + xy}
%                    {+ xy + xy + xy + xy}
%        }
%        {z}
%  \]
%  \end{verbatim}
%  \[
%  a=\frac{
%      \splitfrac{xy + xy + xy + xy + xy}
%                {+ xy + xy + xy + xy}
%    }
%    {z}
%  =\frac{
%      \splitdfrac{xy + xy + xy + xy + xy}
%                {+ xy + xy + xy + xy}
%    }
%    {z}
%  \]
%
%
%
%
%
%
%
%
%
%
%
%
%
%
%
%
%
%
%
%
%
%
%
%
%
%
%
%
%
%
%
%
%
%
%
%
%
%
%
%
%
%
%
%
%
%
%
%
%
%
%
%
%
%
%
%
%
%
%
%
%
%
%
%
%
%
%
%
%
%
%
%
%
%
%
%
%
%
%
%
%
%
%
%
%
% \begin{thebibliography}{9}
%  \bibitem{Perlis01}
%    Alexander R. Perlis,
%    \emph{A complement to \cs{smash}, \cs{llap}, and \cs{rlap}},
%    TUGboat 22(4) (2001).
%  \bibitem{Ams99}
%    American Mathematical Society and Michael Downes,
%    \emph{Technical notes on the \pkg{amsmath} package} Version 2.0,
%    1999/10/29.
%    (Available from CTAN as file \texttt{technote.tex}.)
%  \bibitem{Ams00}
%    Frank Mittelbach, Rainer Sch\"opf, Michael Downes, and David M.~Jones,
%    \emph{The \pkg{amsmath} package} Version 2.13,
%    2000/07/18.
%    (Available from CTAN as file \texttt{amsmath.dtx}.)
%  \bibitem{A-W:MG04}
%    Frank Mittelbach and Michel Goossens.
%     \emph{The {\LaTeX} Companion}.
%    Tools and Techniques for Computer Typesetting. Addison-Wesley,
%    Boston, Massachusetts, 2 edition, 2004.
%    With Johannes Braams, David Carlisle, and Chris Rowley.
%
% \bibitem{Carl99}
%   David Carlisle,
%   \emph{The \pkg{keyval} Package},
%   Version 1.13, 1999/03/16.
%   (Available from CTAN as file \texttt{keyval.dtx}.)
%
%  \bibitem{Voss:2004}
%    Herbert Vo\ss,
%    \emph{Math mode}, Version 1.71,
%    2004/07/06.
%    (Available from CTAN as file \texttt{Voss-Mathmode.pdf}.)
%  
%   \bibitem{Swanson} 
%     Ellen Swanson,
%     \emph{Mathematics into type}.
%     American Mathematical Society, updated edition, 1999.
%     Updated by Arlene O'Sean and Antoinette Schleyer.
%  \end{thebibliography}
%
%
%  \StopEventually{}  
%
%
%  \section{Options and package loading}
%
%
%  Lets start the package.
%    \begin{macrocode}
%<*package>
\ProvidesPackage{mathtools}%
  [2012/04/24 v1.12 mathematical typesetting tools]
%    \end{macrocode}
% \changes{v1.10}{2011/02/12}{Might as well make sure that we need the
% latest version of \texttt{mhsetup}}
%    \begin{macrocode}
\RequirePackage{keyval,calc}
\RequirePackage{mhsetup}[2010/01/21]
\MHInternalSyntaxOn
%    \end{macrocode}
%
%  \begin{macro}{\MT_options_name:}
%  \begin{macro}{\mathtoolsset}
%  The name for the options and a user interface for setting keys.
%    \begin{macrocode}
\def\MT_options_name:{mathtools}
\newcommand*\mathtoolsset[1]{\setkeys{\MT_options_name:}{#1}}
%    \end{macrocode}
%  \end{macro}
%  \end{macro}
%
%  Fix \pkg{amsmath} bugs (strongly recommended!). It requires a
%  great deal of typing to avoid fixing the bugs. He he.
%    \begin{macrocode}
\MH_new_boolean:n {fixamsmath}
\DeclareOption{fixamsmath}{
  \MH_set_boolean_T:n {fixamsmath}
}
\DeclareOption{donotfixamsmathbugs}{
  \MH_set_boolean_F:n {fixamsmath}
}
%    \end{macrocode}
%  Disallow spaces before optional arguments in certain \pkg{amsmath}
%  building blocks.
%    \begin{macrocode}
\DeclareOption{allowspaces}{
  \MH_let:NwN \MaybeMHPrecedingSpacesOff
              \relax
    \MH_let:NwN \MH_maybe_nospace_ifnextchar:Nnn \kernel@ifnextchar
}
\DeclareOption{disallowspaces}{
  \MH_let:NwN \MaybeMHPrecedingSpacesOff
              \MHPrecedingSpacesOff
  \MH_let:NwN \MH_maybe_nospace_ifnextchar:Nnn \MH_nospace_ifnextchar:Nnn
}
%    \end{macrocode}
%  Pass all other options directly to \pkg{amsmath}.
%    \begin{macrocode}
\DeclareOption*{
  \PassOptionsToPackage{\CurrentOption}{amsmath}
}
\ExecuteOptions{fixamsmath,disallowspaces}
\ProcessOptions\relax
%    \end{macrocode}
%  We have to turn off the new syntax when \pkg{amstext} is loaded.
%    \begin{macrocode}
\MHInternalSyntaxOff
\RequirePackage{amsmath}[2000/07/18]
\MHInternalSyntaxOn
\AtEndOfPackage{\MHInternalSyntaxOff}
%    \end{macrocode}
%  \begin{macro}{\MT_true_false_error:}
%  Make sure the user selects either `true' or `false' when asked too.
%    \begin{macrocode}
\def\MT_true_false_error:{
  \PackageError{mathtools}
    {You~ have~ to~ select~ either~ `true'~ or~ `false'}
    {I'll~ assume~ you~ chose~ `false'~ for~ now.}
}
%    \end{macrocode}
%  \end{macro}
%
%  \section{Macros I got ideas for myself}
%
%
%
%  \subsection{Tag forms}
%  This is quite simple, but why isn't it then a part of some widely
%  distributed package? Beats me.
%
%  \begin{macro}{\MT_define_tagform:nwnn}
%  We start out by defining a command that will allow us to define
%  commands similar to \cs{tagform@} only this will give us tag form
%  \emph{types}. The actual code is very similar to the one in
%  \pkg{amsmath}.
%    \begin{macrocode}
\def\MT_define_tagform:nwnn #1[#2]#3#4{
  \@namedef{MT_tagform_#1:n}##1
    {\maketag@@@{#3\ignorespaces#2{##1}\unskip\@@italiccorr#4}}
}
%    \end{macrocode}
%  \end{macro}
%
%  \begin{macro}{\newtagform}
%  Similar to \cs{newcommand}. Check if defined and scan for presence
%  of optional argument. Then call generic command.
%    \begin{macrocode}
\providecommand*\newtagform[1]{%
  \@ifundefined{MT_tagform_#1:n}
  {\@ifnextchar[%
    {\MT_define_tagform:nwnn #1}%
    {\MT_define_tagform:nwnn #1[]}%
  }{\PackageError{mathtools}
  {The~ tag~ form~ `#1'~ is~ already~ defined\MessageBreak
  You~ probably~ want~ to~ look~ up~ \@backslashchar renewtagform~
  instead}
  {I~ will~ just~ ignore~ your~ wish~ for~ now.}}
}
%    \end{macrocode}
%  Provide a default tag form which---surprise, surprise---is
%  identical to the standard definition.
%    \begin{macrocode}
\newtagform{default}{(}{)}
%    \end{macrocode}
%  \end{macro}
%  \begin{macro}{\renewtagform}
%  Similar to \cs{renewcommand}.
%    \begin{macrocode}
\providecommand*\renewtagform[1]{%
  \@ifundefined{MT_tagform_#1:n}
  {\PackageError{mathtools}
  {The~ tag~ form~ `#1'~ is~ not~ defined\MessageBreak
  You~ probably~ want~ to~ look~ up~ \@backslashchar newtagform~ instead}
  {I~ will~ just~ ignore~ your~ wish~ for~ now.}}
  {\@ifnextchar[%
    {\MT_define_tagform:nwnn #1}%
    {\MT_define_tagform:nwnn #1[]}%
  }
}
%    \end{macrocode}
%  \end{macro}
%  \begin{macro}{\usetagform}
%  Then the activator. Test if the tag form is defined and then
%  activate it by redefining \cs{tagform@}.
%    \begin{macrocode}
\providecommand*\usetagform[1]{%
  \@ifundefined{MT_tagform_#1:n}
    {
      \PackageError{mathtools}{%
        You~ have~ chosen~ the~ tag~ form~ `#1'\MessageBreak
        but~ it~ appears~ to~ be~ undefined}
        {I~ will~ use~ the~ default~ tag~ form~ instead.}%
        \@namedef{tagform@}{\@nameuse{MT_tagform_default:n}}
      }
  { \@namedef{tagform@}{\@nameuse{MT_tagform_#1:n}} }
%    \end{macrocode}
%  Here we patch if we're using the special ``show only referenced
%  equations'' feature.
%    \begin{macrocode}
  \MH_if_boolean:nT {show_only_refs}{
    \MH_let:NwN \MT_prev_tagform:n \tagform@
    \def\tagform@##1{\MT_extended_tagform:n {##1}}
  }
}
%    \end{macrocode}
%  \end{macro}
%
%  \subsubsection{Showing only referenced tags}
%  A little more interesting is the way to print only the equation
%  numbers that are actually referenced.
%
%  A few booleans to help determine which situations we're in.
%    \begin{macrocode}
\MH_new_boolean:n {manual_tag}
\MH_new_boolean:n {raw_maketag}
%    \end{macrocode}
%  \begin{macro}{\MT_AmS_tag_in_align:}
%  \begin{macro}{\tag@in@align}
%  \begin{macro}{\tag@in@display}
%  We'll need to know when the user has put in a manual tag, and since
%  \cs{tag} is \cs{let} to all sorts of things inside the \pkg{amsmath}
%  code it is safer to provide a small hack to the functions it is copied
%  from. Note that we can't use \cs{iftag@}.
%    \begin{macrocode}
\MH_let:NwN \MT_AmS_tag_in_align: \tag@in@align
\def\tag@in@align{
  \global\MH_set_boolean_T:n {manual_tag}
  \MT_AmS_tag_in_align:
}
\def\tag@in@display#1#{
  \relax
  \global\MH_set_boolean_T:n {manual_tag}
  \tag@in@display@a{#1}
}
%    \end{macrocode}
%  \end{macro}
%  \end{macro}
%  \end{macro}
%
%  \begin{macro}{\MT_extended_tagform:n}
%  \changes{v1.01}{2004/08/03}{Simplified quite a bit}
%  The extended version of \cs{tagform@}.
%    \begin{macrocode}
\def\MT_extended_tagform:n #1{
  \MH_set_boolean_F:n {raw_maketag}
%    \end{macrocode}
% We test if the equation was labelled. We already know if it was
% tagged manually. Have to watch out for \TeX\ inserting a blank line
% so do not let the tag have width zero. Rememeber
% \cs{@safe@activestrue/false} in order to handle active chars in labels.
% \changes{v1.12}{2012/04/24}{Added \cs{@safe@activestrue/false}}
%    \begin{macrocode}
  \if_meaning:NN \df@label\@empty
    \MH_if_boolean:nTF {manual_tag}% this was \MH_if_boolean:nT before
    { \MH_if_boolean:nTF {show_manual_tags}
      { \MT_prev_tagform:n {#1} }
      { \stepcounter{equation}  }
    }{\kern1sp}% this last {\kern1sp} is new.
  \else:
    \MH_if_boolean:nTF {manual_tag}
      { \MH_if_boolean:nTF {show_manual_tags}
          { \MT_prev_tagform:n {#1} }
          { \@safe@activestrue
            \@ifundefined{MT_r_\df@label}
%    \end{macrocode}
% Next we need to remember to deactivate the manual tags switch. This
% is usually done using \verb|\MT_extended_maketag:n|, but this is not
% the case if the show manual tags is false and the manual tag is not
% referred to. 
% \changes{v1.12}{2011/06/08}{Added the falsification of manual tag
% when show manual tags is off and maual tag is not referred to}
%    \begin{macrocode}
              { \global\MH_set_boolean_F:n {manual_tag} }
              { \MT_prev_tagform:n {#1} }
              \@safe@activesfalse
          }
      }
      { 
        \@safe@activestrue
        \@ifundefined{MT_r_\df@label}
          { }
          { \refstepcounter{equation}\MT_prev_tagform:n {#1} }
        \@safe@activesfalse
      }
  \fi:
  \global\MH_set_boolean_T:n {raw_maketag}
}
%    \end{macrocode}
%  \end{macro}
%  \begin{macro}{\MT_extended_maketag:n}
%  The extended version of \cs{maketag@@@}.
% \changes{v1.12}{2012/04/24}{Added \cs{@safe@activestrue/false}}
%    \begin{macrocode}
\def\MT_extended_maketag:n #1{
  \ifx\df@label\@empty
    \MT_maketag:n {#1}
  \else:
    \MH_if_boolean:nTF {raw_maketag}
      {
        \MH_if_boolean:nTF {show_manual_tags}
          { \MT_maketag:n {#1} }
          { \@safe@activestrue
            \@ifundefined{MT_r_\df@label}
              { }
              { \MT_maketag:n {#1}     }
            \@safe@activesfalse
          }
      }
      { \MT_maketag:n {#1} }
  \fi:
%    \end{macrocode}
%  As this function is always called we let it set the marker for a manual
%  tag false when exiting (well actually not true, see above).
%    \begin{macrocode}
  \global\MH_set_boolean_F:n {manual_tag}
}
%    \end{macrocode}
%  \end{macro}
%  \begin{macro}{\MT_extended_eqref:n}
%  \changes{v1.01}{2004/08/03}{Make it robust}
%  We let \cs{eqref} write the label to the \file{aux} file, which is
%  read at the beginning of the next run. Then we print the equation
%  number as usual.
%    \begin{macrocode}
\def\MT_extended_eqref:n #1{
  \protected@write\@auxout{}
    {\string\MT@newlabel{#1}}
  \textup{\MT_prev_tagform:n {\ref{#1}}}
}
%    \end{macrocode}
%  \end{macro}
%
%  \begin{macro}{\refeq}
%  \begin{macro}{\MT_extended_refeq:n}
%  Similar to \cs{eqref} and \cs{MT_extended_eqref:n}.
%    \begin{macrocode}
\newcommand*\refeq[1]{
  \textup{\ref{#1}}
}
\def\MT_extended_refeq:n #1{
  \protected@write\@auxout{}
    {\string\MT@newlabel{#1}}
  \textup{\ref{#1}}
}
%    \end{macrocode}
%  \end{macro}
%  \end{macro}
%
%  \begin{macro}{\MT@newlabel}
%  We can't use |:| or |_| in the command name (yet). We define the
%  special labels for the equations that have been referenced in the
%  previous run.
%    \begin{macrocode}
\newcommand*\MT@newlabel[1]{  \global\@namedef{MT_r_#1}{}  }
%    \end{macrocode}
%  \end{macro}
%    \begin{macrocode}
\MH_new_boolean:n {show_only_refs}
\MH_new_boolean:n {show_manual_tags}
\define@key{\MT_options_name:}{showmanualtags}[true]{
  \@ifundefined{boolean_show_manual_tags_#1:}
    { \MT_true_false_error:
      \@nameuse{boolean_show_manual_tags_false:}
    }
    { \@nameuse{boolean_show_manual_tags_#1:} }
}
%    \end{macrocode}
%  \begin{macro}{\MT_showonlyrefs_true:}
%  The implementation is based on the idea that \cs{tagform@} can be
%  called in two circumstances: when the tag is being printed in the
%  equation and when it is being printed during a reference.
%    \begin{macrocode}
\newcommand*\MT_showonlyrefs_true:{
  \MH_if_boolean:nF {show_only_refs}{
    \MH_set_boolean_T:n {show_only_refs}
%    \end{macrocode}
%  Save the definitions of the original commands.
%    \begin{macrocode}
    \MH_let:NwN \MT_incr_eqnum: \incr@eqnum
    \MH_let:NwN \incr@eqnum \@empty
    \MH_let:NwN \MT_array_parbox_restore: \@arrayparboxrestore
    \@xp\def\@xp\@arrayparboxrestore\@xp{\@arrayparboxrestore
      \MH_let:NwN \incr@eqnum \@empty
    }
    \MH_let:NwN \MT_prev_tagform:n \tagform@
    \MH_let:NwN \MT_eqref:n \eqref
    \MH_let:NwN \MT_refeq:n \refeq
    \MH_let:NwN \MT_maketag:n \maketag@@@
    \MH_let:NwN \maketag@@@ \MT_extended_maketag:n
%    \end{macrocode}
%  We redefine \cs{tagform@}.
%    \begin{macrocode}
    \def\tagform@##1{\MT_extended_tagform:n {##1}}
%    \end{macrocode}
%  Then \cs{eqref}:
%    \begin{macrocode}
    \MH_let:NwN \eqref \MT_extended_eqref:n
    \MH_let:NwN \refeq \MT_extended_refeq:n
  }
}
%    \end{macrocode}
%  \end{macro}
%  \begin{macro}{\MT_showonlyrefs_false:}
%  This macro reverts the settings.
%    \begin{macrocode}
\def\MT_showonlyrefs_false: {
  \MH_if_boolean:nT {show_only_refs}{
    \MH_set_boolean_F:n {show_only_refs}
    \MH_let:NwN \tagform@  \MT_prev_tagform:n
    \MH_let:NwN \eqref \MT_eqref:n
    \MH_let:NwN \refeq \MT_refeq:n
    \MH_let:NwN \maketag@@@ \MT_maketag:n
    \MH_let:NwN \incr@eqnum \MT_incr_eqnum:
    \MH_let:NwN \@arrayparboxrestore \MT_array_parbox_restore:
  }
}
\define@key{\MT_options_name:}{showonlyrefs}[true]{
  \@nameuse{MT_showonlyrefs_#1:}
}
%    \end{macrocode}
%  \end{macro}
%
%
%  \begin{macro}{\nonumber}
%  \changes{v1.01}{2004/08/03}{Fixed using \cs{notag} or \cs{nonumber}
%  with the \key{showonlyrefs} feature}
%  We have to redefine \cs{nonumber} else it will subtract one from the
%  equation number where we don't want it. This is probably not needed
%  since \cs{nonumber} is unnecessary when \key{showonlyrefs} is in
%  effect, but now you can use it with old documents as well.
%    \begin{macrocode}
\renewcommand\nonumber{
  \if@eqnsw
    \if_meaning:NN \incr@eqnum\@empty
%    \end{macrocode}
%  Only subtract the number if |show_only_refs| is false.
%    \begin{macrocode}
      \MH_if_boolean:nF {show_only_refs}
        {\addtocounter{equation}\m@ne}
    \fi:
  \fi:
  \MH_let:NwN \print@eqnum\@empty \MH_let:NwN \incr@eqnum\@empty
  \global\@eqnswfalse
}
%    \end{macrocode}
%  \end{macro}
%
%  \begin{macro}{\noeqref}
%   \changes{v1.04}{2008/03/26}{Added \cs{noeqref} (daleif)}
%   \changes{v1.12}{2012/04/20}{Labels containing active chars (babel)
%   are now allowed}
%   \changes{v1.12}{2012/04/24}{\cs{noeqref} will now make a reference
%   warning if users use undefined labels in \cs{noeqref}, requested
%   by Tue Christensen}
%   Macro for adding numbers to non-referred equations. Syntax similar
%   to \cs{nocite}.
%    \begin{macrocode}
\MHInternalSyntaxOff
\newcommand\noeqref[1]{\@bsphack
  \@for\@tempa:=#1\do{%
    \@safe@activestrue%
    \edef\@tempa{\expandafter\@firstofone\@tempa}%
    \@ifundefined{r@\@tempa}{%
      \protect\G@refundefinedtrue%
      \@latex@warning{Reference `\@tempa' on page \thepage \space
        undefined (\string\noeqref)}%
    }{}%
    \if@filesw\protected@write\@auxout{}%
    {\string\MT@newlabel{\@tempa}}\fi%
  \@safe@activesfalse}
  \@esphack}

%    \end{macrocode}
%  \end{macro}
%    
% \begin{macro}{\@safe@activestrue}
% \begin{macro}{\@safe@activesfalse}
%   These macros are provided by babel. We \emph{provide} them here,
%   just to make sure they exist.
%    \begin{macrocode}
\providecommand\@safe@activestrue{}%
\providecommand\@safe@activesfalse{}%

\MHInternalSyntaxOn
%    \end{macrocode}
%   
% \end{macro}
% \end{macro}
%
%  \subsection{Extensible arrows etc.}
%
%  \begin{macro}{\xleftrightarrow}
%  \begin{macro}{\MT_leftrightarrow_fill:}
%  \begin{macro}{\xLeftarrow}
%  \begin{macro}{\xRightarrow}
%  \begin{macro}{\xLeftrightarrow}
%
%  These are straight adaptions from \pkg{amsmath}.
%    \begin{macrocode}
\providecommand*\xleftrightarrow[2][]{%
  \ext@arrow 3095\MT_leftrightarrow_fill:{#1}{#2}}
\def\MT_leftrightarrow_fill:{%
  \arrowfill@\leftarrow\relbar\rightarrow}
\providecommand*\xLeftarrow[2][]{%
  \ext@arrow 0055{\Leftarrowfill@}{#1}{#2}}
\providecommand*\xRightarrow[2][]{%
  \ext@arrow 0055{\Rightarrowfill@}{#1}{#2}}
\providecommand*\xLeftrightarrow[2][]{%
  \ext@arrow 0055{\Leftrightarrowfill@}{#1}{#2}}
%    \end{macrocode}
%  \end{macro}
%  \end{macro}
%  \end{macro}
%  \end{macro}
%  \end{macro}
%  \begin{macro}{\MT_rightharpoondown_fill:}
%  \begin{macro}{\MT_rightharpoonup_fill:}
%  \begin{macro}{\MT_leftharpoondown_fill:}
%  \begin{macro}{\MT_leftharpoonup_fill:}
%  \begin{macro}{\xrightharpoondown}
%  \begin{macro}{\xrightharpoonup}
%  \begin{macro}{\xleftharpoondown}
%  \begin{macro}{\xleftharpoonup}
%  \begin{macro}{\xleftrightharpoons}
%  \begin{macro}{\xrightleftharpoons}
%  The harpoons.
%    \begin{macrocode}
\def\MT_rightharpoondown_fill:{%
  \arrowfill@\relbar\relbar\rightharpoondown}
\def\MT_rightharpoonup_fill:{%
  \arrowfill@\relbar\relbar\rightharpoonup}
\def\MT_leftharpoondown_fill:{%
  \arrowfill@\leftharpoondown\relbar\relbar}
\def\MT_leftharpoonup_fill:{%
  \arrowfill@\leftharpoonup\relbar\relbar}
\providecommand*\xrightharpoondown[2][]{%
  \ext@arrow 0359\MT_rightharpoondown_fill:{#1}{#2}}
\providecommand*\xrightharpoonup[2][]{%
  \ext@arrow 0359\MT_rightharpoonup_fill:{#1}{#2}}
\providecommand*\xleftharpoondown[2][]{%
  \ext@arrow 3095\MT_leftharpoondown_fill:{#1}{#2}}
\providecommand*\xleftharpoonup[2][]{%
  \ext@arrow 3095\MT_leftharpoonup_fill:{#1}{#2}}
\providecommand*\xleftrightharpoons[2][]{\mathrel{%
  \raise.22ex\hbox{%
    $\ext@arrow 3095\MT_leftharpoonup_fill:{\phantom{#1}}{#2}$}%
  \setbox0=\hbox{%
    $\ext@arrow 0359\MT_rightharpoondown_fill:{#1}{\phantom{#2}}$}%
  \kern-\wd0 \lower.22ex\box0}}
\providecommand*\xrightleftharpoons[2][]{\mathrel{%
  \raise.22ex\hbox{%
    $\ext@arrow 0359\MT_rightharpoonup_fill:{\phantom{#1}}{#2}$}%
  \setbox0=\hbox{%
    $\ext@arrow 3095\MT_leftharpoondown_fill:{#1}{\phantom{#2}}$}%
  \kern-\wd0 \lower.22ex\box0}}
%    \end{macrocode}
%  \end{macro}
%  \end{macro}
%  \end{macro}
%  \end{macro}
%  \end{macro}
%  \end{macro}
%  \end{macro}
%  \end{macro}
%  \end{macro}
%  \end{macro}
%  \begin{macro}{\xhookleftarrow}
%  \begin{macro}{\xhookrightarrow}
%  \begin{macro}{\MT_hookright_fill:}
%  The hooks.
%    \begin{macrocode}
\providecommand*\xhookleftarrow[2][]{%
  \ext@arrow 3095\MT_hookleft_fill:{#1}{#2}}
\def\MT_hookleft_fill:{%
  \arrowfill@\leftarrow\relbar{\relbar\joinrel\rhook}}
\providecommand*\xhookrightarrow[2][]{%
  \ext@arrow 3095\MT_hookright_fill:{#1}{#2}}
\def\MT_hookright_fill:{%
  \arrowfill@{\lhook\joinrel\relbar}\relbar\rightarrow}
%    \end{macrocode}
%  \end{macro}
%  \end{macro}
%  \end{macro}
%  \begin{macro}{\xmapsto}
%  \begin{macro}{\MT_mapsto_fill:}
%  The maps-to arrow.
%    \begin{macrocode}
\providecommand*\xmapsto[2][]{%
  \ext@arrow 0395\MT_mapsto_fill:{#1}{#2}}
\def\MT_mapsto_fill:{%
  \arrowfill@{\mapstochar\relbar}\relbar\rightarrow}
%    \end{macrocode}
%  \end{macro}
%  \end{macro}
%  \subsection{Underbrackets etc.}
%  \begin{macro}{\underbracket}
%  \begin{macro}{\MT_underbracket_I:w}
%  \begin{macro}{\MT_underbracket_II:w}
%  \begin{macro}{\upbracketfill}
%  \begin{macro}{\upbracketend}
%  The \cs{underbracket} macro. Scan for two optional arguments. When
%  \pkg{xparse} becomes the standard this will be so much easier.
%    \begin{macrocode}
\providecommand*\underbracket{
  \@ifnextchar[
    {\MT_underbracket_I:w}
    {\MT_underbracket_I:w[\l_MT_bracketheight_fdim]}}
\def\MT_underbracket_I:w[#1]{
  \@ifnextchar[
    {\MT_underbracket_II:w[#1]}
    {\MT_underbracket_II:w[#1][.7\fontdimen5\textfont2]}}
\def\MT_underbracket_II:w[#1][#2]#3{%
  \mathop{\vtop{\m@th\ialign{##
    \crcr
      $\hfil\displaystyle{#3}\hfil$%
    \crcr
      \noalign{\kern .2\fontdimen5\textfont2 \nointerlineskip}%
      \upbracketfill {#1}{#2}%
    \crcr}}}
  \limits}
\def\upbracketfill#1#2{%
  \sbox\z@{$\braceld$}
  \edef\l_MT_bracketheight_fdim{\the\ht\z@}%
  \upbracketend{#1}{#2}
  \leaders \vrule \@height \z@ \@depth #1 \hfill
  \upbracketend{#1}{#2}%
}
\def\upbracketend#1#2{\vrule \@height #2 \@width #1\relax}
%    \end{macrocode}
%  \end{macro}
%  \end{macro}
%  \end{macro}
%  \end{macro}
%  \end{macro}
%  \begin{macro}{\overbracket}
%  \begin{macro}{\MT_overbracket_I:w}
%  \begin{macro}{\MT_overbracket_II:w}
%  \begin{macro}{\downbracketfill}
%  \begin{macro}{\downbracketend}
%  The overbracket is quite similar.
%    \begin{macrocode}
\providecommand*\overbracket{
  \@ifnextchar[
    {\MT_overbracket_I:w}
    {\MT_overbracket_I:w[\l_MT_bracketheight_fdim]}}
\def\MT_overbracket_I:w[#1]{
  \@ifnextchar[
    {\MT_overbracket_II:w[#1]}
    {\MT_overbracket_II:w[#1][.7\fontdimen5\textfont2]}}
\def\MT_overbracket_II:w[#1][#2]#3{%
  \mathop{\vbox{\m@th\ialign{##
        \crcr
          \downbracketfill{#1}{#2}%
        \crcr
          \noalign{\kern .2\fontdimen5\textfont2 \nointerlineskip}%
          $\hfil\displaystyle{#3}\hfil$
        \crcr}}}%
  \limits}
\def\downbracketfill#1#2{%
  \sbox\z@{$\braceld$}\edef\l_MT_bracketheight_fdim{\the\ht\z@}
  \downbracketend{#1}{#2}
  \leaders \vrule \@height #1 \@depth \z@ \hfill
  \downbracketend{#1}{#2}%
}
\def\downbracketend#1#2{\vrule \@width #1\@depth #2\relax}
%    \end{macrocode}
%  \end{macro}
%  \end{macro}
%  \end{macro}
%  \end{macro}
%  \end{macro}
%  \begin{macro}{\LaTeXunderbrace}
%  \begin{macro}{\underbrace}
%  Redefinition of \cs{underbrace} and \cs{overbrace}.
%    \begin{macrocode}
\MH_let:NwN \LaTeXunderbrace \underbrace
\def\underbrace#1{\mathop{\vtop{\m@th\ialign{##\crcr
   $\hfil\displaystyle{#1}\hfil$\crcr
   \noalign{\kern.7\fontdimen5\textfont2\nointerlineskip}%
%    \end{macrocode}
%  |.5\fontdimen5\textfont2| is the height of the tip of the brace.
%  the remaining |.2\fontdimen5\textfont2| is for space between
%    \begin{macrocode}
   \upbracefill\crcr\noalign{\kern.5\fontdimen5\textfont2}}}}\limits}
%    \end{macrocode}
%  \end{macro}
%  \end{macro}
%  \begin{macro}{\LaTeXoverbrace}
%  \begin{macro}{\overbrace}
%  Same technique for \cs{overbrace}.
%    \begin{macrocode}
\MH_let:NwN \LaTeXoverbrace \overbrace
\def\overbrace#1{\mathop{\vbox{\m@th\ialign{##\crcr
  \noalign{\kern.5\fontdimen5\textfont2}%
%    \end{macrocode}
%  Adjust for tip height
%    \begin{macrocode}
  \downbracefill\crcr
  \noalign{\kern.7\fontdimen5\textfont2\nointerlineskip}%
%    \end{macrocode}
%  |.5\fontdimen5\textfont2| is the height of the tip of the brace.
%  The remaining |.2\fontdimen5\textfont2| is for space between
%    \begin{macrocode}
  $\hfil\displaystyle{#1}\hfil$\crcr}}}\limits}
%    \end{macrocode}
%  \end{macro}
%  \end{macro}
%
%
%
%
%
%  \subsection{Special symbols}
%
%  \subsubsection{Command names for parentheses}
%  \begin{macro}{\lparen}
%  \begin{macro}{\rparen}
%  Just an addition to the \LaTeXe\ kernel.
%    \begin{macrocode}
\providecommand*\lparen{(}
\providecommand*\rparen{)}
%    \end{macrocode}
%  \end{macro}
%  \end{macro}

%  \subsubsection{Vertically centered colon}
%
%  \begin{macro}{\vcentcolon}
%  \begin{macro}{\ordinarycolon}
%  \begin{macro}{\MT_active_colon_true:}
%  \begin{macro}{\MT_active_colon_false:}
%  This is from the hands of Donald Arseneau. Somehow it is not
%  distributed, so I include it here. Here's the original text by
%  Donald:
%  \begin{verbatim}
%  centercolon.sty                 Dec 7, 2000
%  Donald Arseneau                 asnd@triumf.ca
%  Public domain.
%  Vertically center colon characters (:) in math mode.
%  Particularly useful for $ a:=b$, and still correct for
%  $f : x\to y$.  May be used in any TeX.
%  \end{verbatim}
%
% Slight change: the colon meaning is given only if \verb|centercolon|
% is explicitly requested (before it was always assigned even if : remained
% catcode 12). This allows better interaction with packages like babel
% that also make colon active.
%    \begin{macrocode}
\def\vcentcolon{\mathrel{\mathop\ordinarycolon}}
\providecommand\ordinarycolon{:}
\begingroup
  \catcode`\:=\active
  \lowercase{\endgroup
\def\MT_activate_colon{%
    \ifnum\mathcode`\:=32768\relax
      \let\ordinarycolon= :%
    \else
      \mathchardef\ordinarycolon\mathcode`\: %
    \fi 
    \let :\vcentcolon
  }
}
%    \end{macrocode}
% Option processing.
% The `false' branch can only be requested if the option has previously been set `true'.
% (By default neither are set.)
%    \begin{macrocode}
\MH_new_boolean:n {center_colon}
\define@key{\MT_options_name:}{centercolon}[true]{
  \@ifundefined{MT_active_colon_#1:}
    { \MT_true_false_error:n
      \@nameuse{MT_active_colon_false:}
    }
    { \@nameuse{MT_active_colon_#1:} }
}
\def\MT_active_colon_true: {
  \MT_activate_colon
  \MH_if_boolean:nF {center_colon}{
    \MH_set_boolean_T:n {center_colon}
    \edef\MT_active_colon_false:
      {\mathcode`\noexpand\:=\the\mathcode`\:\relax}
    \mathcode`\:=32768
  }
}
%    \end{macrocode}
%  \end{macro}
%  \end{macro}
%  \end{macro}
%  \end{macro}
%  \begin{macro}{\dblcolon}
%  \begin{macro}{\coloneqq}
%  \begin{macro}{\Coloneqq}
%  \begin{macro}{\coloneq}
%  \begin{macro}{\Coloneq}
%  \begin{macro}{\eqqcolon}
%  \begin{macro}{\Eqqcolon}
%  \begin{macro}{\eqcolon}
%  \begin{macro}{\Eqcolon}
%  \begin{macro}{\colonapprox}
%  \begin{macro}{\Colonapprox}
%  \begin{macro}{\colonsim}
%  \begin{macro}{\Colonsim}
%  This is just to simulate all the \cs{..colon..} symbols from
%  \pkg{txfonts} and \pkg{pxfonts}.
% \changes{v1.08c}{2010/11/17}{Enclosed all in \cs{mathrel}}
%    \begin{macrocode}
\AtBeginDocument{
  \providecommand*\dblcolon{\mathrel{\vcentcolon\mkern-.9mu\vcentcolon}}
  \providecommand*\coloneqq{\mathrel{\vcentcolon\mkern-1.2mu=}}
  \providecommand*\Coloneqq{\mathrel{\dblcolon\mkern-1.2mu=}}
  \providecommand*\coloneq{\mathrel{\vcentcolon\mkern-1.2mu\mathrel{-}}}
  \providecommand*\Coloneq{\mathrel{\dblcolon\mkern-1.2mu\mathrel{-}}}
  \providecommand*\eqqcolon{\mathrel{=\mkern-1.2mu\vcentcolon}}
  \providecommand*\Eqqcolon{\mathrel{=\mkern-1.2mu\dblcolon}}
  \providecommand*\eqcolon{\mathrel{\mathrel{-}\mkern-1.2mu\vcentcolon}}
  \providecommand*\Eqcolon{\mathrel{\mathrel{-}\mkern-1.2mu\dblcolon}}
  \providecommand*\colonapprox{\mathrel{\vcentcolon\mkern-1.2mu\approx}}
  \providecommand*\Colonapprox{\mathrel{\dblcolon\mkern-1.2mu\approx}}
  \providecommand*\colonsim{\mathrel{\vcentcolon\mkern-1.2mu\sim}}
  \providecommand*\Colonsim{\mathrel{\dblcolon\mkern-1.2mu\sim}}
}
%    \end{macrocode}
%  \end{macro}
%  \end{macro}
%  \end{macro}
%  \end{macro}
%  \end{macro}
%  \end{macro}
%  \end{macro}
%  \end{macro}
%  \end{macro}
%  \end{macro}
%  \end{macro}
%  \end{macro}
%  \end{macro}
%
%
%  \subsection{Multlined}
%
%  \begin{macro}{\g_MT_multlinerow_int}
%  \begin{macro}{\l_MT_multwidth_dim}
%  Helpers.
%    \begin{macrocode}
\let \AMS@math@cr@@ \math@cr@@
\MH_new_boolean:n {mult_firstline}
\MH_new_boolean:n {outer_mult}
\newcount\g_MT_multlinerow_int
\newdimen\l_MT_multwidth_dim
%    \end{macrocode}
%  \end{macro}
%  \end{macro}
%  \begin{macro}{\MT_test_for_tcb_other:nnnnn}
%  This tests if the token(s) is/are equal to either t, c, or~b, or
%  something entirely different.
%    \begin{macrocode}
\newcommand*\MT_test_for_tcb_other:nnnnn [1]{
  \if:w t#1\relax
    \expandafter\MH_use_choice_i:nnnn
  \else:
    \if:w c#1\relax
      \expandafter\expandafter\expandafter\MH_use_choice_ii:nnnn
    \else:
      \if:w b#1\relax
        \expandafter\expandafter\expandafter
        \expandafter\expandafter\expandafter\expandafter
        \MH_use_choice_iii:nnnn
      \else:
        \expandafter\expandafter\expandafter
        \expandafter\expandafter\expandafter\expandafter
        \MH_use_choice_iv:nnnn
      \fi:
    \fi:
  \fi:
}
%    \end{macrocode}
%  \end{macro}
%  \begin{macro}{\MT_mult_invisible_line:}
%  An invisible line.
%    \begin{macrocode}
\def\MT_mult_invisible_line: {
  \crcr
  \global\MH_set_boolean_F:n {mult_firstline}
  \hbox to \l_MT_multwidth_dim{}\crcr
  \noalign{\vskip-\baselineskip \vskip-\normallineskip}
}
%    \end{macrocode}
%  \end{macro}
%  \begin{macro}{\MT_mult_mathcr_atat:w}
%  The normal \cs{math@cr@@} with our hooks.
%    \begin{macrocode}
\def\MT_mult_mathcr_atat:w [#1]{%
  \if_num:w 0=`{\fi: \iffalse}\fi:
  \MH_if_boolean:nT {mult_firstline}{
    \kern\l_MT_mult_left_fdim
    \MT_mult_invisible_line:
  }
  \crcr
  \noalign{\vskip#1\relax}
  \global\advance\g_MT_multlinerow_int\@ne
  \if_num:w \g_MT_multlinerow_int=\l_MT_multline_lastline_fint
    \MH_let:NwN \math@cr@@\MT_mult_last_mathcr:w
  \fi:
}
%    \end{macrocode}
%  \end{macro}
%  \begin{macro}{\MT_mult_firstandlast_mathcr:w}
%  The special case where there is a two-line \env{multlined}. We
%  insert the first kern, then the invisible line of the desired
%  width, the optional vertical space and then the last kern.
%    \begin{macrocode}
\def\MT_mult_firstandlast_mathcr:w [#1]{%
  \if_num:w 0=`{\fi: \iffalse}\fi:
  \kern\l_MT_mult_left_fdim
  \MT_mult_invisible_line:
  \noalign{\vskip#1\relax}
  \kern\l_MT_mult_right_fdim
}
%    \end{macrocode}
%  \end{macro}
%  \begin{macro}{\MT_mult_last_mathcr:w}
%  The normal last \cs{math@cr@@} which inserts the last kern.
%    \begin{macrocode}
\def\MT_mult_last_mathcr:w [#1]{
  \if_num:w 0=`{\fi: \iffalse}\fi:\math@cr@@@
  \noalign{\vskip#1\relax}
  \kern\l_MT_mult_right_fdim}
%    \end{macrocode}
%  \end{macro}
%  \begin{macro}{\MT_start_mult:N}
%  Setup for \env{multlined}. Finds the position.
%    \begin{macrocode}
\newcommand\MT_start_mult:N [1]{
  \MT_test_for_tcb_other:nnnnn {#1}
    { \MH_let:NwN \MT_next:\vtop }
    { \MH_let:NwN \MT_next:\vcenter }
    { \MH_let:NwN \MT_next:\vbox }
    {
      \PackageError{mathtools}
        {Invalid~ position~ specifier.~ I'll~ try~ to~ recover~ with~
        `c'}\@ehc
    }
  \collect@body\MT_mult_internal:n
}
%    \end{macrocode}
%  \end{macro}
%  \begin{macro}{\MT_shoveright:wn}
%  \begin{macro}{\MT_shoveleft:wn}
%  Extended versions of \cs{shoveleft} and \cs{shoveright}.
%    \begin{macrocode}
\newcommand*\MT_shoveright:wn [2][0pt]{%
  #2\hfilneg
  \setlength\@tempdima{#1}
  \kern\@tempdima
}
\newcommand*\MT_shoveleft:wn [2][0pt]{%
  \hfilneg
  \setlength\@tempdima{#1}
  \kern\@tempdima
  #2
}
%    \end{macrocode}
%  \end{macro}
%  \end{macro}
%  \begin{macro}{\MT_mult_internal:n}
%  \changes{v1.01a}{2004/10/10}{Added Ord atom to beginning of each line}
%  The real internal \env{multlined}.
%    \begin{macrocode}
\newcommand*\MT_mult_internal:n [1]{
 \MH_if_boolean:nF {outer_mult}{\null\,}
  \MT_next:
  \bgroup
%    \end{macrocode}
%  Restore the meaning of \cmd{\\} inside \env{multlined}, else it
%  wouldn't work in the \env{equation} environment. Set the fake row
%  counter to zero.
%    \begin{macrocode}
    \Let@
    \def\l_MT_multline_lastline_fint{0 }
    \chardef\dspbrk@context\@ne \restore@math@cr
%    \end{macrocode}
%  Use private versions.
%    \begin{macrocode}
    \MH_let:NwN \math@cr@@\MT_mult_mathcr_atat:w
    \MH_let:NwN \shoveleft\MT_shoveleft:wn
    \MH_let:NwN \shoveright\MT_shoveright:wn
    \spread@equation
    \MH_set_boolean_F:n {mult_firstline}
%    \end{macrocode}
%  Do some measuring.
%    \begin{macrocode}
    \MT_measure_mult:n {#1}
%    \end{macrocode}
%  Make sure the box is wide enough.
%    \begin{macrocode}
    \if_dim:w \l_MT_multwidth_dim<\l_MT_multline_measure_fdim
      \MH_setlength:dn \l_MT_multwidth_dim{\l_MT_multline_measure_fdim}
    \fi
    \MH_set_boolean_T:n {mult_firstline}
%    \end{macrocode}
%  Tricky bit: If we only encountered one \cmd{\\} then use a very
%  special \cs{math@cr@@} that inserts everything needed.
%    \begin{macrocode}
    \if_num:w \l_MT_multline_lastline_fint=\@ne
      \MH_let:NwN \math@cr@@ \MT_mult_firstandlast_mathcr:w
    \fi:
%    \end{macrocode}
%  Do the typesetting.
%    \begin{macrocode}
    \ialign\bgroup
      \hfil\strut@$\m@th\displaystyle{}##$\hfil
      \crcr
      \hfilneg
      #1
}
%    \end{macrocode}
%  \end{macro}
%  \begin{macro}{\MT_measure_mult:n}
%  \changes{v1.01a}{2004/10/10}{Added Ord atom to beginning of each line}
%  Measuring. Disable all labelling and check the number of lines.
%    \begin{macrocode}
\newcommand\MT_measure_mult:n [1]{
  \begingroup
    \g_MT_multlinerow_int\@ne
    \MH_let:NwN \label\MT_gobblelabel:w
    \MH_let:NwN \tag\gobble@tag
    \setbox\z@\vbox{
      \ialign{\strut@$\m@th\displaystyle{}##$
        \crcr
        #1
        \crcr
      }
    }
    \xdef\l_MT_multline_measure_fdim{\the\wdz@}
    \advance\g_MT_multlinerow_int\m@ne
    \xdef\l_MT_multline_lastline_fint{\number\g_MT_multlinerow_int}
  \endgroup
  \g_MT_multlinerow_int\@ne
}
%    \end{macrocode}
%  \end{macro}
%  \begin{macro}{\MT_multlined_second_arg:w}
%  Scan for a second optional argument.
%    \begin{macrocode}
\MaybeMHPrecedingSpacesOff
\newcommand*\MT_multlined_second_arg:w [1][\@empty]{
  \MT_test_for_tcb_other:nnnnn {#1}
    {\def\MT_mult_default_pos:{#1}}
    {\def\MT_mult_default_pos:{#1}}
    {\def\MT_mult_default_pos:{#1}}
    {
      \if_meaning:NN \@empty#1\@empty
      \else:
        \setlength \l_MT_multwidth_dim{#1}
      \fi:
    }
  \MT_start_mult:N \MT_mult_default_pos:
}
%    \end{macrocode}
%  \end{macro}
%  \begin{environment}{multlined}
%  The user environment. Scan for an optional argument.
%    \begin{macrocode}
\newenvironment{multlined}[1][]
  {\MH_group_align_safe_begin:
  \MT_test_for_tcb_other:nnnnn {#1}
    {\def\MT_mult_default_pos:{#1}}
    {\def\MT_mult_default_pos:{#1}}
    {\def\MT_mult_default_pos:{#1}}
    {
      \if_meaning:NN \@empty#1\@empty
      \else:
        \setlength \l_MT_multwidth_dim{#1}
      \fi:
    }
    \MT_multlined_second_arg:w
  }
  {
    \hfilneg  \endaligned \MH_group_align_safe_end:
  }
\MHPrecedingSpacesOn
%    \end{macrocode}
%  \end{environment}
%  The keys needed.
%    \begin{macrocode}
\define@key{\MT_options_name:}
  {firstline-afterskip}{\def\l_MT_mult_left_fdim{#1}}
\define@key{\MT_options_name:}
  {lastline-preskip}{\def\l_MT_mult_right_fdim{#1}}
\define@key{\MT_options_name:}
  {multlined-width}{\setlength \l_MT_multwidth_dim{#1}}
\define@key{\MT_options_name:}
  {multlined-pos}{\def\MT_mult_default_pos:{#1}}
\setkeys{\MT_options_name:}{
  firstline-afterskip=\multlinegap,
  lastline-preskip=\multlinegap,
  multlined-width=0pt,
  multlined-pos=c,
}
%    \end{macrocode}
%  \begin{macro}{\MT_gobblelabel:w}
%  Better than to assume that \cs{label} has exactly one mandatory
%  argument, hence the \texttt{w} specifier.
%    \begin{macrocode}
\def\MT_gobblelabel:w #1{}
%    \end{macrocode}
%  \end{macro}
%
%
%
%
%  \section{Macros suggested/requested by Lars Madsen}
%
%  The macros in this section are all requests made by Lars Madsen.
%
%  \subsection{Paired delimiters}
%
%  \changes{v1.13}{2012/05/10}{Extended \cs{DeclarePairedDelimiter(X)}}
%  \begin{macro}{\MT_delim_default_inner_wrappers:n}
%  In some cases users may want to control the internals a bit more. We
%  therefore create two call back macros each time the
%  \cs{DeclarePaired...} macro is issued. The default value of these
%  call backs are provided by |\MT_delim_default_inner_wrappers:n|
%    \begin{macrocode}
\newcommand\MT_delim_default_inner_wrappers:n [1]{
   \@namedef{MT_delim_\MH_cs_to_str:N #1 _star_wrapper:nnn}##1##2##3{
      \mathopen{}\mathclose\bgroup ##1 ##2 \aftergroup\egroup ##3
    }
    \@namedef{MT_delim_\MH_cs_to_str:N #1 _nostar_wrapper:nnn}##1##2##3{
      \mathopen{##1}##2\mathclose{##3}
    }
  }

%    \end{macrocode}
% \begin{macro}{\reDeclarePairedDelimiterInnerWrapper}
%   Macro enabling the user to alter an existing call back. Note that
%   currently no checks are performed. First argument is the name of
%   the macro we are altering (as defined via \cs{DeclarePaired...}),
%   the second is \texttt{star} or \texttt{nostar}. In the last
%   argument \texttt{\#1} and \texttt{\#3} respectively refer to the
%   scaled fences and \texttt{\#3} refer to whatever come between.
%    \begin{macrocode}
\newcommand\reDeclarePairedDelimiterInnerWrapper[3]{
  \@namedef{MT_delim_\MH_cs_to_str:N #1 _ #2 _wrapper:nnn}##1##2##3{
    #3
  }
}

%    \end{macrocode}
%   
% \end{macro}
%  \end{macro}
%  \begin{macro}{\DeclarePairedDelimiter}
%  \changes{v1.06}{2008/08/01}{Made user command robust}
%  This macro defines |#1| to be a control sequence that takes either
%  a star or an optional argument.
%    \begin{macrocode}
\newcommand*\DeclarePairedDelimiter[3]{%
  \@ifdefinable{#1}{
%    \end{macrocode}
%  Define the starred command to just put \cs{left} and \cs{right}
%  before the delimiters.
%  \changes{1.08e}{2010/09/02}{`Fixed' \cs{left}\dots\cs{right} bad spacing}
%  \changes{1.08e}{2010/09/14}{redid the \cs{left}\dots\cs{right} fix,
%  see \cs{DeclarePairedDelimiterX} for details.}
%  \changes{v1.13}{2012/05/10}{Using call back instead}
%    \begin{macrocode}
    \MT_delim_default_inner_wrappers:n{#1}
    \@namedef{MT_delim_\MH_cs_to_str:N #1 _star:}##1
      %{\mathopen{}\mathclose\bgroup\left#2 ##1 \aftergroup\egroup\right #3}%
      { \@nameuse{MT_delim_\MH_cs_to_str:N #1 _star_wrapper:nnn}%
           {\left#2}{##1}{\right#3} }%
%    \end{macrocode}
%  The command with optional argument. It should be \cs{bigg} or
%  alike.
%    \begin{macrocode}
    \@xp\@xp\@xp
      \newcommand
        \@xp\csname MT_delim_\MH_cs_to_str:N #1 _nostar:\endcsname
        [2][\\@gobble]
        { 
%    \end{macrocode}
%  With the default optional argument we wind up with \cs{relax},
%  else we get \cs{biggr} and \cs{biggl} etc.
%  \changes{v1.13}{2012/05/10}{Using call back instead}
%    \begin{macrocode}
          %\mathopen{\@nameuse {\MH_cs_to_str:N ##1 l} #2} ##2 
          %\mathclose{\@nameuse {\MH_cs_to_str:N ##1 r} #3}}
          \@nameuse{MT_delim_\MH_cs_to_str:N #1 _nostar_wrapper:nnn}%
             {\@nameuse {\MH_cs_to_str:N ##1 l} #2}
             {##2}
             {\@nameuse {\MH_cs_to_str:N ##1 r} #3}
        }
%    \end{macrocode}
%  The user command comes here. Just check for the star and choose
%  the right internal command.
%    \begin{macrocode}
    \DeclareRobustCommand{#1}{
      \@ifstar
        {\@nameuse{MT_delim_\MH_cs_to_str:N #1 _star:}}
        {\@nameuse{MT_delim_\MH_cs_to_str:N #1 _nostar:}}
    }
  }
}
%    \end{macrocode}
%  \end{macro}
%
% \begin{macro}{\DeclarePairedDelimiterX}
%  \changes{v1.08}{2010/06/10}{Added \cs{DeclarePairedDelimiterX}}
%  It has turned out that it is convenient to have a more general
%  version of \cs{DeclarePairedDelimiter}. In this version the user
%  can specify the number of arguments the created macro has, and they
%  can specify the code for the inner part of the macro. Other than
%  that the code is fairly similar to \cs{DeclarePairedDelimiter}
%  \changes{v1.08e}{2010/09/02}{Provided better implementation of 
%  \cs{DeclarePairedDelimiterX}}
%    \begin{macrocode}
\def\MHempty{}
\def\DeclarePairedDelimiterX#1[#2]#3#4#5{%
  \@ifdefinable{#1}{
%    \end{macrocode}
% The constructor takes five arguments, the name of the macro, the
% number of arguments (1-9), the left and right delimiter, the inner
% code for the two macros. First we verify that the number of arguments fit.
%    \begin{macrocode}
    \ifnum#2>9\relax
      \PackageError{mathtools}{No~ more~ than~ 9~ arguments}{}
    \else
      \ifnum#2<1\relax
        \PackageError{mathtools}{Macro~ need~ 1~ or~ more~ arguments}{}
      \fi
    \fi
%    \end{macrocode}
%  \changes{v1.13}{2012/05/10}{Using call back instead}
%  Initiate the default call backs.
%    \begin{macrocode}
    \MT_delim_default_inner_wrappers:n{#1}
%    \end{macrocode}
% We make sure to store the delimiter size in the local variable
% \cs{delimsize}. Then users can refer to the size in the fifth
% argument. In the starred version it will refer to \cs{middle} and in
% the normal version it will hold the provided optional argument.
%    \begin{macrocode}
    \@xp\@xp\@xp
      \newcommand
        \@xp\csname MT_delim_\MH_cs_to_str:N #1 _star:\endcsname
        [#2]
        {
          \begingroup
            \def\delimsize{\middle}
%    \end{macrocode}
% This is slightly controversial, \cs{left}\dots\cs{right} are known
% to produce an inner atom, thus may cause different spacing than
% normal delimiters. We `fix' this by introducing \cs{mathopen} and
% \cs{mathclose}. This change is now factored out into call backs. 
% \changes{v1.08e}{2010/09/14}{redid the left/right fix, inspired by
% ctt thread named `spacing after \cs{right}) and before \cs{left})'
% started 2010-08-12.}
% \changes{v1.13}{2012/05/10}{Using call back instead}
%    \begin{macrocode}
            %\mathopen{}\mathclose\bgroup\left#3 #5 \aftergroup\egroup\right#4
            \@nameuse{MT_delim_\MH_cs_to_str:N #1 _star_wrapper:nnn}
              {\left#3}{#5}{\right#4}
          \endgroup
        }
%    \end{macrocode}
% In order for the starred and non-starred version to have the same
% arguments, we need to introduce an extra macro to catch the optional
% argument (this means that the non-starred version can actually
% support ten arguments!).
% Here we do things a little differently than with
% \cs{DeclarePairedDelimiter}. The optional argument have \cs{MHempty}
% as the default. This is locally changed when we scale the
% delimiters, such that it can eat the l/r if needed.
%    \begin{macrocode}
    \@xp\@xp\@xp
      \newcommand
        \@xp\csname MT_delim_\MH_cs_to_str:N #1 _nostar:\endcsname
        [1][\MHempty]
      {
%    \end{macrocode}
% We need to introduce a local group in order to support nesting. It
% is ended inside \verb|\MT_delim_\MH_cs_to_str:N #1 _nostar_inner:|
%    \begin{macrocode}
        \begingroup
        \def\delimsize{##1}
        \@nameuse{MT_delim_\MH_cs_to_str:N #1 _nostar_inner:}
      } 
%    \end{macrocode}
% Next we provide the inner workhorse. We need a bit of expansion
% magic to get \cs{delimsize} to work.
% \changes{v1.13}{2012/05/10}{Using call back instead}
%    \begin{macrocode}
    \@xp\@xp\@xp
      \newcommand
        \@xp\csname MT_delim_\MH_cs_to_str:N #1 _nostar_inner:\endcsname
        [#2]
        {
          %\mathopen{%
          %  \let\MHempty\@gobble
          %  \@xp\@xp\@xp\csname\@xp\MH_cs_to_str:N \delimsize l\endcsname #3} 
          %#5
          %\mathclose{%
          %  \let\MHempty\@gobble
          %  \@xp\@xp\@xp\csname\@xp\MH_cs_to_str:N \delimsize r\endcsname #4}
          \@nameuse{MT_delim_\MH_cs_to_str:N #1 _nostar_wrapper:nnn}
          {
            \let\MHempty\@gobble
            \@xp\@xp\@xp\csname\@xp\MH_cs_to_str:N \delimsize l\endcsname #3
          }
          {#5}
          {
            \let\MHempty\@gobble
            \@xp\@xp\@xp\csname\@xp\MH_cs_to_str:N \delimsize r\endcsname #4
          }
          \endgroup
        }
    \DeclareRobustCommand{#1}{
      \@ifstar
        {\@nameuse{MT_delim_\MH_cs_to_str:N #1 _star:}}
        {\@nameuse{MT_delim_\MH_cs_to_str:N #1 _nostar:}}
    }
  }
}
%    \end{macrocode}
% \end{macro}
%
%  \subsection{A \texttt{\textbackslash displaystyle} \env{cases} environment}
%
%  \begin{macro}{\MT_start_cases:nnn}
%  We define a single command that does all the hard work.
%  \changes{v1.08}{2010/06/10}{made \cs{MT_start_cases:nnnn} more general}
%    \begin{macrocode}
\def\MT_start_cases:nnnn #1#2#3#4{ % #1=sep,#2=lpreamble,#3=rpreamble,#4=delim
 \RIfM@\else
   \nonmatherr@{\begin{\@currenvir}}
 \fi
 \MH_group_align_safe_begin:
 \left#4
 \vcenter \bgroup
     \Let@ \chardef\dspbrk@context\@ne \restore@math@cr
     \let  \math@cr@@\AMS@math@cr@@
     \spread@equation
     \ialign\bgroup
%    \end{macrocode}
%  Set the first column flush left in \cs{displaystyle} math and the
%  second as specified by the second argument. The first argument is
%  the separation between the columns. It could be a \cs{quad} or
%  something entirely different.
%    \begin{macrocode}
       \strut@#2 &#1\strut@
       #3
       \crcr
}
%    \end{macrocode}
%  \end{macro}
% \begin{macro}{\MH_end_cases:}
%    \begin{macrocode}
\def\MH_end_cases:{\crcr\egroup
 \restorecolumn@
 \egroup
 \MH_group_align_safe_end:
}
%    \end{macrocode}
% \end{macro}
%  \begin{macro}{\newcases}
%  \begin{macro}{\renewcases}
%  Easy creation of new \env{cases}-like environments.
%  \changes{v1.08}{2010/06/10}{changed to match the change in \cs{MT_start_cases:nnnn}}
%    \begin{macrocode}
\newcommand*\newcases[6]{% #1=name, #2=sep, #3=preamble, #4=left, #5=right
 \newenvironment{#1}
   {\MT_start_cases:nnnn {#2}{#3}{#4}{#5}}
   {\MH_end_cases:\right#6}
}
\newcommand*\renewcases[6]{
 \renewenvironment{#1}
   {\MT_start_cases:nnnn {#2}{#3}{#4}{#5}}
   {\MH_end_cases:\right#6}
}
%    \end{macrocode}
%  \begin{environment}{dcases}
%  \begin{environment}{dcases*}
%  \begin{environment}{rcases}
%  \begin{environment}{rcases*}
%  \begin{environment}{drcases}
%  \begin{environment}{drcases*}
%  \begin{environment}{cases*}
%  \env{dcases} is a traditional cases with display style math in
%  both columns, while \env{dcases*} has text in the second column.
%  \changes{v1.08}{2010/06/10}{changed to match the change in
%  \cs{newcases} plus added rcases and drcases}
%    \begin{macrocode}
\newcases{dcases}{\quad}{%
  $\m@th\displaystyle{##}$\hfil}{$\m@th\displaystyle{##}$\hfil}{\lbrace}{.}
\newcases{dcases*}{\quad}{%
  $\m@th\displaystyle{##}$\hfil}{{##}\hfil}{\lbrace}{.}
\newcases{rcases}{\quad}{%
  $\m@th{##}$\hfil}{$\m@th{##}$\hfil}{.}{\rbrace}
\newcases{rcases*}{\quad}{%
  $\m@th{##}$\hfil}{{##}\hfil}{.}{\rbrace}
\newcases{drcases}{\quad}{%
  $\m@th\displaystyle{##}$\hfil}{$\m@th\displaystyle{##}$\hfil}{.}{\rbrace}
\newcases{drcases*}{\quad}{%
  $\m@th\displaystyle{##}$\hfil}{{##}\hfil}{.}{\rbrace}
\newcases{cases*}{\quad}{%
  $\m@th{##}$\hfil}{{##}\hfil}{\lbrace}{.}
%    \end{macrocode}
%  \end{environment}
%  \end{environment}
%  \end{environment}
%  \end{environment}
%  \end{environment}
%  \end{environment}
%  \end{environment}
%  \end{macro}
%  \end{macro}
%
%  \subsection{New matrix environments}
%  \begin{macro}{\MT_matrix_begin:N}
%  \begin{macro}{\MT_matrix_end:}
%  Here are a few helpers for the matrices. \cs{MT_matrix_begin:N}
%  takes one argument specifying the column type for the array inside
%  the matrix. and \cs{MT_matrix_end:} inserts the correct ending.
%    \begin{macrocode}
\def\MT_matrix_begin:N #1{%
  \hskip -\arraycolsep
  \MH_let:NwN \@ifnextchar \MH_nospace_ifnextchar:Nnn
  \array{*\c@MaxMatrixCols #1}}
\def\MT_matrix_end:{\endarray \hskip -\arraycolsep}
%    \end{macrocode}
%  \end{macro}
%  \end{macro}
%  Before we define the environments we better make sure that spaces
%  before the optional argument is disallowed. Else a user who types
%  \begin{verbatim}
%  \[
%    \begin{pmatrix*}
%      [c] & a \\
%       b  & d
%    \end{pmatrix*}
%  \]
%  \end{verbatim}
%  will lose the \texttt{[c]}!
%    \begin{macrocode}
\MaybeMHPrecedingSpacesOff
%    \end{macrocode}
%  \begin{environment}{matrix*}
%  This environment is just like \env{matrix} only it takes an
%  optional argument specifying the column type.
%    \begin{macrocode}
\newenvironment{matrix*}[1][c]
  {\MT_matrix_begin:N #1}
  {\MT_matrix_end:}
%    \end{macrocode}
%  \end{environment}
%  \begin{environment}{pmatrix*}
%  \begin{environment}{bmatrix*}
%  \begin{environment}{Bmatrix*}
%  \begin{environment}{vmatrix*}
%  \begin{environment}{Vmatrix*}
%  Then starred versions of the other \AmS{} matrices.
%    \begin{macrocode}
\newenvironment{pmatrix*}[1][c]
  {\left(\MT_matrix_begin:N #1}
  {\MT_matrix_end:\right)}
\newenvironment{bmatrix*}[1][c]
  {\left[\MT_matrix_begin:N #1}
  {\MT_matrix_end:\right]}
\newenvironment{Bmatrix*}[1][c]
  {\left\lbrace\MT_matrix_begin:N #1}
  {\MT_matrix_end:\right\rbrace}
\newenvironment{vmatrix*}[1][c]
  {\left\lvert\MT_matrix_begin:N #1}
  {\MT_matrix_end:\right\rvert}
\newenvironment{Vmatrix*}[1][c]
  {\left\lVert\MT_matrix_begin:N #1}
  {\MT_matrix_end:\right\lVert}
%    \end{macrocode}
%  \end{environment}
%  \end{environment}
%  \end{environment}
%  \end{environment}
%  \end{environment}
%
% \changes{v1.10}{2011/02/12}{Added the code below, courtesy of Rasmus Villemoes}
% Now we are at it why not provide fenced versions of the
% \env{smallmatrix} construction as well. We will only provide a
% version that can be adjusted as \env{matrix*} above, thus we keep
% the * in the name. The implementation is courtesy of Rasmus
% Villemoes. Rasmus also suggested making the default alignment in
% these environments globally adjustable, so we did
% (\texttt{smallmatrix-align=c} by default). It \emph{is} possible to
% do something similar with the large matrix environments, but that
% might cause problems with the \texttt{array} package, thus for now
% we lease that feature alone.
%
% The base code is a variation over the original \env{smallmatrix}
% environmetn fround in \texttt{amsmath}, thus we will not comment it further.
% 
% TODO: make the code check that the optional argument is either
% \texttt{c}, \texttt{l} or \texttt{r}.
%    \begin{macrocode}
\def\MT_smallmatrix_begin:N #1{%
  \Let@\restore@math@cr\default@tag
  \baselineskip6\ex@ \lineskip1.5\ex@ \lineskiplimit\lineskip
  \csname MT_smallmatrix_#1_begin:\endcsname
}
\def\MT_smallmatrix_end:{\crcr\egroup\egroup\MT_smallmatrix_inner_space:}
\def\MT_smallmatrix_l_begin:{\null\MT_smallmatrix_inner_space:\vcenter\bgroup
  \ialign\bgroup$\m@th\scriptstyle##$\hfil&&\thickspace
  $\m@th\scriptstyle##$\hfil\crcr
}
\def\MT_smallmatrix_c_begin:{\null\MT_smallmatrix_inner_space:\vcenter\bgroup
  \ialign\bgroup\hfil$\m@th\scriptstyle##$\hfil&&\thickspace\hfil
  $\m@th\scriptstyle##$\hfil\crcr
}
\def\MT_smallmatrix_r_begin:{\null\MT_smallmatrix_inner_space:\vcenter\bgroup
  \ialign\bgroup\hfil$\m@th\scriptstyle##$&&\thickspace\hfil
  $\m@th\scriptstyle##$\crcr
}
\newenvironment{smallmatrix*}[1][\MT_smallmatrix_default_align:]
  {\MT_smallmatrix_begin:N #1}
  {\MT_smallmatrix_end:}
%    \end{macrocode}
% We would like to keep to the tradition of the \verb?Xmatrix? and
% \verb?Xmatrix*? macros we added earlier, since most code is similar
% we define them using a constructor macro. We also apply the trick
% used within \verb?\DeclarePairedDelimiter(X)? such that \verb?\left?
% \verb?\right?  constructions produce spacings corresponding to
% \verb?\mathopen? and \verb?\mathclose?.
%    \begin{macrocode}
\def\MT_fenced_sm_generator:nnn #1#2#3{%
  \@ifundefined{#1}{%
    \newenvironment{#1}
    {\@nameuse{#1hook}\mathopen{}\mathclose\bgroup\left#2\MT_smallmatrix_begin:N c}%
      {\MT_smallmatrix_end:\aftergroup\egroup\right#3}%
  }{}%
  \@ifundefined{#1*}{%
    \newenvironment{#1*}[1][\MT_smallmatrix_default_align:]%
    {\@nameuse{#1hook}\mathopen{}\mathclose\bgroup\left#2\MT_smallmatrix_begin:N ##1}%
      {\MT_smallmatrix_end:\aftergroup\egroup\right#3}%
  }{}%
}
\MT_fenced_sm_generator:nnn{psmallmatrix}()
\MT_fenced_sm_generator:nnn{bsmallmatrix}[]
\MT_fenced_sm_generator:nnn{Bsmallmatrix}\lbrace\rbrace
\MT_fenced_sm_generator:nnn{vsmallmatrix}\lvert\rvert
\MT_fenced_sm_generator:nnn{Vsmallmatrix}\lVert\rVert
%    \end{macrocode}
% 
% The options associated with this.
%    \begin{macrocode}
\define@key{\MT_options_name:}
  {smallmatrix-align}{\def\MT_smallmatrix_default_align:{#1}}
\define@key{\MT_options_name:}
  {smallmatrix-inner-space}{\def\MT_smallmatrix_inner_space:{#1}}
\setkeys{\MT_options_name:}{
  smallmatrix-align=c,
  smallmatrix-inner-space=\,
}

%    \end{macrocode}
%  Restore the usual spacing behavior.
%    \begin{macrocode}
\MHPrecedingSpacesOn
%    \end{macrocode}
%
%  \subsection{Smashing an operator with limits}
%
%  \begin{macro}{\smashoperator}
%  The user command. Define \cs{MT_smop_use:NNNNN} to be one of the
%  specialized commands \cs{MT_smop_smash_l:NNNNN},
%  \cs{MT_smop_smash_r:NNNNN}, or the default
%  \cs{MT_smop_smash_lr:NNNNN}.
%    \begin{macrocode}
\newcommand*\smashoperator[2][lr]{
  \def\MT_smop_use:NNNNN {\@nameuse{MT_smop_smash_#1:NNNNN}}
  \toks@{#2}
  \expandafter\MT_smop_get_args:wwwNnNn
    \the\toks@\@nil\@nil\@nil\@nil\@nil\@nil\@@nil
}
%    \end{macrocode}
%  \end{macro}
%  \begin{macro}{\MT_smop_remove_nil_vi:N}
%  \begin{macro}{\MT_smop_mathop:n}
%  \begin{macro}{\MT_smop_limits:}
%  Some helper functions.
%    \begin{macrocode}
\def\MT_smop_remove_nil_vi:N #1\@nil\@nil\@nil\@nil\@nil\@nil{#1}
\def\MT_smop_mathop:n {\mathop}
\def\MT_smop_limits: {\limits}
%    \end{macrocode}
%  \end{macro}
%  \end{macro}
%  \end{macro}
%  Some conditionals.
%    \begin{macrocode}
\MH_new_boolean:n {smop_one}
\MH_new_boolean:n {smop_two}
%    \end{macrocode}
%  \begin{macro}{\MT_smop_get_args:wwwNnNn}
%  The argument stripping. There are three different valid types of
%  input:
%  \begin{enumerate}
%    \item An operator with neither subscript nor superscript.
%    \item An operator with one subscript or superscript.
%    \item An operator with both subscript and superscript.
%  \end{enumerate}
%  Additionally an operator can be either a single macro as in
%  \cs{sum} or in \cs{mathop}\arg{A} and people might be tempted to
%  put a \cs{limits} after the operator, even though it's not
%  necessary. Thus the input with most tokens would be something like
%  \begin{verbatim}
%    \mathop{TTT}\limits_{sub}^{sup}
%  \end{verbatim}
%  Therefore we have to scan for seven arguments, but there might
%  only be one actually. So let's list the possible situations:
%  \begin{enumerate}
%    \item \verb|\mathop{TTT}\limits_{subsub}^{supsup}|
%    \item \verb|\mathop{TTT}_{subsub}^{supsup}|
%    \item \verb|\sum\limits_{subsub}^{supsup}|
%    \item \verb|\sum_{subsub}^{supsup}|
%  \end{enumerate}
%  Furthermore the |_{subsub}^{supsup}| part can also just be
%  |_{subsub}| or empty.
%    \begin{macrocode}
\def\MT_smop_get_args:wwwNnNn #1#2#3#4#5#6#7\@@nil{%
  \begingroup
    \def\MT_smop_arg_A: {#1} \def\MT_smop_arg_B: {#2}
    \def\MT_smop_arg_C: {#3} \def\MT_smop_arg_D: {#4}
    \def\MT_smop_arg_E: {#5} \def\MT_smop_arg_F: {#6}
    \def\MT_smop_arg_G: {#7}
%    \end{macrocode}
%  Check if A is \cs{mathop}. If it is, we know that B is the argument
%  of the \cs{mathop}.
%    \begin{macrocode}
    \if_meaning:NN \MT_smop_arg_A: \MT_smop_mathop:n
%    \end{macrocode}
%  If A was \cs{mathop} we check if C is \cs{limits}
%    \begin{macrocode}
      \if_meaning:NN \MT_smop_arg_C:\MT_smop_limits:
        \def\MT_smop_final_arg_A:{#1{#2}}%
%    \end{macrocode}
%  Now we have something like \verb|\mathop{TTT}\limits|. Then check
%  if D is \cs{@nil}.
%    \begin{macrocode}
        \if_meaning:NN \MT_smop_arg_D: \@nnil
        \else:
          \MH_set_boolean_T:n {smop_one}
          \MH_let:NwN \MT_smop_final_arg_B: \MT_smop_arg_D:
          \MH_let:NwN \MT_smop_final_arg_C: \MT_smop_arg_E:
          \if_meaning:NN \MT_smop_arg_F: \@nnil
          \else:
            \MH_set_boolean_T:n {smop_two}
            \MH_let:NwN \MT_smop_final_arg_D: \MT_smop_arg_F:
            \edef\MT_smop_final_arg_E:
              {\expandafter\MT_smop_remove_nil_vi:N \MT_smop_arg_G: }
          \fi:
        \fi:
      \else:
%    \end{macrocode}
%  Here we have something like \verb|\mathop{TTT}|. Still check
%  if D is \cs{@nil}.
%    \begin{macrocode}
        \def\MT_smop_final_arg_A:{#1{#2}}%
        \if_meaning:NN \MT_smop_arg_D: \@nnil
        \else:
          \MH_set_boolean_T:n {smop_one}
          \MH_let:NwN \MT_smop_final_arg_B: \MT_smop_arg_C:
          \MH_let:NwN \MT_smop_final_arg_C: \MT_smop_arg_D:
          \if_meaning:NN \MT_smop_arg_F: \@nnil
          \else:
            \MH_set_boolean_T:n {smop_two}
            \MH_let:NwN \MT_smop_final_arg_D: \MT_smop_arg_E:
            \MH_let:NwN \MT_smop_final_arg_E: \MT_smop_arg_F:
          \fi:
        \fi:
      \fi:
%    \end{macrocode}
%  If A was not \cs{mathop}, it is an operator in itself, so we check
%  if B is \cs{limits}
%    \begin{macrocode}
    \else:
      \if_meaning:NN \MT_smop_arg_B:\MT_smop_limits:
        \def\MT_smop_final_arg_A:{#1}%
        \if_meaning:NN \MT_smop_arg_D: \@nnil
        \else:
          \MH_set_boolean_T:n {smop_one}
          \MH_let:NwN \MT_smop_final_arg_B: \MT_smop_arg_C:
          \MH_let:NwN \MT_smop_final_arg_C: \MT_smop_arg_D:
          \if_meaning:NN \MT_smop_arg_F: \@nnil
          \else:
            \MH_set_boolean_T:n {smop_two}
            \MH_let:NwN \MT_smop_final_arg_D: \MT_smop_arg_E:
            \MH_let:NwN \MT_smop_final_arg_E: \MT_smop_arg_F:
          \fi:
        \fi:
      \else:
%    \end{macrocode}
%  No \cs{limits} was found, so we already have the right input. Just
%  forget about the last two arguments.
%    \begin{macrocode}
        \def\MT_smop_final_arg_A:{#1}%
        \if_meaning:NN \MT_smop_arg_C: \@nnil
        \else:
          \MH_set_boolean_T:n {smop_one}
          \MH_let:NwN \MT_smop_final_arg_B: \MT_smop_arg_B:
          \MH_let:NwN \MT_smop_final_arg_C: \MT_smop_arg_C:
          \if_meaning:NN \MT_smop_arg_D: \@nnil
          \else:
            \MH_set_boolean_T:n {smop_two}
            \MH_let:NwN \MT_smop_final_arg_D: \MT_smop_arg_D:
            \MH_let:NwN \MT_smop_final_arg_E: \MT_smop_arg_E:
          \fi:
        \fi:
      \fi:
    \fi:
%    \end{macrocode}
%  No reason to measure if there's no sub or sup.
%    \begin{macrocode}
    \MH_if_boolean:nT {smop_one}{
      \MT_smop_measure:NNNNN
      \MT_smop_final_arg_A: \MT_smop_final_arg_B: \MT_smop_final_arg_C:
      \MT_smop_final_arg_D: \MT_smop_final_arg_E:
    }
    \MT_smop_use:NNNNN
      \MT_smop_final_arg_A: \MT_smop_final_arg_B: \MT_smop_final_arg_C:
      \MT_smop_final_arg_D: \MT_smop_final_arg_E:
  \endgroup
}
%    \end{macrocode}
%  \end{macro}
%  Typeset what is necessary and ignore width of sub and sup:
%    \begin{macrocode}
\def\MT_smop_needed_args:NNNNN #1#2#3#4#5{%
  \displaystyle #1
  \MH_if_boolean:nT {smop_one}{
%    \end{macrocode}
%  Let's use the internal versions of \cs{crampedclap} now that we now
%  it is set in \cs{scriptstyle}.
%    \begin{macrocode}
    \limits#2{\MT_cramped_clap_internal:Nn \scriptstyle{#3}}
    \MH_if_boolean:nT {smop_two}{
      #4{\MT_cramped_clap_internal:Nn \scriptstyle{#5}}
    }
  }
}
%    \end{macrocode}
%  Measure the natural width. \cs{@tempdima} holds the dimen we need to
%  adjust it all with.
%    \begin{macrocode}
\def\MT_smop_measure:NNNNN #1#2#3#4#5{%
  \MH_let:NwN \MT_saved_mathclap:Nn \MT_cramped_clap_internal:Nn
  \MH_let:NwN \MT_cramped_clap_internal:Nn \@secondoftwo
  \sbox\z@{$\m@th\MT_smop_needed_args:NNNNN #1#2#3#4#5$}
  \MH_let:NwN \MT_cramped_clap_internal:Nn \MT_saved_mathclap:Nn
  \sbox\tw@{$\m@th\displaystyle#1$}
  \@tempdima=.5\wd0
  \advance\@tempdima-.5\wd2
}
%    \end{macrocode}
%  The `l' variant
%    \begin{macrocode}
\def\MT_smop_smash_l:NNNNN #1#2#3#4#5{
  \MT_smop_needed_args:NNNNN #1#2#3#4#5\kern\@tempdima
}
%    \end{macrocode}
%  The `r' variant
%    \begin{macrocode}
\def\MT_smop_smash_r:NNNNN #1#2#3#4#5{
  \kern\@tempdima\MT_smop_needed_args:NNNNN #1#2#3#4#5
}
%    \end{macrocode}
%  The `lr' variant
%    \begin{macrocode}
\def\MT_smop_smash_lr:NNNNN #1#2#3#4#5{
  \MT_smop_needed_args:NNNNN #1#2#3#4#5
}
%    \end{macrocode}
%
%
%  \subsection{Adjusting limits}
%
%
%  \begin{macro}{\MT_vphantom:Nn}
%  \begin{macro}{\MT_hphantom:Nn}
%  \begin{macro}{\MT_phantom:Nn}
%  \begin{macro}{\MT_internal_phantom:N}
%  The main advantage of \cs{phantom} et al., is the ability to
%  choose the right size automatically, but it requires the input to
%  be typeset four times. Since we will need to have a \cs{cramped}
%  inside a \cs{vphantom} it is much, much faster to choose the style
%  ourselves (we already know it). These macros make it possible.
%    \begin{macrocode}
\def\MT_vphantom:Nn {\v@true\h@false\MT_internal_phantom:N}
\def\MT_hphantom:Nn {\v@false\h@true\MT_internal_phantom:N}
\def\MT_phantom:Nn {\v@true\h@true\MT_internal_phantom:N}
\def\MT_internal_phantom:N #1{
  \ifmmode
    \expandafter\mathph@nt\expandafter#1
  \else
    \expandafter\makeph@nt
  \fi
}
%    \end{macrocode}
%  \end{macro}
%  \end{macro}
%  \end{macro}
%  \end{macro}
%
%  \begin{macro}{\adjustlimits}
%  This is for making sure limits line up on two consecutive
%  operators.
%    \begin{macrocode}
\newcommand*\adjustlimits[6]{
%    \end{macrocode}
%  We measure the two operators and save the difference of their
%  depths.
%    \begin{macrocode}
  \sbox\z@{$\m@th \displaystyle #1$}
  \sbox\tw@{$\m@th \displaystyle #4$}
  \@tempdima=\dp\z@ \advance\@tempdima-\dp\tw@
%    \end{macrocode}
%  We force \cs{displaystyle} for the operator and \cs{scripstyle}
%  for the limit. If we make use of the regular \cs{smash},
%  \cs{vphantom}, and \cs{cramped} macros, and let \TeX{} choose the
%  right style for each one of them, we get a lot of redundant code
%  as we have no need for the combination
%  $(\cs{displaystyle},\cs{textstyle})$ etc. Only
%  $(\cs{scriptstyle},\cs{scriptstyle})$ is useful.
%    \begin{macrocode}
  \if_dim:w \@tempdima>\z@
    \mathop{#1}\limits#2{#3}
  \else:
    \mathop{#1\MT_vphantom:Nn \displaystyle{#4}}\limits
    #2{
        \def\finsm@sh{\ht\z@\z@ \box\z@}
        \mathsm@sh\scriptstyle{\MT_cramped_internal:Nn \scriptstyle{#3}}
        \MT_vphantom:Nn \scriptstyle
          {\MT_cramped_internal:Nn \scriptstyle{#6}}
    }
  \fi:
  \if_dim:w \@tempdima>\z@
    \mathop{#4\MT_vphantom:Nn \displaystyle{#1}}\limits
    #5
    {
      \MT_vphantom:Nn \scriptstyle
        {\MT_cramped_internal:Nn \scriptstyle{#3}}
      \def\finsm@sh{\ht\z@\z@ \box\z@}
      \mathsm@sh\scriptstyle{\MT_cramped_internal:Nn \scriptstyle{#6}}
    }
  \else:
    \mathop{#4}\limits#5{#6}
  \fi:
}
%    \end{macrocode}
%  \end{macro}
% 
%  \subsection{Swapping above display skip}
%
% \begin{macro}{\SwapAboveDisplaySkip}
%   This macro is intended to be used at the start of \AmS\
%   environments, in order to force it to use
%   \cs{abovedisplayshortskip} instead of \cs{abovedisplayskip} above
%   the displayed math. Because of the use of \cs{noalign} it will not
%   work inside \env{equation} or \env{multline}.
%    \begin{macrocode}
\newcommand\SwapAboveDisplaySkip{%
  \noalign{\vskip-\abovedisplayskip\vskip\abovedisplayshortskip}
}

%    \end{macrocode}
%   
% \end{macro}
%
%  \subsection{An aid to alignment}
%
% \begin{macro}{\MoveEqLeft}
%   \changes{v1.05}{2008/06/05}{Added \cs{MoveEqLeft} (daleif)}
%   \changes{v1.05b}{2008/06/18}{We don't need \cs{setlength} here
%     (daleif), after discussion about \cs{global} and \cs{setlength}
%     on ctt} 
%   \changes{v1.12}{2011/06/12}{We don't even need lengths. GL
%   suggested on ctt to just apply them directly.}
%   This is a very simple macro, we `move' a line in an
%   alignment backwards in order to simulate that all subsequent lines
%   have been indented. Note that simply using \verb+\kern-2m+ after
%   the \verb+&+ is not enough, then the alignemnt environment never
%   detects that there is anything (though simulated) in the cell
%   before the \verb+&+.
%    \begin{macrocode}
\newcommand\MoveEqLeft[1][2]{\kern #1em  &   \kern -#1em}
%    \end{macrocode}
% \end{macro}
%
% \begin{macro}{\Aboxed}
% \changes{v1.08}{2010/06/29}{Added \cs{Aboxed}}
% The idea from \cs{MoveEqLeft} can be used for other things. Here we
% create a macro that will allow a user to box an equation inside an
% alignment.
% \changes{v1.12}{2011/08/17}{\cs{Aboxed} reimplemented, cudos to GL}
%    \begin{macrocode}
\newcommand\Aboxed[1]{\let\bgroup{\romannumeral-`}\@Aboxed#1&&\ENDDNE}
%    \end{macrocode}
% The macro has been reimplemented courtesy of Florent Chervet out of
% a posting on ctt, \url{https://groups.google.com/group/comp.text.tex/browse_thread/thread/5d66395f2a1b5134/93fd9661484bd8d8?#93fd9661484bd8d8}
%    \begin{macrocode}
\def\@Aboxed#1&#2&#3\ENDDNE{%
  \ifnum0=`{}\fi \setbox \z@
    \hbox{$\displaystyle#1{}\m@th$\kern\fboxsep \kern\fboxrule }%
    \edef\@tempa {\kern  \wd\z@ &\kern -\the\wd\z@ \fboxsep
        \the\fboxsep \fboxrule \the\fboxrule }\@tempa \boxed {#1#2}%
} 
%    \end{macrocode}
% \end{macro}
%
% \begin{macro}{\ArrowBetweenLines}
%   \changes{v1.05}{2008/06/05}{Added \cs{ArrowBetweenLines} as it
%   belongs here and not just in my \LaTeX book (daleif)}
%   ????Implementation notes are needed????
%    \begin{macrocode}
\MHInternalSyntaxOff
\def\ArrowBetweenLines{\relax
  \iffalse{\fi\ifnum0=`}\fi
  \@ifstar{\ArrowBetweenLines@auxI{00}}{\ArrowBetweenLines@auxI{01}}}
\def\ArrowBetweenLines@auxI#1{%
  \@ifnextchar[%
  {\ArrowBetweenLines@auxII{#1}}%
  {\ArrowBetweenLines@auxII{#1}[\Updownarrow]}}
\def\ArrowBetweenLines@auxII#1[#2]{%
  \ifnum0=`{\fi \iffalse}\fi
%    \end{macrocode}
% It turns out that for some reason the \cs{crcr} (next) removes the
% automatic equation number replacement. The replacement hack seems to
% the trick, though I have no idea why \cs{crcr} broke things (/daleif).
% \changes{v1.08}{2010/06/15}{fixed eq num replacement bug}
%    \begin{macrocode}
%  \crcr
    \expandafter\in@\expandafter{\@currenvir}%
      {alignedat,aligned,gathered}%
      \ifin@ \else
      \notag
      \fi%
   \\
  \noalign{\nobreak\vskip-\baselineskip\vskip-\lineskip}%
  \noalign{\expandafter\in@\expandafter{\@currenvir}%
      {alignedat,aligned,gathered}%
      \ifin@ \else\notag\fi%
  }%
  \if#1 &&\quad #2\else #2\quad\fi
  \\\noalign{\nobreak\vskip-\lineskip}}

\MHInternalSyntaxOn
%    \end{macrocode}
% \end{macro}
%
%
% \subsection{Centered vertical dots}
% 
% Doing a \verb?\vdots? centered within a different sized box, is
% rather easy with the tools available. Note that it does \emph{not}
% check for the style we are running in, thus do not expect this to
% work well within \verb?\scriptstyle? and smaller. Basically we
% create a box of a width corresponding to \verb?{}#1{}? and center
% the \verb?\vdots? within it.
%    \begin{macrocode}
\newcommand\vdotswithin[1]{%
  {\mathmakebox[\widthof{\ensuremath{{}#1{}}}][c]{{\vdots}}}}
%    \end{macrocode}
% Next we are inspired by \verb?\ArrowBetweenLines? and provide a
% costruction to be used within alignments with much less vertical
% space above and below. 
%
% First in order to support \env{spreadlines} we need to store the
% original value of \verb?\jot? (and hope the user does not mess with it).
%    \begin{macrocode}
\newlength\origjot
\setlength\origjot{\jot}
%    \end{macrocode}
% Next define how much we spacing we flush out, and make this user adjustable.
%    \begin{macrocode}
\newdimen\l_MT_shortvdotswithinadjustabove_dim
\newdimen\l_MT_shortvdotswithinadjustbelow_dim
\define@key{\MT_options_name:}
  {shortvdotsadjustabove}{\setlength\l_MT_shortvdotswithinadjustabove_dim{#1}}
\define@key{\MT_options_name:}
  {shortvdotsadjustbelow}{\setlength\l_MT_shortvdotswithinadjustbelow_dim{#1}}
%    \end{macrocode}
% The actual defaults we found by trail and error.
%    \begin{macrocode}
\setkeys{\MT_options_name:}{
  shortvdotsadjustabove=2.15\origjot,
  shortvdotsadjustbelow=\origjot
}
%    \end{macrocode}
% The user macro comes in two versions, starred version corresponding
% to alignment \emph{before} and \verb?&? and a non-starred version
% with alignment \emph{after} \verb?&?.
%    \begin{macrocode}
\def\shortvdotswithin{\relax
  \@ifstar{\MT_svwi_aux:nn{00}}{\MT_svwi_aux:nn{01}}}
\def\MT_svwi_aux:nn #1#2{
  \MTFlushSpaceAbove
  \if#1 \vdotswithin{#2}& \else &\vdotswithin{#2}  \fi
  \MTFlushSpaceBelow
}
%    \end{macrocode}
% We will need a way to remove any tags (eq. numbers) on the
% \verb?\vdots? line. We cannot use the method used by
% \verb?\ArrowBetweenLines? so we use inspiration from
% \texttt{etoolbox}.  
%    \begin{macrocode}
\def\MT_remove_tag_unless_inner:n #1{%
  \begingroup
  \def\etb@tempa##1|#1|##2\MT@END{\endgroup
    \ifx\@empty##2\@empty\notag\fi}%
  \expandafter\etb@tempa\expandafter|alignedat|aligned|split|#1|\MT@END}
%    \end{macrocode}
% These macros take care of removing the space above or below. Since
% these may be useful for the user in very special cases, we provide
% them as separate macros.
%    \begin{macrocode}
\newcommand\MTFlushSpaceAbove{  
  \expandafter\MT_remove_tag_unless_inner:n\expandafter{\@currenvir}
  \\
  \noalign{%
    \nobreak\vskip-\baselineskip\vskip-\lineskip%
      \vskip-\l_MT_shortvdotswithinadjustabove_dim 
      \vskip-\origjot
      \vskip\jot
  }%
  \noalign{
    \expandafter\MT_remove_tag_unless_inner:n\expandafter{\@currenvir}
  }
}
\newcommand\MTFlushSpaceBelow{
  \\\noalign{%
    \nobreak\vskip-\lineskip
    \vskip-\l_MT_shortvdotswithinadjustbelow_dim
    \vskip-\origjot
    \vskip\jot
  }
}

%    \end{macrocode}
%
%
%  \section{A few extra symbols}
%
%  Most math font sets are missing three symbols: \cs{nuparrow},
%  \cs{ndownarrow} and \cs{bigtimes}. We provide \emph{simulated}
%  versions of these symbols in case they are missing. 
%
%  \subsection{Negated up- and down arrows} 
%  
%  Note that the \cs{nuparrow} and the \cs{ndownarrow} are made from
%  \cs{nrightarrow} and \cs{nleftarrow}, so these have to be
%  present. If they are not, we throw an error at use. The
%  implementation details are due to Enrico Gregorio
%  (\url{http://groups.google.com/group/comp.text.tex/msg/689cc8bd604fdb51}),
%  the basic idea is to reflect and rotate existing negated
%  arrows. Note that the reflection and rotation will not show up i
%  most DVI previewers.
% \begin{macro}{\MH_nrotarrow:NN}
% \changes{v1.07}{2008/08/11}{Added support for \cs{nuparrow} and \cs{ndownaddow}}
%  First a common construction macro.
%    \begin{macrocode}
\def\MH_nrotarrow:NN #1#2{%
  \setbox0=\hbox{$\m@th#1\uparrow$}\dimen0=\dp0
  \setbox0=\hbox{%
    \reflectbox{\rotatebox[origin=c]{90}{$\m@th#1\mkern2.22mu #2$}}}%
  \dp0=\dimen0 \box0 \mkern2.3965mu
}
%    \end{macrocode}
% \end{macro}
% The negated arrows are then made using this macro on respectively
% \cs{nrightarrow} and \cs{nleftarrow}
% \begin{macro}{\MH_nuparrow:}
% \begin{macro}{\MH_ndownarrow:}
%    \begin{macrocode}
\def\MH_nuparrow: {%
  \mathrel{\mathpalette\MH_nrotarrow:NN\nrightarrow} }
\def\MH_ndownarrow: {%
  \mathrel{\mathpalette\MH_nrotarrow:NN\nleftarrow} }
%    \end{macrocode}
% \end{macro}
% \end{macro}
% \begin{macro}{\nuparrow}
% \begin{macro}{\ndownarrow}
%   Next we provide \cs{nuparrow} and \cs{ndownarrow} at begin
%   document. Since they depend on \cs{nrightarrow} and
%   \cs{nleftarrow} we test for these and let the macros throw an
%   error if they are missing.
% \changes{v1.08b}{2010/07/21}{Moved graphicx loading down here such
% that we do not get into option clash problems}
%    \begin{macrocode}
\AtBeginDocument{%
  \RequirePackage{graphicx}%
  \@ifundefined{nrightarrow}{%
    \providecommand\nuparrow{%
      \PackageError{mathtools}{\string\nuparrow\space~ is~
        constructed~ from~ \string\nrightarrow,~ which~ is~ not~
        provided.~ Please~ load~ the~ amssymb~ package~ or~ similar}{}
    }}{ \providecommand\nuparrow{\MH_nuparrow:}}
  \@ifundefined{nleftarrow}{%
    \providecommand\ndownarrow{%
      \PackageError{mathtools}{\string\ndownarrow\space~ is~
        constructed~ from~ \string\nleftarrow,~ which~ is~ not~
        provided.~ Please~ load~ the~ amssymb~ package~ or~ similar}{}
    }}{ \providecommand\ndownarrow{\MH_ndownarrow:}} }
%    \end{macrocode}
% \end{macro}
% \end{macro}
% 
%
%  \subsection{Providing bigtimes}
%
%  The idea is to use the original \cs{times} and then scale it
%  accordingly. Again the implementation details have been improved by
%  Enrico Gregorio
%  (\url{http://groups.google.com/group/comp.text.tex/msg/9685c9405df2ff94}). 
%
% \begin{macro}{\MH_bigtimes_scaler:N}
% \begin{macro}{\MH_bigtimes_inner:}
% \begin{macro}{\MH_csym_bigtimes:}
% \changes{v1.07}{2008/08/11}{Added support for \cs{bigtimes}}
%    \begin{macrocode}
\def\MH_bigtimes_scaler:N #1{%
  \vcenter{\hbox{#1$\m@th\mkern-2mu\times\mkern-2mu$}}}
%    \end{macrocode}
%  This is then combined with \cs{mathchoice} to form the inner parts
%  of the macro
%    \begin{macrocode}
\def\MH_bigtimes_inner: {
  \mathchoice{\MH_bigtimes_scaler:N \huge}         % display style
             {\MH_bigtimes_scaler:N \LARGE}        % text style
             {\MH_bigtimes_scaler:N {}}            % script style
             {\MH_bigtimes_scaler:N \footnotesize} % script script style
}
%    \end{macrocode}
%  And thus the internal prepresentaion of the \cs{bigtimes} macro.
%    \begin{macrocode}
\def\MH_csym_bigtimes: {\mathop{\MH_bigtimes_inner:}\displaylimits}
%    \end{macrocode}
% \end{macro}
% \end{macro}
% \end{macro}
% \begin{macro}{\bigtimes}
%   In the end we provide \cs{bigtimes} if otherwise not defined.
%    \begin{macrocode}
\AtBeginDocument{
  \providecommand\bigtimes{\MH_csym_bigtimes:}
}
%    \end{macrocode}
% \end{macro}
%
%  \section{Macros by other people}
%
%  \subsection{Intertext and short intertext}
%
%  It turns out that \cs{intertext} use a bit too much
%  space. Especially noticable if combined with the \env{spreadlines}
%  environment, the extra space is also applied above and below
%  \cs{intertext}, which ends up looking unproffesional.  Chung-chieh
%  Shan
%  (\url{http://conway.rutgers.edu/~ccshan/wiki/blog/posts/Beyond_amsmath/})
%  via Tobias Weh suggested a fix. We apply it here, but also keep the
%  original \cs{intertext} in case a user would rather want it.
% \begin{macro}{\MT_orig_intertext:}
% \begin{macro}{\MT_intertext:}
% \changes{v1.13}{2012/08/19}{\cs{l_MT_X_intertext_dim} renamed to 
% \cs{l_MT_X_intertext_sep}}
% \begin{macro}{\l_MT_above_intertext_sep}
% \begin{macro}{\l_MT_below_intertext_sep}
%  First store the originam Ams version.
%    \begin{macrocode}
\MH_let:NwN \MT_orig_intertext: \intertext@
%    \end{macrocode}
% And then for some reconfiguration. First a few lengths
% \changes{v1.13}{2012/08/19}{Fixed typos and changed the names}
%    \begin{macrocode}
\newdimen\l_MT_above_intertext_sep
\newdimen\l_MT_below_intertext_sep
\define@key{\MT_options_name:}
  {aboveintertextdim}{\setlength\l_MT_above_intertext_sep{#1}}
\define@key{\MT_options_name:}
  {belowintertextdim}{\setlength\l_MT_below_intertext_sep{#1}}
\define@key{\MT_options_name:}
  {above-intertext-dim}{\setlength\l_MT_above_intertext_sep{#1}}
\define@key{\MT_options_name:}
  {below-intertext-dim}{\setlength\l_MT_below_intertext_sep{#1}}
\define@key{\MT_options_name:}
  {above-intertext-sep}{\setlength\l_MT_above_intertext_sep{#1}}
\define@key{\MT_options_name:}
  {below-intertext-sep}{\setlength\l_MT_below_intertext_sep{#1}}
%    \end{macrocode}
% Their default values are zero. Now for our extended version of
% CCShan's solution.  
%    \begin{macrocode}
\def\MT_intertext: {%
  \def\intertext##1{%
    \ifvmode\else\\\@empty\fi
    \noalign{%
      \penalty\postdisplaypenalty\vskip\belowdisplayskip
      \vskip-\lineskiplimit      % CCS
      \vskip\normallineskiplimit % CCS
      \vskip\l_MT_above_intertext_sep
       \vbox{\normalbaselines
        \ifdim\linewidth=\columnwidth
        \else \parshape\@ne \@totalleftmargin \linewidth
        \fi
        \noindent##1\par}%
      \penalty\predisplaypenalty\vskip\abovedisplayskip%
      \vskip-\lineskiplimit      % CCS
      \vskip\normallineskiplimit % CCS
      \vskip\l_MT_above_intertext_sep
   }%
}}
%    \end{macrocode}
% And provide a key to switch
%    \begin{macrocode}
\def\MT_orig_intertext_true:  { \MH_let:NwN \intertext@ \MT_orig_intertext: }
\def\MT_orig_intertext_false: { \MH_let:NwN \intertext@ \MT_intertext: }
\define@key{\MT_options_name:}{original-intertext}[true]{
  \@nameuse{MT_orig_intertext_#1:}
}
%    \end{macrocode}
% And use the new version as default.
%    \begin{macrocode}
\setkeys{\MT_options_name:}{
  original-intertext=false
}
%    \end{macrocode}
% 
% \end{macro}
% \end{macro}
% \end{macro}
% \end{macro}
%
%  Gabriel Zachmann, Donald Arseneau on comp.text.tex 2000/05/12-13
%  \begin{macro}{\shortintertext}
%  \begin{macro}{\MT_orig_shortintertext}
%  \begin{macro}{\MT_shortintertext}
%  \begin{macro}{\l_above_shortintertext_sep}
%  \begin{macro}{\l_below_shortintertext_sep}
%  This is like \cs{intertext} but uses shorter skips between the
%  math. Again this turned out to have the same problem as
%  \cs{intertext}, so we provide two versions.
%    \begin{macrocode}
\def\MT_orig_shortintertext:n #1{%
  \ifvmode\else\\\@empty\fi
  \noalign{%
    \penalty\postdisplaypenalty\vskip\abovedisplayshortskip
    \vbox{\normalbaselines
      \if_dim:w \linewidth=\columnwidth
      \else:
        \parshape\@ne \@totalleftmargin \linewidth
      \fi:
      \noindent#1\par}%
    \penalty\predisplaypenalty\vskip\abovedisplayshortskip%
  }%
}
%    \end{macrocode}
% Lengths like above
% \changes{v1.13}{2012/08/19}{The option was named differently in the
% manual. Also renamed to use the postfix \emph{sep} instead. Though
% the old names remain for compatibility.}
%    \begin{macrocode}
\newdimen\l_MT_above_shortintertext_sep
\newdimen\l_MT_below_shortintertext_sep
\define@key{\MT_options_name:}
  {aboveshortintertextdim}{\setlength \l_MT_above_shortintertext_sep{#1}}
\define@key{\MT_options_name:}
  {belowshortintertextdim}{\setlength \l_MT_below_shortintertext_sep{#1}}
\define@key{\MT_options_name:}
  {above-short-intertext-dim}{\setlength \l_MT_above_shortintertext_sep{#1}}
\define@key{\MT_options_name:}
  {below-short-intertext-dim}{\setlength \l_MT_below_shortintertext_sep{#1}}
\define@key{\MT_options_name:}
  {above-short-intertext-sep}{\setlength \l_MT_above_shortintertext_sep{#1}}
\define@key{\MT_options_name:}
  {below-short-intertext-sep}{\setlength \l_MT_below_shortintertext_sep{#1}}
%    \end{macrocode}
% Looks best with the `old' values of the original \cs{jot}
% setting. So we set them to 3pt each.
%    \begin{macrocode}
\setkeys{\MT_options_name:}{
  aboveshortintertextdim=3pt,
  belowshortintertextdim=3pt
}
%    \end{macrocode}
%  Next, just add the same as we did for \cs{intertext}
%    \begin{macrocode}
\def\MT_shortintertext:n #1{%
  \ifvmode\else\\\@empty\fi
  \noalign{%
    \penalty\postdisplaypenalty\vskip\abovedisplayshortskip
    \vskip-\lineskiplimit      
    \vskip\normallineskiplimit 
    \vskip\l_MT_above_shortintertext_sep
    \vbox{\normalbaselines
      \if_dim:w \linewidth=\columnwidth
      \else:
        \parshape\@ne \@totalleftmargin \linewidth
      \fi:
      \noindent#1\par}%
    \penalty\predisplaypenalty\vskip\abovedisplayshortskip%
    \vskip-\lineskiplimit      
    \vskip\normallineskiplimit 
    \vskip\l_MT_below_shortintertext_sep
  }%
}
%    \end{macrocode}
% Next we need to be able to switch.
%    \begin{macrocode}
\def\MT_orig_shortintertext_true:  { \MH_let:NwN \shortintertext \MT_orig_shortintertext:n }
\def\MT_orig_shortintertext_false: { \MH_let:NwN \shortintertext \MT_shortintertext:n }
\define@key{\MT_options_name:}{original-shortintertext}[true]{
  \@nameuse{MT_orig_shortintertext_#1:}
}
%    \end{macrocode}
% With the updated one as the default.
%    \begin{macrocode}
\setkeys{\MT_options_name:}{
  original-shortintertext=false
}
%    \end{macrocode}
%  \end{macro}
%  \end{macro}
%  \end{macro}
%  \end{macro}
%  \end{macro}
%
%  \subsection{Fine-tuning mathematical layout}
%
%  \subsubsection{A complement to \texttt{\textbackslash smash},
%  \texttt{\textbackslash llap}, and \texttt{\textbackslash rlap}}
%  \begin{macro}{\clap}
%  \begin{macro}{\mathllap}
%  \begin{macro}{\mathrlap}
%  \begin{macro}{\mathclap}
%  \begin{macro}{\MT_mathllap:Nn}
%  \begin{macro}{\MT_mathrlap:Nn}
%  \begin{macro}{\MT_mathclap:Nn}
%  First we'll \cs{provide} those macros (they are so simple that I
%  think other packages might define them as well).
%    \begin{macrocode}
\providecommand*\clap[1]{\hb@xt@\z@{\hss#1\hss}}
\providecommand*\mathllap[1][\@empty]{
  \ifx\@empty#1\@empty
    \expandafter \mathpalette \expandafter \MT_mathllap:Nn
  \else
    \expandafter \MT_mathllap:Nn \expandafter #1
  \fi
}
\providecommand*\mathrlap[1][\@empty]{
  \ifx\@empty#1\@empty
    \expandafter \mathpalette \expandafter \MT_mathrlap:Nn
  \else
    \expandafter \MT_mathrlap:Nn \expandafter #1
  \fi
}
\providecommand*\mathclap[1][\@empty]{
  \ifx\@empty#1\@empty
    \expandafter \mathpalette \expandafter \MT_mathclap:Nn
  \else
    \expandafter \MT_mathclap:Nn \expandafter #1
  \fi
}
%    \end{macrocode}
%  We have to insert |{}| because we otherwise risk triggering a
%  ``feature'' in \TeX.
%    \begin{macrocode}
\def\MT_mathllap:Nn #1#2{{}\llap{$\m@th#1{#2}$}}
\def\MT_mathrlap:Nn #1#2{{}\rlap{$\m@th#1{#2}$}}
\def\MT_mathclap:Nn #1#2{{}\clap{$\m@th#1{#2}$}}
%    \end{macrocode}
%  \end{macro}
%  \end{macro}
%  \end{macro}
%  \end{macro}
%  \end{macro}
%  \end{macro}
%  \end{macro}
%  \begin{macro}{\mathmbox}
%  \begin{macro}{\MT_mathmbox:nn}
%  \begin{macro}{\mathmakebox}
%  \begin{macro}{\MT_mathmakebox_I:w}
%  \begin{macro}{\MT_mathmakebox_II:w}
%  \begin{macro}{\MT_mathmakebox_III:w}
%  Then the \cs{mathmbox}\marg{arg} and
%  \cs{mathmakebox}\oarg{width}\oarg{pos}\marg{arg} macros which are
%  very similar to \cs{mbox} and \cs{makebox}. The differences are:
%  \begin{itemize}
%    \item \meta{arg} is set in math mode of course.
%    \item No need for \cs{leavevmode} as we're in math mode.
%    \item No need to make them \cs{long} (we're still in math mode).
%    \item No need to support a picture version.
%  \end{itemize}
%  The first is easy.
%    \begin{macrocode}
\providecommand*\mathmbox{\mathpalette\MT_mathmbox:nn}
\def\MT_mathmbox:nn #1#2{\mbox{$\m@th#1#2$}}
%    \end{macrocode}
%  We scan for the optional arguments first.
%    \begin{macrocode}
\providecommand*\mathmakebox{
  \@ifnextchar[  \MT_mathmakebox_I:w
                 \mathmbox}
\def\MT_mathmakebox_I:w[#1]{%
  \@ifnextchar[  {\MT_mathmakebox_II:w[#1]}
                 {\MT_mathmakebox_II:w[#1][c]}}
%    \end{macrocode}
%  We had to get the optional arguments out of the way before calling
%  upon the powers of \cs{mathpalette}.
%    \begin{macrocode}
\def\MT_mathmakebox_II:w[#1][#2]{
  \mathpalette{\MT_mathmakebox_III:w[#1][#2]}}
\def\MT_mathmakebox_III:w[#1][#2]#3#4{%
  \@begin@tempboxa\hbox{$\m@th#3#4$}%
    \setlength\@tempdima{#1}%
    \hbox{\hb@xt@\@tempdima{\csname bm@#2\endcsname}}%
  \@end@tempboxa}
%    \end{macrocode}
%  \end{macro}
%  \end{macro}
%  \end{macro}
%  \end{macro}
%  \end{macro}
%  \end{macro}
%  \begin{macro}{\mathsm@sh}
%  Fix \cs{smash}.
%    \begin{macrocode}
\def\mathsm@sh#1#2{%
  \setbox\z@\hbox{$\m@th#1{#2}$}{}\finsm@sh}
%    \end{macrocode}
%  \end{macro}
%
%
%  \subsubsection{A cramped style}
%
%  comp.text.tex on 1992/07/21 by Michael Herschorn.
%  With speed-ups by the Grand Wizard himself as shown on
%  \begin{quote}\rightskip-\leftmargini
%  \url{http://www.tug.org/tex-archive/digests/tex-implementors/042}
%  \end{quote}
%  The (better) user interface by the author.
%
%  \begin{macro}{\cramped}
%  Make sure the expansion is timed correctly.
%    \begin{macrocode}
\providecommand*\cramped[1][\@empty]{
  \ifx\@empty#1\@empty
    \expandafter \mathpalette \expandafter \MT_cramped_internal:Nn
  \else
    \expandafter \MT_cramped_internal:Nn \expandafter #1
  \fi
}
%    \end{macrocode}
%  \end{macro}
%  \begin{macro}{\MT_cramped_internal:Nn}
%  The internal command.
%    \begin{macrocode}
\def\MT_cramped_internal:Nn #1#2{
%    \end{macrocode}
%  Create a box containing the math and force a cramped style by
%  issuing a non-existing radical.
%    \begin{macrocode}
  \sbox\z@{$\m@th#1\nulldelimiterspace=\z@\radical\z@{#2}$}
%    \end{macrocode}
%  Then make sure the height is correct.
%    \begin{macrocode}
    \ifx#1\displaystyle
      \dimen@=\fontdimen8\textfont3
      \advance\dimen@ .25\fontdimen5\textfont2
    \else
      \dimen@=1.25\fontdimen8
      \ifx#1\textstyle\textfont
      \else
        \ifx#1\scriptstyle
          \scriptfont
        \else
          \scriptscriptfont
        \fi
      \fi
      3
    \fi
    \advance\dimen@-\ht\z@ \ht\z@=-\dimen@
    \box\z@
}
%    \end{macrocode}
%  \end{macro}
%
%  \subsubsection{Cramped versions of \texttt{\textbackslash
%  mathllap}, \texttt{\textbackslash mathclap}, and
%  \texttt{\textbackslash mathrlap}}
%  \begin{macro}{\crampedllap}
%  \begin{macro}{\MT_cramped_llap_internal:Nn}
%  \begin{macro}{\crampedclap}
%  \begin{macro}{\MT_cramped_clap_internal:Nn}
%  \begin{macro}{\crampedrlap}
%  \begin{macro}{\MT_cramped_rlap_internal:Nn}
%  Cramped versions of \cs{mathXlap} (for speed). Made by the author.
%    \begin{macrocode}
\providecommand*\crampedllap[1][\@empty]{
  \ifx\@empty#1\@empty
    \expandafter \mathpalette \expandafter \MT_cramped_llap_internal:Nn
  \else
    \expandafter \MT_cramped_llap_internal:Nn \expandafter #1
  \fi
}
\def\MT_cramped_llap_internal:Nn #1#2{
  {}\llap{\MT_cramped_internal:Nn #1{#2}}
}
\providecommand*\crampedclap[1][\@empty]{
  \ifx\@empty#1\@empty
    \expandafter \mathpalette \expandafter \MT_cramped_clap_internal:Nn
  \else
    \expandafter \MT_cramped_clap_internal:Nn \expandafter #1
  \fi
}
\def\MT_cramped_clap_internal:Nn #1#2{
  {}\clap{\MT_cramped_internal:Nn #1{#2}}
}
\providecommand*\crampedrlap[1][\@empty]{
  \ifx\@empty#1\@empty
    \expandafter \mathpalette \expandafter \MT_cramped_rlap_internal:Nn
  \else
    \expandafter \MT_cramped_rlap_internal:Nn \expandafter #1
  \fi
}
\def\MT_cramped_rlap_internal:Nn #1#2{
  {}\rlap{\MT_cramped_internal:Nn #1{#2}}
}
%    \end{macrocode}
%  \end{macro}
%  \end{macro}
%  \end{macro}
%  \end{macro}
%  \end{macro}
%  \end{macro}
%
%
%  \section{Macros by Michael J.~Downes}
%
%  The macros in this section are all by Michael J.~Downes. Either
%  they are straight copies of his original macros or inspired and
%  extended here.
%
%
%  \subsection{Prescript}
%  \begin{macro}{\prescript}
%  This command is taken from a posting to comp.text.tex on
%  December~20th 2000 by Michael J.~Downes. The comments are his. I
%  have added some formatting options to the arguments so that a user
%  can emulate the \pkg{isotope} package.
%
% \changes{v.1.12}{2012/04/19}{Extended \cs{prescript} to change style
% if used in say S context. Requestd by Oliver Buerschaper.}
%  Update 2012: One drawback from MJD's original implementation, is
%  that the math style is hardwired, such that if used in say
%  \cs{scriptstyle} context, then the style/size of the prescript
%  remain the same size. A slightly expensive fix, is to use the
%  \cs{mathchoice} construction. First exdent MJD's code a little
%  (keeping his comments)
% \begin{macro}{\MT_prescript_inner:}
% We make the style an extra forth argument
%    \begin{macrocode}
\newcommand{\MT_prescript_inner:}[4]{
%    \end{macrocode}
%  Put the sup in box 0 and the sub in box 2.
%    \begin{macrocode}
  \@mathmeasure\z@#4{\MT_prescript_sup:{#1}}
  \@mathmeasure\tw@#4{\MT_prescript_sub:{#2}}
  \if_dim:w \wd\tw@>\wd\z@
    \setbox\z@\hbox to\wd\tw@{\hfil\unhbox\z@}
  \else:
    \setbox\tw@\hbox to\wd\z@{\hfil\unhbox\tw@}
  \fi:
%    \end{macrocode}
%  Do not let a preceding mathord symbol approach without any
%  intervening space.
%    \begin{macrocode}
  \mathop{}
%    \end{macrocode}
%  Use \cs{mathopen} to suppress space between the prescripts and the
%  base object even when the latter is not of type ord.
%    \begin{macrocode}
  \mathopen{\vphantom{\MT_prescript_arg:{#3}}}^{\box\z@}\sb{\box\tw@}
  \MT_prescript_arg:{#3}
}
%    \end{macrocode}
% \end{macro}
% Next create \cs{prescript} using \cs{mathchoice} 
%    \begin{macrocode}
\DeclareRobustCommand{\prescript}[3]{
  \mathchoice
%    \end{macrocode}
%  In D and T style, we use MJD's default:
%    \begin{macrocode}
    {\MT_prescript_inner:{#1}{#2}{#3}{\scriptstyle}}
    {\MT_prescript_inner:{#1}{#2}{#3}{\scriptstyle}}
%    \end{macrocode}
%  In the others we step one style down. Of couse in SS style, using
%  \cs{scriptscript} may seem wrong, but there is no lower style.
%    \begin{macrocode}
    {\MT_prescript_inner:{#1}{#2}{#3}{\scriptscriptstyle}}
    {\MT_prescript_inner:{#1}{#2}{#3}{\scriptscriptstyle}}
}
%    \end{macrocode}
%  \end{macro}
%  Then the named arguments. Can you see I'm preparing for templates?
%    \begin{macrocode}
\define@key{\MT_options_name:}
  {prescript-sup-format}{\def\MT_prescript_sup:{#1}}
\define@key{\MT_options_name:}
  {prescript-sub-format}{\def\MT_prescript_sub:{#1}}
\define@key{\MT_options_name:}
  {prescript-arg-format}{\def\MT_prescript_arg:{#1}}
\setkeys{\MT_options_name:}{
  prescript-sup-format={},
  prescript-sub-format={},
  prescript-arg-format={},
}
%    \end{macrocode}
%
%  \subsection{Math sizes}
%  \begin{macro}{\@DeclareMathSizes}
%  This command is taken from a posting to comp.text.tex on
%  October~17th 2002 by Michael J.~Downes. The purpose is to be able
%  to put dimensions on the last three arguments of
%  \cs{DeclareMathSizes}.
%    \begin{macrocode}
\def\@DeclareMathSizes #1#2#3#4#5{%
  \@defaultunits\dimen@ #2pt\relax\@nnil
  \if:w $#3$%
    \MH_let:cN {S@\strip@pt\dimen@}\math@fontsfalse
  \else:
    \@defaultunits\dimen@ii #3pt\relax\@nnil
    \@defaultunits\@tempdima #4pt\relax\@nnil
    \@defaultunits\@tempdimb #5pt\relax\@nnil
    \toks@{#1}%
    \expandafter\xdef\csname S@\strip@pt\dimen@\endcsname{%
      \gdef\noexpand\tf@size{\strip@pt\dimen@ii}%
      \gdef\noexpand\sf@size{\strip@pt\@tempdima}%
      \gdef\noexpand\ssf@size{\strip@pt\@tempdimb}%
      \the\toks@
    }%
  \fi:
}
%    \end{macrocode}
%  \end{macro}
%
%  \subsection{Mathematics within italic text}
%  mathic: Michael J.~Downes on comp.text.tex, 1998/05/14.
%  \begin{macro}{\MT_mathic_true:}
%  \begin{macro}{\MT_mathic_false:}
%  Renew \cs{(} so that it detects the slant of the font and inserts
%  an italic correction.
%    \begin{macrocode}
\def\MT_mathic_true: {
  \MH_if_boolean:nF {math_italic_corr}{
    \MH_set_boolean_T:n {math_italic_corr}
%    \end{macrocode}
%  Save the original meaning if you need to go back.
%    \begin{macrocode}
    \MH_let:NwN \MT_begin_inlinemath: \(
    \renewcommand*\({\relax\ifmmode\@badmath\else
      \ifhmode
        \if_dim:w \fontdimen\@ne\font>\z@
          \if_dim:w \lastskip>\z@
            \skip@\lastskip\unskip
            \@@italiccorr
            \hskip\skip@
          \else:
            \@@italiccorr
          \fi:
        \fi:
      \fi:
      $\fi:
    }
  }
}
%$ for emacs coloring ;-)
%    \end{macrocode}
%  Just for restoring the old behavior.
%    \begin{macrocode}
\def\MT_mathic_false: {
  \MH_if_boolean:nT {math_italic_corr}{
    \MH_set_boolean_F:n {math_italic_corr}
    \MH_let:NwN \( \MT_begin_inlinemath:
  }
}
\MH_new_boolean:n {math_italic_corr}
\define@key{\MT_options_name:}{mathic}[true]{
  \@ifundefined{MT_mathic_#1:}
    { \MT_true_false_error:
      \@nameuse{MT_mathic_false:}
    }
    { \@nameuse{MT_mathic_#1:} }
}
%    \end{macrocode}
%  \end{macro}
%  \end{macro}
%
%  \subsection{Spreading equations}
%
%  Michael J.~Downes on comp.text.tex 1999/08/25
%  \begin{environment}{spreadlines}
%  This is meant to be used outside math, just like
%  \env{subequations}.
%    \begin{macrocode}
\newenvironment{spreadlines}[1]{
  \setlength{\jot}{#1}
  \ignorespaces
}{ \ignorespacesafterend }
%    \end{macrocode}
%  \end{environment}
%
%  \subsection{Gathered}
%
%  Inspired by Michael J.~Downes on comp.text.tex 2002/01/17.
%  \begin{environment}{MT_gathered_env}
%  Just like the normal \env{gathered}, only here we're allowed to
%  specify actions before and after each line.
%    \begin{macrocode}
\MaybeMHPrecedingSpacesOff
\newenvironment{MT_gathered_env}[1][c]{%
    \RIfM@\else
        \nonmatherr@{\begin{\@currenvir}}%
    \fi
    \null\,%
    \if #1t\vtop \else \if#1b\vbox \else \vcenter \fi\fi \bgroup
        \Let@ \chardef\dspbrk@context\@ne \restore@math@cr
        \spread@equation
        \ialign\bgroup
            \MT_gathered_pre:
            \strut@$\m@th\displaystyle##$
            \MT_gathered_post:
            \crcr
}{%
  \endaligned
  \MT_gathered_env_end:
}
\MHPrecedingSpacesOn
%    \end{macrocode}
%  \end{environment}
%  \begin{macro}{\newgathered}
%  \begin{macro}{\renewgathered}
%  \begin{environment}{lgathered}
%  \begin{environment}{rgathered}
%  \begin{environment}{gathered}
%  An easier interface.
%    \begin{macrocode}
\newcommand*\newgathered[4]{
  \newenvironment{#1}
    { \def\MT_gathered_pre:{#2}
      \def\MT_gathered_post:{#3}
      \def\MT_gathered_env_end:{#4}
      \MT_gathered_env
    }{\endMT_gathered_env}
}
\newcommand*\renewgathered[4]{
  \renewenvironment{#1}
    { \def\MT_gathered_pre:{#2}
      \def\MT_gathered_post:{#3}
      \def\MT_gathered_env_end:{#4}
      \MT_gathered_env
    }{\endMT_gathered_env}
}
\newgathered{lgathered}{}{\hfil}{}
\newgathered{rgathered}{\hfil}{}{}
\renewgathered{gathered}{\hfil}{\hfil}{}
%    \end{macrocode}
%  \end{environment}
%  \end{environment}
%  \end{environment}
%  \end{macro}
%  \end{macro}
%
%  \subsection{Split fractions}
%
%  Michael J.~Downes on comp.text.tex 2001/12/06.
%  \begin{macro}{\splitfrac}
%  \begin{macro}{\splitdfrac}
%  These commands use \cs{genfrac} to typeset a split fraction. The
%  thickness of the fraction rule is simply set to zero.
%    \begin{macrocode}
\newcommand*\splitfrac[2]{%
  \genfrac{}{}{0pt}{1}%
    {\textstyle#1\quad\hfill}%
    {\textstyle\hfill\quad\mathstrut#2}%
}
\newcommand*\splitdfrac[2]{%
  \genfrac{}{}{0pt}{0}{#1\quad\hfill}{\hfill\quad\mathstrut #2}%
}
%    \end{macrocode}
%  \end{macro}
%  \end{macro}
%
%
%  \section{Bug fixes for \pkg{amsmath}}
%  The following fixes some bugs in \pkg{amsmath}, but only if the
%  switch is true.
%    \begin{macrocode}
\MH_if_boolean:nT {fixamsmath}{
%    \end{macrocode}
%  \begin{macro}{\place@tag}
%  This corrects a bug in \pkg{amsmath} affecting tag placement in
%  \env{flalign}.\footnote{See
%  \url{http://www.latex-project.org/cgi-bin/ltxbugs2html?pr=amslatex/3591}}
%    \begin{macrocode}
\def\place@tag{%
  \iftagsleft@
    \kern-\tagshift@
%    \end{macrocode}
%  The addition. If we're in \env{flalign} (meaning
%  $\cs{xatlevel@}=\cs{tw@}$) we skip back by an amount of
%  \cs{@mathmargin}. This test is also true for the \env{xxalignat}
%  environment, but it doesn't matter because a)~it's not
%  supported/described in the documentation anymore so new users
%  won't know about it and b)~it forbids the use of \cs{tag}
%  anyway.
%    \begin{macrocode}
    \if@fleqn
      \if_num:w \xatlevel@=\tw@
        \kern-\@mathmargin
      \fi:
    \fi:
%    \end{macrocode}
%  End of additions.
%    \begin{macrocode}
    \if:w 1\shift@tag\row@\relax
      \rlap{\vbox{%
        \normalbaselines
        \boxz@
        \vbox to\lineht@{}%
        \raise@tag
      }}%
    \else:
      \rlap{\boxz@}%
    \fi:
    \kern\displaywidth@
  \else:
    \kern-\tagshift@
    \if:w 1\shift@tag\row@\relax
      \llap{\vtop{%
        \raise@tag
        \normalbaselines
        \setbox\@ne\null
        \dp\@ne\lineht@
        \box\@ne
        \boxz@
      }}%
    \else:
      \llap{\boxz@}%
    \fi:
  \fi:
}
%    \end{macrocode}
%  \end{macro}
%
%  \begin{macro}{\x@calc@shift@lf}
%  This corrects a bug\footnote{See
%  \url{http://www.latex-project.org/cgi-bin/ltxbugs2html?pr=amslatex/3614}}
%  in \pkg{amsmath} that could cause a non-positive value of the dimension
%  \cs{@mathmargin} to cause an
%  \begin{verbatim}
%  ! Arithmetic overflow.
%  <recently read> \@tempcntb
%  \end{verbatim}
%  when in \mode{fleqn,leqno} mode. Not very comprehensible for the user.
%    \begin{macrocode}
\def\x@calc@shift@lf{%
  \if_dim:w \eqnshift@=\z@
    \global\eqnshift@\@mathmargin\relax
      \alignsep@\displaywidth
      \advance\alignsep@-\totwidth@
%    \end{macrocode}
%  The addition: If \cs{@tempcntb} is zero we avoid division.
%    \begin{macrocode}
      \if_num:w \@tempcntb=0
      \else:
        \global\divide\alignsep@\@tempcntb % original line
      \fi:
%    \end{macrocode}
%  Addition end.
%    \begin{macrocode}
      \if_dim:w \alignsep@<\minalignsep\relax
        \global\alignsep@\minalignsep\relax
      \fi:
  \fi:
  \if_dim:w \tag@width\row@>\@tempdima
    \saveshift@1%
  \else:
    \saveshift@0%
  \fi:}%
%    \end{macrocode}
%  \end{macro}
%    \begin{macrocode}
}
%    \end{macrocode}
%  End of bug fixing.
%
%  \subsection{Making environments safer}
%
%  \begin{macro}{\aligned@a}
%  Here we make the \pkg{amsmath} inner environments disallow spaces
%  before their optional positioning specifier.
%    \begin{macrocode}
\MaybeMHPrecedingSpacesOff
\renewcommand\aligned@a[1][c]{\start@aligned{#1}\m@ne}
\MHPrecedingSpacesOn
%    \end{macrocode}
%  \end{macro}
%
%  This is the end of the \pkg{mathtools} package.
%    \begin{macrocode}
%</package>
%    \end{macrocode}
%
%  \Finale
\endinput

%        (quote the arguments according to the demands of your shell)
%
% Documentation:
%    (a) If mathtools.drv is present:
%           latex mathtools.drv
%    (b) Without mathtools.drv:
%           latex mathtools.dtx; ...
%    The class ltxdoc loads the configuration file ltxdoc.cfg
%    if available. Here you can specify further options, e.g.
%    use A4 as paper format:
%       \PassOptionsToClass{a4paper}{article}
%
%    Programm calls to get the documentation (example):
%       pdflatex mathtools.dtx
%       makeindex -s gind.ist mathtools.idx
%       pdflatex mathtools.dtx
%       makeindex -s gind.ist mathtools.idx
%       pdflatex mathtools.dtx
%
% Installation:
%    TDS:tex/latex/mh/mathtools.sty
%    TDS:doc/latex/mh/mathtools.pdf
%    TDS:source/latex/mh/mathtools.dtx
%
%<*ignore>
\begingroup
  \def\x{LaTeX2e}
\expandafter\endgroup
\ifcase 0\ifx\install y1\fi\expandafter
         \ifx\csname processbatchFile\endcsname\relax\else1\fi
         \ifx\fmtname\x\else 1\fi\relax
\else\csname fi\endcsname
%</ignore>
%<*install>
\input docstrip.tex
\Msg{************************************************************************}
\Msg{* Installation}
\Msg{* Package: mathtools 2012/05/10 v1.12}
\Msg{************************************************************************}

\keepsilent
\askforoverwritefalse

\preamble

This is a generated file.

Copyright (C) 2002-2011 by Morten Hoegholm

This work may be distributed and/or modified under the
conditions of the LaTeX Project Public License, either
version 1.3 of this license or (at your option) any later
version. The latest version of this license is in
   http://www.latex-project.org/lppl.txt
and version 1.3 or later is part of all distributions of
LaTeX version 2005/12/01 or later.

This work has the LPPL maintenance status "maintained".

This Current Maintainer of this work is  
Lars Madsen, Will Robertson and Joseph Wright.

This work consists of the main source file mathtools.dtx
and the derived files
   mathtools.sty, mathtools.pdf, mathtools.ins, mathtools.drv.

\endpreamble

\generate{%
  \file{mathtools.ins}{\from{mathtools.dtx}{install}}%
  \file{mathtools.drv}{\from{mathtools.dtx}{driver}}%
  \usedir{tex/latex/mh}%
  \file{mathtools.sty}{\from{mathtools.dtx}{package}}%
}

\obeyspaces
\Msg{************************************************************************}
\Msg{*}
\Msg{* To finish the installation you have to move the following}
\Msg{* file into a directory searched by TeX:}
\Msg{*}
\Msg{*     mathtools.sty}
\Msg{*}
\Msg{* To produce the documentation run the file `mathtools.drv'}
\Msg{* through LaTeX.}
\Msg{*}
\Msg{* Happy TeXing!}
\Msg{*}
\Msg{************************************************************************}

\endbatchfile
%</install>
%<*ignore>
\fi
%</ignore>
%<*driver>
\NeedsTeXFormat{LaTeX2e}
\ProvidesFile{mathtools.drv}%
  [2012/05/10 v1.12 mathematical typesetting tools]
\documentclass{ltxdoc}
\IfFileExists{fourier.sty}{\usepackage{fourier}}{}
\addtolength\marginparwidth{-25pt}
\usepackage{mathtools}

\setcounter{IndexColumns}{2}

\providecommand*\pkg[1]{\textsf{#1}}
\providecommand*\env[1]{\texttt{#1}}
\providecommand*\email[1]{\href{mailto:#1}{\texttt{#1}}}
\providecommand*\mode[1]{\texttt{[#1]}}
\providecommand*\file[1]{\texttt{#1}}
\usepackage{xcolor,varioref,amssymb}
\makeatletter
\newcommand*\thinfbox[2][black]{\fboxsep0pt\textcolor{#1}{\rulebox{{\normalcolor#2}}}}
\newcommand*\thinboxed[2][black]{\thinfbox[#1]{\ensuremath{\displaystyle#2}}}
\newcommand*\rulebox[1]{%
  \sbox\z@{\ensuremath{\displaystyle#1}}%
  \@tempdima\dp\z@
  \hbox{%
    \lower\@tempdima\hbox{%
      \vbox{\hrule height\fboxrule\box\z@\hrule height\fboxrule}%
    }%
  }%
}

\newenvironment{codesyntax}
    {\par\small\addvspace{4.5ex plus 1ex}%
     \vskip -\parskip
     \noindent
     \begin{tabular}{|l|}\hline\ignorespaces}%
    {\\\hline\end{tabular}\nobreak\par\nobreak
     \vspace{2.3ex}\vskip -\parskip\noindent\ignorespacesafterend}
\makeatletter

\newcommand*\FeatureRequest[2]{%
  \hskip1sp
  \marginpar{%
    \parbox[b]{\marginparwidth}{\small\sffamily\raggedright
      \strut Feature request by\\#1\\#2%
    }
  }%
}


\newcommand*\ProvidedBy[2]{%
  \hskip1sp
  \marginpar{%
    \parbox[b]{\marginparwidth}{\small\sffamily\raggedright
      \strut Feature provided by\\#1\\#2%
    }
  }%
}

\newcommand*\cttPosting[2]{%
  \hskip1sp
  \marginpar{%
    \parbox[b]{\marginparwidth}{\small\sffamily\raggedright
     \strut Posted on \texttt{comp.text.tex} \\#1\\#2%
    }%
  }%
}

\newcommand*\tsxPosting[2]{%
  \hskip1sp
  \marginpar{%
    \parbox[b]{\marginparwidth}{\small\sffamily\raggedright
     \strut Posted on \texttt{\small tex.stackexchange.com} \\#1\\#2%
    }%
  }%
}


\expandafter\def\expandafter\MakePrivateLetters\expandafter{%
  \MakePrivateLetters  \catcode`\_=11\relax
}

\providecommand*\SpecialOptIndex[1]{%
  \@bsphack
  \index{#1\actualchar{\protect\ttfamily #1}
          (option)\encapchar usage}%
      \index{options:\levelchar#1\actualchar{\protect\ttfamily #1}\encapchar
            usage}\@esphack}
\providecommand*\opt[1]{\texttt{#1}}

\providecommand*\SpecialKeyIndex[1]{%
  \@bsphack
  \index{#1\actualchar{\protect\ttfamily #1}
          (key)\encapchar usage}%
      \index{keys:\levelchar#1\actualchar{\protect\ttfamily #1}\encapchar
            usage}\@esphack}
\providecommand*\key[1]{\textsf{#1}}

\providecommand*\eTeX{$\m@th\varepsilon$-\TeX}

\def\MTmeta#1{%
     \ensuremath\langle
     \ifmmode \expandafter \nfss@text \fi
     {%
      \meta@font@select
      \edef\meta@hyphen@restore
        {\hyphenchar\the\font\the\hyphenchar\font}%
      \hyphenchar\font\m@ne
      \language\l@nohyphenation
      #1\/%
      \meta@hyphen@restore
     }\ensuremath\rangle
     \endgroup
}
\makeatother
\DeclareRobustCommand\meta{\begingroup\MakePrivateLetters\MTmeta}%
\def\MToarg#1{{\ttfamily[}\meta{#1}{\ttfamily]}\endgroup}
\DeclareRobustCommand\oarg{\begingroup\MakePrivateLetters\MToarg}%
\def\MHmarg#1{{\ttfamily\char`\{}\meta{#1}{\ttfamily\char`\}}\endgroup}
\DeclareRobustCommand\marg{\begingroup\MakePrivateLetters\MHmarg}%
\def\MHarg#1{{\ttfamily\char`\{#1\ttfamily\char`\}}\endgroup}
\DeclareRobustCommand\arg{\begingroup\MakePrivateLetters\MHarg}%
\def\MHcs#1{\texttt{\char`\\#1}\endgroup}
\DeclareRobustCommand\cs{\begingroup\MakePrivateLetters\MHcs}

\def\endverbatim{\if@newlist
\leavevmode\fi\endtrivlist\vspace{-\baselineskip}}
\expandafter\let\csname endverbatim*\endcsname =\endverbatim

\let\MTtheindex\theindex
\def\theindex{\MTtheindex\MakePrivateLetters}

\usepackage[final,hyperindex=false]{hyperref}
\renewcommand*\usage[1]{\textit{\hyperpage{#1}}}

\OnlyDescription
\begin{document}
  \DocInput{mathtools.dtx}
\end{document}
%</driver>
%  \fi
%
%  \changes{v1.0}{2004/07/26}{Initial release}
%
%  \GetFileInfo{mathtools.drv}
%
%  \CheckSum{2836}
%
%  \title{The \pkg{mathtools} package\thanks{This file has version number
%  \fileversion, last revised \filedate.}}
%
%  \author{Lars Madsen, Will Robertson and Joseph
%  Wright\\ (maintainers)\thanks{The maintainers would like to thank
%  Morten H\o gholm for his contributions to this package}}
%  \date{\filedate}
%
%  \maketitle
%
%  \begin{abstract}
%    The \pkg{mathtools} package is an extension package to
%    \pkg{amsmath}. There are two things on \pkg{mathtools}' agenda:
%    1)~correct various bugs/defeciencies in \pkg{amsmath} until
%    these are fixed by the \AmS{} and 2)~provide useful tools
%    for mathematical typesetting, be it a small macro for
%    typesetting a prescript or an underbracket, or entirely new
%    display math constructs such as a \env{multlined} environment.
%  \end{abstract}
%
%  \tableofcontents
%
%  \section{Introduction}
%
%  Although \pkg{amsmath} provides many handy tools for mathematical
%  typesetting, it is nonetheless a static package. This is not a bad
%  thing, because what it does, it mostly does quite well and having
%  a stable math typesetting package is ``a good thing.'' However,
%  \pkg{amsmath} does not fulfill all the needs of the mathematical
%  part of the \LaTeX{} community, resulting in many authors writing
%  small snippets of code for tweaking the mathematical layout. Some
%  of these snippets has also been posted to newsgroups and mailing
%  lists over the years, although more often than not without being
%  released as stand-alone packages.
%
%
%  The \pkg{mathtools} package is exactly what its name implies: tools
%  for mathematical typesetting. It is a collection of many of these
%  often needed small tweaks---with some big tweaks added as well. It
%  can only do so by having harvesting newsgroups for code and/or you
%  writing the maintainers with wishes for code to be included, so if
%  you have any good macros or just macros that help you when writing
%  mathematics, then don't hesitate to report them to us. We can be
%  reached at
%  \begin{quote}\email{mh.ctan@gmail.com}\end{quote}
%  This is of course also the address to use in case of bug reports.
%
%  \section{Package loading}
%
%
%  The \pkg{mathtools} package requires \pkg{amsmath} but is able to
%  pass options to it as well. Thus a line like
%  \begin{verbatim}
%    \usepackage[fleqn,tbtags]{mathtools}
%  \end{verbatim}
%  is equivalent to
%  \begin{verbatim}
%    \usepackage[fleqn,tbtags]{amsmath}
%    \usepackage{mathtools}
%  \end{verbatim}
%
%
%  \subsection{Special \pkg{mathtools} options}
%
%  \begin{codesyntax}
%  \SpecialOptIndex{fixamsmath}\opt{fixamsmath}\texttt{~~~~}
%  \SpecialOptIndex{donotfixamsmathbugs}\opt{donotfixamsmathbugs}
%  \end{codesyntax}
%  The option \opt{fixamsmath} (default) fixes two bugs in
%  \pkg{amsmath}.\footnote{See the online \LaTeX{} bugs database
%  \url{http://www.latex-project.org/cgi-bin/ltxbugs2html} under
%  \AmS\LaTeX{} problem reports 3591 and 3614.} Should you for some
%  reason not want to fix these bugs then just add the option
%  \opt{donotfixamsmathbugs} (if you can do it without typos). The
%  reason for this extremely long name is that I really don't see why
%  you wouldn't want these bugs to be fixed, so I've made it slightly
%  difficult not to fix them.
%
%  \begin{codesyntax}
%  \SpecialOptIndex{allowspaces}\opt{allowspaces}\texttt{~~~~}
%  \SpecialOptIndex{disallowspaces}\opt{disallowspaces}
%  \end{codesyntax}
%  Sometimes \pkg{amsmath} gives you nasty surprises, as here where
%  things look seemingly innocent:
%  \begin{verbatim}
%  \[
%      \begin{gathered}
%        [p] = 100 \\
%        [v] = 200
%      \end{gathered}
%  \]
%  \end{verbatim}
%  Without \pkg{mathtools} this will result in this output:
%  \[
%      \begin{gathered}[c]
%        = 100 \\
%        [v] = 200
%      \end{gathered}
%  \]
%  Yes, the \texttt{[p]} has been gobbled without any warning
%  whatsoever.\footnote{\pkg{amsmath} thought the \texttt[p] was an
%  optional argument, checked if it was \texttt{t} or \texttt{b} and
%  when both tests failed, assumed it was a \texttt{c}.} This is
%  hardly what you'd expect as an end user, as the desired output was
%  probably something like this instead:
%  \[
%      \begin{gathered}[c]
%        [p] = 100 \\
%        [v] = 200
%      \end{gathered}
%  \]
%  With the option \opt{disallowspaces} (default) \pkg{mathtools}
%  disallows spaces in front of optional arguments where it could
%  possibly cause problems just as \pkg{amsmath} does with |\\|
%  inside the display environments. This includes the environments
%  \env{gathered} (and also those shown in \S
%  \vref{subsec:gathered}), \env{aligned}, \env{multlined}, and the
%  extended \env{matrix}-environments (\S \vref{subsubsec:matrices}).
%  If you however want to preserve the more dangerous standard
%  optional spaces, simply choose the option \opt{allowspaces}.
%
%
%  \section{Tools for mathematical typesetting}
%
%  \begin{codesyntax}
%    \SpecialUsageIndex{\mathtoolsset}\cs{mathtoolsset}\marg{key val list}
%  \end{codesyntax}
%  Many of the tools shown in this manual can be turned on and off by
%  setting a switch to either true or false. In all cases it is done
%  with the command \cs{mathtoolsset}. A typical use could be something like
%  \begin{verbatim}
%    \mathtoolsset{
%      showonlyrefs,
%      mathic % or mathic = true
%    }
%  \end{verbatim}
%  More information on the keys later on.
%
%  \subsection{Fine-tuning mathematical layout}
%
%  Sometimes you need to tweak the layout of formulas a little to get
%  the best result and this part of the manual describes the various
%  macros \pkg{mathtools} provides for this.
%
%  \subsubsection{A complement to \texttt{\textbackslash smash},
%  \texttt{\textbackslash llap}, and \texttt{\textbackslash rlap}}
%
%  \begin{codesyntax}
%    \SpecialUsageIndex{\mathllap}
%    \cs{mathllap}\oarg{mathstyle}\marg{math}\texttt{~~}
%    \SpecialUsageIndex{\mathclap}
%    \cs{mathclap}\oarg{mathstyle}\marg{math}\\
%    \SpecialUsageIndex{\mathrlap}
%    \cs{mathrlap}\oarg{mathstyle}\marg{math}\texttt{~~}
%    \SpecialUsageIndex{\clap}
%    \cs{clap}\marg{text}\\
%    \SpecialUsageIndex{\mathmbox}
%    \cs{mathmbox}\marg{math}\phantom{\meta{mathstyle}}\texttt{~~~~}
%    \SpecialUsageIndex{\mathmakebox}
%    \cs{mathmakebox}\oarg{width}\oarg{pos}\marg{math}
%  \end{codesyntax}
%  In \cite{Perlis01}, Alexander R.~Perlis describes some simple yet
%  useful macros for use in math displays. For example the display
%  \begin{verbatim}
%    \[
%      X = \sum_{1\le i\le j\le n} X_{ij}
%    \]
%  \end{verbatim}
%  \[
%    X = \sum_{1\le i\le j\le n} X_{ij}
%  \]
%  contains a lot of excessive white space.  The idea that comes to
%  mind is to fake the width of the subscript. The command
%  \cs{mathclap} puts its argument in a zero width box and centers
%  it, so it could possibly be of use here.
%  \begin{verbatim}
%    \[
%      X = \sum_{\mathclap{1\le i\le j\le n}} X_{ij}
%    \]
%  \end{verbatim}
%  \[
%    X = \sum_{\mathclap{1\le i\le j\le n}} X_{ij}
%  \]
%  For an in-depth discussion of
%  these macros I find it better to read the article; an online
%  version can be found at
%  \begin{quote}
%    \url{http://www.tug.org/TUGboat/Articles/tb22-4/tb72perlS.pdf}
%  \end{quote}
%  Note that the definitions shown in the article do not exactly
%  match the definitions in \pkg{mathtools}. Besides providing an
%  optional argument for specifying the desired math style, these
%  versions also work around a most unfortunate \TeX{}
%  ``feature.''\footnote{The faulty reboxing procedure.} The
%  \cs{smash} macro is fixed too.
%
%
%  \subsubsection{Forcing a cramped style}
%
%  \begin{codesyntax}
%    \SpecialUsageIndex{\cramped}
%    \cs{cramped}\oarg{mathstyle}\marg{math}
%  \end{codesyntax}
%  \cttPosting{Michael Herschorn}{1992/07/21}
%  Let's look at another example where we have used \cs{mathclap}:
%  \begin{verbatim}
%    \begin{equation}\label{eq:mathclap}
%      \sum_{\mathclap{a^2<b^2<c}}\qquad
%      \sum_{a^2<b^2<c}
%    \end{equation}
%  \end{verbatim}
%  \begin{equation}\label{eq:mathclap}
%    \sum_{\mathclap{a^2<b^2<c}}\qquad
%    \sum_{a^2<b^2<c}
%  \end{equation}
%  Do you see the difference? Maybe if I zoomed in a bit:
%  \begingroup \fontsize{24}{\baselineskip}\selectfont
%  \[
%    \sum_{\mathclap{a^2<b^2<c}}\qquad
%    \sum_{a^2<b^2<c}
%  \]
%  \endgroup
%  Notice how the limit of the right summation sign is typeset in a
%  more compact style than the left. It is because \TeX{} sets the
%  limits of operators in a \emph{cramped} style. For each of \TeX'
%  four math styles (\cs{displaystyle}, \cs{textstyle},
%  \cs{scriptstyle}, and \cs{scriptscriptstyle}), there also exists a
%  cramped style that doesn't raise exponents as much. Besides in the
%  limits of operators, \TeX{} also automatically uses these cramped
%  styles in radicals such as \cs{sqrt} and in the denominators of
%  fractions, but unfortunately there are no primitive commands that
%  allows you to detect crampedness or switch to it.
%
%  \pkg{mathtools} offers the command \cs{cramped} which forces a
%  cramped style in normal un-cramped math. Additionally you can
%  choose which of the four styles you want it in as well by
%  specifying it as the optional argument:
%  \begin{verbatim}
%    \[
%      \cramped{x^2}               \leftrightarrow x^2    \quad
%      \cramped[\scriptstyle]{x^2} \leftrightarrow {\scriptstyle x^2}
%    \]
%  \end{verbatim}
%  \[
%    \cramped{x^2}               \leftrightarrow x^2    \quad
%    \cramped[\scriptstyle]{x^2} \leftrightarrow {\scriptstyle x^2}
%  \]
%  You may be surprised how often the cramped style can be
%  beneficial yo your output. Take a look at this example:
%  \begin{verbatim}
%    \begin{quote}
%      The 2005 Euro\TeX{} conference is held in Abbaye des
%      Pr\'emontr\'es, France, marking the 16th ($2^{2^2}$) anniversary
%      of both Dante and GUTenberg (the German and French \TeX{} users
%      group resp.).
%    \end{quote}
%  \end{verbatim}
%  \begin{quote}
%    The 2005 Euro\TeX{} conference is held in Abbaye des
%    Pr\'emontr\'es, France, marking the 16th ($2^{2^2}$) anniversary
%    of both Dante and GUTenberg (the German and French \TeX{} users
%    group resp.).
%  \end{quote}
%  Typesetting on a grid is generally considered quite desirable, but
%  as the second line of the example shows, the exponents of $2$
%  causes the line to be too tall for the normal value of
%  \cs{baselineskip}, so \TeX{} inserts a \cs{lineskip} (normal value
%  is \the\lineskip). In order to circumvent the problem, we can
%  force a cramped style so that the exponents aren't raised as much:
%  \begin{verbatim}
%    \begin{quote}
%      The 2005 Euro\TeX{} ... 16th ($\cramped{2^{2^2}}$) ...
%    \end{quote}
%  \end{verbatim}
%  \begin{quote}
%    The 2005 Euro\TeX{} conference is held in Abbaye des
%    Pr\'emontr\'es, France, marking the 16th ($\cramped{2^{2^2}}$)
%    anniversary of both Dante and GUTenberg (the German and French
%    \TeX{} users group resp.).
%  \end{quote}
%
%  \begin{codesyntax}
%    \SpecialUsageIndex{\crampedllap}
%    \cs{crampedllap}\oarg{mathstyle}\marg{math}\texttt{~~}
%    \SpecialUsageIndex{\crampedclap}
%    \cs{crampedclap}\oarg{mathstyle}\marg{math}\\
%    \SpecialUsageIndex{\crampedrlap}
%    \cs{crampedrlap}\oarg{mathstyle}\marg{math}
%  \end{codesyntax}
%  The commands \cs{crampedllap}, \cs{crampedclap}, and
%  \cs{crampedrlap} are identical to the three \cs{mathXlap} commands
%  described earlier except the argument is typeset in cramped style.
%  You need this in order to typeset \eqref{eq:mathclap} correctly
%  while still faking the width of the limit.
%  \begin{verbatim}
%    \begin{equation*}\label{eq:mathclap-b}
%      \sum_{\crampedclap{a^2<b^2<c}}
%      \tag{\ref{eq:mathclap}*}
%    \end{equation*}
%  \end{verbatim}
%  \begin{equation*}\label{eq:mathclap-b}
%    \sum_{\crampedclap{a^2<b^2<c}}
%    \tag{\ref{eq:mathclap}*}
%  \end{equation*}
%  Of course you could just type
%  \begin{verbatim}
%    \sum_{\mathclap{\cramped{a^2<b^2<c}}}
%  \end{verbatim}
%  but it has one major disadvantage: In order for \cs{mathXlap} and
%  \cs{cramped} to get the right size, \TeX{} has to process them
%  four times, meaning that nesting them as shown above will cause
%  \TeX{} to typeset $4^2$ instances before choosing the right one.
%  In this situation however, we will of course need the same style
%  for both commands so it makes sense to combine the commands in
%  one, thus letting \TeX{} make the choice only once rather than
%  twice.
%
%
%
%  \subsubsection{Smashing an operator}
%
%
%
%  \begin{codesyntax}
%    \SpecialUsageIndex{\smashoperator}
%    \cs{smashoperator}\oarg{pos}\marg{operator with limits}
%  \end{codesyntax}
%  \FeatureRequest{Lars Madsen}{2004/05/04}
%  Above we shoved how to get \LaTeX{} to ignore the width of the
%  subscript of an operator. However this approach takes a lot of
%  extra typing, especially if you have a wide superscript, meaning
%  you have to put in \cs{crampedclap} in both sub- and superscript.
%  To make things easier, \pkg{mathtools} provides a
%  \cs{smashoperator} command, which simply ignores the width of the
%  sub- and superscript. It also takes an optional argument,
%  \texttt{l}, \texttt{r}, or \texttt{lr} (default), denoting which
%  side of the operator should be ignored (smashed).
%  \begin{verbatim}
%    \[
%      V = \sum_{1\le i\le j\le n}^{\infty} V_{ij}                  \quad
%      X = \smashoperator{\sum_{1\le i\le j\le n}^{3456}} X_{ij}    \quad
%      Y = \smashoperator[r]{\sum\limits_{1\le i\le j\le n}} Y_{ij} \quad
%      Z = \smashoperator[l]{\mathop{T}_{1\le i\le j\le n}} Z_{ij}
%    \]
%  \end{verbatim}
%    \[
%      V = \sum_{1\le i\le j\le n}^{\infty} V_{ij}                  \quad
%      X = \smashoperator{\sum_{1\le i\le j\le n}^{3456}} X_{ij}    \quad
%      Y = \smashoperator[r]{\sum\limits_{1\le i\le j\le n}} Y_{ij} \quad
%      Z = \smashoperator[l]{\mathop{T}_{1\le i\le j\le n}} Z_{ij}
%    \]
%  Note that \cs{smashoperator} always sets its argument in display
%  style and with limits even if you have used the \opt{nosumlimits}
%  option of \pkg{amsmath}. If you wish, you can use shorthands for
%  \texttt{\_} and \texttt{\textasciicircum} such as \cs{sb} and
%  \cs{sp}.
%
%
%  \subsubsection{Adjusting limits of operators}
%
%  \begin{codesyntax}
%    \SpecialUsageIndex{\adjustlimits}
%    \cs{adjustlimits}\marg{operator$\sb1$}\texttt{\_}\marg{limit$\sb1$}
%                     \marg{operator$\sb2$}\texttt{\_}\marg{limit$\sb2$}
%  \end{codesyntax}
%  \FeatureRequest{Lars Madsen}{2004/07/09}
%  When typesetting two consecutive operators with limits one often
%  wishes the limits of the operators were better aligned. Look
%  closely at these examples:
%  \begin{verbatim}
%    \[
%      \text{a)} \lim_{n\to\infty} \max_{p\ge n} \quad
%      \text{b)} \lim_{n\to\infty} \max_{p^2\ge n} \quad
%      \text{c)} \lim_{n\to\infty} \sup_{p^2\ge nK} \quad
%      \text{d)} \limsup_{n\to\infty} \max_{p\ge n}
%    \]
%  \end{verbatim}
%  \[
%    \text{a)} \lim_{n\to\infty} \max_{p\ge n} \quad
%    \text{b)} \lim_{n\to\infty} \max_{p^2\ge n} \quad
%    \text{c)} \lim_{n\to\infty} \sup_{p^2\ge nK} \quad
%    \text{d)} \limsup_{n\to\infty} \max_{p\ge n}
%  \]
%  a) looks okay, but b) is not quite as good because the second
%  limit ($\cramped{p^2\ge n}$) is significantly taller than the
%  first ($n\to\infty$). With c)~things begin to look really bad,
%  because the second operator has a descender while the first
%  doesn't, and finally we have d)~which looks just as bad as~c). The
%  command \cs{adjustlimits} is useful in these cases, as you can
%  just put it in front of these consecutive operators and it'll make
%  the limits line up.
%  \medskip\par\noindent
%  \begin{minipage}{\textwidth}
%  \begin{verbatim}
%    \[
%      \text{a)} \adjustlimits\lim_{n\to\infty} \max_{p\ge n} \quad
%      \text{b)} \adjustlimits\lim_{n\to\infty} \max_{p^2\ge n} \quad
%      \text{c)} \adjustlimits\lim_{n\to\infty} \sup_{p^2\ge nK} \quad
%      \text{d)} \adjustlimits\limsup_{n\to\infty} \max_{p\ge n}
%    \]
%  \end{verbatim}
%  \end{minipage}
%  \[
%      \text{a)} \adjustlimits\lim_{n\to\infty} \max_{p\ge n} \quad
%      \text{b)} \adjustlimits\lim_{n\to\infty} \max_{p^2\ge n} \quad
%      \text{c)} \adjustlimits\lim_{n\to\infty} \sup_{p^2\ge nK} \quad
%      \text{d)} \adjustlimits\limsup_{n\to\infty} \max_{p\ge n}
%    \]
%  The use of \cs{sb} instead of \texttt{\_} is allowed.
%
%
%  \subsubsection{Swapping space above \texorpdfstring{\AmS}{AMS} display math environments }
%  \label{sec:swapping}
%
%  One feature that the plain old \env{equation} environment has that
%  the \AmS\ environments does not (because of thechnical reasons), is
%  the feature of using less space above the equation if the situation
%  presents itself. The \AmS\ environments cannot do this, but one can
%  manually, using  
%  \begin{codesyntax}
%    \SpecialUsageIndex{\SwapAboveDisplaySkip}
%    \cs{SwapAboveDisplaySkip}
%  \end{codesyntax}
%  as the very first content within an \AmS\ display math
%  environment. It will then issue an \cs{abovedisplayshortskip}
%  instead of the normal \cs{abovedisplayskip}.
%
%  Note it will not work with the \env{equation} or \env{multline} environments.
%  
%  Here is an example of the effect
% \begin{verbatim}
%  \noindent\rule\textwidth{1pt}
%  \begin{align*}  A &= B \end{align*}
%  \noindent\rule\textwidth{1pt}
%  \begin{align*}  
%  \SwapAboveDisplaySkip
%  A &= B 
%  \end{align*}
% \end{verbatim}
%  \noindent\rule\textwidth{1pt}
%  \begin{align*}  A &= B \end{align*}
%  \noindent\rule\textwidth{1pt}
%  \begin{align*}  
%  \SwapAboveDisplaySkip
%  A &= B 
%  \end{align*}
%  
%
%  \subsection{Controlling tags}
%
%  In this section various tools for altering the appearance of tags
%  are shown. All of the tools here can be used at any point in the
%  document but they should probably be affect the whole document, so
%  the preamble is the best place to issue them.
%
%  \subsubsection{The appearance of tags}
%  \begin{codesyntax}
%    \SpecialUsageIndex{\newtagform}
%    \cs{newtagform}\marg{name}\oarg{inner_format}\marg{left}\marg{right}\\
%    \SpecialUsageIndex{\renewtagform}
%    \cs{renewtagform}\marg{name}\oarg{inner_format}\marg{left}\marg{right}\\
%    \SpecialUsageIndex{\usetagform}
%    \cs{usetagform}\marg{name}
%  \end{codesyntax}
%  Altering the layout of equation numbers in \pkg{amsmath} is not
%  very user friendly (it involves a macro with three \texttt{@}'s in
%  its name), so \pkg{mathtools} provides an interface somewhat
%  reminiscent of the page style concept. This way you can define
%  several different tag forms and then choose the one you prefer.
%
%  As an example let's try to define a tag form which puts the
%  equation number in square brackets. First we define a brand new tag
%  form:
%  \begin{verbatim}
%    \newtagform{brackets}{[}{]}
%  \end{verbatim}
%  Then we activate it:
%  \begin{verbatim}
%    \usetagform{brackets}
%  \end{verbatim}
%  The result is then
%  \newtagform{brackets}{[}{]}
%  \usetagform{brackets}
%   \begin{equation}
%     E \neq m c^3
%   \end{equation}
%
%  Similarly you could define a second version of the brackets that
%  prints the equation number in bold face instead
%  \begin{verbatim}
%    \newtagform{brackets2}[\textbf]{[}{]}
%    \usetagform{brackets2}
%    \begin{equation}
%      E \neq m c^3
%    \end{equation}
%  \end{verbatim}
%  \newtagform{brackets2}[\textbf]{[}{]}
%  \usetagform{brackets2}
%  \begin{equation}
%    E \neq m c^3
%  \end{equation}
%  When you reference an equation with \cs{eqref}, the tag form in
%  effect at the time of referencing controls the formatting, so be
%  careful if you use different tag forms throughout your document.
%
%  If you want to renew a tag form, then use the command
%  \cs{renewtagform}. Should you want to
%  return to the standard setting then choose\usetagform{default}
%  \begin{verbatim}
%    \usetagform{default}
%  \end{verbatim}
%
%  \changes{v1.12}{2012/05/09}{Added caveat}
%  \noindent\textbf{Caveat regarding \pkg{ntheorem}}: If you like to
%  change the appearence of the tags \emph{and} you are also using the
%  \pkg{ntheorem} package, then please postpone the change of
%  appearance until \emph{after} loading \pkg{ntheorem}. (In order to
%  do its thing, \pkg{ntheorem} has to mess with the tags\dots)
%
%  \subsubsection{Showing only referenced tags}
%
%  \begin{codesyntax}
%    \SpecialKeyIndex{showonlyrefs}$\key{showonlyrefs}=\texttt{true}\vert\texttt{false}$\\
%    \SpecialKeyIndex{showmanualtags}$\key{showmanualtags}=\texttt{true}\vert\texttt{false}$\\
%    \SpecialUsageIndex{\refeq}\cs{refeq}\marg{label}
%  \end{codesyntax}
%  An equation where the tag is produced with a manual \cs{tag*}
%  shouldn't be referenced with the normal \cs{eqref} because that
%  would format it according to the current tag format. Using just
%  \cs{ref} on the other hand may not be a good solution either as
%  the argument of \cs{tag*} is always set in upright shape in the
%  equation and you may be referencing it in italic text. In the
%  example below, the command \cs{refeq} is used to avoid what could
%  possibly lead to confusion in cases where the tag font has very
%  different form in upright and italic shape (here we switch to
%  Palatino in the example):
%    \begin{verbatim}
%    \begin{quote}\renewcommand*\rmdefault{ppl}\normalfont\itshape
%    \begin{equation*}
%      a=b \label{eq:example}\tag*{Q\&A}
%    \end{equation*}
%    See \ref{eq:example} or is it better with \refeq{eq:example}?
%    \end{quote}
%  \end{verbatim}
%    \begin{quote}\renewcommand*\rmdefault{ppl}\normalfont\itshape
%    \begin{equation*}
%      a=b \label{eq:example}\tag*{Q\&A}
%    \end{equation*}
%    See \ref{eq:example} or is it better with \refeq{eq:example}?
%  \end{quote}
%
%
%  Another problem sometimes faced is the need for showing the
%  equation numbers for only those equations actually referenced. In
%  \pkg{mathtools} this can be done by setting the key
%  \key{showonlyrefs} to either true or false by using
%  \cs{mathtoolsset}. You can also choose whether or not to show the
%  manual tags specified with \cs{tag} or \cs{tag*} by setting the
%  option \key{showmanualtags} to true or false.\footnote{I recommend
%  setting \key{showmanualtags} to true, else the whole idea of using
%  \cs{tag} doesn't really make sense, does it?} For both keys just
%  typing the name of it chooses true as shown in the following
%  example.
%
%  \begin{verbatim}
%  \mathtoolsset{showonlyrefs,showmanualtags}
%  \usetagform{brackets}
%  \begin{gather}
%    a=a \label{eq:a} \\
%    b=b \label{eq:b} \tag{**}
%  \end{gather}
%  This should refer to the equation containing $a=a$: \eqref{eq:a}.
%  Then a switch of tag forms.
%  \usetagform{default}
%  \begin{align}
%    c&=c \label{eq:c} \\
%    d&=d \label{eq:d}
%  \end{align}
%  This should refer to the equation containing $d=d$: \eqref{eq:d}.
%  \begin{equation}
%    e=e
%  \end{equation}
%  Back to normal.\mathtoolsset{showonlyrefs=false}
%  \begin{equation}
%    f=f
%  \end{equation}
%  \end{verbatim}
%  \mathtoolsset{showonlyrefs,showmanualtags}
%  \usetagform{brackets}
%  \begin{gather}
%    a=a \label{eq:a} \\
%    b=b \label{eq:b} \tag{**}
%  \end{gather}
%  This should refer to the equation containing $a=a$: \eqref{eq:a}.
%  Then a switch of tag forms.
%  \usetagform{default}
%  \begin{align}
%    c&=c \label{eq:c} \\
%    d&=d \label{eq:d}
%  \end{align}
%  This should refer to the equation containing $d=d$: \eqref{eq:d}.
%  \begin{equation}
%    e=e
%  \end{equation}
%  Back to normal.\mathtoolsset{showonlyrefs=false}
%  \begin{equation}
%    f=f
%  \end{equation}
%
%  Note that this feature only works if you use \cs{eqref} or
%  \cs{refeq} to reference your equations.
%
%  When using \key{showonlyrefs} it might be useful to be able to
%  actually add a few equation numbers without directly referring to
%  them.
%  \begin{codesyntax}
%    \SpecialUsageIndex{\noeqref}\cs{noeqref}\marg{label,label,\dots}
%  \end{codesyntax}
%  \FeatureRequest{Rasmus Villemoes}{2008/03/26}
%  The syntax is somewhat similar to \cs{nocite}. If a label in the
%  list is undefined we will throw a warning in the same manner as
%  \cs{ref}. 
%
%  \medskip\noindent\textbf{BUG 1:} Unfortunately the use of the
%  \key{showonlyref} introduce a bug within amsmath's typesetting
%  of formula versus equation number. This bug manifest itself by
%  allowing formulas to be typeset close to or over the equation
%  number.  Currently no general fix is known, other than making sure
%  that one's formulas are not long enough to touch the equation
%  number.
%
%  To make a long story stort, amsmath typesets its math environments
%  twice, one time for measuring and one time for the actual
%  typesetting. In the measuring part, the width of the equation
%  number is recorded such that the formula or the equation number can
%  be moved (if necessary) in the typesetting part. When
%  \key{showonlyref} is enabled, the width of the equation number
%  depend on whether or not this number is referred~to. To determine
%  this, we need to know the current label. But the current label is
%  \emph{not} known in the measuring phase. Thus the measured width is
%  always zero (because no label equals not referred to) and therefore
%  the typesetting phase does not take the equation number into
%  account.
%
% \medskip\noindent\textbf{BUG 2:} Currently there is a bug between
% \key{showonlyrefs} and the \pkg{ntheorem} package, when the
% \pkg{ntheorem} option \key{thmmarks} is active. The shown equation
% numbers may come out wrong (seems to be multiplied by 2). The
% easiest fix is to add the following line
% \begin{verbatim}
% \usepackage[overload,ntheorem]{empheq}
% \end{verbatim}
% before loading \pkg{ntheorem}. The \pkg{empheq} package fixes some
% problems with \pkg{ntheorem} and lets \pkg{mathtools} get correct
% access to the equation numbers again.
% 
%  \subsection{Extensible symbols}
%
%  The number of horizontally extensible symbols in standard \LaTeX{}
%  and \pkg{amsmath} is somewhat low. This part of the manual
%  describes what \pkg{mathtools} does to help this situation.
%
%  \subsubsection{Arrow-like symbols}
%
%
%  \begin{codesyntax}
%    \SpecialUsageIndex{\xleftrightarrow}
%    \cs{xleftrightarrow}\oarg{sub}\marg{sup}\texttt{~~~~~~~~~}
%    \SpecialUsageIndex{\xRightarrow}
%    \cs{xRightarrow}\oarg{sub}\marg{sup}\\
%    \SpecialUsageIndex{\xLeftarrow}
%    \cs{xLeftarrow}\oarg{sub}\marg{sup}\texttt{~~~~~~~~~~~~~~}
%    \SpecialUsageIndex{\xLeftrightarrow}
%    \cs{xLeftrightarrow}\oarg{sub}\marg{sup}\\
%    \SpecialUsageIndex{\xhookleftarrow}
%    \cs{xhookleftarrow}\oarg{sub}\marg{sup}\texttt{~~~~~~~~~~}
%    \SpecialUsageIndex{\xhookrightarrow}
%    \cs{xhookrightarrow}\oarg{sub}\marg{sup}\\
%    \SpecialUsageIndex{\xmapsto}
%    \cs{xmapsto}\oarg{sub}\marg{sup}
%  \end{codesyntax}
%  Extensible arrows are part of \pkg{amsmath} in the form of the
%  commands
%  \begin{quote}
%    \cs{xrightarrow}\oarg{subscript}\marg{superscript}\quad and\\
%    \cs{xleftarrow}\oarg{subscript}\marg{superscript}
%  \end{quote}
%  But what about extensible versions of say, \cs{leftrightarrow} or
%  \cs{Longleftarrow}? It turns out that the above mentioned
%  extensible arrows are the only two of their kind defined by
%  \pkg{amsmath}, but luckily \pkg{mathtools} helps with that. The
%  extensible arrow-like symbols in \pkg{mathtools} follow the same
%  naming scheme as the one's in \pkg{amsmath} so to get an extensible
%  \cs{Leftarrow} you simply do a
%  \begin{verbatim}
%    \[
%      A \xLeftarrow[under]{over} B
%    \]
%  \end{verbatim}
%    \[
%      A \xLeftarrow[under]{over} B
%    \]
%  \begin{codesyntax}
%    \SpecialUsageIndex{\xrightharpoondown}
%    \cs{xrightharpoondown}\oarg{sub}\marg{sup}\texttt{~~~~}
%    \SpecialUsageIndex{\xrightharpoonup}
%    \cs{xrightharpoonup}\oarg{sub}\marg{sup}\\
%    \SpecialUsageIndex{\xleftharpoondown}
%    \cs{xleftharpoondown}\oarg{sub}\marg{sup}\texttt{~~~~~}
%    \SpecialUsageIndex{\xleftharpoonup}
%    \cs{xleftharpoonup}\oarg{sub}\marg{sup}\\
%    \SpecialUsageIndex{\xrightleftharpoons}
%    \cs{xrightleftharpoons}\oarg{sub}\marg{sup}\texttt{~~~}
%    \SpecialUsageIndex{\xleftrightharpoons}
%    \cs{xleftrightharpoons}\oarg{sub}\marg{sup}
%  \end{codesyntax}
%  \pkg{mathtools} also provides the extensible harpoons shown above.
%  They're taken from~\cite{Voss:2004}.
%
%  \subsubsection{Braces and brackets}
%
%  \LaTeX{} defines other kinds of extensible symbols like
%  \cs{overbrace} and \cs{underbrace}, but sometimes you may want
%  another symbol, say, a bracket.
%  \begin{codesyntax}
%    \SpecialUsageIndex{\underbracket}\cs{underbracket}\oarg{rule thickness}
%      \oarg{bracket height}\marg{arg}\\
%    \SpecialUsageIndex{\overbracket}\cs{overbracket}\oarg{rule thickness}
%      \oarg{bracket height}\marg{arg}
%  \end{codesyntax}
%  The commands \cs{underbracket} and \cs{overbracket} are inspired
%  by \cite{Voss:2004}, although the implementation here is slightly
%  different.
%  Used without the optional arguments the bracket commands produce this:
%  \begin{quote}
%   |$\underbracket {foo\ bar}_{baz}$|\quad  $\underbracket {foo\ bar}_{baz}$ \\
%   |$\overbracket {foo\ bar}^{baz}$ |\quad  $\overbracket {foo\ bar}^{baz}$
%  \end{quote}
%  The default rule thickness is equal to that of \cs{underbrace}
%  (app.~$5/18$\,ex) while the default bracket height is equal to
%  app.~$0.7$\,ex. These values give really pleasing results in all
%  font sizes, but feel free to use the optional arguments. That way
%  you may get ``beauties'' like
%  \begin{verbatim}
%    \[
%      \underbracket[3pt]{xxx\  yyy}_{zzz} \quad \text{and} \quad
%      \underbracket[1pt][7pt]{xxx\  yyy}_{zzz}
%    \]
%  \end{verbatim}
%    \[
%      \underbracket[3pt]{xxx\  yyy}_{zzz} \quad \text{and} \quad
%      \underbracket[1pt][7pt]{xxx\  yyy}_{zzz}
%    \]
%  \begin{codesyntax}
%    \SpecialUsageIndex{\underbrace}\cs{underbrace}\marg{arg}\texttt{~~}
%    \SpecialUsageIndex{\LaTeXunderbrace}\cs{LaTeXunderbrace}\marg{arg}\\
%    \SpecialUsageIndex{\overbrace}\cs{overbrace}\marg{arg}\texttt{~~~}
%    \SpecialUsageIndex{\LaTeXoverbrace}\cs{LaTeXoverbrace}\marg{arg}
%  \end{codesyntax}
%  The standard implementation of the math operators \cs{underbrace}
%  and \cs{overbrace} in \LaTeX{} has some deficiencies. For example,
%  all lengths used internally are \emph{fixed} and optimized for
%  10\,pt typesetting. As a direct consequence thereof, using font
%  sizes other than 10 will produce less than optimal results.
%  Another unfortunate feature is the size of the braces. In the
%  example below, notice how the math operator \cs{sum} places its
%  limit compared to \cs{underbrace}.
%  \[
%    \mathop{\thinboxed[blue]{\sum}}_{n}
%    \mathop{\thinboxed[blue]{\LaTeXunderbrace{\thinboxed[green]{foof}}}}_{zzz}
%  \]
%  The blue lines indicate the dimensions of the math operator and
%  the green lines the dimensions of $foof$. As you can see, there
%  seems to be too much space between the brace and the $zzz$ whereas
%  the space between brace and $foof$ is okay. Let's see what happens
%  when we use a bigger font size:\par\Huge\vskip-\baselineskip
%  \[
%    \mathop{\thinboxed[blue]{\sum}}_{n}
%    \mathop{\thinboxed[blue]{\LaTeXunderbrace{\thinboxed[green]{foof}}}}_{zzz}
%  \]
%  \normalsize Now there's too little space between the brace and the
%  $zzz$ and also too little space between the brace and the $foof$.
%  If you use Computer Modern you'll actually see that the $f$
%  overlaps with the brace! Let's try in \cs{footnotesize}:
%  \par\footnotesize
%  \[
%    \mathop{\thinboxed[blue]{\sum}}_{n}
%    \mathop{\thinboxed[blue]{\LaTeXunderbrace{\thinboxed[green]{foof}}}}_{zzz}
%  \]\normalsize
%  Here the spacing above and below the brace is quite excessive.
%
%  As \cs{overbrace} has the exact same problems, there are good
%  reasons for \pkg{mathtools} to make redefinitions of
%  \cs{underbrace} and \cs{overbrace}. These new versions work
%  equally well in all font sizes and fixes the spacing issues and
%  apart from working with the default Computer Modern fonts, they
%  also work with the packages \pkg{mathpazo}, \pkg{pamath},
%  \pkg{fourier}, \pkg{eulervm}, \pkg{cmbright}, and \pkg{mathptmx}.
%  If you use the \pkg{ccfonts} to get the full Concrete fonts, the
%  original version saved under the names \cs{LaTeXunderbrace} and
%  \cs{LaTeXoverbrace} are better, due to of the special design of
%  the Concrete extensible braces. In that case you should probably
%  just add the lines
%  \begin{verbatim}
%    \let\underbrace\LaTeXunderbrace
%    \let\overbrace\LaTeXoverbrace
%  \end{verbatim}
%  to your preamble after loading \pkg{mathtools} which will restore
%  the original definitions of \cs{overbrace} and \cs{underbrace}.
%
%
%  
%
%  \subsection{New mathematical building blocks}
%
%  In this part of the manual, various mathematical environments are
%  described.
%
%  \subsubsection{Matrices}\label{subsubsec:matrices}
%
%  \begin{codesyntax}
%    \SpecialEnvIndex{matrix*}\cs{begin}\arg{matrix*}\texttt{ }\oarg{col}
%        \meta{contents} \cs{end}\arg{matrix*}\\
%    \SpecialEnvIndex{pmatrix*}\cs{begin}\arg{pmatrix*}\oarg{col}
%        \meta{contents} \cs{end}\arg{pmatrix*}\\
%    \SpecialEnvIndex{bmatrix*}\cs{begin}\arg{bmatrix*}\oarg{col}
%        \meta{contents} \cs{end}\arg{bmatrix*}\\
%    \SpecialEnvIndex{Bmatrix*}\cs{begin}\arg{Bmatrix*}\oarg{col}
%        \meta{contents} \cs{end}\arg{Bmatrix*}\\
%    \SpecialEnvIndex{vmatrix*}\cs{begin}\arg{vmatrix*}\oarg{col}
%        \meta{contents} \cs{end}\arg{vmatrix*}\\
%    \SpecialEnvIndex{Vmatrix*}\cs{begin}\arg{Vmatrix*}\oarg{col}
%        \meta{contents} \cs{end}\arg{Vmatrix*}
%  \end{codesyntax}
%  \FeatureRequest{Lars Madsen}{2004/04/05}
%  All of the \pkg{amsmath} \env{matrix} environments center the
%  columns by default, which is not always what you want. Thus
%  \pkg{mathtools} provides a starred version for each of the original
%  environments. These starred environments take an optional argument
%  specifying the alignment of the columns, so that
%  \begin{verbatim}
%    \[
%      \begin{pmatrix*}[r]
%        -1 & 3 \\
%        2  & -4
%      \end{pmatrix*}
%    \]
%  \end{verbatim}
%  yields
%    \[
%      \begin{pmatrix*}[r]
%        -1 & 3 \\
%        2  & -4
%      \end{pmatrix*}
%    \]
%  The optional argument (default is \texttt{[c]}) can be any column
%  type valid in the usual \env{array} environment.
%
%  While we are at it, we also provide fenced versions of the
%  \env{smallmatrix} environment, To keep up with the naming of the
%  large matrix environments, we provide both a starred and a
%  non-starred version. Since \env{smallmatrix} is defined in a
%  different manner than the \env{matrix} environment, the option to
%  say \env{smallmatrix*} \emph{has} to be either \texttt{c},
%  \texttt{l} \emph{or}~\texttt{r}. The default is \texttt{c}, which
%  can be changed globally using the \key{smallmatrix-align}=\meta{c,l
%    or r}.
%  \begin{codesyntax}
%    \SpecialEnvIndex{smallmatrix*}\cs{begin}\arg{smallmatrix*}\texttt{ }\oarg{col}
%        \meta{contents} \cs{end}\arg{smallmatrix*}\\
%    \SpecialEnvIndex{psmallmatrix}\cs{begin}\arg{psmallmatrix}
%        \meta{contents} \cs{end}\arg{psmallmatrix}\\
%    \SpecialEnvIndex{psmallmatrix*}\cs{begin}\arg{psmallmatrix*}\oarg{col}
%        \meta{contents} \cs{end}\arg{psmallmatrix*}\\
%    \SpecialEnvIndex{bsmallmatrix}\cs{begin}\arg{bsmallmatrix}
%        \meta{contents} \cs{end}\arg{bsmallmatrix}\\
%    \SpecialEnvIndex{bsmallmatrix*}\cs{begin}\arg{bsmallmatrix*}\oarg{col}
%        \meta{contents} \cs{end}\arg{bsmallmatrix*}\\
%    \SpecialEnvIndex{Bsmallmatrix}\cs{begin}\arg{Bsmallmatrix}
%        \meta{contents} \cs{end}\arg{Bsmallmatrix}\\
%    \SpecialEnvIndex{Bsmallmatrix*}\cs{begin}\arg{Bsmallmatrix*}\oarg{col}
%        \meta{contents} \cs{end}\arg{Bsmallmatrix*}\\
%    \SpecialEnvIndex{vsmallmatrix}\cs{begin}\arg{vsmallmatrix}
%        \meta{contents} \cs{end}\arg{vsmallmatrix}\\
%    \SpecialEnvIndex{vsmallmatrix*}\cs{begin}\arg{vsmallmatrix*}\oarg{col}
%        \meta{contents} \cs{end}\arg{vsmallmatrix*}\\
%    \SpecialEnvIndex{Vsmallmatrix}\cs{begin}\arg{Vsmallmatrix}
%        \meta{contents} \cs{end}\arg{Vsmallmatrix}\\
%    \SpecialEnvIndex{Vsmallmatrix*}\cs{begin}\arg{Vsmallmatrix*}\oarg{col}
%        \meta{contents} \cs{end}\arg{Vsmallmatrix*}\\
%    \SpecialKeyIndex{smallmatrix-align}\makebox{$\key{smallmatrix-align}=\meta{c,l or r}$}\\
%    \SpecialKeyIndex{smallmatrix-inner-space}\makebox{$\key{smallmatrix-inner-space}=\cs{,}$}
%  \end{codesyntax}
%  \ProvidedBy{Rasmus Villemoes}{2011/01/17}
% \begin{verbatim}
% \[
% \begin{bsmallmatrix}    a & -b \\ -c & d \end{bsmallmatrix}
% \begin{bsmallmatrix*}[r] a & -b \\ -c & d \end{bsmallmatrix*}
% \]
% \end{verbatim}
%  yields
% \[
% \begin{bsmallmatrix} a & -b \\ -c & d \end{bsmallmatrix}
% \begin{bsmallmatrix*}[r] a & -b \\ -c & d \end{bsmallmatrix*}
% \]
% Inside the \verb?Xsmallmatrix? construction a small space is
% inserted between the fences and the contents, the size of it can be
% changed using \key{smallmatrix-align}=\meta{some spacing command},
% the default is \cs{,}.
%
% As an extra trick the fences will behave as open and closing fences
% in constract to their auto-scaling nature.\footnote{\cs{left} and
%   \cs{right} do \emph{not} produce open and closing fences, thus
%   the space before or after may be too large. Inside this
%   construction they behave.}
% 
%  \subsubsection{The \env{multlined} environment}
%
%  \begin{codesyntax}
%    \SpecialEnvIndex{multlined}\cs{begin}\arg{multlined}\oarg{pos}\oarg{width}
%        \meta{contents} \cs{end}\arg{multlined}\\
%    \SpecialUsageIndex{\shoveleft}\cs{shoveleft}\oarg{dimen}\marg{arg}\texttt{~~}
%    \SpecialUsageIndex{\shoveright}\cs{shoveright}\oarg{dimen}\marg{arg}\\
%    \makeatletter\settowidth\@tempdimc{\cs{shoveleft}\oarg{dimen}\marg{arg}}\global\@tempdimc\@tempdimc
%    \SpecialKeyIndex{firstline-afterskip}\makebox[\@tempdimc][l]{$\key{firstline-afterskip}=\meta{dimen}$}\texttt{~~}
%    \SpecialKeyIndex{lastline-preskip}$\key{lastline-preskip}=\meta{dimen}$\\
%    \makeatletter\SpecialKeyIndex{multlined-width}\makebox[\@tempdimc][l]{$\key{multlined-width}=\meta{dimen}$}\texttt{~~}
%    \SpecialKeyIndex{multlined-pos}$\key{multlined-pos}=\texttt{c}\vert\texttt{b}\vert\texttt{t}$
%  \end{codesyntax}
%  Some of the \pkg{amsmath} environments exist in two forms: an
%  outer and an inner environment. One example is the pair
%  \env{gather} \& \env{gathered}. There is one important omission on
%  this list however, as there is no inner \env{multlined}
%  environment, so this is where \pkg{mathtools} steps in.
%
%  One might wonder what the sensible behavior should be. We want it
%  to be an inner environment so that it is not wider than necessary,
%  but on the other hand we would like to be able to control the
%  width. The current implementation of \env{multlined} handles both
%  cases. The idea is this: Set the first line flush left and add a
%  hard space after it; this space is governed by the
%  \key{firstline-afterskip} key. The last line should be set flush
%  right and preceded by a hard space of size \key{lastline-preskip}.
%  Both these hard spaces have a default value of \cs{multlinegap}.
%  Here we use a `t' in the first optional argument denoting a
%  top-aligned building block (the default is `c').
%  \begin{verbatim}
%    \[
%      A = \begin{multlined}[t]
%            \framebox[4cm]{first} \\
%            \framebox[4cm]{last}
%          \end{multlined} B
%    \]
%  \end{verbatim}
%    \[
%      A = \begin{multlined}[t]
%            \framebox[4cm]{first} \\
%            \framebox[4cm]{last}
%          \end{multlined} B
%    \]
%  Note also that \env{multlined} gives you access to an extended
%  syntax for \cs{shoveleft} and \cs{shoveright} as shown in the
%  example below.
%  \begin{verbatim}
%    \[
%      \begin{multlined}
%        \framebox[.65\columnwidth]{First line}        \\
%        \framebox[.5\columnwidth]{Second line}        \\
%        \shoveleft{L+E+F+T}                           \\
%        \shoveright{R+I+G+H+T}                        \\
%        \shoveleft[1cm]{L+E+F+T}                      \\
%        \shoveright[\widthof{$R+I+G+H+T$}]{R+I+G+H+T} \\
%        \framebox[.65\columnwidth]{Last line}
%      \end{multlined}
%    \]
%  \end{verbatim}
%  \[
%    \begin{multlined}
%      \framebox[.65\columnwidth]{First line} \\
%      \framebox[.5\columnwidth]{Second line} \\
%      \shoveleft{L+E+F+T}         \\
%      \shoveright{R+I+G+H+T}         \\
%      \shoveleft[1cm]{L+E+F+T}         \\
%      \shoveright[\widthof{$R+I+G+H+T$}]{R+I+G+H+T}         \\
%      \framebox[.65\columnwidth]{Last line}
%    \end{multlined}
%  \]
%
%  You can also choose the width yourself by specifying it as an
%  optional argument:
%  \begin{verbatim}
%    \[
%      \begin{multlined}[b][7cm]
%        \framebox[4cm]{first} \\
%        \framebox[4cm]{last}
%      \end{multlined} = B
%    \]
%  \end{verbatim}
%    \[
%       \begin{multlined}[b][7cm]
%            \framebox[4cm]{first} \\
%            \framebox[4cm]{last}
%          \end{multlined} = B
%    \]
%  There can be two optional arguments (position and width) and
%  they're interchangeable.
%
%  \subsubsection{More \env{cases}-like environments}
%
%  \begin{codesyntax}
%    \SpecialEnvIndex{dcases}
%    \cs{begin}\arg{dcases}\texttt{~}  \meta{math_column} |&| \meta{math_column}
%    \cs{end}\arg{dcases}\\
%    \SpecialEnvIndex{dcases*}
%    \cs{begin}\arg{dcases*}  \meta{math_column} |&| \makebox[\widthof{\meta{math\_column}}][l]{\meta{text\_column}}
%    \cs{end}\arg{dcases*}\\
%    \SpecialEnvIndex{rcases}
%    \cs{begin}\arg{rcases}\texttt{~~}  \meta{math_column} |&| \makebox[\widthof{\meta{math\_column}}][l]{\meta{math\_column}}
%    \cs{end}\arg{rcases}\\
%    \SpecialEnvIndex{rcases*}
%    \cs{begin}\arg{rcases*}\texttt{~}  \meta{math_column} |&| \makebox[\widthof{\meta{math\_column}}][l]{\meta{text\_column}}
%    \cs{end}\arg{rcases*}\\
%    \SpecialEnvIndex{drcases}
%    \cs{begin}\arg{drcases}\texttt{~}  \meta{math_column} |&| \makebox[\widthof{\meta{math\_column}}][l]{\meta{math\_column}}
%    \cs{end}\arg{drcases}\\
%    \SpecialEnvIndex{drcases*}
%    \cs{begin}\arg{drcases*}  \meta{math_column} |&| \makebox[\widthof{\meta{math\_column}}][l]{\meta{text\_column}}
%    \cs{end}\arg{drcases*}\\
%    \SpecialEnvIndex{cases*}
%    \cs{begin}\arg{cases*}\texttt{~}  \meta{math_column} |&| \makebox[\widthof{\meta{math\_column}}][l]{\meta{text\_column}}
%    \cs{end}\arg{cases*}
%  \end{codesyntax}
%  \FeatureRequest{Lars Madsen}{2004/07/01}
%  Anyone who have tried to use an integral in the regular
%  \env{cases} environment from \pkg{amsmath} will have noticed that
%  it is set as
%  \[
%    a=\begin{cases}
%      E = m c^2     & \text{Nothing to see here} \\
%      \int x-3\, dx & \text{Integral is text style}
%    \end{cases}
%  \]
%  \pkg{mathtools} provides two environments similar to \env{cases}.
%  Using the \env{dcases} environment you get the same output as with
%  \env{cases} except that the rows are set in display style.
%  \begin{verbatim}
%  \[
%    \begin{dcases}
%      E = m c^2     & c \approx 3.00\times 10^{8}\,\mathrm{m}/\mathrm{s} \\
%      \int x-3\, dx & \text{Integral is display style}
%    \end{dcases}
%  \]
%  \end{verbatim}
%  \[
%    \begin{dcases}
%      E = m c^2     & c \approx 3.00\times 10^{8}\,\mathrm{m}/\mathrm{s} \\
%      \int x-3\, dx & \text{Integral is display style}
%    \end{dcases}
%  \]
%  Additionally the environment \env{dcases*} acts just the same, but
%  the second column is set in the normal roman font of the
%  document.\footnote{Or rather: it inherits the font characteristics
%  active just before the \env{dcases*} environment.}
%  \begin{verbatim}
%  \[
%    a= \begin{dcases*}
%      E = m c^2     & Nothing to see here \\
%      \int x-3\, dx & Integral is display style
%    \end{dcases*}
%  \]
%  \end{verbatim}
%  \[
%    a= \begin{dcases*}
%      E = m c^2     & Nothing to see here \\
%      \int x-3\, dx & Integral is display style
%    \end{dcases*}
%  \]
%  The environments \env{rcases}, \env{rcases*}, \env{drcases} and
%  \env{drcases*} are equivalent to \env{cases} and \env{dcases}, but
%  here the brace is placed on the right instead of on the left.
% \begin{verbatim}
% \[
% \begin{rcases*}
%   x^2 & for $x>0$\\ 
%   x^3 & else
% \end{rcases*} \quad \Rightarrow \cdots
% \]
% \end{verbatim}
% \[
% \begin{rcases*}
%   x^2 & for $x>0$\\ 
%   x^3 & else
% \end{rcases*} \quad \Rightarrow\cdots
% \]
%
%
%  \subsubsection{Emulating indented lines in alignments}
%  \begin{codesyntax}
%    \SpecialEnvIndex{\MoveEqLeft}\cs{MoveEqLeft}\oarg{number}
%  \end{codesyntax}
%  \ProvidedBy{Lars Madsen}{2008/06/05} In \cite{Swanson}, Ellen
%  Swanson recommends that when ever one has a long displayed formula,
%  spanning several lines, and it is unfeasible to align against a
%  relation within the first line, then all lines in the display
%  should be aligned at the left most edge of the first line, and all
%  subsequent lines should be indented by 2\,em (or if needed by a
%  smaller amount). That is we are talking about displayes that end up
%  looking like this
%  \begin{align*}
%    \MoveEqLeft \framebox[10cm][c]{Long first line}\\
%    & = \framebox[6cm][c]{ \hphantom{g} 2nd line}\\ 
%    & \leq \dots
%  \end{align*}
%  Traditionally one could do this by starting subsequent lines by
%  \verb+&\qquad ...+, but that is tedious. Instead the example above
%  was made using \cs{MoveEqLeft}:
%  \begin{verbatim}
%  \begin{align*}
%    \MoveEqLeft \framebox[10cm][c]{Long first line}\\
%    & = \framebox[6cm][c]{ \hphantom{g} 2nd line}\\ 
%    & \leq \dots
%  \end{align*}
%  \end{verbatim}
%  \cs{MoveEqLeft} is placed instead of the \verb+&+ on the first
%  line, and will effectively \emph{move} the entire first line
%  \oarg{number} of ems to the left (default is 2). If you choose to
%  align to the right of the relation, use \cs{MoveEqLeft}\verb+[3]+
%  to accommodate the extra distance to the alignment point:
%  \begin{verbatim}
%  \begin{align*}
%    \MoveEqLeft[3] \framebox[10cm][c]{Part 1}\\
%     = {} & \framebox[8cm][c]{2nd line}\\ 
%          & + \framebox[4cm][c]{ last part}
%  \end{align*}
%  \end{verbatim}
%  \begin{align*}
%    \MoveEqLeft[3] \framebox[10cm][c]{Long first line}\\
%     = {} & \framebox[6cm][c]{  2nd line}\\ 
%          & + \framebox[4cm][c]{ last part}
%  \end{align*}
%
%  \subsubsection{Boxing a single line in an alignment}
%  
%  The \texttt{amsmath} package provie the \cs{boxed} macro to box
%  material in math mode. But this of course will not work if the box
%  should cross an alignment point. We provide a macro that
%  can.\footnote{Note that internally \cs{Aboxed} does use \cs{boxed}.}
%    \hskip1sp
%   \marginpar{%
%    \parbox[b]{\marginparwidth}{\small\sffamily\raggedright
%      \strut Evolved from a request by\\Merciadri Luca\\
%       2010/06/28\\on comp.text.tex%
%    }\strut
%  }%
%   \marginpar{\strut\\%
%    \parbox[b]{\marginparwidth}{\small\sffamily\raggedright
%      \strut Reimplemented by\\Florent Chervet (GL) \\
%       2011/06/11\\on comp.text.tex%
%    }\strut
%  }%
%  \begin{codesyntax}
%    \SpecialEnvIndex{\Aboxed}\cs{Aboxed}\marg{left hand side 
%     \quad\texttt{\textnormal{\&}}\quad right hand side}
%  \end{codesyntax}
%  Example
% \begin{verbatim}
% \begin{align*}
%   \Aboxed{ f(x) & = \int h(x)\, dx} \\
%                 & = g(x)
% \end{align*}
% \end{verbatim}
% Resulting in:
% \begin{align*}
%   \Aboxed{ f(x) & = \int h(x)\, dx} \\
%                 & = g(x)
% \end{align*}
% One can have multiple boxes on each line, and the
% >>\texttt{\textnormal{\&}}\quad right hand side<< can even be
% missing. Here is an example of how the padding in the box can be changed
% \begin{verbatim}
% \begin{align*}
%   \setlength\fboxsep{1em}
%   \Aboxed{ f(x) &= 0 } & \Aboxed{ g(x) &= b} \\
%   \Aboxed{ h(x) }      & \Aboxed{ i(x) }   
% \end{align*}
% \end{verbatim}
% \begin{align*}
%   \setlength\fboxsep{1em}
%   \Aboxed{ f(x) &= 0 } & \Aboxed{ g(x) &= b} \\
%   \Aboxed{ h(x) }      & \Aboxed{ i(x) }   
% \end{align*}
% Note how the \cs{fboxsep} change only affect the box coming
% immediately after it.  
%
%  \subsubsection{Adding arrows between lines in an alignment}
%
%  This first macro is a bit misleading, it is only intended to be
%  used in combination with the \env{alignat(*)} environment.
%  \begin{codesyntax}
%    \SpecialEnvIndex{\ArrowBetweenLines}\cs{ArrowBetweenLines}\oarg{symbol}\\
%    \SpecialEnvIndex{\ArrowBetweenLines*}\cs{ArrowBetweenLines*}\oarg{symbol}
%  \end{codesyntax}
%    \hskip1sp
%   \marginpar{%
%    \parbox[b]{\marginparwidth}{\small\sffamily\raggedright
%      \strut Evolved from a request by\\Christian
%      Bohr-Halling\\2004/03/31\\on dk.edb.tekst%
%    }
%  }%
%  To add, say $\Updownarrow$ between two lines in an alignment use
%  \cs{ArrowBetweenLines} and the \env{alignat} environment (note the
%  extra pair of  \texttt{\&}'s in front):
%  \begin{verbatim}
%  \begin{alignat}{2}
%    && \framebox[1.5cm]{} &= \framebox[3cm]{}\\
%    \ArrowBetweenLines % \Updownarrow is the default
%    && \framebox[1.5cm]{} &= \framebox[2cm]{}
%  \end{alignat}
%  \end{verbatim}
%  resulting in
%  \begin{alignat}{2}
%    && \framebox[1.5cm]{} &= \framebox[3cm]{}\\
%    \ArrowBetweenLines 
%    && \framebox[1.5cm]{} &= \framebox[2cm]{}
%  \end{alignat}
%  Note the use of \verb+&&+ starting each \emph{regular} line of
%  math. For adding the arrow on the right, use
%  \cs{ArrowBetweenLines*}\oarg{symbol}, and end each line of math
%  with \verb+&&+.
%  \begin{verbatim}
%  \begin{alignat*}{2}
%    \framebox[1.5cm]{} &= \framebox[3cm]{}  &&\\
%    \ArrowBetweenLines*[\Downarrow] 
%    \framebox[1.5cm]{} &= \framebox[2cm]{}  &&
%  \end{alignat*}
%  \end{verbatim}
%  resulting in
%  \begin{alignat*}{2}
%    \framebox[1.5cm]{} &= \framebox[3cm]{}  &&\\
%    \ArrowBetweenLines*[\Downarrow] 
%    \framebox[1.5cm]{} &= \framebox[2cm]{}  &&
%  \end{alignat*}
%
%
% \subsubsection{Centered \cs{vdots}}
%
%  If one want to mark a vertical continuation, there is
%  the \verb?\vdots? command, but combine this with an alignment and
%  we get something rather suboptimal
% \FeatureRequest{Bruno Le Floch \\(and many others)}{2011/01/25}
%  \begin{align*}
%    \framebox[1.5cm]{} &= \framebox[3cm]{}\\
%                       & \vdots\\
%                       &= \framebox[3cm]{} 
%  \end{align*}
%  It would be nice to have (1) a \verb?\vdots? centered within the
%  width of another symbol, and (2) a construction similar to
%  \verb?\ArrowBetweenLines? that does not take up so much space. 
%  We provide both.
%  \begin{codesyntax}
%    \SpecialUsageIndex{\vdotswithin}\cs{vdotswithin}\marg{symbol}\\
%    \SpecialUsageIndex{\shortvdotswithin}\cs{shortvdotswithin}\marg{symbol}\\
%    \SpecialUsageIndex{\shortvdotswithin*}\cs{shortvdotswithin*}\marg{symbol}\\
%    \SpecialUsageIndex{\MTFlushSpaceAbove}\cs{MTFlushSpaceAbove}\\
%    \SpecialUsageIndex{\MTFlushSpaceBelow}\cs{MTFlushSpaceBelow}\\
%    \SpecialKeyIndex{shortvdotsadjustabove}\makebox{$\key{shortvdotsadjustabove}=\meta{length}$}\\
%    \SpecialKeyIndex{shortvdotsadjustbelow}\makebox{$\key{shortvdotsadjustbelow}=\meta{length}$}
%  \end{codesyntax}
%  Two examples in one
% \begin{verbatim}
% \begin{align*}
%   a &= b              \\
%     & \vdotswithin{=} \\
%     & = c             \\
%     \shortvdotswithin{=}
%     & = d
% \end{align*}
% \end{verbatim}
% yielding
% \begin{align*}
%   a &= b              \\
%     & \vdotswithin{=} \\
%     & = c             \\
%     \shortvdotswithin{=}
%     & = d
% \end{align*}
% Thus \verb?\vdotswithin{=}? create a box corersponding to
% \verb?{}={}? and typeset a >>$\vdots$<< centered inside it. When doing
% this as a normal line in an alignment leaves us with excessive space
% which \verb?\shortvdotswithin{=}? takes care with for us.
%
% \verb?\shortvdotswithin{=}? corresponds to
% \begin{verbatim}
% \MTFlushSpaceAbove
% & \vdotswithin{=} \\
% \MTFlushSpaceBelow
% \end{verbatim}
% whereas \verb?\shortvdotswithin*{=}? is the case with 
% \verb?\vdotswithin{=} & \\?. This also means one cannot write more
% on the line when using \verb?\shortvdotswithin? or the starred
% version. But one can de-construct the macro and arrive at
% \begin{verbatim}
% \begin{alignat*}{3}
%   A&+ B &&= C &&+ D \\
%   \MTFlushSpaceAbove
%   &\vdotswithin{+} &&&& \vdotswithin{+}
%   \MTFlushSpaceBelow
%   C &+ D &&= Y &&+K
% \end{alignat*}
% \end{verbatim}
% yielding
% \begin{alignat*}{3}
%   A&+ B &&= C &&+ D \\
%   \MTFlushSpaceAbove
%   &\vdotswithin{+} &&&& \vdotswithin{+}
%   \MTFlushSpaceBelow
%   C &+ D &&= Y &&+K
% \end{alignat*}
% If one has the need for such a construction.
%
% The de-spaced version does support the \env{spreadlines}
% environment. The actual amount of space being \emph{flushed} above
% and below can be controlled by the user using the two options
% indicated. Their original values are \verb?2.15\origjot? and
% \verb?\origjot? respectively (\verb?\origjot? is usually 3pt). 
%
%  \subsection{Intertext and short intertext}
%
%
%  \begin{codesyntax}
%    \SpecialUsageIndex{\shortintertext}\cs{shortintertext}\marg{text}
%  \end{codesyntax}
%  \cttPosting{Gabriel Zachmann and Donald Arseneau}{2000/05/12--13}
%  \pkg{amsmath} provides the command \cs{intertext} for interrupting
%  a multiline display while still maintaining the alignment points.
%  However the spacing often seems quite excessive as seen below.
%  \begin{verbatim}
%    \begin{align}
%      a&=b \intertext{Some text}
%      c&=d
%    \end{align}
%  \end{verbatim}
%    \begin{align}
%      a&=b \intertext{Some text}
%      c&=d
%    \end{align}
%
%  Using the command \cs{shortintertext} alleviates this situation
%  somewhat:
%  \begin{verbatim}
%    \begin{align}
%      a&=b \shortintertext{Some text}
%      c&=d
%    \end{align}
%  \end{verbatim}
%  \begin{align}
%    a&=b \shortintertext{Some text}
%    c&=d
%  \end{align}
%
%  \noindent
%  It turns out that both \cs{shortintertext} and the original
%  \cs{intertext} from \pkg{amsmath} has a slight problem. If we use
%  the \env{spreadlines} (see section~\ref{sec:spread}) to open up
%  the equations in a multiline calculation, then this opening up
%  value also applies to the spacing above and below the original
%  \cs{shortintertext} and \cs{intertext}.  \tsxPosting{Tobias Weh
%    \\(referring to a suggestion by Chung-chieh Shan)}{2011/05/29}
% It can be illustrated using the following example, an interested
% reader, can apply it with and with out the original \cs{intertext}
% and \cs{shortintertext}.
% \begin{verbatim}
% % the original \intertext and \shortintertext
% \mathtoolsset{original-intertext,original-shortintertext}
% \newcommand\myline{\par\noindent\rule{\textwidth}{1mm}} 
% \myline
% \begin{spreadlines}{1em}
%   \begin{align*}
%     AA\\  BB\\  \intertext{\myline}
%     AA\\  BB\\  \shortintertext{\myline}
%     AA\\  BB
%   \end{align*}
% \end{spreadlines}
% \myline
% \end{verbatim}
%
%  We now fix this internaly for both \cs{intertext} and
%  \cs{shortintertext}, plus we add the posibility to fine tune
%  spacing around these constructions. The original versions can be
%  brought back using the \texttt{original-x} keys below.
%  \begin{codesyntax}
%    \SpecialUsageIndex{\intertext}\cs{intertext}\marg{text}\\
%    \SpecialUsageIndex{\shortintertext}\cs{shortintertext}\marg{text}\\
%    \SpecialKeyIndex{original-intertext}$\key{original-intertext}=\texttt{true}\vert\texttt{false}$ \quad(default: \texttt{false})\\
%    \SpecialKeyIndex{original-shortintertext}$\key{original-shortintertext}=\texttt{true}\vert\texttt{false}$ 
%    \quad(default: \texttt{false})\\ 
%    \SpecialKeyIndex{above-intertext-sep}$\key{above-intertext-sep}=\meta{dimen}$ \quad(default: 0pt)\\
%    \SpecialKeyIndex{below-intertext-sep}$\key{below-intertext-sep}=\meta{dimen}$ \quad(default: 0pt)\\
%    \SpecialKeyIndex{above-shortintertext-sep}$\key{above-shortintertext-sep}=\meta{dimen}$ \quad(default: 3pt)\\
%    \SpecialKeyIndex{below-shortintertext-sep}$\key{below-shortintertext-sep}=\meta{dimen}$ \quad(default: 3pt)
%  \end{codesyntax}
%  The updated \cs{shortintertext} will look like the original version
%  unless for areas with an enlarged \cs{jot} value (see for example
%  the \env{spreadlines}, section~\ref{sec:spread}). Whereas \cs{intertext}
%  will have a slightly smaller value above and below (corresponding
%  to about 3pt less space above and below), the spacing around
%  \cs{intertext} should now match the normal spacing going into and
%  out of an \env{align}.
%
% \textbf{Tip:} \cs{intertext} and \cs{shortintertext} also works
% within \env{gather}.
%
%  \subsection{Paired delimiters}
%
%
%  \begin{codesyntax}
%    \SpecialUsageIndex{\DeclarePairedDelimiter}
%    \cs{DeclarePairedDelimiter}\marg{cmd}\marg{left_delim}\marg{right_delim}
%  \end{codesyntax}
%  \FeatureRequest{Lars Madsen}{2004/06/25}
%  In the \pkg{amsmath} documentation it is shown how to define a few
%  commands for typesetting the absolute value and norm. These
%  definitions are:
%  \begin{verbatim}
%    \newcommand*\abs[1]{\lvert#1\rvert}
%    \newcommand*\norm[1]{\lVert#1\rVert}
%  \end{verbatim}
%  \DeclarePairedDelimiter\abs\lvert\rvert
%  While they produce correct horizontal spacing you have to be
%  careful about the vertical spacing if the argument is just a
%  little taller than usual as in
%  \[
%    \abs{\frac{a}{b}}
%  \]
%  Here it won't give a nice result, so you have to manually put in
%  either \cs{left}--\cs{right} pair or a \cs{bigl}--\cs{bigr} pair.
%  Both methods mean that you have to delete your \cs{abs} command,
%  which may not sound like an ideal solution.
%
%  With the command \cs{DeclarePairedDelimiter} you can combine all
%  these features in one easy to use command. Let's show an example:
%  \begin{verbatim}
%    \DeclarePairedDelimiter\abs{\lvert}{\rvert}
%  \end{verbatim}
%  This defines the command \cs{abs} just like in the
%  \pkg{amsmath} documentation but with a few additions:
%  \begin{itemize}
%    \item A starred variant: \cs{abs*} produces delimiters that are preceded
%  by \cs{left} and \cs{right} resp.:
%  \begin{verbatim}
%  \[
%    \abs*{\frac{a}{b}}
%  \]
%  \end{verbatim}
%      \[
%        \abs*{\frac{a}{b}}
%      \]
%  \item A variant with an optional argument:
%  \cs{abs}\oarg{size_cmd}, where
%  \meta{size_cmd} is either \cs{big}, \cs{Big}, \cs{bigg}, or
%  \cs{Bigg} (if you have any bigggger versions you can use them
%  too).
%  \begin{verbatim}
%  \[
%    \abs[\Bigg]{\frac{a}{b}}
%  \]
%  \end{verbatim}
%      \[
%        \abs[\Bigg]{\frac{a}{b}}
%      \]
%  \end{itemize}
%
%  \begin{codesyntax}
%    \SpecialUsageIndex{\DeclarePairedDelimiterX}
%    \cs{DeclarePairedDelimiterX}\marg{cmd}\oarg{num args}\marg{left_delim}\marg{right_delim}\marg{code}\\
%     \cs{delimsize}
%  \end{codesyntax}
%  \ProvidedBy{Lars Madsen}{2010/06/15} 
%  Sometimes \cs{DeclarePairedDelimiter} just is not enough.  One
%  might want to have the capabilities of \cs{DeclarePairedDelimiter},
%  but also want a macro the takes more than one argument.
%
%  \cs{DeclarePairedDelimiterX} extends the features of
%  \cs{DeclarePairedDelimiter} such that the user will get a macro
%  which is fenced off at either end, plus the capability to provide
%  the code for what ever the macro should do within these fences. 
%
%  Inside the \meta{code} part, the macro \cs{delimsize} refer to the
%  size of the outer fences. It can then be used inside \meta{code} to
%  scale any inner fences.
%
%  In this setting
%  \cs{DeclarePairedDelimiter}\marg{cmd}\marg{left_delim}\marg{right_delim} is the same thing as 
%  \begin{center}
%    \cs{DeclarePairedDelimiterX}\marg{cmd}\verb|[1]|\marg{left_delim}\marg{right_delim}\verb|{#1}|
%  \end{center}
%  %
%  Let us do some examples. First we want to prepare a macro for inner
%  products, with two arguments such that we can hide the character
%  separating the arguments (a journal style might require a
%  semi-colon, so we will save a lot of hand editing). This can be
%  done via
% \begin{verbatim}
% \DeclarePairedDelimiterX\innerp[2]{\langle}{\rangle}{#1,#2}
% \end{verbatim}
% More interestingly we can refer to the size inside the
% \meta{code}. Here we do a weird three argument `braket'
% \begin{verbatim}
% \DeclarePairedDelimiterX\braket[3]{\langle}{\rangle}%
% {#1\,\delimsize\vert\,#2\,\delimsize\vert\,#3}
% \end{verbatim}
% Then we can get
% \DeclarePairedDelimiterX\innerp[2]{\langle}{\rangle}{#1,#2}
% \DeclarePairedDelimiterX\braket[3]{\langle}{\rangle}%
% {#1\,\delimsize\vert\,#2\,\delimsize\vert\,#3}
% \begin{verbatim}
% \[
% \innerp*{A}{ \frac{1}{2} } \quad
% \braket[\Big]{B}{\sum_{k} f_k}{C}
% \]
% \end{verbatim}
% \[
% \innerp*{A}{ \frac{1}{2} } \quad
% \braket[\Big]{B}{\sum_{k}}{C}
% \]
% \iffalse
% \bigskip
% 
% \noindent
% \textbf{\textit{Side note:}} We have changed the internal code of
% \cs{DeclarePairedDelimiter} and \cs{DeclarePairedDelimiterX} such
% that the starred version does no longer give the odd spacings that
% \cs{left}\dots\cs{right} sometimes give. Compare this
% \begin{verbatim}
% \[
% 2\innerp{A}{B} \quad  2\innerp*{A}{B} \quad 2\left\langle A,B \right\rangle
% \]
% \end{verbatim}
% \[
% \sin\innerp{A}{B} \quad
% \sin\innerp*{A}{B} \quad
% \sin\left\langle A,B \right\rangle
% \]
% The spacing in the last one does not behave like other fences (this
% is a feature of the \cs{left}\dots\cs{right} construction).
% \fi
%
%
%  \subsubsection{Expert use}
%
%  Within the starred version of \cs{DeclarePairedDelimiter} and
%  \cs{DeclarePairedDelimiterX} we make a few changes such that the
%  auto scaled \cs{left} and \cs{right} fences behave as opening and
%  closing fences, i.e.\ $\sin(x)$ vs. $\sin\left(x\right)$ (the later
%  made via \verb|$\sin\left(x\right)$|), notice the gab between
%  '$\sin$' and '('.  In some special cases it may be useful to be
%  able to tinker with the behavior.
%  \begin{codesyntax}
%    \SpecialUsageIndex{\reDeclarePairedDelimiterInnerWrapper}\cs{reDeclarePairedDelimiterInnerWrapper}\marg{macro name}\marg{\textnormal{\texttt{star}} or \textnormal{\texttt{nostar}}}\marg{code}
%  \end{codesyntax}
%  Internally several macros are created, including two call backs
%  that take care of the formatting. There is one internal macro for
%  the starred version, labeled \texttt{star}, the other one is
%  labeled \texttt{nostar}. Within \meta{code}, \texttt{\#1} will be
%  replaced by the scaled left fence, \texttt{\#3} the corresponding
%  scaled right fence, and \texttt{\#2} the stuff in between. For
%  example, here is how one might turn the content into \cs{mathinner}:
% \begin{verbatim}
% \DeclarePairedDelimiter\abs\lvert\rvert
% \reDeclarePairedDelimiterInnerWrapper\abs{star}{#1#2#3}
% \reDeclarePairedDelimiterInnerWrapper\abs{nostar}{\mathinner{#1#2#3}}
% \end{verbatim}
%  The default values for the call backs corresponds to
% \begin{verbatim}
% star:   \mathopen{}\mathclose\bgroup #1#2\aftergroup\egroup #3
% nostar: \mathopen{#1}#2\mathclose{#3}
% \end{verbatim}
%
%
%  \subsection{Special symbols}
%
%  This part of the manual is about special symbols. So far only one
%  technique is covered, but more will come.
%
%  \subsubsection{Left and right parentheses}
%
%  \begin{codesyntax}
%    \SpecialUsageIndex{\lparen}\cs{lparen}\texttt{~~}
%    \SpecialUsageIndex{\rparen}\cs{rparen}
%  \end{codesyntax}
%  When you want a big parenthesis or bracket in a math display you
%  usually just type
%  \begin{quote}
%    |\left( ... \right)|\quad  or\quad |\left[ ... \right]|
%  \end{quote}
%  \LaTeX{} also defines the macro names \cs{lbrack} and \cs{rbrack}
%  to be shorthands for the left and right square bracket resp., but
%  doesn't provide similar definitions for the parentheses. Some
%  packages need command names to work with\footnote{The \pkg{empheq}
%  package needs command names for delimiters in order to make
%  auto-scaling versions.} so \pkg{mathtools} defines the commands
%  \cs{lparen} and \cs{rparen} to represent the left and right
%  parenthesis resp.
%
%
%  \subsubsection{Vertically centered colon}
%
%  \begin{codesyntax}
%    \SpecialKeyIndex{centercolon}$\key{centercolon}=\texttt{true}\vert\texttt{false}$\\
%    \SpecialUsageIndex{\vcentcolon}\cs{vcentcolon}\texttt{~~}
%    \SpecialUsageIndex{\ordinarycolon}\cs{ordinarycolon}
%  \end{codesyntax}
%  \cttPosting{Donald Arseneau}{2000/12/07}
%  When trying to show assignment operations as in $ a := b $, one
%  quickly notices that the colon is not centered on the math axis as
%  the equal sign, leading to an odd-looking output. The command
%  \cs{vcentcolon} is a shorthand for such a vertically centered
%  colon, and can be used as in |$a \vcentcolon= b$| and results in
%  the desired output:  $a \vcentcolon= b$. % for now
%
%  Typing \cs{vcentcolon} every time is quite tedious, so one can use
%  the key \key{centercolon} to make the colon active instead.
%  \begin{verbatim}
%  \mathtoolsset{centercolon}
%  \[
%    a := b
%  \]
%  \mathtoolsset{centercolon=false}
%  \end{verbatim}
%  \[\mathtoolsset{centercolon}
%    a := b
%  \]
%  In this case the command \cs{ordinarycolon} typesets an~\ldots\
%  ordinary colon (what a surprise).
%
%  \medskip
%  \noindent\textbf{Warning:} \texttt{centercolon} \emph{does not}
%  work with languages that make use of an active colon, most notably
%  \emph{French}. Sadly the \texttt{babel} package does not distinguish
%  between text and math when it comes to active characters. Nor does
%  it provide any hooks to deal with math. So currently no general
%  solution exists for this problem.
%
%  \begin{codesyntax}
%    \SpecialUsageIndex{\coloneqq}\cs{coloneqq}\texttt{~~~~~}
%    \SpecialUsageIndex{\Coloneqq}\cs{Coloneqq}\texttt{~~~~~}
%    \SpecialUsageIndex{\coloneq}\cs{coloneq}\texttt{~~~}
%    \SpecialUsageIndex{\Coloneq}\cs{Coloneqq}\\
%    \SpecialUsageIndex{\eqqcolon}\cs{eqqcolon}\texttt{~~~~~}
%    \SpecialUsageIndex{\Eqqcolon}\cs{Eqqcolon}\texttt{~~~~~}
%    \SpecialUsageIndex{\eqcolon}\cs{eqcolon}\texttt{~~~}
%    \SpecialUsageIndex{\Eqcolon}\cs{Eqcolon}\\
%    \SpecialUsageIndex{\colonapprox}\cs{colonapprox}\texttt{~~}
%    \SpecialUsageIndex{\Colonapprox}\cs{Colonapprox}\texttt{~~}
%    \SpecialUsageIndex{\colonsim}\cs{colonsim}\texttt{~~}
%    \SpecialUsageIndex{\Colonsim}\cs{Colonsim}\\
%    \SpecialUsageIndex{\dblcolon}\cs{dblcolon}
%  \end{codesyntax}
%  The font packages \pkg{txfonts} and \pkg{pxfonts} provides various
%  symbols that include a vertically centered colon but with tighter
%  spacing. For example, the combination |:=| exists as the symbol
%  \cs{coloneqq} which typesets as $\coloneqq$ instead of
%  $\vcentcolon=$. The primary disadvantage of using these fonts are
%  the support packages' lack of support for \pkg{amsmath} (and thus
%  \pkg{mathtools}) and worse yet, the side-bearings are way too
%  tight; see~\cite{A-W:MG04} for examples. If you're not using these
%  fonts, \pkg{mathtools} provides the symbols for you. Here are a few
%  examples:
%  \begin{verbatim}
%  \[
%    a \coloneqq b \quad c \Colonapprox d \quad e \dblcolon f
%  \]
%  \end{verbatim}
%  \[
%    a \coloneqq b \quad c \Colonapprox d \quad e \dblcolon f
%  \]
%
%
%
%  \subsubsection{A few missing symbols}
%
%  Most provided math font sets are missing the symbols \cs{nuparrow}
%  and \cs{ndownarrow} (i.e.\ negated up- and downarrow) plus a `big'
%  version of \cs{times}. Therefore we will provide constructed
%  versions of these whenever they are not already available.
%  \begin{codesyntax}
%    \SpecialUsageIndex{\nuparrow}\cs{nuparrow}\\
%    \SpecialUsageIndex{\ndownarrow}\cs{ndownarrow}\\
%    \SpecialUsageIndex{\bigtimes}\cs{bigtimes}
%  \end{codesyntax}
%
%  \noindent
%  \textbf{Note:} that these symbols are constructed via
%  features from the \pkg{graphicx} package, and thus may not display
%  correctly in most DVI previewers. Also note that \cs{nuparrow} and
%  \cs{ndownarrow} are constructed via \cs{nrightarrow} and
%  \cs{nleftarrrow} respectively, so these needs to be
%  present. Usually this is done via \pkg{amssymb}, but some packages
%  may be incompatible with \pkg{amssymb} so the user will have to
%  load \pkg{amssymb} or a similar package, that provides
%  \cs{nrightarrow} and \cs{nleftarrow}, themselves. 
%
%  With those requirements in place, we have
%  \begin{verbatim}
%    \[
%       \lim_{a\ndownarrow 0} f(a) \neq \bigtimes_n X_n \qquad
%       \frac{ \bigtimes_{k=1}^7 B_k \nuparrow \Omega }{2}     
%    \]
%  \end{verbatim}
%    \[
%       \lim_{a\ndownarrow 0} f(a) \neq \bigtimes_n X_n \qquad
%       \frac{ \bigtimes_{k=1}^7 B_k \nuparrow \Omega }{2}     
%    \]
%
%
%
%
%  \section{A tribute to Michael J.~Downes}
%
%  Michael J.~Downes (1958--2003) was one of the major architects
%  behind \pkg{amsmath} and member of the \LaTeX{} Team. He made many
%  great contributions to the \TeX{} community; not only by the means
%  of widely spread macro packages such as \pkg{amsmath} but also in
%  the form of actively giving advice on newsgroups. Some of
%  Michael's macro solutions on the newsgroups never made it into
%  publicly available macro packages although they certainly deserved
%  it, so \pkg{mathtools} tries to rectify this matter. The macros
%  described in this section are either straight copies or heavily
%  inspired by his original postings.
%
%  \subsection{Mathematics within italic text}
%
%  \begin{codesyntax}
%    \SpecialKeyIndex{mathic}$\key{mathic}=\texttt{true}\vert\texttt{false}$
%  \end{codesyntax}
%  \cttPosting{Michael J.~Downes}{1998/05/14}
%  \TeX{} usually takes care of italic corrections in text, but fails
%  when it comes to math. If you use the \LaTeX{} inline math
%  commands \cs{(} and \cs{)} you can however work around it by
%  setting the key \key{mathic} to true as shown below.
%  \begin{verbatim}
%    \begin{quote}\itshape
%    Compare these lines: \par
%    \mathtoolsset{mathic} % or \mathtoolsset{mathic=true}
%    Subset of \(V\) and subset of \(A\). \par
%    \mathtoolsset{mathic=false}
%    Subset of \(V\) and subset of \(A\).
%    \par
%    \end{quote}
%  \end{verbatim}
%  \begin{quote}\itshape
%  Compare these lines: \par
%  \mathtoolsset{mathic}
%  Subset of \(V\) and subset of \(A\). \par
%  \mathtoolsset{mathic=false}
%  Subset of \(V\) and subset of \(A\).
%  \par
%  \end{quote}
%
%  \noindent
%  It is recommended to load the \pkg{fixltx2e} package \emph{after}
%  \pkg{mathtools} and the \verb|\mathtoolsset{mathic=true}| option,
%  as it will make \verb|\(...\)| robust, while maintaining the added
%  italic correction.
%
%  \subsection{Left sub/superscripts}
%
%  \begin{codesyntax}
%    \SpecialUsageIndex{\prescript}
%        \cs{prescript}\marg{sup}\marg{sub}\marg{arg}\texttt{~~}
%    \SpecialKeyIndex{prescript-sup-format}
%        $\key{prescript-sup-format}=\meta{cmd}$\\
%    \SpecialKeyIndex{prescript-sub-format}
%        $\key{prescript-sub-format}=\meta{cmd}$\hfill
%    \SpecialKeyIndex{prescript-arg-format}
%        \rlap{$\key{prescript-arg-format}=\meta{cmd}$}^^A
%        \phantom{$\key{prescript-sup-format}=\meta{cmd}$}
%  \end{codesyntax}
%  \cttPosting{Michael J.~Downes}{2000/12/20}
%  Sometimes one wants to put a sub- or superscript on the left of
%  the argument. The \cs{prescript} command does just that:
%  \begin{verbatim}
%    \[
%      {}^{4}_{12}\mathbf{C}^{5+}_{2}          \quad
%      \prescript{14}{2}{\mathbf{C}}^{5+}_{2}  \quad
%      \prescript{4}{12}{\mathbf{C}}^{5+}_{2}  \quad
%      \prescript{14}{}{\mathbf{C}}^{5+}_{2}   \quad
%      \prescript{}{2}{\mathbf{C}}^{5+}_{2}
%    \]
%  \end{verbatim}
%  \[
%    {}^{4}_{12}\mathbf{C}^{5+}_{2}          \quad
%    \prescript{14}{2}{\mathbf{C}}^{5+}_{2}  \quad
%    \prescript{4}{12}{\mathbf{C}}^{5+}_{2}  \quad
%    \prescript{14}{}{\mathbf{C}}^{5+}_{2}   \quad
%    \prescript{}{2}{\mathbf{C}}^{5+}_{2}
%  \]
%
%  The formatting of the arguments is controlled by three keys. This
%  silly example shows you how to use them:
%  \begin{verbatim}
%  \newcommand*\myisotope[3]{%
%    \begingroup % to keep changes local. We cannot use a brace group
%                % as it affects spacing!
%      \mathtoolsset{
%        prescript-sup-format=\mathit,
%        prescript-sub-format=\mathbf,
%        prescript-arg-format=\mathrm,
%      }%
%    \prescript{#1}{#2}{#3}%
%    \endgroup
%  }
%  \[
%    \myisotope{A}{Z}{X}\to \myisotope{A-4}{Z-2}{Y}+
%    \myisotope{4}{2}{\alpha}
%  \]
%  \end{verbatim}
%  \newcommand*\myisotope[3]{%
%    \begingroup
%      \mathtoolsset{
%        prescript-sup-format=\mathit,
%        prescript-sub-format=\mathbf,
%        prescript-arg-format=\mathrm,
%      }%
%    \prescript{#1}{#2}{#3}%
%    \endgroup
%  }
%  \[
%    \myisotope{A}{Z}{X}\to \myisotope{A-4}{Z-2}{Y}+
%    \myisotope{4}{2}{\alpha}
%  \]
% (Though a package like \pkg{mhchem} might be more suitable for this
% type of material.)
%
%  \subsection{Declaring math sizes}
%
%  \begin{codesyntax}
%    \SpecialUsageIndex{\DeclareMathSizes}
%    \cs{DeclareMathSizes}\marg{dimen}\marg{dimen}\marg{dimen}\marg{dimen}
%  \end{codesyntax}
%  \cttPosting{Michael J.~Downes}{2002/10/17}
%  If you don't know about \cs{DeclareMathSizes}, then skip the rest
%  of this text. If you do know, then all that is needed to say is
%  that with \pkg{mathtools} it is patched so that all regular
%  dimension suffixes are now valid in the last three arguments. Thus
%  a declaration such as
%  \begin{verbatim}
%    \DeclareMathSize{9.5dd}{9.5dd}{7.5dd}{6.5dd}
%  \end{verbatim}
%  will now work (it doesn't in standard \LaTeX). When this bug has
%  been fixed in \LaTeX, this fix will be removed from
%  \pkg{mathtools}.
%
%  \subsection{Spreading equations}\label{sec:spread}
%
%  \begin{codesyntax}
%    \SpecialEnvIndex{spreadlines}
%    \cs{begin}\arg{spreadlines}\marg{dimen} \meta{contents}
%    \cs{end}\arg{spreadlines}
%  \end{codesyntax}
%  \cttPosting{Michael J.~Downes}{2002/10/17}
%  The spacing between lines in a multiline math environment such as
%  \env{gather} is governed by the dimension \cs{jot}. The
%  \env{spreadlines} environment takes one argument denoting the
%  value of \cs{jot} inside the environment:
%  \begin{verbatim}
%    \begin{spreadlines}{20pt}
%    Large spaces between the lines.
%    \begin{gather}
%      a=b\\
%      c=d
%    \end{gather}
%    \end{spreadlines}
%    Back to normal spacing.
%    \begin{gather}
%      a=b\\
%      c=d
%    \end{gather}
%  \end{verbatim}
%    \begin{spreadlines}{20pt}
%    Large spaces between the lines.
%    \begin{gather}
%      a=b\\
%      c=d
%    \end{gather}
%    \end{spreadlines}
%    Back to normal spacing.
%    \begin{gather}
%      a=b\\
%      c=d
%    \end{gather}
%
%
%  \subsection{Gathered environments}\label{subsec:gathered}
%
%  \begin{codesyntax}
%    \SpecialEnvIndex{lgathered}\cs{begin}\arg{lgathered}\oarg{pos}
%    \meta{contents}  \cs{end}\arg{lgathered} \\
%    \SpecialEnvIndex{rgathered}\cs{begin}\arg{rgathered}\oarg{pos}
%    \meta{contents}  \cs{end}\arg{rgathered} \\
%    \SpecialUsageIndex{\newgathered}\cs{newgathered}\marg{name}\marg{pre_line}\marg{post_line}\marg{after}\\
%    \SpecialUsageIndex{\renewgathered}\cs{renewgathered}\marg{name}\marg{pre_line}\marg{post_line}\marg{after}
%  \end{codesyntax}
%  \cttPosting{Michael J.~Downes}{2001/01/17}
%  In a document set in \opt{fleqn}, you might sometimes want an
%  inner \env{gathered} environment that doesn't center its lines but
%  puts them flush left. The \env{lgathered} environment works just
%  like the standard \env{gathered} except that it flushes its
%  contents left:
%  \begin{verbatim}
%    \begin{equation}
%      \begin{lgathered}
%        x=1,\quad x+1=2 \\
%        y=2
%      \end{lgathered}
%    \end{equation}
%  \end{verbatim}
%  \begin{equation}
%    \begin{lgathered}
%        x=1,\quad x+1=2 \\
%        y=2
%    \end{lgathered}
%  \end{equation}
%  Similarly the \env{rgathered} puts it contents flush right.
%
%  More interesting is probably the command \cs{newgathered}. In this
%  example we define a gathered version that centers the lines and
%  also prints a star and a number at the left of each line.
%  \begin{verbatim}
%    \newcounter{steplinecnt}
%    \newcommand\stepline{\stepcounter{steplinecnt}\thesteplinecnt}
%    \newgathered{stargathered}
%                {\llap{\stepline}$*$\quad\hfil}% \hfil for centering
%                {\hfil}%                         \hfil for centering
%                {\setcounter{steplinecnt}{0}}%   reset counter
%  \end{verbatim}
%  \newcounter{steplinecnt}
%  \newcommand\stepline{\stepcounter{steplinecnt}\thesteplinecnt}
%  \newgathered{stargathered}{\llap{\stepline}$*$\quad\hfil}{\hfil}{\setcounter{steplinecnt}{0}}
%  With these definitions we can get something like this:
%  \begin{verbatim}
%    \begin{gather}
%      \begin{stargathered}
%        x=1,\quad x+1=2 \\
%        y=2
%      \end{stargathered}
%    \end{gather}
%  \end{verbatim}
%  \begin{gather}
%    \begin{stargathered}
%      x=1,\quad x+1=2 \\
%      y=2
%    \end{stargathered}
%  \end{gather}
%  \cs{renewgathered} renews a gathered environment of course.
%
%  In all fairness it should be stated that the original concept by
%  Michael has been extended quite a bit in \pkg{mathtools}. Only the
%  end product of \env{lgathered} is the same.
%
%  \subsection{Split fractions}
%
%  \begin{codesyntax}
%    \SpecialUsageIndex{\splitfrac}\cs{splitfrac}\marg{numer}\marg{denom}\texttt{~~}
%    \SpecialUsageIndex{\splitdfrac}\cs{splitdfrac}\marg{numer}\marg{denom}
%  \end{codesyntax}
%  \cttPosting{Michael J.~Downes}{2001/12/06}
%  These commands provide split fractions e.g., multiline fractions:
%  \begin{verbatim}
%    \[
%      a=\frac{
%          \splitfrac{xy + xy + xy + xy + xy}
%                    {+ xy + xy + xy + xy}
%        }
%        {z}
%      =\frac{
%          \splitdfrac{xy + xy + xy + xy + xy}
%                    {+ xy + xy + xy + xy}
%        }
%        {z}
%  \]
%  \end{verbatim}
%  \[
%  a=\frac{
%      \splitfrac{xy + xy + xy + xy + xy}
%                {+ xy + xy + xy + xy}
%    }
%    {z}
%  =\frac{
%      \splitdfrac{xy + xy + xy + xy + xy}
%                {+ xy + xy + xy + xy}
%    }
%    {z}
%  \]
%
%
%
%
%
%
%
%
%
%
%
%
%
%
%
%
%
%
%
%
%
%
%
%
%
%
%
%
%
%
%
%
%
%
%
%
%
%
%
%
%
%
%
%
%
%
%
%
%
%
%
%
%
%
%
%
%
%
%
%
%
%
%
%
%
%
%
%
%
%
%
%
%
%
%
%
%
%
%
%
%
%
%
%
%
% \begin{thebibliography}{9}
%  \bibitem{Perlis01}
%    Alexander R. Perlis,
%    \emph{A complement to \cs{smash}, \cs{llap}, and \cs{rlap}},
%    TUGboat 22(4) (2001).
%  \bibitem{Ams99}
%    American Mathematical Society and Michael Downes,
%    \emph{Technical notes on the \pkg{amsmath} package} Version 2.0,
%    1999/10/29.
%    (Available from CTAN as file \texttt{technote.tex}.)
%  \bibitem{Ams00}
%    Frank Mittelbach, Rainer Sch\"opf, Michael Downes, and David M.~Jones,
%    \emph{The \pkg{amsmath} package} Version 2.13,
%    2000/07/18.
%    (Available from CTAN as file \texttt{amsmath.dtx}.)
%  \bibitem{A-W:MG04}
%    Frank Mittelbach and Michel Goossens.
%     \emph{The {\LaTeX} Companion}.
%    Tools and Techniques for Computer Typesetting. Addison-Wesley,
%    Boston, Massachusetts, 2 edition, 2004.
%    With Johannes Braams, David Carlisle, and Chris Rowley.
%
% \bibitem{Carl99}
%   David Carlisle,
%   \emph{The \pkg{keyval} Package},
%   Version 1.13, 1999/03/16.
%   (Available from CTAN as file \texttt{keyval.dtx}.)
%
%  \bibitem{Voss:2004}
%    Herbert Vo\ss,
%    \emph{Math mode}, Version 1.71,
%    2004/07/06.
%    (Available from CTAN as file \texttt{Voss-Mathmode.pdf}.)
%  
%   \bibitem{Swanson} 
%     Ellen Swanson,
%     \emph{Mathematics into type}.
%     American Mathematical Society, updated edition, 1999.
%     Updated by Arlene O'Sean and Antoinette Schleyer.
%  \end{thebibliography}
%
%
%  \StopEventually{}  
%
%
%  \section{Options and package loading}
%
%
%  Lets start the package.
%    \begin{macrocode}
%<*package>
\ProvidesPackage{mathtools}%
  [2012/04/24 v1.12 mathematical typesetting tools]
%    \end{macrocode}
% \changes{v1.10}{2011/02/12}{Might as well make sure that we need the
% latest version of \texttt{mhsetup}}
%    \begin{macrocode}
\RequirePackage{keyval,calc}
\RequirePackage{mhsetup}[2010/01/21]
\MHInternalSyntaxOn
%    \end{macrocode}
%
%  \begin{macro}{\MT_options_name:}
%  \begin{macro}{\mathtoolsset}
%  The name for the options and a user interface for setting keys.
%    \begin{macrocode}
\def\MT_options_name:{mathtools}
\newcommand*\mathtoolsset[1]{\setkeys{\MT_options_name:}{#1}}
%    \end{macrocode}
%  \end{macro}
%  \end{macro}
%
%  Fix \pkg{amsmath} bugs (strongly recommended!). It requires a
%  great deal of typing to avoid fixing the bugs. He he.
%    \begin{macrocode}
\MH_new_boolean:n {fixamsmath}
\DeclareOption{fixamsmath}{
  \MH_set_boolean_T:n {fixamsmath}
}
\DeclareOption{donotfixamsmathbugs}{
  \MH_set_boolean_F:n {fixamsmath}
}
%    \end{macrocode}
%  Disallow spaces before optional arguments in certain \pkg{amsmath}
%  building blocks.
%    \begin{macrocode}
\DeclareOption{allowspaces}{
  \MH_let:NwN \MaybeMHPrecedingSpacesOff
              \relax
    \MH_let:NwN \MH_maybe_nospace_ifnextchar:Nnn \kernel@ifnextchar
}
\DeclareOption{disallowspaces}{
  \MH_let:NwN \MaybeMHPrecedingSpacesOff
              \MHPrecedingSpacesOff
  \MH_let:NwN \MH_maybe_nospace_ifnextchar:Nnn \MH_nospace_ifnextchar:Nnn
}
%    \end{macrocode}
%  Pass all other options directly to \pkg{amsmath}.
%    \begin{macrocode}
\DeclareOption*{
  \PassOptionsToPackage{\CurrentOption}{amsmath}
}
\ExecuteOptions{fixamsmath,disallowspaces}
\ProcessOptions\relax
%    \end{macrocode}
%  We have to turn off the new syntax when \pkg{amstext} is loaded.
%    \begin{macrocode}
\MHInternalSyntaxOff
\RequirePackage{amsmath}[2000/07/18]
\MHInternalSyntaxOn
\AtEndOfPackage{\MHInternalSyntaxOff}
%    \end{macrocode}
%  \begin{macro}{\MT_true_false_error:}
%  Make sure the user selects either `true' or `false' when asked too.
%    \begin{macrocode}
\def\MT_true_false_error:{
  \PackageError{mathtools}
    {You~ have~ to~ select~ either~ `true'~ or~ `false'}
    {I'll~ assume~ you~ chose~ `false'~ for~ now.}
}
%    \end{macrocode}
%  \end{macro}
%
%  \section{Macros I got ideas for myself}
%
%
%
%  \subsection{Tag forms}
%  This is quite simple, but why isn't it then a part of some widely
%  distributed package? Beats me.
%
%  \begin{macro}{\MT_define_tagform:nwnn}
%  We start out by defining a command that will allow us to define
%  commands similar to \cs{tagform@} only this will give us tag form
%  \emph{types}. The actual code is very similar to the one in
%  \pkg{amsmath}.
%    \begin{macrocode}
\def\MT_define_tagform:nwnn #1[#2]#3#4{
  \@namedef{MT_tagform_#1:n}##1
    {\maketag@@@{#3\ignorespaces#2{##1}\unskip\@@italiccorr#4}}
}
%    \end{macrocode}
%  \end{macro}
%
%  \begin{macro}{\newtagform}
%  Similar to \cs{newcommand}. Check if defined and scan for presence
%  of optional argument. Then call generic command.
%    \begin{macrocode}
\providecommand*\newtagform[1]{%
  \@ifundefined{MT_tagform_#1:n}
  {\@ifnextchar[%
    {\MT_define_tagform:nwnn #1}%
    {\MT_define_tagform:nwnn #1[]}%
  }{\PackageError{mathtools}
  {The~ tag~ form~ `#1'~ is~ already~ defined\MessageBreak
  You~ probably~ want~ to~ look~ up~ \@backslashchar renewtagform~
  instead}
  {I~ will~ just~ ignore~ your~ wish~ for~ now.}}
}
%    \end{macrocode}
%  Provide a default tag form which---surprise, surprise---is
%  identical to the standard definition.
%    \begin{macrocode}
\newtagform{default}{(}{)}
%    \end{macrocode}
%  \end{macro}
%  \begin{macro}{\renewtagform}
%  Similar to \cs{renewcommand}.
%    \begin{macrocode}
\providecommand*\renewtagform[1]{%
  \@ifundefined{MT_tagform_#1:n}
  {\PackageError{mathtools}
  {The~ tag~ form~ `#1'~ is~ not~ defined\MessageBreak
  You~ probably~ want~ to~ look~ up~ \@backslashchar newtagform~ instead}
  {I~ will~ just~ ignore~ your~ wish~ for~ now.}}
  {\@ifnextchar[%
    {\MT_define_tagform:nwnn #1}%
    {\MT_define_tagform:nwnn #1[]}%
  }
}
%    \end{macrocode}
%  \end{macro}
%  \begin{macro}{\usetagform}
%  Then the activator. Test if the tag form is defined and then
%  activate it by redefining \cs{tagform@}.
%    \begin{macrocode}
\providecommand*\usetagform[1]{%
  \@ifundefined{MT_tagform_#1:n}
    {
      \PackageError{mathtools}{%
        You~ have~ chosen~ the~ tag~ form~ `#1'\MessageBreak
        but~ it~ appears~ to~ be~ undefined}
        {I~ will~ use~ the~ default~ tag~ form~ instead.}%
        \@namedef{tagform@}{\@nameuse{MT_tagform_default:n}}
      }
  { \@namedef{tagform@}{\@nameuse{MT_tagform_#1:n}} }
%    \end{macrocode}
%  Here we patch if we're using the special ``show only referenced
%  equations'' feature.
%    \begin{macrocode}
  \MH_if_boolean:nT {show_only_refs}{
    \MH_let:NwN \MT_prev_tagform:n \tagform@
    \def\tagform@##1{\MT_extended_tagform:n {##1}}
  }
}
%    \end{macrocode}
%  \end{macro}
%
%  \subsubsection{Showing only referenced tags}
%  A little more interesting is the way to print only the equation
%  numbers that are actually referenced.
%
%  A few booleans to help determine which situations we're in.
%    \begin{macrocode}
\MH_new_boolean:n {manual_tag}
\MH_new_boolean:n {raw_maketag}
%    \end{macrocode}
%  \begin{macro}{\MT_AmS_tag_in_align:}
%  \begin{macro}{\tag@in@align}
%  \begin{macro}{\tag@in@display}
%  We'll need to know when the user has put in a manual tag, and since
%  \cs{tag} is \cs{let} to all sorts of things inside the \pkg{amsmath}
%  code it is safer to provide a small hack to the functions it is copied
%  from. Note that we can't use \cs{iftag@}.
%    \begin{macrocode}
\MH_let:NwN \MT_AmS_tag_in_align: \tag@in@align
\def\tag@in@align{
  \global\MH_set_boolean_T:n {manual_tag}
  \MT_AmS_tag_in_align:
}
\def\tag@in@display#1#{
  \relax
  \global\MH_set_boolean_T:n {manual_tag}
  \tag@in@display@a{#1}
}
%    \end{macrocode}
%  \end{macro}
%  \end{macro}
%  \end{macro}
%
%  \begin{macro}{\MT_extended_tagform:n}
%  \changes{v1.01}{2004/08/03}{Simplified quite a bit}
%  The extended version of \cs{tagform@}.
%    \begin{macrocode}
\def\MT_extended_tagform:n #1{
  \MH_set_boolean_F:n {raw_maketag}
%    \end{macrocode}
% We test if the equation was labelled. We already know if it was
% tagged manually. Have to watch out for \TeX\ inserting a blank line
% so do not let the tag have width zero. Rememeber
% \cs{@safe@activestrue/false} in order to handle active chars in labels.
% \changes{v1.12}{2012/04/24}{Added \cs{@safe@activestrue/false}}
%    \begin{macrocode}
  \if_meaning:NN \df@label\@empty
    \MH_if_boolean:nTF {manual_tag}% this was \MH_if_boolean:nT before
    { \MH_if_boolean:nTF {show_manual_tags}
      { \MT_prev_tagform:n {#1} }
      { \stepcounter{equation}  }
    }{\kern1sp}% this last {\kern1sp} is new.
  \else:
    \MH_if_boolean:nTF {manual_tag}
      { \MH_if_boolean:nTF {show_manual_tags}
          { \MT_prev_tagform:n {#1} }
          { \@safe@activestrue
            \@ifundefined{MT_r_\df@label}
%    \end{macrocode}
% Next we need to remember to deactivate the manual tags switch. This
% is usually done using \verb|\MT_extended_maketag:n|, but this is not
% the case if the show manual tags is false and the manual tag is not
% referred to. 
% \changes{v1.12}{2011/06/08}{Added the falsification of manual tag
% when show manual tags is off and maual tag is not referred to}
%    \begin{macrocode}
              { \global\MH_set_boolean_F:n {manual_tag} }
              { \MT_prev_tagform:n {#1} }
              \@safe@activesfalse
          }
      }
      { 
        \@safe@activestrue
        \@ifundefined{MT_r_\df@label}
          { }
          { \refstepcounter{equation}\MT_prev_tagform:n {#1} }
        \@safe@activesfalse
      }
  \fi:
  \global\MH_set_boolean_T:n {raw_maketag}
}
%    \end{macrocode}
%  \end{macro}
%  \begin{macro}{\MT_extended_maketag:n}
%  The extended version of \cs{maketag@@@}.
% \changes{v1.12}{2012/04/24}{Added \cs{@safe@activestrue/false}}
%    \begin{macrocode}
\def\MT_extended_maketag:n #1{
  \ifx\df@label\@empty
    \MT_maketag:n {#1}
  \else:
    \MH_if_boolean:nTF {raw_maketag}
      {
        \MH_if_boolean:nTF {show_manual_tags}
          { \MT_maketag:n {#1} }
          { \@safe@activestrue
            \@ifundefined{MT_r_\df@label}
              { }
              { \MT_maketag:n {#1}     }
            \@safe@activesfalse
          }
      }
      { \MT_maketag:n {#1} }
  \fi:
%    \end{macrocode}
%  As this function is always called we let it set the marker for a manual
%  tag false when exiting (well actually not true, see above).
%    \begin{macrocode}
  \global\MH_set_boolean_F:n {manual_tag}
}
%    \end{macrocode}
%  \end{macro}
%  \begin{macro}{\MT_extended_eqref:n}
%  \changes{v1.01}{2004/08/03}{Make it robust}
%  We let \cs{eqref} write the label to the \file{aux} file, which is
%  read at the beginning of the next run. Then we print the equation
%  number as usual.
%    \begin{macrocode}
\def\MT_extended_eqref:n #1{
  \protected@write\@auxout{}
    {\string\MT@newlabel{#1}}
  \textup{\MT_prev_tagform:n {\ref{#1}}}
}
%    \end{macrocode}
%  \end{macro}
%
%  \begin{macro}{\refeq}
%  \begin{macro}{\MT_extended_refeq:n}
%  Similar to \cs{eqref} and \cs{MT_extended_eqref:n}.
%    \begin{macrocode}
\newcommand*\refeq[1]{
  \textup{\ref{#1}}
}
\def\MT_extended_refeq:n #1{
  \protected@write\@auxout{}
    {\string\MT@newlabel{#1}}
  \textup{\ref{#1}}
}
%    \end{macrocode}
%  \end{macro}
%  \end{macro}
%
%  \begin{macro}{\MT@newlabel}
%  We can't use |:| or |_| in the command name (yet). We define the
%  special labels for the equations that have been referenced in the
%  previous run.
%    \begin{macrocode}
\newcommand*\MT@newlabel[1]{  \global\@namedef{MT_r_#1}{}  }
%    \end{macrocode}
%  \end{macro}
%    \begin{macrocode}
\MH_new_boolean:n {show_only_refs}
\MH_new_boolean:n {show_manual_tags}
\define@key{\MT_options_name:}{showmanualtags}[true]{
  \@ifundefined{boolean_show_manual_tags_#1:}
    { \MT_true_false_error:
      \@nameuse{boolean_show_manual_tags_false:}
    }
    { \@nameuse{boolean_show_manual_tags_#1:} }
}
%    \end{macrocode}
%  \begin{macro}{\MT_showonlyrefs_true:}
%  The implementation is based on the idea that \cs{tagform@} can be
%  called in two circumstances: when the tag is being printed in the
%  equation and when it is being printed during a reference.
%    \begin{macrocode}
\newcommand*\MT_showonlyrefs_true:{
  \MH_if_boolean:nF {show_only_refs}{
    \MH_set_boolean_T:n {show_only_refs}
%    \end{macrocode}
%  Save the definitions of the original commands.
%    \begin{macrocode}
    \MH_let:NwN \MT_incr_eqnum: \incr@eqnum
    \MH_let:NwN \incr@eqnum \@empty
    \MH_let:NwN \MT_array_parbox_restore: \@arrayparboxrestore
    \@xp\def\@xp\@arrayparboxrestore\@xp{\@arrayparboxrestore
      \MH_let:NwN \incr@eqnum \@empty
    }
    \MH_let:NwN \MT_prev_tagform:n \tagform@
    \MH_let:NwN \MT_eqref:n \eqref
    \MH_let:NwN \MT_refeq:n \refeq
    \MH_let:NwN \MT_maketag:n \maketag@@@
    \MH_let:NwN \maketag@@@ \MT_extended_maketag:n
%    \end{macrocode}
%  We redefine \cs{tagform@}.
%    \begin{macrocode}
    \def\tagform@##1{\MT_extended_tagform:n {##1}}
%    \end{macrocode}
%  Then \cs{eqref}:
%    \begin{macrocode}
    \MH_let:NwN \eqref \MT_extended_eqref:n
    \MH_let:NwN \refeq \MT_extended_refeq:n
  }
}
%    \end{macrocode}
%  \end{macro}
%  \begin{macro}{\MT_showonlyrefs_false:}
%  This macro reverts the settings.
%    \begin{macrocode}
\def\MT_showonlyrefs_false: {
  \MH_if_boolean:nT {show_only_refs}{
    \MH_set_boolean_F:n {show_only_refs}
    \MH_let:NwN \tagform@  \MT_prev_tagform:n
    \MH_let:NwN \eqref \MT_eqref:n
    \MH_let:NwN \refeq \MT_refeq:n
    \MH_let:NwN \maketag@@@ \MT_maketag:n
    \MH_let:NwN \incr@eqnum \MT_incr_eqnum:
    \MH_let:NwN \@arrayparboxrestore \MT_array_parbox_restore:
  }
}
\define@key{\MT_options_name:}{showonlyrefs}[true]{
  \@nameuse{MT_showonlyrefs_#1:}
}
%    \end{macrocode}
%  \end{macro}
%
%
%  \begin{macro}{\nonumber}
%  \changes{v1.01}{2004/08/03}{Fixed using \cs{notag} or \cs{nonumber}
%  with the \key{showonlyrefs} feature}
%  We have to redefine \cs{nonumber} else it will subtract one from the
%  equation number where we don't want it. This is probably not needed
%  since \cs{nonumber} is unnecessary when \key{showonlyrefs} is in
%  effect, but now you can use it with old documents as well.
%    \begin{macrocode}
\renewcommand\nonumber{
  \if@eqnsw
    \if_meaning:NN \incr@eqnum\@empty
%    \end{macrocode}
%  Only subtract the number if |show_only_refs| is false.
%    \begin{macrocode}
      \MH_if_boolean:nF {show_only_refs}
        {\addtocounter{equation}\m@ne}
    \fi:
  \fi:
  \MH_let:NwN \print@eqnum\@empty \MH_let:NwN \incr@eqnum\@empty
  \global\@eqnswfalse
}
%    \end{macrocode}
%  \end{macro}
%
%  \begin{macro}{\noeqref}
%   \changes{v1.04}{2008/03/26}{Added \cs{noeqref} (daleif)}
%   \changes{v1.12}{2012/04/20}{Labels containing active chars (babel)
%   are now allowed}
%   \changes{v1.12}{2012/04/24}{\cs{noeqref} will now make a reference
%   warning if users use undefined labels in \cs{noeqref}, requested
%   by Tue Christensen}
%   Macro for adding numbers to non-referred equations. Syntax similar
%   to \cs{nocite}.
%    \begin{macrocode}
\MHInternalSyntaxOff
\newcommand\noeqref[1]{\@bsphack
  \@for\@tempa:=#1\do{%
    \@safe@activestrue%
    \edef\@tempa{\expandafter\@firstofone\@tempa}%
    \@ifundefined{r@\@tempa}{%
      \protect\G@refundefinedtrue%
      \@latex@warning{Reference `\@tempa' on page \thepage \space
        undefined (\string\noeqref)}%
    }{}%
    \if@filesw\protected@write\@auxout{}%
    {\string\MT@newlabel{\@tempa}}\fi%
  \@safe@activesfalse}
  \@esphack}

%    \end{macrocode}
%  \end{macro}
%    
% \begin{macro}{\@safe@activestrue}
% \begin{macro}{\@safe@activesfalse}
%   These macros are provided by babel. We \emph{provide} them here,
%   just to make sure they exist.
%    \begin{macrocode}
\providecommand\@safe@activestrue{}%
\providecommand\@safe@activesfalse{}%

\MHInternalSyntaxOn
%    \end{macrocode}
%   
% \end{macro}
% \end{macro}
%
%  \subsection{Extensible arrows etc.}
%
%  \begin{macro}{\xleftrightarrow}
%  \begin{macro}{\MT_leftrightarrow_fill:}
%  \begin{macro}{\xLeftarrow}
%  \begin{macro}{\xRightarrow}
%  \begin{macro}{\xLeftrightarrow}
%
%  These are straight adaptions from \pkg{amsmath}.
%    \begin{macrocode}
\providecommand*\xleftrightarrow[2][]{%
  \ext@arrow 3095\MT_leftrightarrow_fill:{#1}{#2}}
\def\MT_leftrightarrow_fill:{%
  \arrowfill@\leftarrow\relbar\rightarrow}
\providecommand*\xLeftarrow[2][]{%
  \ext@arrow 0055{\Leftarrowfill@}{#1}{#2}}
\providecommand*\xRightarrow[2][]{%
  \ext@arrow 0055{\Rightarrowfill@}{#1}{#2}}
\providecommand*\xLeftrightarrow[2][]{%
  \ext@arrow 0055{\Leftrightarrowfill@}{#1}{#2}}
%    \end{macrocode}
%  \end{macro}
%  \end{macro}
%  \end{macro}
%  \end{macro}
%  \end{macro}
%  \begin{macro}{\MT_rightharpoondown_fill:}
%  \begin{macro}{\MT_rightharpoonup_fill:}
%  \begin{macro}{\MT_leftharpoondown_fill:}
%  \begin{macro}{\MT_leftharpoonup_fill:}
%  \begin{macro}{\xrightharpoondown}
%  \begin{macro}{\xrightharpoonup}
%  \begin{macro}{\xleftharpoondown}
%  \begin{macro}{\xleftharpoonup}
%  \begin{macro}{\xleftrightharpoons}
%  \begin{macro}{\xrightleftharpoons}
%  The harpoons.
%    \begin{macrocode}
\def\MT_rightharpoondown_fill:{%
  \arrowfill@\relbar\relbar\rightharpoondown}
\def\MT_rightharpoonup_fill:{%
  \arrowfill@\relbar\relbar\rightharpoonup}
\def\MT_leftharpoondown_fill:{%
  \arrowfill@\leftharpoondown\relbar\relbar}
\def\MT_leftharpoonup_fill:{%
  \arrowfill@\leftharpoonup\relbar\relbar}
\providecommand*\xrightharpoondown[2][]{%
  \ext@arrow 0359\MT_rightharpoondown_fill:{#1}{#2}}
\providecommand*\xrightharpoonup[2][]{%
  \ext@arrow 0359\MT_rightharpoonup_fill:{#1}{#2}}
\providecommand*\xleftharpoondown[2][]{%
  \ext@arrow 3095\MT_leftharpoondown_fill:{#1}{#2}}
\providecommand*\xleftharpoonup[2][]{%
  \ext@arrow 3095\MT_leftharpoonup_fill:{#1}{#2}}
\providecommand*\xleftrightharpoons[2][]{\mathrel{%
  \raise.22ex\hbox{%
    $\ext@arrow 3095\MT_leftharpoonup_fill:{\phantom{#1}}{#2}$}%
  \setbox0=\hbox{%
    $\ext@arrow 0359\MT_rightharpoondown_fill:{#1}{\phantom{#2}}$}%
  \kern-\wd0 \lower.22ex\box0}}
\providecommand*\xrightleftharpoons[2][]{\mathrel{%
  \raise.22ex\hbox{%
    $\ext@arrow 0359\MT_rightharpoonup_fill:{\phantom{#1}}{#2}$}%
  \setbox0=\hbox{%
    $\ext@arrow 3095\MT_leftharpoondown_fill:{#1}{\phantom{#2}}$}%
  \kern-\wd0 \lower.22ex\box0}}
%    \end{macrocode}
%  \end{macro}
%  \end{macro}
%  \end{macro}
%  \end{macro}
%  \end{macro}
%  \end{macro}
%  \end{macro}
%  \end{macro}
%  \end{macro}
%  \end{macro}
%  \begin{macro}{\xhookleftarrow}
%  \begin{macro}{\xhookrightarrow}
%  \begin{macro}{\MT_hookright_fill:}
%  The hooks.
%    \begin{macrocode}
\providecommand*\xhookleftarrow[2][]{%
  \ext@arrow 3095\MT_hookleft_fill:{#1}{#2}}
\def\MT_hookleft_fill:{%
  \arrowfill@\leftarrow\relbar{\relbar\joinrel\rhook}}
\providecommand*\xhookrightarrow[2][]{%
  \ext@arrow 3095\MT_hookright_fill:{#1}{#2}}
\def\MT_hookright_fill:{%
  \arrowfill@{\lhook\joinrel\relbar}\relbar\rightarrow}
%    \end{macrocode}
%  \end{macro}
%  \end{macro}
%  \end{macro}
%  \begin{macro}{\xmapsto}
%  \begin{macro}{\MT_mapsto_fill:}
%  The maps-to arrow.
%    \begin{macrocode}
\providecommand*\xmapsto[2][]{%
  \ext@arrow 0395\MT_mapsto_fill:{#1}{#2}}
\def\MT_mapsto_fill:{%
  \arrowfill@{\mapstochar\relbar}\relbar\rightarrow}
%    \end{macrocode}
%  \end{macro}
%  \end{macro}
%  \subsection{Underbrackets etc.}
%  \begin{macro}{\underbracket}
%  \begin{macro}{\MT_underbracket_I:w}
%  \begin{macro}{\MT_underbracket_II:w}
%  \begin{macro}{\upbracketfill}
%  \begin{macro}{\upbracketend}
%  The \cs{underbracket} macro. Scan for two optional arguments. When
%  \pkg{xparse} becomes the standard this will be so much easier.
%    \begin{macrocode}
\providecommand*\underbracket{
  \@ifnextchar[
    {\MT_underbracket_I:w}
    {\MT_underbracket_I:w[\l_MT_bracketheight_fdim]}}
\def\MT_underbracket_I:w[#1]{
  \@ifnextchar[
    {\MT_underbracket_II:w[#1]}
    {\MT_underbracket_II:w[#1][.7\fontdimen5\textfont2]}}
\def\MT_underbracket_II:w[#1][#2]#3{%
  \mathop{\vtop{\m@th\ialign{##
    \crcr
      $\hfil\displaystyle{#3}\hfil$%
    \crcr
      \noalign{\kern .2\fontdimen5\textfont2 \nointerlineskip}%
      \upbracketfill {#1}{#2}%
    \crcr}}}
  \limits}
\def\upbracketfill#1#2{%
  \sbox\z@{$\braceld$}
  \edef\l_MT_bracketheight_fdim{\the\ht\z@}%
  \upbracketend{#1}{#2}
  \leaders \vrule \@height \z@ \@depth #1 \hfill
  \upbracketend{#1}{#2}%
}
\def\upbracketend#1#2{\vrule \@height #2 \@width #1\relax}
%    \end{macrocode}
%  \end{macro}
%  \end{macro}
%  \end{macro}
%  \end{macro}
%  \end{macro}
%  \begin{macro}{\overbracket}
%  \begin{macro}{\MT_overbracket_I:w}
%  \begin{macro}{\MT_overbracket_II:w}
%  \begin{macro}{\downbracketfill}
%  \begin{macro}{\downbracketend}
%  The overbracket is quite similar.
%    \begin{macrocode}
\providecommand*\overbracket{
  \@ifnextchar[
    {\MT_overbracket_I:w}
    {\MT_overbracket_I:w[\l_MT_bracketheight_fdim]}}
\def\MT_overbracket_I:w[#1]{
  \@ifnextchar[
    {\MT_overbracket_II:w[#1]}
    {\MT_overbracket_II:w[#1][.7\fontdimen5\textfont2]}}
\def\MT_overbracket_II:w[#1][#2]#3{%
  \mathop{\vbox{\m@th\ialign{##
        \crcr
          \downbracketfill{#1}{#2}%
        \crcr
          \noalign{\kern .2\fontdimen5\textfont2 \nointerlineskip}%
          $\hfil\displaystyle{#3}\hfil$
        \crcr}}}%
  \limits}
\def\downbracketfill#1#2{%
  \sbox\z@{$\braceld$}\edef\l_MT_bracketheight_fdim{\the\ht\z@}
  \downbracketend{#1}{#2}
  \leaders \vrule \@height #1 \@depth \z@ \hfill
  \downbracketend{#1}{#2}%
}
\def\downbracketend#1#2{\vrule \@width #1\@depth #2\relax}
%    \end{macrocode}
%  \end{macro}
%  \end{macro}
%  \end{macro}
%  \end{macro}
%  \end{macro}
%  \begin{macro}{\LaTeXunderbrace}
%  \begin{macro}{\underbrace}
%  Redefinition of \cs{underbrace} and \cs{overbrace}.
%    \begin{macrocode}
\MH_let:NwN \LaTeXunderbrace \underbrace
\def\underbrace#1{\mathop{\vtop{\m@th\ialign{##\crcr
   $\hfil\displaystyle{#1}\hfil$\crcr
   \noalign{\kern.7\fontdimen5\textfont2\nointerlineskip}%
%    \end{macrocode}
%  |.5\fontdimen5\textfont2| is the height of the tip of the brace.
%  the remaining |.2\fontdimen5\textfont2| is for space between
%    \begin{macrocode}
   \upbracefill\crcr\noalign{\kern.5\fontdimen5\textfont2}}}}\limits}
%    \end{macrocode}
%  \end{macro}
%  \end{macro}
%  \begin{macro}{\LaTeXoverbrace}
%  \begin{macro}{\overbrace}
%  Same technique for \cs{overbrace}.
%    \begin{macrocode}
\MH_let:NwN \LaTeXoverbrace \overbrace
\def\overbrace#1{\mathop{\vbox{\m@th\ialign{##\crcr
  \noalign{\kern.5\fontdimen5\textfont2}%
%    \end{macrocode}
%  Adjust for tip height
%    \begin{macrocode}
  \downbracefill\crcr
  \noalign{\kern.7\fontdimen5\textfont2\nointerlineskip}%
%    \end{macrocode}
%  |.5\fontdimen5\textfont2| is the height of the tip of the brace.
%  The remaining |.2\fontdimen5\textfont2| is for space between
%    \begin{macrocode}
  $\hfil\displaystyle{#1}\hfil$\crcr}}}\limits}
%    \end{macrocode}
%  \end{macro}
%  \end{macro}
%
%
%
%
%
%  \subsection{Special symbols}
%
%  \subsubsection{Command names for parentheses}
%  \begin{macro}{\lparen}
%  \begin{macro}{\rparen}
%  Just an addition to the \LaTeXe\ kernel.
%    \begin{macrocode}
\providecommand*\lparen{(}
\providecommand*\rparen{)}
%    \end{macrocode}
%  \end{macro}
%  \end{macro}

%  \subsubsection{Vertically centered colon}
%
%  \begin{macro}{\vcentcolon}
%  \begin{macro}{\ordinarycolon}
%  \begin{macro}{\MT_active_colon_true:}
%  \begin{macro}{\MT_active_colon_false:}
%  This is from the hands of Donald Arseneau. Somehow it is not
%  distributed, so I include it here. Here's the original text by
%  Donald:
%  \begin{verbatim}
%  centercolon.sty                 Dec 7, 2000
%  Donald Arseneau                 asnd@triumf.ca
%  Public domain.
%  Vertically center colon characters (:) in math mode.
%  Particularly useful for $ a:=b$, and still correct for
%  $f : x\to y$.  May be used in any TeX.
%  \end{verbatim}
%
% Slight change: the colon meaning is given only if \verb|centercolon|
% is explicitly requested (before it was always assigned even if : remained
% catcode 12). This allows better interaction with packages like babel
% that also make colon active.
%    \begin{macrocode}
\def\vcentcolon{\mathrel{\mathop\ordinarycolon}}
\providecommand\ordinarycolon{:}
\begingroup
  \catcode`\:=\active
  \lowercase{\endgroup
\def\MT_activate_colon{%
    \ifnum\mathcode`\:=32768\relax
      \let\ordinarycolon= :%
    \else
      \mathchardef\ordinarycolon\mathcode`\: %
    \fi 
    \let :\vcentcolon
  }
}
%    \end{macrocode}
% Option processing.
% The `false' branch can only be requested if the option has previously been set `true'.
% (By default neither are set.)
%    \begin{macrocode}
\MH_new_boolean:n {center_colon}
\define@key{\MT_options_name:}{centercolon}[true]{
  \@ifundefined{MT_active_colon_#1:}
    { \MT_true_false_error:n
      \@nameuse{MT_active_colon_false:}
    }
    { \@nameuse{MT_active_colon_#1:} }
}
\def\MT_active_colon_true: {
  \MT_activate_colon
  \MH_if_boolean:nF {center_colon}{
    \MH_set_boolean_T:n {center_colon}
    \edef\MT_active_colon_false:
      {\mathcode`\noexpand\:=\the\mathcode`\:\relax}
    \mathcode`\:=32768
  }
}
%    \end{macrocode}
%  \end{macro}
%  \end{macro}
%  \end{macro}
%  \end{macro}
%  \begin{macro}{\dblcolon}
%  \begin{macro}{\coloneqq}
%  \begin{macro}{\Coloneqq}
%  \begin{macro}{\coloneq}
%  \begin{macro}{\Coloneq}
%  \begin{macro}{\eqqcolon}
%  \begin{macro}{\Eqqcolon}
%  \begin{macro}{\eqcolon}
%  \begin{macro}{\Eqcolon}
%  \begin{macro}{\colonapprox}
%  \begin{macro}{\Colonapprox}
%  \begin{macro}{\colonsim}
%  \begin{macro}{\Colonsim}
%  This is just to simulate all the \cs{..colon..} symbols from
%  \pkg{txfonts} and \pkg{pxfonts}.
% \changes{v1.08c}{2010/11/17}{Enclosed all in \cs{mathrel}}
%    \begin{macrocode}
\AtBeginDocument{
  \providecommand*\dblcolon{\mathrel{\vcentcolon\mkern-.9mu\vcentcolon}}
  \providecommand*\coloneqq{\mathrel{\vcentcolon\mkern-1.2mu=}}
  \providecommand*\Coloneqq{\mathrel{\dblcolon\mkern-1.2mu=}}
  \providecommand*\coloneq{\mathrel{\vcentcolon\mkern-1.2mu\mathrel{-}}}
  \providecommand*\Coloneq{\mathrel{\dblcolon\mkern-1.2mu\mathrel{-}}}
  \providecommand*\eqqcolon{\mathrel{=\mkern-1.2mu\vcentcolon}}
  \providecommand*\Eqqcolon{\mathrel{=\mkern-1.2mu\dblcolon}}
  \providecommand*\eqcolon{\mathrel{\mathrel{-}\mkern-1.2mu\vcentcolon}}
  \providecommand*\Eqcolon{\mathrel{\mathrel{-}\mkern-1.2mu\dblcolon}}
  \providecommand*\colonapprox{\mathrel{\vcentcolon\mkern-1.2mu\approx}}
  \providecommand*\Colonapprox{\mathrel{\dblcolon\mkern-1.2mu\approx}}
  \providecommand*\colonsim{\mathrel{\vcentcolon\mkern-1.2mu\sim}}
  \providecommand*\Colonsim{\mathrel{\dblcolon\mkern-1.2mu\sim}}
}
%    \end{macrocode}
%  \end{macro}
%  \end{macro}
%  \end{macro}
%  \end{macro}
%  \end{macro}
%  \end{macro}
%  \end{macro}
%  \end{macro}
%  \end{macro}
%  \end{macro}
%  \end{macro}
%  \end{macro}
%  \end{macro}
%
%
%  \subsection{Multlined}
%
%  \begin{macro}{\g_MT_multlinerow_int}
%  \begin{macro}{\l_MT_multwidth_dim}
%  Helpers.
%    \begin{macrocode}
\let \AMS@math@cr@@ \math@cr@@
\MH_new_boolean:n {mult_firstline}
\MH_new_boolean:n {outer_mult}
\newcount\g_MT_multlinerow_int
\newdimen\l_MT_multwidth_dim
%    \end{macrocode}
%  \end{macro}
%  \end{macro}
%  \begin{macro}{\MT_test_for_tcb_other:nnnnn}
%  This tests if the token(s) is/are equal to either t, c, or~b, or
%  something entirely different.
%    \begin{macrocode}
\newcommand*\MT_test_for_tcb_other:nnnnn [1]{
  \if:w t#1\relax
    \expandafter\MH_use_choice_i:nnnn
  \else:
    \if:w c#1\relax
      \expandafter\expandafter\expandafter\MH_use_choice_ii:nnnn
    \else:
      \if:w b#1\relax
        \expandafter\expandafter\expandafter
        \expandafter\expandafter\expandafter\expandafter
        \MH_use_choice_iii:nnnn
      \else:
        \expandafter\expandafter\expandafter
        \expandafter\expandafter\expandafter\expandafter
        \MH_use_choice_iv:nnnn
      \fi:
    \fi:
  \fi:
}
%    \end{macrocode}
%  \end{macro}
%  \begin{macro}{\MT_mult_invisible_line:}
%  An invisible line.
%    \begin{macrocode}
\def\MT_mult_invisible_line: {
  \crcr
  \global\MH_set_boolean_F:n {mult_firstline}
  \hbox to \l_MT_multwidth_dim{}\crcr
  \noalign{\vskip-\baselineskip \vskip-\normallineskip}
}
%    \end{macrocode}
%  \end{macro}
%  \begin{macro}{\MT_mult_mathcr_atat:w}
%  The normal \cs{math@cr@@} with our hooks.
%    \begin{macrocode}
\def\MT_mult_mathcr_atat:w [#1]{%
  \if_num:w 0=`{\fi: \iffalse}\fi:
  \MH_if_boolean:nT {mult_firstline}{
    \kern\l_MT_mult_left_fdim
    \MT_mult_invisible_line:
  }
  \crcr
  \noalign{\vskip#1\relax}
  \global\advance\g_MT_multlinerow_int\@ne
  \if_num:w \g_MT_multlinerow_int=\l_MT_multline_lastline_fint
    \MH_let:NwN \math@cr@@\MT_mult_last_mathcr:w
  \fi:
}
%    \end{macrocode}
%  \end{macro}
%  \begin{macro}{\MT_mult_firstandlast_mathcr:w}
%  The special case where there is a two-line \env{multlined}. We
%  insert the first kern, then the invisible line of the desired
%  width, the optional vertical space and then the last kern.
%    \begin{macrocode}
\def\MT_mult_firstandlast_mathcr:w [#1]{%
  \if_num:w 0=`{\fi: \iffalse}\fi:
  \kern\l_MT_mult_left_fdim
  \MT_mult_invisible_line:
  \noalign{\vskip#1\relax}
  \kern\l_MT_mult_right_fdim
}
%    \end{macrocode}
%  \end{macro}
%  \begin{macro}{\MT_mult_last_mathcr:w}
%  The normal last \cs{math@cr@@} which inserts the last kern.
%    \begin{macrocode}
\def\MT_mult_last_mathcr:w [#1]{
  \if_num:w 0=`{\fi: \iffalse}\fi:\math@cr@@@
  \noalign{\vskip#1\relax}
  \kern\l_MT_mult_right_fdim}
%    \end{macrocode}
%  \end{macro}
%  \begin{macro}{\MT_start_mult:N}
%  Setup for \env{multlined}. Finds the position.
%    \begin{macrocode}
\newcommand\MT_start_mult:N [1]{
  \MT_test_for_tcb_other:nnnnn {#1}
    { \MH_let:NwN \MT_next:\vtop }
    { \MH_let:NwN \MT_next:\vcenter }
    { \MH_let:NwN \MT_next:\vbox }
    {
      \PackageError{mathtools}
        {Invalid~ position~ specifier.~ I'll~ try~ to~ recover~ with~
        `c'}\@ehc
    }
  \collect@body\MT_mult_internal:n
}
%    \end{macrocode}
%  \end{macro}
%  \begin{macro}{\MT_shoveright:wn}
%  \begin{macro}{\MT_shoveleft:wn}
%  Extended versions of \cs{shoveleft} and \cs{shoveright}.
%    \begin{macrocode}
\newcommand*\MT_shoveright:wn [2][0pt]{%
  #2\hfilneg
  \setlength\@tempdima{#1}
  \kern\@tempdima
}
\newcommand*\MT_shoveleft:wn [2][0pt]{%
  \hfilneg
  \setlength\@tempdima{#1}
  \kern\@tempdima
  #2
}
%    \end{macrocode}
%  \end{macro}
%  \end{macro}
%  \begin{macro}{\MT_mult_internal:n}
%  \changes{v1.01a}{2004/10/10}{Added Ord atom to beginning of each line}
%  The real internal \env{multlined}.
%    \begin{macrocode}
\newcommand*\MT_mult_internal:n [1]{
 \MH_if_boolean:nF {outer_mult}{\null\,}
  \MT_next:
  \bgroup
%    \end{macrocode}
%  Restore the meaning of \cmd{\\} inside \env{multlined}, else it
%  wouldn't work in the \env{equation} environment. Set the fake row
%  counter to zero.
%    \begin{macrocode}
    \Let@
    \def\l_MT_multline_lastline_fint{0 }
    \chardef\dspbrk@context\@ne \restore@math@cr
%    \end{macrocode}
%  Use private versions.
%    \begin{macrocode}
    \MH_let:NwN \math@cr@@\MT_mult_mathcr_atat:w
    \MH_let:NwN \shoveleft\MT_shoveleft:wn
    \MH_let:NwN \shoveright\MT_shoveright:wn
    \spread@equation
    \MH_set_boolean_F:n {mult_firstline}
%    \end{macrocode}
%  Do some measuring.
%    \begin{macrocode}
    \MT_measure_mult:n {#1}
%    \end{macrocode}
%  Make sure the box is wide enough.
%    \begin{macrocode}
    \if_dim:w \l_MT_multwidth_dim<\l_MT_multline_measure_fdim
      \MH_setlength:dn \l_MT_multwidth_dim{\l_MT_multline_measure_fdim}
    \fi
    \MH_set_boolean_T:n {mult_firstline}
%    \end{macrocode}
%  Tricky bit: If we only encountered one \cmd{\\} then use a very
%  special \cs{math@cr@@} that inserts everything needed.
%    \begin{macrocode}
    \if_num:w \l_MT_multline_lastline_fint=\@ne
      \MH_let:NwN \math@cr@@ \MT_mult_firstandlast_mathcr:w
    \fi:
%    \end{macrocode}
%  Do the typesetting.
%    \begin{macrocode}
    \ialign\bgroup
      \hfil\strut@$\m@th\displaystyle{}##$\hfil
      \crcr
      \hfilneg
      #1
}
%    \end{macrocode}
%  \end{macro}
%  \begin{macro}{\MT_measure_mult:n}
%  \changes{v1.01a}{2004/10/10}{Added Ord atom to beginning of each line}
%  Measuring. Disable all labelling and check the number of lines.
%    \begin{macrocode}
\newcommand\MT_measure_mult:n [1]{
  \begingroup
    \g_MT_multlinerow_int\@ne
    \MH_let:NwN \label\MT_gobblelabel:w
    \MH_let:NwN \tag\gobble@tag
    \setbox\z@\vbox{
      \ialign{\strut@$\m@th\displaystyle{}##$
        \crcr
        #1
        \crcr
      }
    }
    \xdef\l_MT_multline_measure_fdim{\the\wdz@}
    \advance\g_MT_multlinerow_int\m@ne
    \xdef\l_MT_multline_lastline_fint{\number\g_MT_multlinerow_int}
  \endgroup
  \g_MT_multlinerow_int\@ne
}
%    \end{macrocode}
%  \end{macro}
%  \begin{macro}{\MT_multlined_second_arg:w}
%  Scan for a second optional argument.
%    \begin{macrocode}
\MaybeMHPrecedingSpacesOff
\newcommand*\MT_multlined_second_arg:w [1][\@empty]{
  \MT_test_for_tcb_other:nnnnn {#1}
    {\def\MT_mult_default_pos:{#1}}
    {\def\MT_mult_default_pos:{#1}}
    {\def\MT_mult_default_pos:{#1}}
    {
      \if_meaning:NN \@empty#1\@empty
      \else:
        \setlength \l_MT_multwidth_dim{#1}
      \fi:
    }
  \MT_start_mult:N \MT_mult_default_pos:
}
%    \end{macrocode}
%  \end{macro}
%  \begin{environment}{multlined}
%  The user environment. Scan for an optional argument.
%    \begin{macrocode}
\newenvironment{multlined}[1][]
  {\MH_group_align_safe_begin:
  \MT_test_for_tcb_other:nnnnn {#1}
    {\def\MT_mult_default_pos:{#1}}
    {\def\MT_mult_default_pos:{#1}}
    {\def\MT_mult_default_pos:{#1}}
    {
      \if_meaning:NN \@empty#1\@empty
      \else:
        \setlength \l_MT_multwidth_dim{#1}
      \fi:
    }
    \MT_multlined_second_arg:w
  }
  {
    \hfilneg  \endaligned \MH_group_align_safe_end:
  }
\MHPrecedingSpacesOn
%    \end{macrocode}
%  \end{environment}
%  The keys needed.
%    \begin{macrocode}
\define@key{\MT_options_name:}
  {firstline-afterskip}{\def\l_MT_mult_left_fdim{#1}}
\define@key{\MT_options_name:}
  {lastline-preskip}{\def\l_MT_mult_right_fdim{#1}}
\define@key{\MT_options_name:}
  {multlined-width}{\setlength \l_MT_multwidth_dim{#1}}
\define@key{\MT_options_name:}
  {multlined-pos}{\def\MT_mult_default_pos:{#1}}
\setkeys{\MT_options_name:}{
  firstline-afterskip=\multlinegap,
  lastline-preskip=\multlinegap,
  multlined-width=0pt,
  multlined-pos=c,
}
%    \end{macrocode}
%  \begin{macro}{\MT_gobblelabel:w}
%  Better than to assume that \cs{label} has exactly one mandatory
%  argument, hence the \texttt{w} specifier.
%    \begin{macrocode}
\def\MT_gobblelabel:w #1{}
%    \end{macrocode}
%  \end{macro}
%
%
%
%
%  \section{Macros suggested/requested by Lars Madsen}
%
%  The macros in this section are all requests made by Lars Madsen.
%
%  \subsection{Paired delimiters}
%
%  \changes{v1.13}{2012/05/10}{Extended \cs{DeclarePairedDelimiter(X)}}
%  \begin{macro}{\MT_delim_default_inner_wrappers:n}
%  In some cases users may want to control the internals a bit more. We
%  therefore create two call back macros each time the
%  \cs{DeclarePaired...} macro is issued. The default value of these
%  call backs are provided by |\MT_delim_default_inner_wrappers:n|
%    \begin{macrocode}
\newcommand\MT_delim_default_inner_wrappers:n [1]{
   \@namedef{MT_delim_\MH_cs_to_str:N #1 _star_wrapper:nnn}##1##2##3{
      \mathopen{}\mathclose\bgroup ##1 ##2 \aftergroup\egroup ##3
    }
    \@namedef{MT_delim_\MH_cs_to_str:N #1 _nostar_wrapper:nnn}##1##2##3{
      \mathopen{##1}##2\mathclose{##3}
    }
  }

%    \end{macrocode}
% \begin{macro}{\reDeclarePairedDelimiterInnerWrapper}
%   Macro enabling the user to alter an existing call back. Note that
%   currently no checks are performed. First argument is the name of
%   the macro we are altering (as defined via \cs{DeclarePaired...}),
%   the second is \texttt{star} or \texttt{nostar}. In the last
%   argument \texttt{\#1} and \texttt{\#3} respectively refer to the
%   scaled fences and \texttt{\#3} refer to whatever come between.
%    \begin{macrocode}
\newcommand\reDeclarePairedDelimiterInnerWrapper[3]{
  \@namedef{MT_delim_\MH_cs_to_str:N #1 _ #2 _wrapper:nnn}##1##2##3{
    #3
  }
}

%    \end{macrocode}
%   
% \end{macro}
%  \end{macro}
%  \begin{macro}{\DeclarePairedDelimiter}
%  \changes{v1.06}{2008/08/01}{Made user command robust}
%  This macro defines |#1| to be a control sequence that takes either
%  a star or an optional argument.
%    \begin{macrocode}
\newcommand*\DeclarePairedDelimiter[3]{%
  \@ifdefinable{#1}{
%    \end{macrocode}
%  Define the starred command to just put \cs{left} and \cs{right}
%  before the delimiters.
%  \changes{1.08e}{2010/09/02}{`Fixed' \cs{left}\dots\cs{right} bad spacing}
%  \changes{1.08e}{2010/09/14}{redid the \cs{left}\dots\cs{right} fix,
%  see \cs{DeclarePairedDelimiterX} for details.}
%  \changes{v1.13}{2012/05/10}{Using call back instead}
%    \begin{macrocode}
    \MT_delim_default_inner_wrappers:n{#1}
    \@namedef{MT_delim_\MH_cs_to_str:N #1 _star:}##1
      %{\mathopen{}\mathclose\bgroup\left#2 ##1 \aftergroup\egroup\right #3}%
      { \@nameuse{MT_delim_\MH_cs_to_str:N #1 _star_wrapper:nnn}%
           {\left#2}{##1}{\right#3} }%
%    \end{macrocode}
%  The command with optional argument. It should be \cs{bigg} or
%  alike.
%    \begin{macrocode}
    \@xp\@xp\@xp
      \newcommand
        \@xp\csname MT_delim_\MH_cs_to_str:N #1 _nostar:\endcsname
        [2][\\@gobble]
        { 
%    \end{macrocode}
%  With the default optional argument we wind up with \cs{relax},
%  else we get \cs{biggr} and \cs{biggl} etc.
%  \changes{v1.13}{2012/05/10}{Using call back instead}
%    \begin{macrocode}
          %\mathopen{\@nameuse {\MH_cs_to_str:N ##1 l} #2} ##2 
          %\mathclose{\@nameuse {\MH_cs_to_str:N ##1 r} #3}}
          \@nameuse{MT_delim_\MH_cs_to_str:N #1 _nostar_wrapper:nnn}%
             {\@nameuse {\MH_cs_to_str:N ##1 l} #2}
             {##2}
             {\@nameuse {\MH_cs_to_str:N ##1 r} #3}
        }
%    \end{macrocode}
%  The user command comes here. Just check for the star and choose
%  the right internal command.
%    \begin{macrocode}
    \DeclareRobustCommand{#1}{
      \@ifstar
        {\@nameuse{MT_delim_\MH_cs_to_str:N #1 _star:}}
        {\@nameuse{MT_delim_\MH_cs_to_str:N #1 _nostar:}}
    }
  }
}
%    \end{macrocode}
%  \end{macro}
%
% \begin{macro}{\DeclarePairedDelimiterX}
%  \changes{v1.08}{2010/06/10}{Added \cs{DeclarePairedDelimiterX}}
%  It has turned out that it is convenient to have a more general
%  version of \cs{DeclarePairedDelimiter}. In this version the user
%  can specify the number of arguments the created macro has, and they
%  can specify the code for the inner part of the macro. Other than
%  that the code is fairly similar to \cs{DeclarePairedDelimiter}
%  \changes{v1.08e}{2010/09/02}{Provided better implementation of 
%  \cs{DeclarePairedDelimiterX}}
%    \begin{macrocode}
\def\MHempty{}
\def\DeclarePairedDelimiterX#1[#2]#3#4#5{%
  \@ifdefinable{#1}{
%    \end{macrocode}
% The constructor takes five arguments, the name of the macro, the
% number of arguments (1-9), the left and right delimiter, the inner
% code for the two macros. First we verify that the number of arguments fit.
%    \begin{macrocode}
    \ifnum#2>9\relax
      \PackageError{mathtools}{No~ more~ than~ 9~ arguments}{}
    \else
      \ifnum#2<1\relax
        \PackageError{mathtools}{Macro~ need~ 1~ or~ more~ arguments}{}
      \fi
    \fi
%    \end{macrocode}
%  \changes{v1.13}{2012/05/10}{Using call back instead}
%  Initiate the default call backs.
%    \begin{macrocode}
    \MT_delim_default_inner_wrappers:n{#1}
%    \end{macrocode}
% We make sure to store the delimiter size in the local variable
% \cs{delimsize}. Then users can refer to the size in the fifth
% argument. In the starred version it will refer to \cs{middle} and in
% the normal version it will hold the provided optional argument.
%    \begin{macrocode}
    \@xp\@xp\@xp
      \newcommand
        \@xp\csname MT_delim_\MH_cs_to_str:N #1 _star:\endcsname
        [#2]
        {
          \begingroup
            \def\delimsize{\middle}
%    \end{macrocode}
% This is slightly controversial, \cs{left}\dots\cs{right} are known
% to produce an inner atom, thus may cause different spacing than
% normal delimiters. We `fix' this by introducing \cs{mathopen} and
% \cs{mathclose}. This change is now factored out into call backs. 
% \changes{v1.08e}{2010/09/14}{redid the left/right fix, inspired by
% ctt thread named `spacing after \cs{right}) and before \cs{left})'
% started 2010-08-12.}
% \changes{v1.13}{2012/05/10}{Using call back instead}
%    \begin{macrocode}
            %\mathopen{}\mathclose\bgroup\left#3 #5 \aftergroup\egroup\right#4
            \@nameuse{MT_delim_\MH_cs_to_str:N #1 _star_wrapper:nnn}
              {\left#3}{#5}{\right#4}
          \endgroup
        }
%    \end{macrocode}
% In order for the starred and non-starred version to have the same
% arguments, we need to introduce an extra macro to catch the optional
% argument (this means that the non-starred version can actually
% support ten arguments!).
% Here we do things a little differently than with
% \cs{DeclarePairedDelimiter}. The optional argument have \cs{MHempty}
% as the default. This is locally changed when we scale the
% delimiters, such that it can eat the l/r if needed.
%    \begin{macrocode}
    \@xp\@xp\@xp
      \newcommand
        \@xp\csname MT_delim_\MH_cs_to_str:N #1 _nostar:\endcsname
        [1][\MHempty]
      {
%    \end{macrocode}
% We need to introduce a local group in order to support nesting. It
% is ended inside \verb|\MT_delim_\MH_cs_to_str:N #1 _nostar_inner:|
%    \begin{macrocode}
        \begingroup
        \def\delimsize{##1}
        \@nameuse{MT_delim_\MH_cs_to_str:N #1 _nostar_inner:}
      } 
%    \end{macrocode}
% Next we provide the inner workhorse. We need a bit of expansion
% magic to get \cs{delimsize} to work.
% \changes{v1.13}{2012/05/10}{Using call back instead}
%    \begin{macrocode}
    \@xp\@xp\@xp
      \newcommand
        \@xp\csname MT_delim_\MH_cs_to_str:N #1 _nostar_inner:\endcsname
        [#2]
        {
          %\mathopen{%
          %  \let\MHempty\@gobble
          %  \@xp\@xp\@xp\csname\@xp\MH_cs_to_str:N \delimsize l\endcsname #3} 
          %#5
          %\mathclose{%
          %  \let\MHempty\@gobble
          %  \@xp\@xp\@xp\csname\@xp\MH_cs_to_str:N \delimsize r\endcsname #4}
          \@nameuse{MT_delim_\MH_cs_to_str:N #1 _nostar_wrapper:nnn}
          {
            \let\MHempty\@gobble
            \@xp\@xp\@xp\csname\@xp\MH_cs_to_str:N \delimsize l\endcsname #3
          }
          {#5}
          {
            \let\MHempty\@gobble
            \@xp\@xp\@xp\csname\@xp\MH_cs_to_str:N \delimsize r\endcsname #4
          }
          \endgroup
        }
    \DeclareRobustCommand{#1}{
      \@ifstar
        {\@nameuse{MT_delim_\MH_cs_to_str:N #1 _star:}}
        {\@nameuse{MT_delim_\MH_cs_to_str:N #1 _nostar:}}
    }
  }
}
%    \end{macrocode}
% \end{macro}
%
%  \subsection{A \texttt{\textbackslash displaystyle} \env{cases} environment}
%
%  \begin{macro}{\MT_start_cases:nnn}
%  We define a single command that does all the hard work.
%  \changes{v1.08}{2010/06/10}{made \cs{MT_start_cases:nnnn} more general}
%    \begin{macrocode}
\def\MT_start_cases:nnnn #1#2#3#4{ % #1=sep,#2=lpreamble,#3=rpreamble,#4=delim
 \RIfM@\else
   \nonmatherr@{\begin{\@currenvir}}
 \fi
 \MH_group_align_safe_begin:
 \left#4
 \vcenter \bgroup
     \Let@ \chardef\dspbrk@context\@ne \restore@math@cr
     \let  \math@cr@@\AMS@math@cr@@
     \spread@equation
     \ialign\bgroup
%    \end{macrocode}
%  Set the first column flush left in \cs{displaystyle} math and the
%  second as specified by the second argument. The first argument is
%  the separation between the columns. It could be a \cs{quad} or
%  something entirely different.
%    \begin{macrocode}
       \strut@#2 &#1\strut@
       #3
       \crcr
}
%    \end{macrocode}
%  \end{macro}
% \begin{macro}{\MH_end_cases:}
%    \begin{macrocode}
\def\MH_end_cases:{\crcr\egroup
 \restorecolumn@
 \egroup
 \MH_group_align_safe_end:
}
%    \end{macrocode}
% \end{macro}
%  \begin{macro}{\newcases}
%  \begin{macro}{\renewcases}
%  Easy creation of new \env{cases}-like environments.
%  \changes{v1.08}{2010/06/10}{changed to match the change in \cs{MT_start_cases:nnnn}}
%    \begin{macrocode}
\newcommand*\newcases[6]{% #1=name, #2=sep, #3=preamble, #4=left, #5=right
 \newenvironment{#1}
   {\MT_start_cases:nnnn {#2}{#3}{#4}{#5}}
   {\MH_end_cases:\right#6}
}
\newcommand*\renewcases[6]{
 \renewenvironment{#1}
   {\MT_start_cases:nnnn {#2}{#3}{#4}{#5}}
   {\MH_end_cases:\right#6}
}
%    \end{macrocode}
%  \begin{environment}{dcases}
%  \begin{environment}{dcases*}
%  \begin{environment}{rcases}
%  \begin{environment}{rcases*}
%  \begin{environment}{drcases}
%  \begin{environment}{drcases*}
%  \begin{environment}{cases*}
%  \env{dcases} is a traditional cases with display style math in
%  both columns, while \env{dcases*} has text in the second column.
%  \changes{v1.08}{2010/06/10}{changed to match the change in
%  \cs{newcases} plus added rcases and drcases}
%    \begin{macrocode}
\newcases{dcases}{\quad}{%
  $\m@th\displaystyle{##}$\hfil}{$\m@th\displaystyle{##}$\hfil}{\lbrace}{.}
\newcases{dcases*}{\quad}{%
  $\m@th\displaystyle{##}$\hfil}{{##}\hfil}{\lbrace}{.}
\newcases{rcases}{\quad}{%
  $\m@th{##}$\hfil}{$\m@th{##}$\hfil}{.}{\rbrace}
\newcases{rcases*}{\quad}{%
  $\m@th{##}$\hfil}{{##}\hfil}{.}{\rbrace}
\newcases{drcases}{\quad}{%
  $\m@th\displaystyle{##}$\hfil}{$\m@th\displaystyle{##}$\hfil}{.}{\rbrace}
\newcases{drcases*}{\quad}{%
  $\m@th\displaystyle{##}$\hfil}{{##}\hfil}{.}{\rbrace}
\newcases{cases*}{\quad}{%
  $\m@th{##}$\hfil}{{##}\hfil}{\lbrace}{.}
%    \end{macrocode}
%  \end{environment}
%  \end{environment}
%  \end{environment}
%  \end{environment}
%  \end{environment}
%  \end{environment}
%  \end{environment}
%  \end{macro}
%  \end{macro}
%
%  \subsection{New matrix environments}
%  \begin{macro}{\MT_matrix_begin:N}
%  \begin{macro}{\MT_matrix_end:}
%  Here are a few helpers for the matrices. \cs{MT_matrix_begin:N}
%  takes one argument specifying the column type for the array inside
%  the matrix. and \cs{MT_matrix_end:} inserts the correct ending.
%    \begin{macrocode}
\def\MT_matrix_begin:N #1{%
  \hskip -\arraycolsep
  \MH_let:NwN \@ifnextchar \MH_nospace_ifnextchar:Nnn
  \array{*\c@MaxMatrixCols #1}}
\def\MT_matrix_end:{\endarray \hskip -\arraycolsep}
%    \end{macrocode}
%  \end{macro}
%  \end{macro}
%  Before we define the environments we better make sure that spaces
%  before the optional argument is disallowed. Else a user who types
%  \begin{verbatim}
%  \[
%    \begin{pmatrix*}
%      [c] & a \\
%       b  & d
%    \end{pmatrix*}
%  \]
%  \end{verbatim}
%  will lose the \texttt{[c]}!
%    \begin{macrocode}
\MaybeMHPrecedingSpacesOff
%    \end{macrocode}
%  \begin{environment}{matrix*}
%  This environment is just like \env{matrix} only it takes an
%  optional argument specifying the column type.
%    \begin{macrocode}
\newenvironment{matrix*}[1][c]
  {\MT_matrix_begin:N #1}
  {\MT_matrix_end:}
%    \end{macrocode}
%  \end{environment}
%  \begin{environment}{pmatrix*}
%  \begin{environment}{bmatrix*}
%  \begin{environment}{Bmatrix*}
%  \begin{environment}{vmatrix*}
%  \begin{environment}{Vmatrix*}
%  Then starred versions of the other \AmS{} matrices.
%    \begin{macrocode}
\newenvironment{pmatrix*}[1][c]
  {\left(\MT_matrix_begin:N #1}
  {\MT_matrix_end:\right)}
\newenvironment{bmatrix*}[1][c]
  {\left[\MT_matrix_begin:N #1}
  {\MT_matrix_end:\right]}
\newenvironment{Bmatrix*}[1][c]
  {\left\lbrace\MT_matrix_begin:N #1}
  {\MT_matrix_end:\right\rbrace}
\newenvironment{vmatrix*}[1][c]
  {\left\lvert\MT_matrix_begin:N #1}
  {\MT_matrix_end:\right\rvert}
\newenvironment{Vmatrix*}[1][c]
  {\left\lVert\MT_matrix_begin:N #1}
  {\MT_matrix_end:\right\lVert}
%    \end{macrocode}
%  \end{environment}
%  \end{environment}
%  \end{environment}
%  \end{environment}
%  \end{environment}
%
% \changes{v1.10}{2011/02/12}{Added the code below, courtesy of Rasmus Villemoes}
% Now we are at it why not provide fenced versions of the
% \env{smallmatrix} construction as well. We will only provide a
% version that can be adjusted as \env{matrix*} above, thus we keep
% the * in the name. The implementation is courtesy of Rasmus
% Villemoes. Rasmus also suggested making the default alignment in
% these environments globally adjustable, so we did
% (\texttt{smallmatrix-align=c} by default). It \emph{is} possible to
% do something similar with the large matrix environments, but that
% might cause problems with the \texttt{array} package, thus for now
% we lease that feature alone.
%
% The base code is a variation over the original \env{smallmatrix}
% environmetn fround in \texttt{amsmath}, thus we will not comment it further.
% 
% TODO: make the code check that the optional argument is either
% \texttt{c}, \texttt{l} or \texttt{r}.
%    \begin{macrocode}
\def\MT_smallmatrix_begin:N #1{%
  \Let@\restore@math@cr\default@tag
  \baselineskip6\ex@ \lineskip1.5\ex@ \lineskiplimit\lineskip
  \csname MT_smallmatrix_#1_begin:\endcsname
}
\def\MT_smallmatrix_end:{\crcr\egroup\egroup\MT_smallmatrix_inner_space:}
\def\MT_smallmatrix_l_begin:{\null\MT_smallmatrix_inner_space:\vcenter\bgroup
  \ialign\bgroup$\m@th\scriptstyle##$\hfil&&\thickspace
  $\m@th\scriptstyle##$\hfil\crcr
}
\def\MT_smallmatrix_c_begin:{\null\MT_smallmatrix_inner_space:\vcenter\bgroup
  \ialign\bgroup\hfil$\m@th\scriptstyle##$\hfil&&\thickspace\hfil
  $\m@th\scriptstyle##$\hfil\crcr
}
\def\MT_smallmatrix_r_begin:{\null\MT_smallmatrix_inner_space:\vcenter\bgroup
  \ialign\bgroup\hfil$\m@th\scriptstyle##$&&\thickspace\hfil
  $\m@th\scriptstyle##$\crcr
}
\newenvironment{smallmatrix*}[1][\MT_smallmatrix_default_align:]
  {\MT_smallmatrix_begin:N #1}
  {\MT_smallmatrix_end:}
%    \end{macrocode}
% We would like to keep to the tradition of the \verb?Xmatrix? and
% \verb?Xmatrix*? macros we added earlier, since most code is similar
% we define them using a constructor macro. We also apply the trick
% used within \verb?\DeclarePairedDelimiter(X)? such that \verb?\left?
% \verb?\right?  constructions produce spacings corresponding to
% \verb?\mathopen? and \verb?\mathclose?.
%    \begin{macrocode}
\def\MT_fenced_sm_generator:nnn #1#2#3{%
  \@ifundefined{#1}{%
    \newenvironment{#1}
    {\@nameuse{#1hook}\mathopen{}\mathclose\bgroup\left#2\MT_smallmatrix_begin:N c}%
      {\MT_smallmatrix_end:\aftergroup\egroup\right#3}%
  }{}%
  \@ifundefined{#1*}{%
    \newenvironment{#1*}[1][\MT_smallmatrix_default_align:]%
    {\@nameuse{#1hook}\mathopen{}\mathclose\bgroup\left#2\MT_smallmatrix_begin:N ##1}%
      {\MT_smallmatrix_end:\aftergroup\egroup\right#3}%
  }{}%
}
\MT_fenced_sm_generator:nnn{psmallmatrix}()
\MT_fenced_sm_generator:nnn{bsmallmatrix}[]
\MT_fenced_sm_generator:nnn{Bsmallmatrix}\lbrace\rbrace
\MT_fenced_sm_generator:nnn{vsmallmatrix}\lvert\rvert
\MT_fenced_sm_generator:nnn{Vsmallmatrix}\lVert\rVert
%    \end{macrocode}
% 
% The options associated with this.
%    \begin{macrocode}
\define@key{\MT_options_name:}
  {smallmatrix-align}{\def\MT_smallmatrix_default_align:{#1}}
\define@key{\MT_options_name:}
  {smallmatrix-inner-space}{\def\MT_smallmatrix_inner_space:{#1}}
\setkeys{\MT_options_name:}{
  smallmatrix-align=c,
  smallmatrix-inner-space=\,
}

%    \end{macrocode}
%  Restore the usual spacing behavior.
%    \begin{macrocode}
\MHPrecedingSpacesOn
%    \end{macrocode}
%
%  \subsection{Smashing an operator with limits}
%
%  \begin{macro}{\smashoperator}
%  The user command. Define \cs{MT_smop_use:NNNNN} to be one of the
%  specialized commands \cs{MT_smop_smash_l:NNNNN},
%  \cs{MT_smop_smash_r:NNNNN}, or the default
%  \cs{MT_smop_smash_lr:NNNNN}.
%    \begin{macrocode}
\newcommand*\smashoperator[2][lr]{
  \def\MT_smop_use:NNNNN {\@nameuse{MT_smop_smash_#1:NNNNN}}
  \toks@{#2}
  \expandafter\MT_smop_get_args:wwwNnNn
    \the\toks@\@nil\@nil\@nil\@nil\@nil\@nil\@@nil
}
%    \end{macrocode}
%  \end{macro}
%  \begin{macro}{\MT_smop_remove_nil_vi:N}
%  \begin{macro}{\MT_smop_mathop:n}
%  \begin{macro}{\MT_smop_limits:}
%  Some helper functions.
%    \begin{macrocode}
\def\MT_smop_remove_nil_vi:N #1\@nil\@nil\@nil\@nil\@nil\@nil{#1}
\def\MT_smop_mathop:n {\mathop}
\def\MT_smop_limits: {\limits}
%    \end{macrocode}
%  \end{macro}
%  \end{macro}
%  \end{macro}
%  Some conditionals.
%    \begin{macrocode}
\MH_new_boolean:n {smop_one}
\MH_new_boolean:n {smop_two}
%    \end{macrocode}
%  \begin{macro}{\MT_smop_get_args:wwwNnNn}
%  The argument stripping. There are three different valid types of
%  input:
%  \begin{enumerate}
%    \item An operator with neither subscript nor superscript.
%    \item An operator with one subscript or superscript.
%    \item An operator with both subscript and superscript.
%  \end{enumerate}
%  Additionally an operator can be either a single macro as in
%  \cs{sum} or in \cs{mathop}\arg{A} and people might be tempted to
%  put a \cs{limits} after the operator, even though it's not
%  necessary. Thus the input with most tokens would be something like
%  \begin{verbatim}
%    \mathop{TTT}\limits_{sub}^{sup}
%  \end{verbatim}
%  Therefore we have to scan for seven arguments, but there might
%  only be one actually. So let's list the possible situations:
%  \begin{enumerate}
%    \item \verb|\mathop{TTT}\limits_{subsub}^{supsup}|
%    \item \verb|\mathop{TTT}_{subsub}^{supsup}|
%    \item \verb|\sum\limits_{subsub}^{supsup}|
%    \item \verb|\sum_{subsub}^{supsup}|
%  \end{enumerate}
%  Furthermore the |_{subsub}^{supsup}| part can also just be
%  |_{subsub}| or empty.
%    \begin{macrocode}
\def\MT_smop_get_args:wwwNnNn #1#2#3#4#5#6#7\@@nil{%
  \begingroup
    \def\MT_smop_arg_A: {#1} \def\MT_smop_arg_B: {#2}
    \def\MT_smop_arg_C: {#3} \def\MT_smop_arg_D: {#4}
    \def\MT_smop_arg_E: {#5} \def\MT_smop_arg_F: {#6}
    \def\MT_smop_arg_G: {#7}
%    \end{macrocode}
%  Check if A is \cs{mathop}. If it is, we know that B is the argument
%  of the \cs{mathop}.
%    \begin{macrocode}
    \if_meaning:NN \MT_smop_arg_A: \MT_smop_mathop:n
%    \end{macrocode}
%  If A was \cs{mathop} we check if C is \cs{limits}
%    \begin{macrocode}
      \if_meaning:NN \MT_smop_arg_C:\MT_smop_limits:
        \def\MT_smop_final_arg_A:{#1{#2}}%
%    \end{macrocode}
%  Now we have something like \verb|\mathop{TTT}\limits|. Then check
%  if D is \cs{@nil}.
%    \begin{macrocode}
        \if_meaning:NN \MT_smop_arg_D: \@nnil
        \else:
          \MH_set_boolean_T:n {smop_one}
          \MH_let:NwN \MT_smop_final_arg_B: \MT_smop_arg_D:
          \MH_let:NwN \MT_smop_final_arg_C: \MT_smop_arg_E:
          \if_meaning:NN \MT_smop_arg_F: \@nnil
          \else:
            \MH_set_boolean_T:n {smop_two}
            \MH_let:NwN \MT_smop_final_arg_D: \MT_smop_arg_F:
            \edef\MT_smop_final_arg_E:
              {\expandafter\MT_smop_remove_nil_vi:N \MT_smop_arg_G: }
          \fi:
        \fi:
      \else:
%    \end{macrocode}
%  Here we have something like \verb|\mathop{TTT}|. Still check
%  if D is \cs{@nil}.
%    \begin{macrocode}
        \def\MT_smop_final_arg_A:{#1{#2}}%
        \if_meaning:NN \MT_smop_arg_D: \@nnil
        \else:
          \MH_set_boolean_T:n {smop_one}
          \MH_let:NwN \MT_smop_final_arg_B: \MT_smop_arg_C:
          \MH_let:NwN \MT_smop_final_arg_C: \MT_smop_arg_D:
          \if_meaning:NN \MT_smop_arg_F: \@nnil
          \else:
            \MH_set_boolean_T:n {smop_two}
            \MH_let:NwN \MT_smop_final_arg_D: \MT_smop_arg_E:
            \MH_let:NwN \MT_smop_final_arg_E: \MT_smop_arg_F:
          \fi:
        \fi:
      \fi:
%    \end{macrocode}
%  If A was not \cs{mathop}, it is an operator in itself, so we check
%  if B is \cs{limits}
%    \begin{macrocode}
    \else:
      \if_meaning:NN \MT_smop_arg_B:\MT_smop_limits:
        \def\MT_smop_final_arg_A:{#1}%
        \if_meaning:NN \MT_smop_arg_D: \@nnil
        \else:
          \MH_set_boolean_T:n {smop_one}
          \MH_let:NwN \MT_smop_final_arg_B: \MT_smop_arg_C:
          \MH_let:NwN \MT_smop_final_arg_C: \MT_smop_arg_D:
          \if_meaning:NN \MT_smop_arg_F: \@nnil
          \else:
            \MH_set_boolean_T:n {smop_two}
            \MH_let:NwN \MT_smop_final_arg_D: \MT_smop_arg_E:
            \MH_let:NwN \MT_smop_final_arg_E: \MT_smop_arg_F:
          \fi:
        \fi:
      \else:
%    \end{macrocode}
%  No \cs{limits} was found, so we already have the right input. Just
%  forget about the last two arguments.
%    \begin{macrocode}
        \def\MT_smop_final_arg_A:{#1}%
        \if_meaning:NN \MT_smop_arg_C: \@nnil
        \else:
          \MH_set_boolean_T:n {smop_one}
          \MH_let:NwN \MT_smop_final_arg_B: \MT_smop_arg_B:
          \MH_let:NwN \MT_smop_final_arg_C: \MT_smop_arg_C:
          \if_meaning:NN \MT_smop_arg_D: \@nnil
          \else:
            \MH_set_boolean_T:n {smop_two}
            \MH_let:NwN \MT_smop_final_arg_D: \MT_smop_arg_D:
            \MH_let:NwN \MT_smop_final_arg_E: \MT_smop_arg_E:
          \fi:
        \fi:
      \fi:
    \fi:
%    \end{macrocode}
%  No reason to measure if there's no sub or sup.
%    \begin{macrocode}
    \MH_if_boolean:nT {smop_one}{
      \MT_smop_measure:NNNNN
      \MT_smop_final_arg_A: \MT_smop_final_arg_B: \MT_smop_final_arg_C:
      \MT_smop_final_arg_D: \MT_smop_final_arg_E:
    }
    \MT_smop_use:NNNNN
      \MT_smop_final_arg_A: \MT_smop_final_arg_B: \MT_smop_final_arg_C:
      \MT_smop_final_arg_D: \MT_smop_final_arg_E:
  \endgroup
}
%    \end{macrocode}
%  \end{macro}
%  Typeset what is necessary and ignore width of sub and sup:
%    \begin{macrocode}
\def\MT_smop_needed_args:NNNNN #1#2#3#4#5{%
  \displaystyle #1
  \MH_if_boolean:nT {smop_one}{
%    \end{macrocode}
%  Let's use the internal versions of \cs{crampedclap} now that we now
%  it is set in \cs{scriptstyle}.
%    \begin{macrocode}
    \limits#2{\MT_cramped_clap_internal:Nn \scriptstyle{#3}}
    \MH_if_boolean:nT {smop_two}{
      #4{\MT_cramped_clap_internal:Nn \scriptstyle{#5}}
    }
  }
}
%    \end{macrocode}
%  Measure the natural width. \cs{@tempdima} holds the dimen we need to
%  adjust it all with.
%    \begin{macrocode}
\def\MT_smop_measure:NNNNN #1#2#3#4#5{%
  \MH_let:NwN \MT_saved_mathclap:Nn \MT_cramped_clap_internal:Nn
  \MH_let:NwN \MT_cramped_clap_internal:Nn \@secondoftwo
  \sbox\z@{$\m@th\MT_smop_needed_args:NNNNN #1#2#3#4#5$}
  \MH_let:NwN \MT_cramped_clap_internal:Nn \MT_saved_mathclap:Nn
  \sbox\tw@{$\m@th\displaystyle#1$}
  \@tempdima=.5\wd0
  \advance\@tempdima-.5\wd2
}
%    \end{macrocode}
%  The `l' variant
%    \begin{macrocode}
\def\MT_smop_smash_l:NNNNN #1#2#3#4#5{
  \MT_smop_needed_args:NNNNN #1#2#3#4#5\kern\@tempdima
}
%    \end{macrocode}
%  The `r' variant
%    \begin{macrocode}
\def\MT_smop_smash_r:NNNNN #1#2#3#4#5{
  \kern\@tempdima\MT_smop_needed_args:NNNNN #1#2#3#4#5
}
%    \end{macrocode}
%  The `lr' variant
%    \begin{macrocode}
\def\MT_smop_smash_lr:NNNNN #1#2#3#4#5{
  \MT_smop_needed_args:NNNNN #1#2#3#4#5
}
%    \end{macrocode}
%
%
%  \subsection{Adjusting limits}
%
%
%  \begin{macro}{\MT_vphantom:Nn}
%  \begin{macro}{\MT_hphantom:Nn}
%  \begin{macro}{\MT_phantom:Nn}
%  \begin{macro}{\MT_internal_phantom:N}
%  The main advantage of \cs{phantom} et al., is the ability to
%  choose the right size automatically, but it requires the input to
%  be typeset four times. Since we will need to have a \cs{cramped}
%  inside a \cs{vphantom} it is much, much faster to choose the style
%  ourselves (we already know it). These macros make it possible.
%    \begin{macrocode}
\def\MT_vphantom:Nn {\v@true\h@false\MT_internal_phantom:N}
\def\MT_hphantom:Nn {\v@false\h@true\MT_internal_phantom:N}
\def\MT_phantom:Nn {\v@true\h@true\MT_internal_phantom:N}
\def\MT_internal_phantom:N #1{
  \ifmmode
    \expandafter\mathph@nt\expandafter#1
  \else
    \expandafter\makeph@nt
  \fi
}
%    \end{macrocode}
%  \end{macro}
%  \end{macro}
%  \end{macro}
%  \end{macro}
%
%  \begin{macro}{\adjustlimits}
%  This is for making sure limits line up on two consecutive
%  operators.
%    \begin{macrocode}
\newcommand*\adjustlimits[6]{
%    \end{macrocode}
%  We measure the two operators and save the difference of their
%  depths.
%    \begin{macrocode}
  \sbox\z@{$\m@th \displaystyle #1$}
  \sbox\tw@{$\m@th \displaystyle #4$}
  \@tempdima=\dp\z@ \advance\@tempdima-\dp\tw@
%    \end{macrocode}
%  We force \cs{displaystyle} for the operator and \cs{scripstyle}
%  for the limit. If we make use of the regular \cs{smash},
%  \cs{vphantom}, and \cs{cramped} macros, and let \TeX{} choose the
%  right style for each one of them, we get a lot of redundant code
%  as we have no need for the combination
%  $(\cs{displaystyle},\cs{textstyle})$ etc. Only
%  $(\cs{scriptstyle},\cs{scriptstyle})$ is useful.
%    \begin{macrocode}
  \if_dim:w \@tempdima>\z@
    \mathop{#1}\limits#2{#3}
  \else:
    \mathop{#1\MT_vphantom:Nn \displaystyle{#4}}\limits
    #2{
        \def\finsm@sh{\ht\z@\z@ \box\z@}
        \mathsm@sh\scriptstyle{\MT_cramped_internal:Nn \scriptstyle{#3}}
        \MT_vphantom:Nn \scriptstyle
          {\MT_cramped_internal:Nn \scriptstyle{#6}}
    }
  \fi:
  \if_dim:w \@tempdima>\z@
    \mathop{#4\MT_vphantom:Nn \displaystyle{#1}}\limits
    #5
    {
      \MT_vphantom:Nn \scriptstyle
        {\MT_cramped_internal:Nn \scriptstyle{#3}}
      \def\finsm@sh{\ht\z@\z@ \box\z@}
      \mathsm@sh\scriptstyle{\MT_cramped_internal:Nn \scriptstyle{#6}}
    }
  \else:
    \mathop{#4}\limits#5{#6}
  \fi:
}
%    \end{macrocode}
%  \end{macro}
% 
%  \subsection{Swapping above display skip}
%
% \begin{macro}{\SwapAboveDisplaySkip}
%   This macro is intended to be used at the start of \AmS\
%   environments, in order to force it to use
%   \cs{abovedisplayshortskip} instead of \cs{abovedisplayskip} above
%   the displayed math. Because of the use of \cs{noalign} it will not
%   work inside \env{equation} or \env{multline}.
%    \begin{macrocode}
\newcommand\SwapAboveDisplaySkip{%
  \noalign{\vskip-\abovedisplayskip\vskip\abovedisplayshortskip}
}

%    \end{macrocode}
%   
% \end{macro}
%
%  \subsection{An aid to alignment}
%
% \begin{macro}{\MoveEqLeft}
%   \changes{v1.05}{2008/06/05}{Added \cs{MoveEqLeft} (daleif)}
%   \changes{v1.05b}{2008/06/18}{We don't need \cs{setlength} here
%     (daleif), after discussion about \cs{global} and \cs{setlength}
%     on ctt} 
%   \changes{v1.12}{2011/06/12}{We don't even need lengths. GL
%   suggested on ctt to just apply them directly.}
%   This is a very simple macro, we `move' a line in an
%   alignment backwards in order to simulate that all subsequent lines
%   have been indented. Note that simply using \verb+\kern-2m+ after
%   the \verb+&+ is not enough, then the alignemnt environment never
%   detects that there is anything (though simulated) in the cell
%   before the \verb+&+.
%    \begin{macrocode}
\newcommand\MoveEqLeft[1][2]{\kern #1em  &   \kern -#1em}
%    \end{macrocode}
% \end{macro}
%
% \begin{macro}{\Aboxed}
% \changes{v1.08}{2010/06/29}{Added \cs{Aboxed}}
% The idea from \cs{MoveEqLeft} can be used for other things. Here we
% create a macro that will allow a user to box an equation inside an
% alignment.
% \changes{v1.12}{2011/08/17}{\cs{Aboxed} reimplemented, cudos to GL}
%    \begin{macrocode}
\newcommand\Aboxed[1]{\let\bgroup{\romannumeral-`}\@Aboxed#1&&\ENDDNE}
%    \end{macrocode}
% The macro has been reimplemented courtesy of Florent Chervet out of
% a posting on ctt, \url{https://groups.google.com/group/comp.text.tex/browse_thread/thread/5d66395f2a1b5134/93fd9661484bd8d8?#93fd9661484bd8d8}
%    \begin{macrocode}
\def\@Aboxed#1&#2&#3\ENDDNE{%
  \ifnum0=`{}\fi \setbox \z@
    \hbox{$\displaystyle#1{}\m@th$\kern\fboxsep \kern\fboxrule }%
    \edef\@tempa {\kern  \wd\z@ &\kern -\the\wd\z@ \fboxsep
        \the\fboxsep \fboxrule \the\fboxrule }\@tempa \boxed {#1#2}%
} 
%    \end{macrocode}
% \end{macro}
%
% \begin{macro}{\ArrowBetweenLines}
%   \changes{v1.05}{2008/06/05}{Added \cs{ArrowBetweenLines} as it
%   belongs here and not just in my \LaTeX book (daleif)}
%   ????Implementation notes are needed????
%    \begin{macrocode}
\MHInternalSyntaxOff
\def\ArrowBetweenLines{\relax
  \iffalse{\fi\ifnum0=`}\fi
  \@ifstar{\ArrowBetweenLines@auxI{00}}{\ArrowBetweenLines@auxI{01}}}
\def\ArrowBetweenLines@auxI#1{%
  \@ifnextchar[%
  {\ArrowBetweenLines@auxII{#1}}%
  {\ArrowBetweenLines@auxII{#1}[\Updownarrow]}}
\def\ArrowBetweenLines@auxII#1[#2]{%
  \ifnum0=`{\fi \iffalse}\fi
%    \end{macrocode}
% It turns out that for some reason the \cs{crcr} (next) removes the
% automatic equation number replacement. The replacement hack seems to
% the trick, though I have no idea why \cs{crcr} broke things (/daleif).
% \changes{v1.08}{2010/06/15}{fixed eq num replacement bug}
%    \begin{macrocode}
%  \crcr
    \expandafter\in@\expandafter{\@currenvir}%
      {alignedat,aligned,gathered}%
      \ifin@ \else
      \notag
      \fi%
   \\
  \noalign{\nobreak\vskip-\baselineskip\vskip-\lineskip}%
  \noalign{\expandafter\in@\expandafter{\@currenvir}%
      {alignedat,aligned,gathered}%
      \ifin@ \else\notag\fi%
  }%
  \if#1 &&\quad #2\else #2\quad\fi
  \\\noalign{\nobreak\vskip-\lineskip}}

\MHInternalSyntaxOn
%    \end{macrocode}
% \end{macro}
%
%
% \subsection{Centered vertical dots}
% 
% Doing a \verb?\vdots? centered within a different sized box, is
% rather easy with the tools available. Note that it does \emph{not}
% check for the style we are running in, thus do not expect this to
% work well within \verb?\scriptstyle? and smaller. Basically we
% create a box of a width corresponding to \verb?{}#1{}? and center
% the \verb?\vdots? within it.
%    \begin{macrocode}
\newcommand\vdotswithin[1]{%
  {\mathmakebox[\widthof{\ensuremath{{}#1{}}}][c]{{\vdots}}}}
%    \end{macrocode}
% Next we are inspired by \verb?\ArrowBetweenLines? and provide a
% costruction to be used within alignments with much less vertical
% space above and below. 
%
% First in order to support \env{spreadlines} we need to store the
% original value of \verb?\jot? (and hope the user does not mess with it).
%    \begin{macrocode}
\newlength\origjot
\setlength\origjot{\jot}
%    \end{macrocode}
% Next define how much we spacing we flush out, and make this user adjustable.
%    \begin{macrocode}
\newdimen\l_MT_shortvdotswithinadjustabove_dim
\newdimen\l_MT_shortvdotswithinadjustbelow_dim
\define@key{\MT_options_name:}
  {shortvdotsadjustabove}{\setlength\l_MT_shortvdotswithinadjustabove_dim{#1}}
\define@key{\MT_options_name:}
  {shortvdotsadjustbelow}{\setlength\l_MT_shortvdotswithinadjustbelow_dim{#1}}
%    \end{macrocode}
% The actual defaults we found by trail and error.
%    \begin{macrocode}
\setkeys{\MT_options_name:}{
  shortvdotsadjustabove=2.15\origjot,
  shortvdotsadjustbelow=\origjot
}
%    \end{macrocode}
% The user macro comes in two versions, starred version corresponding
% to alignment \emph{before} and \verb?&? and a non-starred version
% with alignment \emph{after} \verb?&?.
%    \begin{macrocode}
\def\shortvdotswithin{\relax
  \@ifstar{\MT_svwi_aux:nn{00}}{\MT_svwi_aux:nn{01}}}
\def\MT_svwi_aux:nn #1#2{
  \MTFlushSpaceAbove
  \if#1 \vdotswithin{#2}& \else &\vdotswithin{#2}  \fi
  \MTFlushSpaceBelow
}
%    \end{macrocode}
% We will need a way to remove any tags (eq. numbers) on the
% \verb?\vdots? line. We cannot use the method used by
% \verb?\ArrowBetweenLines? so we use inspiration from
% \texttt{etoolbox}.  
%    \begin{macrocode}
\def\MT_remove_tag_unless_inner:n #1{%
  \begingroup
  \def\etb@tempa##1|#1|##2\MT@END{\endgroup
    \ifx\@empty##2\@empty\notag\fi}%
  \expandafter\etb@tempa\expandafter|alignedat|aligned|split|#1|\MT@END}
%    \end{macrocode}
% These macros take care of removing the space above or below. Since
% these may be useful for the user in very special cases, we provide
% them as separate macros.
%    \begin{macrocode}
\newcommand\MTFlushSpaceAbove{  
  \expandafter\MT_remove_tag_unless_inner:n\expandafter{\@currenvir}
  \\
  \noalign{%
    \nobreak\vskip-\baselineskip\vskip-\lineskip%
      \vskip-\l_MT_shortvdotswithinadjustabove_dim 
      \vskip-\origjot
      \vskip\jot
  }%
  \noalign{
    \expandafter\MT_remove_tag_unless_inner:n\expandafter{\@currenvir}
  }
}
\newcommand\MTFlushSpaceBelow{
  \\\noalign{%
    \nobreak\vskip-\lineskip
    \vskip-\l_MT_shortvdotswithinadjustbelow_dim
    \vskip-\origjot
    \vskip\jot
  }
}

%    \end{macrocode}
%
%
%  \section{A few extra symbols}
%
%  Most math font sets are missing three symbols: \cs{nuparrow},
%  \cs{ndownarrow} and \cs{bigtimes}. We provide \emph{simulated}
%  versions of these symbols in case they are missing. 
%
%  \subsection{Negated up- and down arrows} 
%  
%  Note that the \cs{nuparrow} and the \cs{ndownarrow} are made from
%  \cs{nrightarrow} and \cs{nleftarrow}, so these have to be
%  present. If they are not, we throw an error at use. The
%  implementation details are due to Enrico Gregorio
%  (\url{http://groups.google.com/group/comp.text.tex/msg/689cc8bd604fdb51}),
%  the basic idea is to reflect and rotate existing negated
%  arrows. Note that the reflection and rotation will not show up i
%  most DVI previewers.
% \begin{macro}{\MH_nrotarrow:NN}
% \changes{v1.07}{2008/08/11}{Added support for \cs{nuparrow} and \cs{ndownaddow}}
%  First a common construction macro.
%    \begin{macrocode}
\def\MH_nrotarrow:NN #1#2{%
  \setbox0=\hbox{$\m@th#1\uparrow$}\dimen0=\dp0
  \setbox0=\hbox{%
    \reflectbox{\rotatebox[origin=c]{90}{$\m@th#1\mkern2.22mu #2$}}}%
  \dp0=\dimen0 \box0 \mkern2.3965mu
}
%    \end{macrocode}
% \end{macro}
% The negated arrows are then made using this macro on respectively
% \cs{nrightarrow} and \cs{nleftarrow}
% \begin{macro}{\MH_nuparrow:}
% \begin{macro}{\MH_ndownarrow:}
%    \begin{macrocode}
\def\MH_nuparrow: {%
  \mathrel{\mathpalette\MH_nrotarrow:NN\nrightarrow} }
\def\MH_ndownarrow: {%
  \mathrel{\mathpalette\MH_nrotarrow:NN\nleftarrow} }
%    \end{macrocode}
% \end{macro}
% \end{macro}
% \begin{macro}{\nuparrow}
% \begin{macro}{\ndownarrow}
%   Next we provide \cs{nuparrow} and \cs{ndownarrow} at begin
%   document. Since they depend on \cs{nrightarrow} and
%   \cs{nleftarrow} we test for these and let the macros throw an
%   error if they are missing.
% \changes{v1.08b}{2010/07/21}{Moved graphicx loading down here such
% that we do not get into option clash problems}
%    \begin{macrocode}
\AtBeginDocument{%
  \RequirePackage{graphicx}%
  \@ifundefined{nrightarrow}{%
    \providecommand\nuparrow{%
      \PackageError{mathtools}{\string\nuparrow\space~ is~
        constructed~ from~ \string\nrightarrow,~ which~ is~ not~
        provided.~ Please~ load~ the~ amssymb~ package~ or~ similar}{}
    }}{ \providecommand\nuparrow{\MH_nuparrow:}}
  \@ifundefined{nleftarrow}{%
    \providecommand\ndownarrow{%
      \PackageError{mathtools}{\string\ndownarrow\space~ is~
        constructed~ from~ \string\nleftarrow,~ which~ is~ not~
        provided.~ Please~ load~ the~ amssymb~ package~ or~ similar}{}
    }}{ \providecommand\ndownarrow{\MH_ndownarrow:}} }
%    \end{macrocode}
% \end{macro}
% \end{macro}
% 
%
%  \subsection{Providing bigtimes}
%
%  The idea is to use the original \cs{times} and then scale it
%  accordingly. Again the implementation details have been improved by
%  Enrico Gregorio
%  (\url{http://groups.google.com/group/comp.text.tex/msg/9685c9405df2ff94}). 
%
% \begin{macro}{\MH_bigtimes_scaler:N}
% \begin{macro}{\MH_bigtimes_inner:}
% \begin{macro}{\MH_csym_bigtimes:}
% \changes{v1.07}{2008/08/11}{Added support for \cs{bigtimes}}
%    \begin{macrocode}
\def\MH_bigtimes_scaler:N #1{%
  \vcenter{\hbox{#1$\m@th\mkern-2mu\times\mkern-2mu$}}}
%    \end{macrocode}
%  This is then combined with \cs{mathchoice} to form the inner parts
%  of the macro
%    \begin{macrocode}
\def\MH_bigtimes_inner: {
  \mathchoice{\MH_bigtimes_scaler:N \huge}         % display style
             {\MH_bigtimes_scaler:N \LARGE}        % text style
             {\MH_bigtimes_scaler:N {}}            % script style
             {\MH_bigtimes_scaler:N \footnotesize} % script script style
}
%    \end{macrocode}
%  And thus the internal prepresentaion of the \cs{bigtimes} macro.
%    \begin{macrocode}
\def\MH_csym_bigtimes: {\mathop{\MH_bigtimes_inner:}\displaylimits}
%    \end{macrocode}
% \end{macro}
% \end{macro}
% \end{macro}
% \begin{macro}{\bigtimes}
%   In the end we provide \cs{bigtimes} if otherwise not defined.
%    \begin{macrocode}
\AtBeginDocument{
  \providecommand\bigtimes{\MH_csym_bigtimes:}
}
%    \end{macrocode}
% \end{macro}
%
%  \section{Macros by other people}
%
%  \subsection{Intertext and short intertext}
%
%  It turns out that \cs{intertext} use a bit too much
%  space. Especially noticable if combined with the \env{spreadlines}
%  environment, the extra space is also applied above and below
%  \cs{intertext}, which ends up looking unproffesional.  Chung-chieh
%  Shan
%  (\url{http://conway.rutgers.edu/~ccshan/wiki/blog/posts/Beyond_amsmath/})
%  via Tobias Weh suggested a fix. We apply it here, but also keep the
%  original \cs{intertext} in case a user would rather want it.
% \begin{macro}{\MT_orig_intertext:}
% \begin{macro}{\MT_intertext:}
% \changes{v1.13}{2012/08/19}{\cs{l_MT_X_intertext_dim} renamed to 
% \cs{l_MT_X_intertext_sep}}
% \begin{macro}{\l_MT_above_intertext_sep}
% \begin{macro}{\l_MT_below_intertext_sep}
%  First store the originam Ams version.
%    \begin{macrocode}
\MH_let:NwN \MT_orig_intertext: \intertext@
%    \end{macrocode}
% And then for some reconfiguration. First a few lengths
% \changes{v1.13}{2012/08/19}{Fixed typos and changed the names}
%    \begin{macrocode}
\newdimen\l_MT_above_intertext_sep
\newdimen\l_MT_below_intertext_sep
\define@key{\MT_options_name:}
  {aboveintertextdim}{\setlength\l_MT_above_intertext_sep{#1}}
\define@key{\MT_options_name:}
  {belowintertextdim}{\setlength\l_MT_below_intertext_sep{#1}}
\define@key{\MT_options_name:}
  {above-intertext-dim}{\setlength\l_MT_above_intertext_sep{#1}}
\define@key{\MT_options_name:}
  {below-intertext-dim}{\setlength\l_MT_below_intertext_sep{#1}}
\define@key{\MT_options_name:}
  {above-intertext-sep}{\setlength\l_MT_above_intertext_sep{#1}}
\define@key{\MT_options_name:}
  {below-intertext-sep}{\setlength\l_MT_below_intertext_sep{#1}}
%    \end{macrocode}
% Their default values are zero. Now for our extended version of
% CCShan's solution.  
%    \begin{macrocode}
\def\MT_intertext: {%
  \def\intertext##1{%
    \ifvmode\else\\\@empty\fi
    \noalign{%
      \penalty\postdisplaypenalty\vskip\belowdisplayskip
      \vskip-\lineskiplimit      % CCS
      \vskip\normallineskiplimit % CCS
      \vskip\l_MT_above_intertext_sep
       \vbox{\normalbaselines
        \ifdim\linewidth=\columnwidth
        \else \parshape\@ne \@totalleftmargin \linewidth
        \fi
        \noindent##1\par}%
      \penalty\predisplaypenalty\vskip\abovedisplayskip%
      \vskip-\lineskiplimit      % CCS
      \vskip\normallineskiplimit % CCS
      \vskip\l_MT_above_intertext_sep
   }%
}}
%    \end{macrocode}
% And provide a key to switch
%    \begin{macrocode}
\def\MT_orig_intertext_true:  { \MH_let:NwN \intertext@ \MT_orig_intertext: }
\def\MT_orig_intertext_false: { \MH_let:NwN \intertext@ \MT_intertext: }
\define@key{\MT_options_name:}{original-intertext}[true]{
  \@nameuse{MT_orig_intertext_#1:}
}
%    \end{macrocode}
% And use the new version as default.
%    \begin{macrocode}
\setkeys{\MT_options_name:}{
  original-intertext=false
}
%    \end{macrocode}
% 
% \end{macro}
% \end{macro}
% \end{macro}
% \end{macro}
%
%  Gabriel Zachmann, Donald Arseneau on comp.text.tex 2000/05/12-13
%  \begin{macro}{\shortintertext}
%  \begin{macro}{\MT_orig_shortintertext}
%  \begin{macro}{\MT_shortintertext}
%  \begin{macro}{\l_above_shortintertext_sep}
%  \begin{macro}{\l_below_shortintertext_sep}
%  This is like \cs{intertext} but uses shorter skips between the
%  math. Again this turned out to have the same problem as
%  \cs{intertext}, so we provide two versions.
%    \begin{macrocode}
\def\MT_orig_shortintertext:n #1{%
  \ifvmode\else\\\@empty\fi
  \noalign{%
    \penalty\postdisplaypenalty\vskip\abovedisplayshortskip
    \vbox{\normalbaselines
      \if_dim:w \linewidth=\columnwidth
      \else:
        \parshape\@ne \@totalleftmargin \linewidth
      \fi:
      \noindent#1\par}%
    \penalty\predisplaypenalty\vskip\abovedisplayshortskip%
  }%
}
%    \end{macrocode}
% Lengths like above
% \changes{v1.13}{2012/08/19}{The option was named differently in the
% manual. Also renamed to use the postfix \emph{sep} instead. Though
% the old names remain for compatibility.}
%    \begin{macrocode}
\newdimen\l_MT_above_shortintertext_sep
\newdimen\l_MT_below_shortintertext_sep
\define@key{\MT_options_name:}
  {aboveshortintertextdim}{\setlength \l_MT_above_shortintertext_sep{#1}}
\define@key{\MT_options_name:}
  {belowshortintertextdim}{\setlength \l_MT_below_shortintertext_sep{#1}}
\define@key{\MT_options_name:}
  {above-short-intertext-dim}{\setlength \l_MT_above_shortintertext_sep{#1}}
\define@key{\MT_options_name:}
  {below-short-intertext-dim}{\setlength \l_MT_below_shortintertext_sep{#1}}
\define@key{\MT_options_name:}
  {above-short-intertext-sep}{\setlength \l_MT_above_shortintertext_sep{#1}}
\define@key{\MT_options_name:}
  {below-short-intertext-sep}{\setlength \l_MT_below_shortintertext_sep{#1}}
%    \end{macrocode}
% Looks best with the `old' values of the original \cs{jot}
% setting. So we set them to 3pt each.
%    \begin{macrocode}
\setkeys{\MT_options_name:}{
  aboveshortintertextdim=3pt,
  belowshortintertextdim=3pt
}
%    \end{macrocode}
%  Next, just add the same as we did for \cs{intertext}
%    \begin{macrocode}
\def\MT_shortintertext:n #1{%
  \ifvmode\else\\\@empty\fi
  \noalign{%
    \penalty\postdisplaypenalty\vskip\abovedisplayshortskip
    \vskip-\lineskiplimit      
    \vskip\normallineskiplimit 
    \vskip\l_MT_above_shortintertext_sep
    \vbox{\normalbaselines
      \if_dim:w \linewidth=\columnwidth
      \else:
        \parshape\@ne \@totalleftmargin \linewidth
      \fi:
      \noindent#1\par}%
    \penalty\predisplaypenalty\vskip\abovedisplayshortskip%
    \vskip-\lineskiplimit      
    \vskip\normallineskiplimit 
    \vskip\l_MT_below_shortintertext_sep
  }%
}
%    \end{macrocode}
% Next we need to be able to switch.
%    \begin{macrocode}
\def\MT_orig_shortintertext_true:  { \MH_let:NwN \shortintertext \MT_orig_shortintertext:n }
\def\MT_orig_shortintertext_false: { \MH_let:NwN \shortintertext \MT_shortintertext:n }
\define@key{\MT_options_name:}{original-shortintertext}[true]{
  \@nameuse{MT_orig_shortintertext_#1:}
}
%    \end{macrocode}
% With the updated one as the default.
%    \begin{macrocode}
\setkeys{\MT_options_name:}{
  original-shortintertext=false
}
%    \end{macrocode}
%  \end{macro}
%  \end{macro}
%  \end{macro}
%  \end{macro}
%  \end{macro}
%
%  \subsection{Fine-tuning mathematical layout}
%
%  \subsubsection{A complement to \texttt{\textbackslash smash},
%  \texttt{\textbackslash llap}, and \texttt{\textbackslash rlap}}
%  \begin{macro}{\clap}
%  \begin{macro}{\mathllap}
%  \begin{macro}{\mathrlap}
%  \begin{macro}{\mathclap}
%  \begin{macro}{\MT_mathllap:Nn}
%  \begin{macro}{\MT_mathrlap:Nn}
%  \begin{macro}{\MT_mathclap:Nn}
%  First we'll \cs{provide} those macros (they are so simple that I
%  think other packages might define them as well).
%    \begin{macrocode}
\providecommand*\clap[1]{\hb@xt@\z@{\hss#1\hss}}
\providecommand*\mathllap[1][\@empty]{
  \ifx\@empty#1\@empty
    \expandafter \mathpalette \expandafter \MT_mathllap:Nn
  \else
    \expandafter \MT_mathllap:Nn \expandafter #1
  \fi
}
\providecommand*\mathrlap[1][\@empty]{
  \ifx\@empty#1\@empty
    \expandafter \mathpalette \expandafter \MT_mathrlap:Nn
  \else
    \expandafter \MT_mathrlap:Nn \expandafter #1
  \fi
}
\providecommand*\mathclap[1][\@empty]{
  \ifx\@empty#1\@empty
    \expandafter \mathpalette \expandafter \MT_mathclap:Nn
  \else
    \expandafter \MT_mathclap:Nn \expandafter #1
  \fi
}
%    \end{macrocode}
%  We have to insert |{}| because we otherwise risk triggering a
%  ``feature'' in \TeX.
%    \begin{macrocode}
\def\MT_mathllap:Nn #1#2{{}\llap{$\m@th#1{#2}$}}
\def\MT_mathrlap:Nn #1#2{{}\rlap{$\m@th#1{#2}$}}
\def\MT_mathclap:Nn #1#2{{}\clap{$\m@th#1{#2}$}}
%    \end{macrocode}
%  \end{macro}
%  \end{macro}
%  \end{macro}
%  \end{macro}
%  \end{macro}
%  \end{macro}
%  \end{macro}
%  \begin{macro}{\mathmbox}
%  \begin{macro}{\MT_mathmbox:nn}
%  \begin{macro}{\mathmakebox}
%  \begin{macro}{\MT_mathmakebox_I:w}
%  \begin{macro}{\MT_mathmakebox_II:w}
%  \begin{macro}{\MT_mathmakebox_III:w}
%  Then the \cs{mathmbox}\marg{arg} and
%  \cs{mathmakebox}\oarg{width}\oarg{pos}\marg{arg} macros which are
%  very similar to \cs{mbox} and \cs{makebox}. The differences are:
%  \begin{itemize}
%    \item \meta{arg} is set in math mode of course.
%    \item No need for \cs{leavevmode} as we're in math mode.
%    \item No need to make them \cs{long} (we're still in math mode).
%    \item No need to support a picture version.
%  \end{itemize}
%  The first is easy.
%    \begin{macrocode}
\providecommand*\mathmbox{\mathpalette\MT_mathmbox:nn}
\def\MT_mathmbox:nn #1#2{\mbox{$\m@th#1#2$}}
%    \end{macrocode}
%  We scan for the optional arguments first.
%    \begin{macrocode}
\providecommand*\mathmakebox{
  \@ifnextchar[  \MT_mathmakebox_I:w
                 \mathmbox}
\def\MT_mathmakebox_I:w[#1]{%
  \@ifnextchar[  {\MT_mathmakebox_II:w[#1]}
                 {\MT_mathmakebox_II:w[#1][c]}}
%    \end{macrocode}
%  We had to get the optional arguments out of the way before calling
%  upon the powers of \cs{mathpalette}.
%    \begin{macrocode}
\def\MT_mathmakebox_II:w[#1][#2]{
  \mathpalette{\MT_mathmakebox_III:w[#1][#2]}}
\def\MT_mathmakebox_III:w[#1][#2]#3#4{%
  \@begin@tempboxa\hbox{$\m@th#3#4$}%
    \setlength\@tempdima{#1}%
    \hbox{\hb@xt@\@tempdima{\csname bm@#2\endcsname}}%
  \@end@tempboxa}
%    \end{macrocode}
%  \end{macro}
%  \end{macro}
%  \end{macro}
%  \end{macro}
%  \end{macro}
%  \end{macro}
%  \begin{macro}{\mathsm@sh}
%  Fix \cs{smash}.
%    \begin{macrocode}
\def\mathsm@sh#1#2{%
  \setbox\z@\hbox{$\m@th#1{#2}$}{}\finsm@sh}
%    \end{macrocode}
%  \end{macro}
%
%
%  \subsubsection{A cramped style}
%
%  comp.text.tex on 1992/07/21 by Michael Herschorn.
%  With speed-ups by the Grand Wizard himself as shown on
%  \begin{quote}\rightskip-\leftmargini
%  \url{http://www.tug.org/tex-archive/digests/tex-implementors/042}
%  \end{quote}
%  The (better) user interface by the author.
%
%  \begin{macro}{\cramped}
%  Make sure the expansion is timed correctly.
%    \begin{macrocode}
\providecommand*\cramped[1][\@empty]{
  \ifx\@empty#1\@empty
    \expandafter \mathpalette \expandafter \MT_cramped_internal:Nn
  \else
    \expandafter \MT_cramped_internal:Nn \expandafter #1
  \fi
}
%    \end{macrocode}
%  \end{macro}
%  \begin{macro}{\MT_cramped_internal:Nn}
%  The internal command.
%    \begin{macrocode}
\def\MT_cramped_internal:Nn #1#2{
%    \end{macrocode}
%  Create a box containing the math and force a cramped style by
%  issuing a non-existing radical.
%    \begin{macrocode}
  \sbox\z@{$\m@th#1\nulldelimiterspace=\z@\radical\z@{#2}$}
%    \end{macrocode}
%  Then make sure the height is correct.
%    \begin{macrocode}
    \ifx#1\displaystyle
      \dimen@=\fontdimen8\textfont3
      \advance\dimen@ .25\fontdimen5\textfont2
    \else
      \dimen@=1.25\fontdimen8
      \ifx#1\textstyle\textfont
      \else
        \ifx#1\scriptstyle
          \scriptfont
        \else
          \scriptscriptfont
        \fi
      \fi
      3
    \fi
    \advance\dimen@-\ht\z@ \ht\z@=-\dimen@
    \box\z@
}
%    \end{macrocode}
%  \end{macro}
%
%  \subsubsection{Cramped versions of \texttt{\textbackslash
%  mathllap}, \texttt{\textbackslash mathclap}, and
%  \texttt{\textbackslash mathrlap}}
%  \begin{macro}{\crampedllap}
%  \begin{macro}{\MT_cramped_llap_internal:Nn}
%  \begin{macro}{\crampedclap}
%  \begin{macro}{\MT_cramped_clap_internal:Nn}
%  \begin{macro}{\crampedrlap}
%  \begin{macro}{\MT_cramped_rlap_internal:Nn}
%  Cramped versions of \cs{mathXlap} (for speed). Made by the author.
%    \begin{macrocode}
\providecommand*\crampedllap[1][\@empty]{
  \ifx\@empty#1\@empty
    \expandafter \mathpalette \expandafter \MT_cramped_llap_internal:Nn
  \else
    \expandafter \MT_cramped_llap_internal:Nn \expandafter #1
  \fi
}
\def\MT_cramped_llap_internal:Nn #1#2{
  {}\llap{\MT_cramped_internal:Nn #1{#2}}
}
\providecommand*\crampedclap[1][\@empty]{
  \ifx\@empty#1\@empty
    \expandafter \mathpalette \expandafter \MT_cramped_clap_internal:Nn
  \else
    \expandafter \MT_cramped_clap_internal:Nn \expandafter #1
  \fi
}
\def\MT_cramped_clap_internal:Nn #1#2{
  {}\clap{\MT_cramped_internal:Nn #1{#2}}
}
\providecommand*\crampedrlap[1][\@empty]{
  \ifx\@empty#1\@empty
    \expandafter \mathpalette \expandafter \MT_cramped_rlap_internal:Nn
  \else
    \expandafter \MT_cramped_rlap_internal:Nn \expandafter #1
  \fi
}
\def\MT_cramped_rlap_internal:Nn #1#2{
  {}\rlap{\MT_cramped_internal:Nn #1{#2}}
}
%    \end{macrocode}
%  \end{macro}
%  \end{macro}
%  \end{macro}
%  \end{macro}
%  \end{macro}
%  \end{macro}
%
%
%  \section{Macros by Michael J.~Downes}
%
%  The macros in this section are all by Michael J.~Downes. Either
%  they are straight copies of his original macros or inspired and
%  extended here.
%
%
%  \subsection{Prescript}
%  \begin{macro}{\prescript}
%  This command is taken from a posting to comp.text.tex on
%  December~20th 2000 by Michael J.~Downes. The comments are his. I
%  have added some formatting options to the arguments so that a user
%  can emulate the \pkg{isotope} package.
%
% \changes{v.1.12}{2012/04/19}{Extended \cs{prescript} to change style
% if used in say S context. Requestd by Oliver Buerschaper.}
%  Update 2012: One drawback from MJD's original implementation, is
%  that the math style is hardwired, such that if used in say
%  \cs{scriptstyle} context, then the style/size of the prescript
%  remain the same size. A slightly expensive fix, is to use the
%  \cs{mathchoice} construction. First exdent MJD's code a little
%  (keeping his comments)
% \begin{macro}{\MT_prescript_inner:}
% We make the style an extra forth argument
%    \begin{macrocode}
\newcommand{\MT_prescript_inner:}[4]{
%    \end{macrocode}
%  Put the sup in box 0 and the sub in box 2.
%    \begin{macrocode}
  \@mathmeasure\z@#4{\MT_prescript_sup:{#1}}
  \@mathmeasure\tw@#4{\MT_prescript_sub:{#2}}
  \if_dim:w \wd\tw@>\wd\z@
    \setbox\z@\hbox to\wd\tw@{\hfil\unhbox\z@}
  \else:
    \setbox\tw@\hbox to\wd\z@{\hfil\unhbox\tw@}
  \fi:
%    \end{macrocode}
%  Do not let a preceding mathord symbol approach without any
%  intervening space.
%    \begin{macrocode}
  \mathop{}
%    \end{macrocode}
%  Use \cs{mathopen} to suppress space between the prescripts and the
%  base object even when the latter is not of type ord.
%    \begin{macrocode}
  \mathopen{\vphantom{\MT_prescript_arg:{#3}}}^{\box\z@}\sb{\box\tw@}
  \MT_prescript_arg:{#3}
}
%    \end{macrocode}
% \end{macro}
% Next create \cs{prescript} using \cs{mathchoice} 
%    \begin{macrocode}
\DeclareRobustCommand{\prescript}[3]{
  \mathchoice
%    \end{macrocode}
%  In D and T style, we use MJD's default:
%    \begin{macrocode}
    {\MT_prescript_inner:{#1}{#2}{#3}{\scriptstyle}}
    {\MT_prescript_inner:{#1}{#2}{#3}{\scriptstyle}}
%    \end{macrocode}
%  In the others we step one style down. Of couse in SS style, using
%  \cs{scriptscript} may seem wrong, but there is no lower style.
%    \begin{macrocode}
    {\MT_prescript_inner:{#1}{#2}{#3}{\scriptscriptstyle}}
    {\MT_prescript_inner:{#1}{#2}{#3}{\scriptscriptstyle}}
}
%    \end{macrocode}
%  \end{macro}
%  Then the named arguments. Can you see I'm preparing for templates?
%    \begin{macrocode}
\define@key{\MT_options_name:}
  {prescript-sup-format}{\def\MT_prescript_sup:{#1}}
\define@key{\MT_options_name:}
  {prescript-sub-format}{\def\MT_prescript_sub:{#1}}
\define@key{\MT_options_name:}
  {prescript-arg-format}{\def\MT_prescript_arg:{#1}}
\setkeys{\MT_options_name:}{
  prescript-sup-format={},
  prescript-sub-format={},
  prescript-arg-format={},
}
%    \end{macrocode}
%
%  \subsection{Math sizes}
%  \begin{macro}{\@DeclareMathSizes}
%  This command is taken from a posting to comp.text.tex on
%  October~17th 2002 by Michael J.~Downes. The purpose is to be able
%  to put dimensions on the last three arguments of
%  \cs{DeclareMathSizes}.
%    \begin{macrocode}
\def\@DeclareMathSizes #1#2#3#4#5{%
  \@defaultunits\dimen@ #2pt\relax\@nnil
  \if:w $#3$%
    \MH_let:cN {S@\strip@pt\dimen@}\math@fontsfalse
  \else:
    \@defaultunits\dimen@ii #3pt\relax\@nnil
    \@defaultunits\@tempdima #4pt\relax\@nnil
    \@defaultunits\@tempdimb #5pt\relax\@nnil
    \toks@{#1}%
    \expandafter\xdef\csname S@\strip@pt\dimen@\endcsname{%
      \gdef\noexpand\tf@size{\strip@pt\dimen@ii}%
      \gdef\noexpand\sf@size{\strip@pt\@tempdima}%
      \gdef\noexpand\ssf@size{\strip@pt\@tempdimb}%
      \the\toks@
    }%
  \fi:
}
%    \end{macrocode}
%  \end{macro}
%
%  \subsection{Mathematics within italic text}
%  mathic: Michael J.~Downes on comp.text.tex, 1998/05/14.
%  \begin{macro}{\MT_mathic_true:}
%  \begin{macro}{\MT_mathic_false:}
%  Renew \cs{(} so that it detects the slant of the font and inserts
%  an italic correction.
%    \begin{macrocode}
\def\MT_mathic_true: {
  \MH_if_boolean:nF {math_italic_corr}{
    \MH_set_boolean_T:n {math_italic_corr}
%    \end{macrocode}
%  Save the original meaning if you need to go back.
%    \begin{macrocode}
    \MH_let:NwN \MT_begin_inlinemath: \(
    \renewcommand*\({\relax\ifmmode\@badmath\else
      \ifhmode
        \if_dim:w \fontdimen\@ne\font>\z@
          \if_dim:w \lastskip>\z@
            \skip@\lastskip\unskip
            \@@italiccorr
            \hskip\skip@
          \else:
            \@@italiccorr
          \fi:
        \fi:
      \fi:
      $\fi:
    }
  }
}
%$ for emacs coloring ;-)
%    \end{macrocode}
%  Just for restoring the old behavior.
%    \begin{macrocode}
\def\MT_mathic_false: {
  \MH_if_boolean:nT {math_italic_corr}{
    \MH_set_boolean_F:n {math_italic_corr}
    \MH_let:NwN \( \MT_begin_inlinemath:
  }
}
\MH_new_boolean:n {math_italic_corr}
\define@key{\MT_options_name:}{mathic}[true]{
  \@ifundefined{MT_mathic_#1:}
    { \MT_true_false_error:
      \@nameuse{MT_mathic_false:}
    }
    { \@nameuse{MT_mathic_#1:} }
}
%    \end{macrocode}
%  \end{macro}
%  \end{macro}
%
%  \subsection{Spreading equations}
%
%  Michael J.~Downes on comp.text.tex 1999/08/25
%  \begin{environment}{spreadlines}
%  This is meant to be used outside math, just like
%  \env{subequations}.
%    \begin{macrocode}
\newenvironment{spreadlines}[1]{
  \setlength{\jot}{#1}
  \ignorespaces
}{ \ignorespacesafterend }
%    \end{macrocode}
%  \end{environment}
%
%  \subsection{Gathered}
%
%  Inspired by Michael J.~Downes on comp.text.tex 2002/01/17.
%  \begin{environment}{MT_gathered_env}
%  Just like the normal \env{gathered}, only here we're allowed to
%  specify actions before and after each line.
%    \begin{macrocode}
\MaybeMHPrecedingSpacesOff
\newenvironment{MT_gathered_env}[1][c]{%
    \RIfM@\else
        \nonmatherr@{\begin{\@currenvir}}%
    \fi
    \null\,%
    \if #1t\vtop \else \if#1b\vbox \else \vcenter \fi\fi \bgroup
        \Let@ \chardef\dspbrk@context\@ne \restore@math@cr
        \spread@equation
        \ialign\bgroup
            \MT_gathered_pre:
            \strut@$\m@th\displaystyle##$
            \MT_gathered_post:
            \crcr
}{%
  \endaligned
  \MT_gathered_env_end:
}
\MHPrecedingSpacesOn
%    \end{macrocode}
%  \end{environment}
%  \begin{macro}{\newgathered}
%  \begin{macro}{\renewgathered}
%  \begin{environment}{lgathered}
%  \begin{environment}{rgathered}
%  \begin{environment}{gathered}
%  An easier interface.
%    \begin{macrocode}
\newcommand*\newgathered[4]{
  \newenvironment{#1}
    { \def\MT_gathered_pre:{#2}
      \def\MT_gathered_post:{#3}
      \def\MT_gathered_env_end:{#4}
      \MT_gathered_env
    }{\endMT_gathered_env}
}
\newcommand*\renewgathered[4]{
  \renewenvironment{#1}
    { \def\MT_gathered_pre:{#2}
      \def\MT_gathered_post:{#3}
      \def\MT_gathered_env_end:{#4}
      \MT_gathered_env
    }{\endMT_gathered_env}
}
\newgathered{lgathered}{}{\hfil}{}
\newgathered{rgathered}{\hfil}{}{}
\renewgathered{gathered}{\hfil}{\hfil}{}
%    \end{macrocode}
%  \end{environment}
%  \end{environment}
%  \end{environment}
%  \end{macro}
%  \end{macro}
%
%  \subsection{Split fractions}
%
%  Michael J.~Downes on comp.text.tex 2001/12/06.
%  \begin{macro}{\splitfrac}
%  \begin{macro}{\splitdfrac}
%  These commands use \cs{genfrac} to typeset a split fraction. The
%  thickness of the fraction rule is simply set to zero.
%    \begin{macrocode}
\newcommand*\splitfrac[2]{%
  \genfrac{}{}{0pt}{1}%
    {\textstyle#1\quad\hfill}%
    {\textstyle\hfill\quad\mathstrut#2}%
}
\newcommand*\splitdfrac[2]{%
  \genfrac{}{}{0pt}{0}{#1\quad\hfill}{\hfill\quad\mathstrut #2}%
}
%    \end{macrocode}
%  \end{macro}
%  \end{macro}
%
%
%  \section{Bug fixes for \pkg{amsmath}}
%  The following fixes some bugs in \pkg{amsmath}, but only if the
%  switch is true.
%    \begin{macrocode}
\MH_if_boolean:nT {fixamsmath}{
%    \end{macrocode}
%  \begin{macro}{\place@tag}
%  This corrects a bug in \pkg{amsmath} affecting tag placement in
%  \env{flalign}.\footnote{See
%  \url{http://www.latex-project.org/cgi-bin/ltxbugs2html?pr=amslatex/3591}}
%    \begin{macrocode}
\def\place@tag{%
  \iftagsleft@
    \kern-\tagshift@
%    \end{macrocode}
%  The addition. If we're in \env{flalign} (meaning
%  $\cs{xatlevel@}=\cs{tw@}$) we skip back by an amount of
%  \cs{@mathmargin}. This test is also true for the \env{xxalignat}
%  environment, but it doesn't matter because a)~it's not
%  supported/described in the documentation anymore so new users
%  won't know about it and b)~it forbids the use of \cs{tag}
%  anyway.
%    \begin{macrocode}
    \if@fleqn
      \if_num:w \xatlevel@=\tw@
        \kern-\@mathmargin
      \fi:
    \fi:
%    \end{macrocode}
%  End of additions.
%    \begin{macrocode}
    \if:w 1\shift@tag\row@\relax
      \rlap{\vbox{%
        \normalbaselines
        \boxz@
        \vbox to\lineht@{}%
        \raise@tag
      }}%
    \else:
      \rlap{\boxz@}%
    \fi:
    \kern\displaywidth@
  \else:
    \kern-\tagshift@
    \if:w 1\shift@tag\row@\relax
      \llap{\vtop{%
        \raise@tag
        \normalbaselines
        \setbox\@ne\null
        \dp\@ne\lineht@
        \box\@ne
        \boxz@
      }}%
    \else:
      \llap{\boxz@}%
    \fi:
  \fi:
}
%    \end{macrocode}
%  \end{macro}
%
%  \begin{macro}{\x@calc@shift@lf}
%  This corrects a bug\footnote{See
%  \url{http://www.latex-project.org/cgi-bin/ltxbugs2html?pr=amslatex/3614}}
%  in \pkg{amsmath} that could cause a non-positive value of the dimension
%  \cs{@mathmargin} to cause an
%  \begin{verbatim}
%  ! Arithmetic overflow.
%  <recently read> \@tempcntb
%  \end{verbatim}
%  when in \mode{fleqn,leqno} mode. Not very comprehensible for the user.
%    \begin{macrocode}
\def\x@calc@shift@lf{%
  \if_dim:w \eqnshift@=\z@
    \global\eqnshift@\@mathmargin\relax
      \alignsep@\displaywidth
      \advance\alignsep@-\totwidth@
%    \end{macrocode}
%  The addition: If \cs{@tempcntb} is zero we avoid division.
%    \begin{macrocode}
      \if_num:w \@tempcntb=0
      \else:
        \global\divide\alignsep@\@tempcntb % original line
      \fi:
%    \end{macrocode}
%  Addition end.
%    \begin{macrocode}
      \if_dim:w \alignsep@<\minalignsep\relax
        \global\alignsep@\minalignsep\relax
      \fi:
  \fi:
  \if_dim:w \tag@width\row@>\@tempdima
    \saveshift@1%
  \else:
    \saveshift@0%
  \fi:}%
%    \end{macrocode}
%  \end{macro}
%    \begin{macrocode}
}
%    \end{macrocode}
%  End of bug fixing.
%
%  \subsection{Making environments safer}
%
%  \begin{macro}{\aligned@a}
%  Here we make the \pkg{amsmath} inner environments disallow spaces
%  before their optional positioning specifier.
%    \begin{macrocode}
\MaybeMHPrecedingSpacesOff
\renewcommand\aligned@a[1][c]{\start@aligned{#1}\m@ne}
\MHPrecedingSpacesOn
%    \end{macrocode}
%  \end{macro}
%
%  This is the end of the \pkg{mathtools} package.
%    \begin{macrocode}
%</package>
%    \end{macrocode}
%
%  \Finale
\endinput

%        (quote the arguments according to the demands of your shell)
%
% Documentation:
%    (a) If mathtools.drv is present:
%           latex mathtools.drv
%    (b) Without mathtools.drv:
%           latex mathtools.dtx; ...
%    The class ltxdoc loads the configuration file ltxdoc.cfg
%    if available. Here you can specify further options, e.g.
%    use A4 as paper format:
%       \PassOptionsToClass{a4paper}{article}
%
%    Programm calls to get the documentation (example):
%       pdflatex mathtools.dtx
%       makeindex -s gind.ist mathtools.idx
%       pdflatex mathtools.dtx
%       makeindex -s gind.ist mathtools.idx
%       pdflatex mathtools.dtx
%
% Installation:
%    TDS:tex/latex/mh/mathtools.sty
%    TDS:doc/latex/mh/mathtools.pdf
%    TDS:source/latex/mh/mathtools.dtx
%
%<*ignore>
\begingroup
  \def\x{LaTeX2e}
\expandafter\endgroup
\ifcase 0\ifx\install y1\fi\expandafter
         \ifx\csname processbatchFile\endcsname\relax\else1\fi
         \ifx\fmtname\x\else 1\fi\relax
\else\csname fi\endcsname
%</ignore>
%<*install>
\input docstrip.tex
\Msg{************************************************************************}
\Msg{* Installation}
\Msg{* Package: mathtools 2021/03/18 v1.25}
\Msg{************************************************************************}

\keepsilent
\askforoverwritefalse

\preamble

This is a generated file.

Copyright (C) 2002-2011 by Morten Hoegholm
Copyright (C) 2012-2019 by Lars Madsen
Copyright (C) 2020-     by Lars Madsen, the LaTeX3 project


This work may be distributed and/or modified under the
conditions of the LaTeX Project Public License, either
version 1.3c of this license or (at your option) any later
version. The latest version of this license is in
   http://www.latex-project.org/lppl.txt
and version 1.3c or later is part of all distributions of
LaTeX version 2008/05/04 or later.

This work has the LPPL maintenance status "maintained".

This Current Maintainer of this work is  
Lars Madsen and the LaTeX3 project

This work consists of the main source file mathtools.dtx
and the derived files
   mathtools.sty, mathtools.pdf, mathtools.ins, mathtools.drv.

\endpreamble

\generate{%
  \file{mathtools.ins}{\from{mathtools.dtx}{install}}%
  \file{mathtools.drv}{\from{mathtools.dtx}{driver}}%
  \usedir{tex/latex/mh}%
  \file{mathtools.sty}{\from{mathtools.dtx}{package}}%
}

\obeyspaces
\Msg{************************************************************************}
\Msg{*}
\Msg{* To finish the installation you have to move the following}
\Msg{* file into a directory searched by TeX:}
\Msg{*}
\Msg{*     mathtools.sty}
\Msg{*}
\Msg{* To produce the documentation run the file `mathtools.drv'}
\Msg{* through LaTeX.}
\Msg{*}
\Msg{* Happy TeXing!}
\Msg{*}
\Msg{************************************************************************}

\endbatchfile
%</install>
%<*ignore>
\fi
%</ignore>
%<*driver>
\NeedsTeXFormat{LaTeX2e}
\ProvidesFile{mathtools.drv}%
  [2021/03/18 v1.25 mathematical typesetting tools]
\documentclass{ltxdoc}
\IfFileExists{fourier.sty}{\usepackage{fourier}}{}
\addtolength\marginparwidth{-25pt}
\usepackage{mathtools}

\setcounter{IndexColumns}{2}

\providecommand*\pkg[1]{\textsf{#1}}
\providecommand*\env[1]{\texttt{#1}}
\providecommand*\email[1]{\href{mailto:#1}{\texttt{#1}}}
\providecommand*\mode[1]{\texttt{[#1]}}
\providecommand*\file[1]{\texttt{#1}}
\usepackage{xcolor,varioref,amssymb}
\makeatletter
\newcommand*\thinfbox[2][black]{\fboxsep0pt\textcolor{#1}{\rulebox{{\normalcolor#2}}}}
\newcommand*\thinboxed[2][black]{\thinfbox[#1]{\ensuremath{\displaystyle#2}}}
\newcommand*\rulebox[1]{%
  \sbox\z@{\ensuremath{\displaystyle#1}}%
  \@tempdima\dp\z@
  \hbox{%
    \lower\@tempdima\hbox{%
      \vbox{\hrule height\fboxrule\box\z@\hrule height\fboxrule}%
    }%
  }%
}

\newenvironment{codesyntax}
    {\par\small\addvspace{4.5ex plus 1ex}%
     \vskip -\parskip
     \noindent
     \begin{tabular}{|l|}\hline\ignorespaces}%
    {\\\hline\end{tabular}\nobreak\par\nobreak
     \vspace{2.3ex}\vskip -\parskip\noindent\ignorespacesafterend}
\makeatletter

\newcommand*\FeatureRequest[2]{%
  \hskip1sp
  \marginpar{%
    \parbox[b]{\marginparwidth}{\small\sffamily\raggedright
      \strut Feature request by\\#1\\#2%
    }
  }%
}


\newcommand*\ProvidedBy[2]{%
  \hskip1sp
  \marginpar{%
    \parbox[b]{\marginparwidth}{\small\sffamily\raggedright
      \strut Feature provided by\\#1\\#2%
    }
  }%
}

\newcommand*\SuggestedBy[2]{%
  \hskip1sp
  \marginpar{%
    \parbox[b]{\marginparwidth}{\small\sffamily\raggedright
      \strut Suggested by\\#1\\#2%
    }
  }%
}



\newcommand*\cttPosting[2]{%
  \hskip1sp
  \marginpar{%
    \parbox[b]{\marginparwidth}{\small\sffamily\raggedright
     \strut Posted on \texttt{comp.text.tex} \\#1\\#2%
    }%
  }%
}

\newcommand*\tsxPosting[2]{%
  \hskip1sp
  \marginpar{%
    \parbox[b]{\marginparwidth}{\small\sffamily\raggedright
     \strut Posted on \texttt{\small tex.stackexchange.com} \\#1\\#2%
    }%
  }%
}

\newcommand*\tsxchatPosting[3][Requested on]{%
  \hskip1sp
  \marginpar{%
    \parbox[b]{\marginparwidth}{\small\sffamily\raggedright
     \strut #1 \texttt{\scriptsize tex.stackexchange.com chat} \\#2\\#3%
    }%
  }%
}



\newcommand*\CommentAdded[1]{%
  \hskip1sp
  \marginpar{%
    \parbox[b]{\marginparwidth}{\small\sffamily\raggedright
     \strut Comment added\\#1%
    }%
  }%
}


\expandafter\def\expandafter\MakePrivateLetters\expandafter{%
  \MakePrivateLetters  \catcode`\_=11\relax
}

\providecommand*\SpecialOptIndex[1]{%
  \@bsphack
  \index{#1\actualchar{\protect\ttfamily #1}
          (option)\encapchar usage}%
      \index{options:\levelchar#1\actualchar{\protect\ttfamily #1}\encapchar
            usage}\@esphack}
\providecommand*\opt[1]{\texttt{#1}}

\providecommand*\SpecialKeyIndex[1]{%
  \@bsphack
  \index{#1\actualchar{\protect\ttfamily #1}
          (key)\encapchar usage}%
      \index{keys:\levelchar#1\actualchar{\protect\ttfamily #1}\encapchar
            usage}\@esphack}
\providecommand*\key[1]{\textsf{#1}}

\providecommand*\eTeX{$\m@th\varepsilon$-\TeX}

\def\MTmeta#1{%
     \ensuremath\langle
     \ifmmode \expandafter \nfss@text \fi
     {%
      \meta@font@select
      \edef\meta@hyphen@restore
        {\hyphenchar\the\font\the\hyphenchar\font}%
      \hyphenchar\font\m@ne
      \language\l@nohyphenation
      #1\/%
      \meta@hyphen@restore
     }\ensuremath\rangle
     \endgroup
}
\makeatother
\DeclareRobustCommand\meta{\begingroup\MakePrivateLetters\MTmeta}%
\def\MToarg#1{{\ttfamily[}\meta{#1}{\ttfamily]}\endgroup}
\DeclareRobustCommand\oarg{\begingroup\MakePrivateLetters\MToarg}%
\def\MHmarg#1{{\ttfamily\char`\{}\meta{#1}{\ttfamily\char`\}}\endgroup}
\DeclareRobustCommand\marg{\begingroup\MakePrivateLetters\MHmarg}%
\def\MHarg#1{{\ttfamily\char`\{#1\ttfamily\char`\}}\endgroup}
\DeclareRobustCommand\arg{\begingroup\MakePrivateLetters\MHarg}%
\def\MHcs#1{\texttt{\char`\\#1}\endgroup}
\DeclareRobustCommand\cs{\begingroup\MakePrivateLetters\MHcs}

\def\endverbatim{\if@newlist
\leavevmode\fi\endtrivlist\vspace{-\baselineskip}}
\expandafter\let\csname endverbatim*\endcsname =\endverbatim

\let\MTtheindex\theindex
\def\theindex{\MTtheindex\MakePrivateLetters}


%\usepackage[draft]{fixme}
%\fxsetup{
% multiuser,
%layout=marginnote,
%}
%\providecommand\fxnote[2][]{}

\usepackage{xurl}

\usepackage[final,
hyperindex=false,
colorlinks,
]{hyperref}
\renewcommand*\usage[1]{\textit{\hyperpage{#1}}}

\OnlyDescription
\begin{document}
  \DocInput{mathtools.dtx}
\end{document}
%</driver>
%  \fi
%
%  \changes{v1.19}{2017/03/31}{Updated to match some macro renaming in
%  \pkg{mhsetup}}
%  \changes{v1.0}{2004/07/26}{Initial release}
%
%  \GetFileInfo{mathtools.drv}
%
%  \CheckSum{3519}
%
%  \title{The \pkg{mathtools} package\thanks{This file has version number
%  \fileversion, last revised \filedate.}}
%
%  \author{Morten Høgholm, Lars Madsen and the \LaTeX3 project}
%  \date{\filedate}
%
%  \maketitle
%
%  \begin{abstract}
%    \noindent The \pkg{mathtools} package is an extension package to
%    \pkg{amsmath}. There are two things on \pkg{mathtools}' agenda:
%    (1)~correct various bugs/defeciencies in \pkg{amsmath} until
%    these are fixed by the \AmS{} and (2)~provide useful tools
%    for mathematical typesetting, be it a small macro for
%    typesetting a prescript or an underbracket, or entirely new
%    display math constructs such as a \env{multlined} environment.
%  \end{abstract}
%
%  \tableofcontents
%
%  \section{Introduction}
%
%  Although \pkg{amsmath} provides many handy tools for mathematical
%  typesetting, it is nonetheless a static package. This is not a bad
%  thing, because what it does, it mostly does quite well and having
%  a stable math typesetting package is ``a good thing.'' However,
%  \pkg{amsmath} does not fulfill all the needs of the mathematical
%  part of the \LaTeX{} community, resulting in many authors writing
%  small snippets of code for tweaking the mathematical layout. Some
%  of these snippets have also been posted to newsgroups and mailing
%  lists over the years, although more often than not without being
%  released as stand-alone packages.
%
%  The \pkg{mathtools} package is exactly what its name implies: tools
%  for mathematical typesetting. It is a collection of many of these
%  often needed small tweaks---with some big tweaks added as well. It
%  can only do so by having harvesting newsgroups for code and/or you
%  writing the maintainers with wishes for code to be included, so if
%  you have any good macros or just macros that help you when writing
%  mathematics, then don't hesitate to report them.
%
%  As of 2020, \pkg{mathtools} (and \pkg{empheq}) is now hosted at the
%  \LaTeX3 team GitHub at
%  \begin{center}
%    \url{https://github.com/latex3/mathtools}
%  \end{center}
%  So if you have any issue, feel free to register an issue there.
%
% \medskip\noindent
% Update 2013: We now make \cs{(}\cs{)} and \cs[\cs]
% robust (can be disabled via \texttt{nonrobust} package option).
% 
%  \section{Package loading}
%
%
%  The \pkg{mathtools} package requires \pkg{amsmath} but is able to
%  pass options to it as well. Thus a line like
%  \begin{verbatim}
%    \usepackage[fleqn,tbtags]{mathtools}
%  \end{verbatim}
%  is equivalent to
%  \begin{verbatim}
%    \usepackage[fleqn,tbtags]{amsmath}
%    \usepackage{mathtools}
%  \end{verbatim}
%
%
%  \subsection{Special \pkg{mathtools} options}
%
%  \begin{codesyntax}
%  \SpecialOptIndex{fixamsmath}\opt{fixamsmath}\texttt{~~~~}
%  \SpecialOptIndex{donotfixamsmathbugs}\opt{donotfixamsmathbugs}
%  \end{codesyntax}
%  The option \opt{fixamsmath} (default) fixes two bugs in
%  \pkg{amsmath}.\footnote{See the online \LaTeX{} bugs database
%  \url{http://www.latex-project.org/cgi-bin/ltxbugs2html} under
%  \AmS\LaTeX{} problem reports 3591 and 3614.} Should you for some
%  reason not want to fix these bugs then just add the option
%  \opt{donotfixamsmathbugs} (if you can do it without typos). The
%  reason for this extremely long name is that I really don't see why
%  you wouldn't want these bugs to be fixed, so I've made it slightly
%  difficult not to fix them.
%
%  \begin{codesyntax}
%  \SpecialOptIndex{allowspaces}\opt{allowspaces}\texttt{~~~~}
%  \SpecialOptIndex{disallowspaces}\opt{disallowspaces}
%  \end{codesyntax}
%  Sometimes \pkg{amsmath} gives you nasty surprises, as here where
%  things look seemingly innocent:
%  \begin{verbatim}
%  \[
%      \begin{gathered}
%        [p] = 100 \\
%        [v] = 200
%      \end{gathered}
%  \]
%  \end{verbatim}
%  Without \pkg{mathtools} this will result in this output:
%  \[
%      \begin{gathered}[c]
%        = 100 \\
%        [v] = 200
%      \end{gathered}
%  \]
%  Yes, the \texttt{[p]} has been gobbled without any warning
%  whatsoever.\footnote{\pkg{amsmath} thought the \texttt[p] was an
%  optional argument, checked if it was \texttt{t} or \texttt{b} and
%  when both tests failed, assumed it was a \texttt{c}.} This is
%  hardly what you'd expect as an end user, as the desired output was
%  probably something like this instead:
%  \[
%      \begin{gathered}[c]
%        [p] = 100 \\
%        [v] = 200
%      \end{gathered}
%  \]
%  With the option \opt{disallowspaces} (default) \pkg{mathtools}
%  disallows spaces in front of optional arguments where it could
%  possibly cause problems just as \pkg{amsmath} does with |\\|
%  inside the display environments. This includes the environments
%  \env{gathered} (and also those shown in \S
%  \vref{subsec:gathered}), \env{aligned}, \env{multlined}, and the
%  extended \env{matrix}-environments (\S \vref{subsubsec:matrices}).
%  If you however want to preserve the more dangerous standard
%  optional spaces, simply choose the option \opt{allowspaces}.
%
%
%  \section{Tools for mathematical typesetting}
%
%  \begin{codesyntax}
%    \SpecialUsageIndex{\mathtoolsset}\cs{mathtoolsset}\marg{key val list}
%  \end{codesyntax}
%  Many of the tools shown in this manual can be turned on and off by
%  setting a switch to either true or false. In all cases it is done
%  with the command \cs{mathtoolsset}. A typical use could be something like
%  \begin{verbatim}
%    \mathtoolsset{
%      showonlyrefs,
%      mathic % or mathic = true
%    }
%  \end{verbatim}
%  More information on the keys later on.
%
%  \subsection{Fine-tuning mathematical layout}
%
%  Sometimes you need to tweak the layout of formulas a little to get
%  the best result and this part of the manual describes the various
%  macros \pkg{mathtools} provides for this.
%
%  \subsubsection{A complement to \texttt{\textbackslash smash},
%  \texttt{\textbackslash llap}, and \texttt{\textbackslash rlap}}
%
%  \begin{codesyntax}
%    \SpecialUsageIndex{\mathllap}
%    \cs{mathllap}\oarg{mathstyle}\marg{math}\texttt{~~}
%    \SpecialUsageIndex{\mathclap}
%    \cs{mathclap}\oarg{mathstyle}\marg{math}\\
%    \SpecialUsageIndex{\mathrlap}
%    \cs{mathrlap}\oarg{mathstyle}\marg{math}\texttt{~~}
%    \SpecialUsageIndex{\clap}
%    \cs{clap}\marg{text}\\
%    \SpecialUsageIndex{\mathmbox}
%    \cs{mathmbox}\marg{math}\phantom{\meta{mathstyle}}\texttt{~~~~}
%    \SpecialUsageIndex{\mathmakebox}
%    \cs{mathmakebox}\oarg{width}\oarg{pos}\marg{math}
%  \end{codesyntax}
%  In \cite{Perlis01}, Alexander R.~Perlis describes some simple yet
%  useful macros for use in math displays. For example the display
%  \begin{verbatim}
%    \[
%      X = \sum_{1\le i\le j\le n} X_{ij}
%    \]
%  \end{verbatim}
%  \[
%    X = \sum_{1\le i\le j\le n} X_{ij}
%  \]
%  contains a lot of excessive white space.  The idea that comes to
%  mind is to fake the width of the subscript. The command
%  \cs{mathclap} puts its argument in a zero width box and centers
%  it, so it could possibly be of use here.
%  \begin{verbatim}
%    \[
%      X = \sum_{\mathclap{1\le i\le j\le n}} X_{ij}
%    \]
%  \end{verbatim}
%  \[
%    X = \sum_{\mathclap{1\le i\le j\le n}} X_{ij}
%  \]
%  For an in-depth discussion of
%  these macros I find it better to read the article; an online
%  version can be found at
%  \begin{quote}
%    \url{http://www.tug.org/TUGboat/Articles/tb22-4/tb72perlS.pdf}
%  \end{quote}
%  Note that the definitions shown in the article do not exactly
%  match the definitions in \pkg{mathtools}. Besides providing an
%  optional argument for specifying the desired math style, these
%  versions also work around a most unfortunate \TeX{}
%  ``feature.''\footnote{The faulty reboxing procedure.} The
%  \cs{smash} macro is fixed too.
%
%
%  \subsubsection{Forcing a cramped style}
%
%  \begin{codesyntax}
%    \SpecialUsageIndex{\cramped}
%    \cs{cramped}\oarg{mathstyle}\marg{math}
%  \end{codesyntax}
%  \cttPosting{Michael Herschorn}{1992/07/21}
%  Let's look at another example where we have used \cs{mathclap}:
%  \begin{verbatim}
%    \begin{equation}\label{eq:mathclap}
%      \sum_{\mathclap{a^2<b^2<c}}\qquad
%      \sum_{a^2<b^2<c}
%    \end{equation}
%  \end{verbatim}
%  \begin{equation}\label{eq:mathclap}
%    \sum_{\mathclap{a^2<b^2<c}}\qquad
%    \sum_{a^2<b^2<c}
%  \end{equation}
%  Do you see the difference? Maybe if I zoomed in a bit:
%  \begingroup \fontsize{24}{\baselineskip}\selectfont
%  \[
%    \sum_{\mathclap{a^2<b^2<c}}\qquad
%    \sum_{a^2<b^2<c}
%  \]
%  \endgroup
%  Notice how the limit of the right summation sign is typeset in a
%  more compact style than the left. It is because \TeX{} sets the
%  limits of operators in a \emph{cramped} style. For each of \TeX'
%  four math styles (\cs{displaystyle}, \cs{textstyle},
%  \cs{scriptstyle}, and \cs{scriptscriptstyle}), there also exists a
%  cramped style that doesn't raise exponents as much. Besides in the
%  limits of operators, \TeX{} also automatically uses these cramped
%  styles in radicals such as \cs{sqrt} and in the denominators of
%  fractions, but unfortunately there are no primitive commands that
%  allows you to detect crampedness or switch to it.
%
%  \pkg{mathtools} offers the command \cs{cramped} which forces a
%  cramped style in normal un-cramped math. Additionally you can
%  choose which of the four styles you want it in as well by
%  specifying it as the optional argument:
%  \begin{verbatim}
%    \[
%      \cramped{x^2}               \leftrightarrow x^2    \quad
%      \cramped[\scriptstyle]{x^2} \leftrightarrow {\scriptstyle x^2}
%    \]
%  \end{verbatim}
%  \[
%    \cramped{x^2}               \leftrightarrow x^2    \quad
%    \cramped[\scriptstyle]{x^2} \leftrightarrow {\scriptstyle x^2}
%  \]
%  You may be surprised how often the cramped style can be
%  beneficial to your output. Take a look at this example:
%  \begin{verbatim}
%    \begin{quote}
%      The 2005 Euro\TeX{} conference is held in Abbaye des
%      Pr\'emontr\'es, France, marking the 16th ($2^{2^2}$) anniversary
%      of both Dante and GUTenberg (the German and French \TeX{} users
%      group resp.).
%    \end{quote}
%  \end{verbatim}
%  \begin{quote}
%    The 2005 Euro\TeX{} conference is held in Abbaye des
%    Pr\'emontr\'es, France, marking the 16th ($2^{2^2}$) anniversary
%    of both Dante and GUTenberg (the German and French \TeX{} users
%    group resp.).
%  \end{quote}
%  Typesetting on a grid is generally considered quite desirable, but
%  as the second line of the example shows, the exponents of $2$
%  causes the line to be too tall for the normal value of
%  \cs{baselineskip}, so \TeX{} inserts a \cs{lineskip} (normal value
%  is \the\lineskip). In order to circumvent the problem, we can
%  force a cramped style so that the exponents aren't raised as much:
%  \begin{verbatim}
%    \begin{quote}
%      The 2005 Euro\TeX{} ... 16th ($\cramped{2^{2^2}}$) ...
%    \end{quote}
%  \end{verbatim}
%  \begin{quote}
%    The 2005 Euro\TeX{} conference is held in Abbaye des
%    Pr\'emontr\'es, France, marking the 16th ($\cramped{2^{2^2}}$)
%    anniversary of both Dante and GUTenberg (the German and French
%    \TeX{} users group resp.).
%  \end{quote}
%
%  \begin{codesyntax}
%    \SpecialUsageIndex{\crampedllap}
%    \cs{crampedllap}\oarg{mathstyle}\marg{math}\texttt{~~}
%    \SpecialUsageIndex{\crampedclap}
%    \cs{crampedclap}\oarg{mathstyle}\marg{math}\\
%    \SpecialUsageIndex{\crampedrlap}
%    \cs{crampedrlap}\oarg{mathstyle}\marg{math}
%  \end{codesyntax}
%  The commands \cs{crampedllap}, \cs{crampedclap}, and
%  \cs{crampedrlap} are identical to the three \cs{mathXlap} commands
%  described earlier except the argument is typeset in cramped style.
%  You need this in order to typeset \eqref{eq:mathclap} correctly
%  while still faking the width of the limit.
%  \begin{verbatim}
%    \begin{equation*}\label{eq:mathclap-b}
%      \sum_{\crampedclap{a^2<b^2<c}}
%      \tag{\ref{eq:mathclap}*}
%    \end{equation*}
%  \end{verbatim}
%  \begin{equation*}\label{eq:mathclap-b}
%    \sum_{\crampedclap{a^2<b^2<c}}
%    \tag{\ref{eq:mathclap}*}
%  \end{equation*}
%  Of course you could just type
%  \begin{verbatim}
%    \sum_{\mathclap{\cramped{a^2<b^2<c}}}
%  \end{verbatim}
%  but it has one major disadvantage: In order for \cs{mathXlap} and
%  \cs{cramped} to get the right size, \TeX{} has to process them
%  four times, meaning that nesting them as shown above will cause
%  \TeX{} to typeset $4^2$ instances before choosing the right one.
%  In this situation however, we will of course need the same style
%  for both commands so it makes sense to combine the commands in
%  one, thus letting \TeX{} make the choice only once rather than
%  twice.
%
%  \begin{codesyntax}
%    \SpecialEnvIndex{crampedsubarray}\cs{begin}\arg{crampedsubarray}\marg{col}
%        \meta{contents} \cs{end}\arg{crampedsubarray}\\
%    \SpecialUsageIndex{\crampedsubstack}
%    \cs{crampedsubstack}\marg{lines separated by \cs{}\cs{}}
% \end{codesyntax}
% \SuggestedBy{Henri Menke}{2019/07/08} If we go back to
% \eqref{eq:mathclap} and apply \cs{substack}, you'll notice that the
% cramped style, the sum would normally apply, is now gone:
% \begin{verbatim}
%  \[
%    \sum_{\substack{a^2<b^2<c}}\qquad
%    \sum_{a^2<b^2<c}
%  \]
% \end{verbatim}
%  \begingroup \fontsize{24}{\baselineskip}\selectfont
%  \[
%    \sum_{\substack{a^2<b^2<c}}\qquad
%    \sum_{a^2<b^2<c}
%  \]
%  \endgroup 
%  We therefore provide a cramped version of
%  \cs{substack}.\footnote{\cs{substack} is internally implemented via
%  the \env{subarray}-env, so our cramped version of \cs{substack} is implemented via a
%  cramped version of this env.}
% \begin{verbatim}
%  \[
%    \sum_{\crampedsubstack{a^2<b^2<c}}\qquad
%    \sum_{a^2<b^2<c}
%  \]
% \end{verbatim}
%  \begingroup \fontsize{24}{\baselineskip}\selectfont
%  \[
%    \sum_{\crampedsubstack{a^2<b^2<c}}\qquad
%    \sum_{a^2<b^2<c}
%  \]
%  \endgroup 
%  Note: We may need to add a similar hook into multlined. 
%
%
%
%
%
%  \subsubsection{Smashing an operator}
%
%
%
%  \begin{codesyntax}
%    \SpecialUsageIndex{\smashoperator}
%    \cs{smashoperator}\oarg{pos}\marg{operator with limits}
%  \end{codesyntax}
%  \FeatureRequest{Lars Madsen}{2004/05/04} Above we showed how to get
%  \LaTeX{} to ignore the width of the subscript of an
%  operator. However this approach takes a lot of extra typing,
%  especially if you have a wide superscript, meaning you have to put
%  in \cs{crampedclap} in both sub- and superscript.  To make things
%  easier, \pkg{mathtools} provides a \cs{smashoperator} command,
%  which simply ignores the width of the sub- and superscript. It also
%  takes an optional argument, \texttt{l}, \texttt{r}, or \texttt{lr}
%  (default, \texttt{rl} is an alias for \texttt{lr}), denoting which
%  side of the operator should be ignored (smashed).
%  \begin{verbatim}
%    \[
%      V = \sum_{1\le i\le j\le n}^{\infty} V_{ij}                  \quad
%      X = \smashoperator{\sum_{1\le i\le j\le n}^{3456}} X_{ij}    \quad
%      Y = \smashoperator[r]{\sum\limits_{1\le i\le j\le n}} Y_{ij} \quad
%      Z = \smashoperator[l]{\mathop{T}_{1\le i\le j\le n}} Z_{ij}
%    \]
%  \end{verbatim}
%    \[
%      V = \sum_{1\le i\le j\le n}^{\infty} V_{ij}                  \quad
%      X = \smashoperator{\sum_{1\le i\le j\le n}^{3456}} X_{ij}    \quad
%      Y = \smashoperator[r]{\sum\limits_{1\le i\le j\le n}} Y_{ij} \quad
%      Z = \smashoperator[l]{\mathop{T}_{1\le i\le j\le n}} Z_{ij}
%    \]
%  Note that \cs{smashoperator} always sets its argument in display
%  style and with limits even if you have used the \opt{nosumlimits}
%  option of \pkg{amsmath}. If you wish, you can use shorthands for
%  \texttt{\_} and \texttt{\textasciicircum} such as \cs{sb} and
%  \cs{sp}.
%
%
%  \subsubsection{Adjusting limits of operators}
%
%  \begin{codesyntax}
%    \SpecialUsageIndex{\adjustlimits}
%    \cs{adjustlimits}\marg{operator$\sb1$}\texttt{\_}\marg{limit$\sb1$}
%                     \marg{operator$\sb2$}\texttt{\_}\marg{limit$\sb2$}
%  \end{codesyntax}
%  \FeatureRequest{Lars Madsen}{2004/07/09}
%  When typesetting two consecutive operators with limits one often
%  wishes the limits of the operators were better aligned. Look
%  closely at these examples:
%  \begin{verbatim}
%    \[
%      \text{a)} \lim_{n\to\infty} \max_{p\ge n} \quad
%      \text{b)} \lim_{n\to\infty} \max_{p^2\ge n} \quad
%      \text{c)} \lim_{n\to\infty} \sup_{p^2\ge nK} \quad
%      \text{d)} \limsup_{n\to\infty} \max_{p\ge n}
%    \]
%  \end{verbatim}
%  \[
%    \text{a)} \lim_{n\to\infty} \max_{p\ge n} \quad
%    \text{b)} \lim_{n\to\infty} \max_{p^2\ge n} \quad
%    \text{c)} \lim_{n\to\infty} \sup_{p^2\ge nK} \quad
%    \text{d)} \limsup_{n\to\infty} \max_{p\ge n}
%  \]
%  a) looks okay, but b) is not quite as good because the second
%  limit ($\cramped{p^2\ge n}$) is significantly taller than the
%  first ($n\to\infty$). With c)~things begin to look really bad,
%  because the second operator has a descender while the first
%  doesn't, and finally we have d)~which looks just as bad as~c). The
%  command \cs{adjustlimits} is useful in these cases, as you can
%  just put it in front of these consecutive operators and it'll make
%  the limits line up.
%  \medskip\par\noindent
%  \begin{minipage}{\textwidth}
%  \begin{verbatim}
%    \[
%      \text{a)} \adjustlimits\lim_{n\to\infty} \max_{p\ge n} \quad
%      \text{b)} \adjustlimits\lim_{n\to\infty} \max_{p^2\ge n} \quad
%      \text{c)} \adjustlimits\lim_{n\to\infty} \sup_{p^2\ge nK} \quad
%      \text{d)} \adjustlimits\limsup_{n\to\infty} \max_{p\ge n}
%    \]
%  \end{verbatim}
%  \end{minipage}
%  \[
%      \text{a)} \adjustlimits\lim_{n\to\infty} \max_{p\ge n} \quad
%      \text{b)} \adjustlimits\lim_{n\to\infty} \max_{p^2\ge n} \quad
%      \text{c)} \adjustlimits\lim_{n\to\infty} \sup_{p^2\ge nK} \quad
%      \text{d)} \adjustlimits\limsup_{n\to\infty} \max_{p\ge n}
%    \]
%  The use of \cs{sb} instead of \texttt{\_} is allowed.
%
%
%  \subsubsection{Swapping space above \texorpdfstring{\AmS}{AMS} display math environments }
%  \label{sec:swapping}
%
%  One feature that the plain old \env{equation} environment has that
%  the \AmS\ environments does not (because of technical reasons), is
%  the feature of using less space above the equation if the situation
%  presents itself. The \AmS\ environments cannot do this, but one can
%  manually, using  
%  \begin{codesyntax}
%    \SpecialUsageIndex{\SwapAboveDisplaySkip}
%    \cs{SwapAboveDisplaySkip}
%  \end{codesyntax}
%  as the very first content within an \AmS\ display math
%  environment. It will then issue an \cs{abovedisplayshortskip}
%  instead of the normal \cs{abovedisplayskip}.
%
%  Note it will not work with the \env{equation} or \env{multline} environments.
%  
%  Here is an example of the effect
% \begin{verbatim}
%  \noindent\rule\textwidth{1pt}
%  \begin{align*}  A &= B \end{align*}
%  \noindent\rule\textwidth{1pt}
%  \begin{align*}  
%  \SwapAboveDisplaySkip
%  A &= B 
%  \end{align*}
% \end{verbatim}
%  \noindent\rule\textwidth{1pt}
%  \begin{align*}  A &= B \end{align*}
%  \noindent\rule\textwidth{1pt}
%  \begin{align*}  
%  \SwapAboveDisplaySkip
%  A &= B 
%  \end{align*}
%  
%
%  \subsection{Controlling tags}
%
%  In this section various tools for altering the appearance of tags
%  are shown. All of the tools here can be used at any point in the
%  document but they should probably be affect the whole document, so
%  the preamble is the best place to issue them.
%
%  \subsubsection{The appearance of tags}
%  \begin{codesyntax}
%    \SpecialUsageIndex{\newtagform}
%    \cs{newtagform}\marg{name}\oarg{inner_format}\marg{left}\marg{right}\\
%    \SpecialUsageIndex{\renewtagform}
%    \cs{renewtagform}\marg{name}\oarg{inner_format}\marg{left}\marg{right}\\
%    \SpecialUsageIndex{\usetagform}
%    \cs{usetagform}\marg{name}
%  \end{codesyntax}
%  Altering the layout of equation numbers in \pkg{amsmath} is not
%  very user friendly (it involves a macro with three \texttt{@}'s in
%  its name), so \pkg{mathtools} provides an interface somewhat
%  reminiscent of the page style concept. This way you can define
%  several different tag forms and then choose the one you prefer.
%
%  As an example let's try to define a tag form which puts the
%  equation number in square brackets. First we define a brand new tag
%  form:
%  \begin{verbatim}
%    \newtagform{brackets}{[}{]}
%  \end{verbatim}
%  Then we activate it:
%  \begin{verbatim}
%    \usetagform{brackets}
%  \end{verbatim}
%  The result is then
%  \newtagform{brackets}{[}{]}
%  \usetagform{brackets}
%   \begin{equation}
%     E \neq m c^3
%   \end{equation}
%
%  Similarly you could define a second version of the brackets that
%  prints the equation number in bold face instead
%  \begin{verbatim}
%    \newtagform{brackets2}[\textbf]{[}{]}
%    \usetagform{brackets2}
%    \begin{equation}
%      E \neq m c^3
%    \end{equation}
%  \end{verbatim}
%  \newtagform{brackets2}[\textbf]{[}{]}
%  \usetagform{brackets2}
%  \begin{equation}
%    E \neq m c^3
%  \end{equation}
%  When you reference an equation with \cs{eqref}, the tag form in
%  effect at the time of referencing controls the formatting, so be
%  careful if you use different tag forms throughout your document.
%
%  If you want to renew a tag form, then use the command
%  \cs{renewtagform}. Should you want to
%  return to the standard setting then choose\usetagform{default}
%  \begin{verbatim}
%    \usetagform{default}
%  \end{verbatim}
%
%  \changes{v1.12}{2012/05/09}{Added caveat}
%  \noindent\textbf{Caveat regarding \pkg{ntheorem}}: If you like to
%  change the appearence of the tags \emph{and} you are also using the
%  \pkg{ntheorem} package, then please postpone the change of
%  appearance until \emph{after} loading \pkg{ntheorem}. (In order to
%  do its thing, \pkg{ntheorem} has to mess with the tags\dots)
%
%  \subsubsection{Showing only referenced tags}
%
%  \begin{codesyntax}
%    \SpecialKeyIndex{showonlyrefs}$\key{showonlyrefs}=\texttt{true}\vert\texttt{false}$\\
%    \SpecialKeyIndex{showmanualtags}$\key{showmanualtags}=\texttt{true}\vert\texttt{false}$\\
%    \SpecialUsageIndex{\refeq}\cs{refeq}\marg{label}
%  \end{codesyntax}
%  An equation where the tag is produced with a manual \cs{tag*}
%  shouldn't be referenced with the normal \cs{eqref} because that
%  would format it according to the current tag format. Using just
%  \cs{ref} on the other hand may not be a good solution either as
%  the argument of \cs{tag*} is always set in upright shape in the
%  equation and you may be referencing it in italic text. In the
%  example below, the command \cs{refeq} is used to avoid what could
%  possibly lead to confusion in cases where the tag font has very
%  different form in upright and italic shape (here we switch to
%  Palatino in the example):
%    \begin{verbatim}
%    \begin{quote}\renewcommand*\rmdefault{ppl}\normalfont\itshape
%    \begin{equation*}
%      a=b \label{eq:example}\tag*{Q\&A}
%    \end{equation*}
%    See \ref{eq:example} or is it better with \refeq{eq:example}?
%    \end{quote}
%  \end{verbatim}
%    \begin{quote}\renewcommand*\rmdefault{ppl}\normalfont\itshape
%    \begin{equation*}
%      a=b \label{eq:example}\tag*{Q\&A}
%    \end{equation*}
%    See \ref{eq:example} or is it better with \refeq{eq:example}?
%  \end{quote}
%
%
%  Another problem sometimes faced is the need for showing the
%  equation numbers for only those equations actually referenced. In
%  \pkg{mathtools} this can be done by setting the key
%  \key{showonlyrefs} to either true or false by using
%  \cs{mathtoolsset}. You can also choose whether or not to show the
%  manual tags specified with \cs{tag} or \cs{tag*} by setting the
%  option \key{showmanualtags} to true or false.\footnote{I recommend
%  setting \key{showmanualtags} to true, else the whole idea of using
%  \cs{tag} doesn't really make sense, does it?} For both keys just
%  typing the name of it chooses true as shown in the following
%  example.
%
%  \begin{verbatim}
%  \mathtoolsset{showonlyrefs,showmanualtags}
%  \usetagform{brackets}
%  \begin{gather}
%    a=a \label{eq:a} \\
%    b=b \label{eq:b} \tag{**}
%  \end{gather}
%  This should refer to the equation containing $a=a$: \eqref{eq:a}.
%  Then a switch of tag forms.
%  \usetagform{default}
%  \begin{align}
%    c&=c \label{eq:c} \\
%    d&=d \label{eq:d}
%  \end{align}
%  This should refer to the equation containing $d=d$: \eqref{eq:d}.
%  \begin{equation}
%    e=e
%  \end{equation}
%  Back to normal.\mathtoolsset{showonlyrefs=false}
%  \begin{equation}
%    f=f
%  \end{equation}
%  \end{verbatim}
%  \mathtoolsset{showonlyrefs,showmanualtags}
%  \usetagform{brackets}
%  \begin{gather}
%    a=a \label{eq:a} \\
%    b=b \label{eq:b} \tag{**}
%  \end{gather}
%  This should refer to the equation containing $a=a$: \eqref{eq:a}.
%  Then a switch of tag forms.
%  \usetagform{default}
%  \begin{align}
%    c&=c \label{eq:c} \\
%    d&=d \label{eq:d}
%  \end{align}
%  This should refer to the equation containing $d=d$: \eqref{eq:d}.
%  \begin{equation}
%    e=e
%  \end{equation}
%  Back to normal.\mathtoolsset{showonlyrefs=false}
%  \begin{equation}
%    f=f
%  \end{equation}
%
%  Note that this feature only works if you use \cs{eqref} or
%  \cs{refeq} to reference your equations.
%
%  When using \key{showonlyrefs} it might be useful to be able to
%  actually add a few equation numbers without directly referring to
%  them.
%  \begin{codesyntax}
%    \SpecialUsageIndex{\noeqref}\cs{noeqref}\marg{label,label,\dots}
%  \end{codesyntax}
%  \FeatureRequest{Rasmus Villemoes}{2008/03/26}
%  The syntax is somewhat similar to \cs{nocite}. If a label in the
%  list is undefined we will throw a warning in the same manner as
%  \cs{ref}. 
%
%  \medskip\noindent\textbf{BUG 1:} Unfortunately the use of the
%  \key{showonlyref} introduce a bug within amsmath's typesetting
%  of formula versus equation number. This bug manifest itself by
%  allowing formulas to be typeset close to or over the equation
%  number.  Currently no general fix is known, other than making sure
%  that one's formulas are not long enough to touch the equation
%  number.
%
%  To make a long story short, amsmath typesets its math environments
%  twice, one time for measuring and one time for the actual
%  typesetting. In the measuring part, the width of the equation
%  number is recorded such that the formula or the equation number can
%  be moved (if necessary) in the typesetting part. When
%  \key{showonlyref} is enabled, the width of the equation number
%  depend on whether or not this number is referred~to. To determine
%  this, we need to know the current label. But the current label is
%  \emph{not} known in the measuring phase. Thus the measured width is
%  always zero (because no label equals not referred to) and therefore
%  the typesetting phase does not take the equation number into
%  account.
%
% \medskip\noindent\textbf{BUG 2:} There is a bug between
% \key{showonlyrefs} and the \pkg{ntheorem} package, when the
% \pkg{ntheorem} option \key{thmmarks} is active. The shown equation
% numbers may come out wrong (seems to be multiplied by 2). Or the end
% marker may be placed in the wrong place if a proof ends with a
% displayed formula, and that formula is not refered to.
%
% There are two possible solutions to this both involving \pkg{empheq}.  
% The easiest fix is to add the following line
% \begin{verbatim}
% \usepackage[overload,ntheorem]{empheq}
% \end{verbatim}
% before loading \pkg{ntheorem}. But the \texttt{overload} option of
% course disables things like \cs{intertext} and \cs{shotintertext}.
%
% The other thing to try is to use drop the \texttt{overload} option
% and use \env{empheq} on the very last expression, as in
% \begin{verbatim}
%   \begin{empheq}{align}
%     A &= B \label {eq:32}\\
%       & = 1. \label{eq:31}
%   \end{empheq}
% \end{proof}
% \end{verbatim}
% 
% The \pkg{empheq} package fixes some
% problems with \pkg{ntheorem} and lets \pkg{mathtools} get correct
% access to the equation numbers again.
% 
%  \subsection{Extensible symbols}
%
%  The number of horizontally extensible symbols in standard \LaTeX{}
%  and \pkg{amsmath} is somewhat low. This part of the manual
%  describes what \pkg{mathtools} does to help this situation.
%
%  \subsubsection{Arrow-like symbols}
%
%
%  \begin{codesyntax}
%    \SpecialUsageIndex{\xleftrightarrow}
%    \cs{xleftrightarrow}\oarg{sub}\marg{sup}\texttt{~~~~~~~~~}
%    \SpecialUsageIndex{\xRightarrow}
%    \cs{xRightarrow}\oarg{sub}\marg{sup}\\
%    \SpecialUsageIndex{\xLeftarrow}
%    \cs{xLeftarrow}\oarg{sub}\marg{sup}\texttt{~~~~~~~~~~~~~~}
%    \SpecialUsageIndex{\xLeftrightarrow}
%    \cs{xLeftrightarrow}\oarg{sub}\marg{sup}\\
%    \SpecialUsageIndex{\xhookleftarrow}
%    \cs{xhookleftarrow}\oarg{sub}\marg{sup}\texttt{~~~~~~~~~~}
%    \SpecialUsageIndex{\xhookrightarrow}
%    \cs{xhookrightarrow}\oarg{sub}\marg{sup}\\
%    \SpecialUsageIndex{\xmapsto}
%    \cs{xmapsto}\oarg{sub}\marg{sup}
%  \end{codesyntax}
%  Extensible arrows are part of \pkg{amsmath} in the form of the
%  commands
%  \begin{quote}
%    \cs{xrightarrow}\oarg{subscript}\marg{superscript}\quad and\\
%    \cs{xleftarrow}\oarg{subscript}\marg{superscript}
%  \end{quote}
%  But what about extensible versions of say, \cs{leftrightarrow} or
%  \cs{Longleftarrow}? It turns out that the above mentioned
%  extensible arrows are the only two of their kind defined by
%  \pkg{amsmath}, but luckily \pkg{mathtools} helps with that. The
%  extensible arrow-like symbols in \pkg{mathtools} follow the same
%  naming scheme as the one's in \pkg{amsmath} so to get an extensible
%  \cs{Leftarrow} you simply do a
%  \begin{verbatim}
%    \[
%      A \xLeftarrow[under]{over} B
%    \]
%  \end{verbatim}
%    \[
%      A \xLeftarrow[under]{over} B
%    \]
%  \begin{codesyntax}
%    \SpecialUsageIndex{\xrightharpoondown}
%    \cs{xrightharpoondown}\oarg{sub}\marg{sup}\texttt{~~~~}
%    \SpecialUsageIndex{\xrightharpoonup}
%    \cs{xrightharpoonup}\oarg{sub}\marg{sup}\\
%    \SpecialUsageIndex{\xleftharpoondown}
%    \cs{xleftharpoondown}\oarg{sub}\marg{sup}\texttt{~~~~~}
%    \SpecialUsageIndex{\xleftharpoonup}
%    \cs{xleftharpoonup}\oarg{sub}\marg{sup}\\
%    \SpecialUsageIndex{\xrightleftharpoons}
%    \cs{xrightleftharpoons}\oarg{sub}\marg{sup}\texttt{~~~}
%    \SpecialUsageIndex{\xleftrightharpoons}
%    \cs{xleftrightharpoons}\oarg{sub}\marg{sup}
%  \end{codesyntax}
%  \pkg{mathtools} also provides the extensible harpoons shown above.
%  They're taken from~\cite{Voss:2004}. For those liking visuals, here
%  are the macros in use in the order listed in the two boxes above.
%  \begin{align*}
%    A &\xleftrightarrow[below]{above} B &
%    A &\xRightarrow[below]{above} B \\
%    A &\xLeftarrow[below]{above} B &
%    A &\xLeftrightarrow[below]{above} B \\
%    A &\xhookleftarrow[below]{above} B &
%    A &\xhookrightarrow[below]{above} B \\
%    A &\xmapsto[below]{above} B \\
%    A &\xrightharpoondown[below]{above} B &
%    A &\xrightharpoonup[below]{above} B \\
%    A &\xleftharpoondown[below]{above} B &
%    A &\xleftharpoonup[below]{above} B \\
%    A &\xrightleftharpoons[below]{above} B &
%    A &\xleftrightharpoons[below]{above} B \\
%  \end{align*}
%  Change in 2020: In \cs{xLeftarrow}, \cs{xRightarrow} and
%  \cs{xLeftrightarrow} we added a space to the argument(s), making
%  the arrow slightly longer and moving the argument away from the
%  large arrow head.
% 
% 
%
%
%
%  \subsubsection{Braces and brackets}
%
%  \LaTeX{} defines other kinds of extensible symbols like
%  \cs{overbrace} and \cs{underbrace}, but sometimes you may want
%  another symbol, say, a bracket.
%  \begin{codesyntax}
%    \SpecialUsageIndex{\underbracket}\cs{underbracket}\oarg{rule thickness}
%      \oarg{bracket height}\marg{arg}\\
%    \SpecialUsageIndex{\overbracket}\cs{overbracket}\oarg{rule thickness}
%      \oarg{bracket height}\marg{arg}
%  \end{codesyntax}
%  The commands \cs{underbracket} and \cs{overbracket} are inspired
%  by \cite{Voss:2004}, although the implementation here is slightly
%  different.
%  Used without the optional arguments the bracket commands produce this:
%  \begin{quote}
%   |$\underbracket {foo\ bar}_{baz}$|\quad  $\underbracket {foo\ bar}_{baz}$ \\
%   |$\overbracket {foo\ bar}^{baz}$ |\quad  $\overbracket {foo\ bar}^{baz}$
%  \end{quote}
%  The default rule thickness is equal to that of \cs{underbrace}
%  (app.~$5/18$\,ex) while the default bracket height is equal to
%  app.~$0.7$\,ex. These values give really pleasing results in all
%  font sizes, but feel free to use the optional arguments. That way
%  you may get ``beauties'' like
%  \begin{verbatim}
%    \[
%      \underbracket[3pt]{xxx\  yyy}_{zzz} \quad \text{and} \quad
%      \underbracket[1pt][7pt]{xxx\  yyy}_{zzz}
%    \]
%  \end{verbatim}
%    \[
%      \underbracket[3pt]{xxx\  yyy}_{zzz} \quad \text{and} \quad
%      \underbracket[1pt][7pt]{xxx\  yyy}_{zzz}
%    \]
%  \begin{codesyntax}
%    \SpecialUsageIndex{\underbrace}\cs{underbrace}\marg{arg}\texttt{~~}
%    \SpecialUsageIndex{\LaTeXunderbrace}\cs{LaTeXunderbrace}\marg{arg}\\
%    \SpecialUsageIndex{\overbrace}\cs{overbrace}\marg{arg}\texttt{~~~}
%    \SpecialUsageIndex{\LaTeXoverbrace}\cs{LaTeXoverbrace}\marg{arg}
%  \end{codesyntax}
%  The standard implementation of the math operators \cs{underbrace}
%  and \cs{overbrace} in \LaTeX{} has some deficiencies. For example,
%  all lengths used internally are \emph{fixed} and optimized for
%  10\,pt typesetting. As a direct consequence thereof, using font
%  sizes other than 10 will produce less than optimal results.
%  Another unfortunate feature is the size of the braces. In the
%  example below, notice how the math operator \cs{sum} places its
%  limit compared to \cs{underbrace}.
%  \[
%    \mathop{\thinboxed[blue]{\sum}}_{n}
%    \mathop{\thinboxed[blue]{\LaTeXunderbrace{\thinboxed[green]{foof}}}}_{zzz}
%  \]
%  The blue lines indicate the dimensions of the math operator and
%  the green lines the dimensions of $foof$. As you can see, there
%  seems to be too much space between the brace and the $zzz$ whereas
%  the space between brace and $foof$ is okay. Let's see what happens
%  when we use a bigger font size:\par\Huge\vskip-\baselineskip
%  \[
%    \mathop{\thinboxed[blue]{\sum}}_{n}
%    \mathop{\thinboxed[blue]{\LaTeXunderbrace{\thinboxed[green]{foof}}}}_{zzz}
%  \]
%  \normalsize Now there's too little space between the brace and the
%  $zzz$ and also too little space between the brace and the $foof$.
%  If you use Computer Modern you'll actually see that the $f$
%  overlaps with the brace! Let's try in \cs{footnotesize}:
%  \par\footnotesize
%  \[
%    \mathop{\thinboxed[blue]{\sum}}_{n}
%    \mathop{\thinboxed[blue]{\LaTeXunderbrace{\thinboxed[green]{foof}}}}_{zzz}
%  \]\normalsize
%  Here the spacing above and below the brace is quite excessive.
%
%  As \cs{overbrace} has the exact same problems, there are good
%  reasons for \pkg{mathtools} to make redefinitions of
%  \cs{underbrace} and \cs{overbrace}. These new versions work
%  equally well in all font sizes and fixes the spacing issues and
%  apart from working with the default Computer Modern fonts, they
%  also work with the packages \pkg{mathpazo}, \pkg{pamath},
%  \pkg{fourier}, \pkg{eulervm}, \pkg{cmbright}, and \pkg{mathptmx}.
%  If you use the \pkg{ccfonts} to get the full Concrete fonts, the
%  original version saved under the names \cs{LaTeXunderbrace} and
%  \cs{LaTeXoverbrace} are better, due to of the special design of
%  the Concrete extensible braces. In that case you should probably
%  just add the lines
%  \begin{verbatim}
%    \let\underbrace\LaTeXunderbrace
%    \let\overbrace\LaTeXoverbrace
%  \end{verbatim}
%  to your preamble after loading \pkg{mathtools} which will restore
%  the original definitions of \cs{overbrace} and \cs{underbrace}.
%
%
%  
%
%  \subsection{New mathematical building blocks}
%
%  In this part of the manual, various mathematical environments are
%  described.
%
%  \subsubsection{Matrices}\label{subsubsec:matrices}
%
%  \begin{codesyntax}
%    \SpecialEnvIndex{matrix*}\cs{begin}\arg{matrix*}\texttt{ }\oarg{col}
%        \meta{contents} \cs{end}\arg{matrix*}\\
%    \SpecialEnvIndex{pmatrix*}\cs{begin}\arg{pmatrix*}\oarg{col}
%        \meta{contents} \cs{end}\arg{pmatrix*}\\
%    \SpecialEnvIndex{bmatrix*}\cs{begin}\arg{bmatrix*}\oarg{col}
%        \meta{contents} \cs{end}\arg{bmatrix*}\\
%    \SpecialEnvIndex{Bmatrix*}\cs{begin}\arg{Bmatrix*}\oarg{col}
%        \meta{contents} \cs{end}\arg{Bmatrix*}\\
%    \SpecialEnvIndex{vmatrix*}\cs{begin}\arg{vmatrix*}\oarg{col}
%        \meta{contents} \cs{end}\arg{vmatrix*}\\
%    \SpecialEnvIndex{Vmatrix*}\cs{begin}\arg{Vmatrix*}\oarg{col}
%        \meta{contents} \cs{end}\arg{Vmatrix*}
%  \end{codesyntax}
%  \FeatureRequest{Lars Madsen}{2004/04/05}
%  All of the \pkg{amsmath} \env{matrix} environments center the
%  columns by default, which is not always what you want. Thus
%  \pkg{mathtools} provides a starred version for each of the original
%  environments. These starred environments take an optional argument
%  specifying the alignment of the columns, so that
%  \begin{verbatim}
%    \[
%      \begin{pmatrix*}[r]
%        -1 & 3 \\
%        2  & -4
%      \end{pmatrix*}
%    \]
%  \end{verbatim}
%  yields
%    \[
%      \begin{pmatrix*}[r]
%        -1 & 3 \\
%        2  & -4
%      \end{pmatrix*}
%    \]
%  The optional argument (default is \texttt{[c]}) can be any column
%  type valid in the usual \env{array} environment.
%
%  While we are at it, we also provide fenced versions of the
%  \env{smallmatrix} environment, To keep up with the naming of the
%  large matrix environments, we provide both a starred and a
%  non-starred version. Since \env{smallmatrix} is defined in a
%  different manner than the \env{matrix} environment, the option to
%  say \env{smallmatrix*} \emph{has} to be either \texttt{c},
%  \texttt{l} \emph{or}~\texttt{r}. The default is \texttt{c}, which
%  can be changed globally using the \key{smallmatrix-align}=\meta{c,l
%    or r}.
%  \begin{codesyntax}
%    \SpecialEnvIndex{smallmatrix*}\cs{begin}\arg{smallmatrix*}\texttt{ }\oarg{col}
%        \meta{contents} \cs{end}\arg{smallmatrix*}\\
%    \SpecialEnvIndex{psmallmatrix}\cs{begin}\arg{psmallmatrix}
%        \meta{contents} \cs{end}\arg{psmallmatrix}\\
%    \SpecialEnvIndex{psmallmatrix*}\cs{begin}\arg{psmallmatrix*}\oarg{col}
%        \meta{contents} \cs{end}\arg{psmallmatrix*}\\
%    \SpecialEnvIndex{bsmallmatrix}\cs{begin}\arg{bsmallmatrix}
%        \meta{contents} \cs{end}\arg{bsmallmatrix}\\
%    \SpecialEnvIndex{bsmallmatrix*}\cs{begin}\arg{bsmallmatrix*}\oarg{col}
%        \meta{contents} \cs{end}\arg{bsmallmatrix*}\\
%    \SpecialEnvIndex{Bsmallmatrix}\cs{begin}\arg{Bsmallmatrix}
%        \meta{contents} \cs{end}\arg{Bsmallmatrix}\\
%    \SpecialEnvIndex{Bsmallmatrix*}\cs{begin}\arg{Bsmallmatrix*}\oarg{col}
%        \meta{contents} \cs{end}\arg{Bsmallmatrix*}\\
%    \SpecialEnvIndex{vsmallmatrix}\cs{begin}\arg{vsmallmatrix}
%        \meta{contents} \cs{end}\arg{vsmallmatrix}\\
%    \SpecialEnvIndex{vsmallmatrix*}\cs{begin}\arg{vsmallmatrix*}\oarg{col}
%        \meta{contents} \cs{end}\arg{vsmallmatrix*}\\
%    \SpecialEnvIndex{Vsmallmatrix}\cs{begin}\arg{Vsmallmatrix}
%        \meta{contents} \cs{end}\arg{Vsmallmatrix}\\
%    \SpecialEnvIndex{Vsmallmatrix*}\cs{begin}\arg{Vsmallmatrix*}\oarg{col}
%        \meta{contents} \cs{end}\arg{Vsmallmatrix*}\\
%    \SpecialKeyIndex{smallmatrix-align}\makebox{$\key{smallmatrix-align}=\meta{c,l or r}$}\\
%    \SpecialKeyIndex{smallmatrix-inner-space}\makebox{$\key{smallmatrix-inner-space}=\cs{,}$}
%  \end{codesyntax}
%  \ProvidedBy{Rasmus Villemoes}{2011/01/17}
% \begin{verbatim}
% \[
% \begin{bsmallmatrix}    a & -b \\ -c & d \end{bsmallmatrix}
% \begin{bsmallmatrix*}[r] a & -b \\ -c & d \end{bsmallmatrix*}
% \]
% \end{verbatim}
%  yields
% \[
% \begin{bsmallmatrix} a & -b \\ -c & d \end{bsmallmatrix}
% \begin{bsmallmatrix*}[r] a & -b \\ -c & d \end{bsmallmatrix*}
% \]
% Inside the \verb?Xsmallmatrix? construction a small space is
% inserted between the fences and the contents, the size of it can be
% changed using \key{smallmatrix-align}=\meta{some spacing command},
% the default is \cs{,}.
%
% As an extra trick the fences will behave as open and closing fences
% in constract to their auto-scaling nature.\footnote{\cs{left} and
% \cs{right} creates an \emph{inner} construction, and not as one
% might expect something where a preceeding \cs{sin} sees an opening
% fence, thus the space before or after may be too large. Inside this
% construction they behave.}
% 
%
%  \subsubsection{The \env{multlined} environment}
%
%  \begin{codesyntax}
%    \SpecialEnvIndex{multlined}\cs{begin}\arg{multlined}\oarg{pos}\oarg{width}
%        \meta{contents} \cs{end}\arg{multlined}\\
%    \SpecialUsageIndex{\shoveleft}\cs{shoveleft}\oarg{dimen}\marg{arg}\texttt{~~}
%    \SpecialUsageIndex{\shoveright}\cs{shoveright}\oarg{dimen}\marg{arg}\\
%    \makeatletter\settowidth\@tempdimc{\cs{shoveleft}\oarg{dimen}\marg{arg}}\global\@tempdimc\@tempdimc
%    \SpecialKeyIndex{firstline-afterskip}\makebox[\@tempdimc][l]{$\key{firstline-afterskip}=\meta{dimen}$}\texttt{~~}
%    \SpecialKeyIndex{lastline-preskip}$\key{lastline-preskip}=\meta{dimen}$\\
%    \makeatletter\SpecialKeyIndex{multlined-width}\makebox[\@tempdimc][l]{$\key{multlined-width}=\meta{dimen}$}\texttt{~~}
%    \SpecialKeyIndex{multlined-pos}$\key{multlined-pos}=\texttt{c}\vert\texttt{b}\vert\texttt{t}$
%  \end{codesyntax}
%  Some of the \pkg{amsmath} environments exist in two forms: an
%  outer and an inner environment. One example is the pair
%  \env{gather} \& \env{gathered}. There is one important omission on
%  this list however, as there is no inner \env{multlined}
%  environment, so this is where \pkg{mathtools} steps in.
%
%  One might wonder what the sensible behavior should be. We want it
%  to be an inner environment so that it is not wider than necessary,
%  but on the other hand we would like to be able to control the
%  width. The current implementation of \env{multlined} handles both
%  cases. The idea is this: Set the first line flush left and add a
%  hard space after it; this space is governed by the
%  \key{firstline-afterskip} key. The last line should be set flush
%  right and preceded by a hard space of size \key{lastline-preskip}.
%  Both these hard spaces have a default value of \cs{multlinegap}.
%  Here we use a `t' in the first optional argument denoting a
%  top-aligned building block (the default is `c').
%  \begin{verbatim}
%    \[
%      A = \begin{multlined}[t]
%            \framebox[4cm]{first} \\
%            \framebox[4cm]{last}
%          \end{multlined} B
%    \]
%  \end{verbatim}
%    \[
%      A = \begin{multlined}[t]
%            \framebox[4cm]{first} \\
%            \framebox[4cm]{last}
%          \end{multlined} B
%    \]
%  Note also that \env{multlined} gives you access to an extended
%  syntax for \cs{shoveleft} and \cs{shoveright} as shown in the
%  example below.
%  \begin{verbatim}
%    \[
%      \begin{multlined}
%        \framebox[.65\columnwidth]{First line}        \\
%        \framebox[.5\columnwidth]{Second line}        \\
%        \shoveleft{L+E+F+T}                           \\
%        \shoveright{R+I+G+H+T}                        \\
%        \shoveleft[1cm]{L+E+F+T}                      \\
%        \shoveright[\widthof{$R+I+G+H+T$}]{R+I+G+H+T} \\
%        \framebox[.65\columnwidth]{Last line}
%      \end{multlined}
%    \]
%  \end{verbatim}
%  \[
%    \begin{multlined}
%      \framebox[.65\columnwidth]{First line} \\
%      \framebox[.5\columnwidth]{Second line} \\
%      \shoveleft{L+E+F+T}         \\
%      \shoveright{R+I+G+H+T}         \\
%      \shoveleft[1cm]{L+E+F+T}         \\
%      \shoveright[\widthof{$R+I+G+H+T$}]{R+I+G+H+T}         \\
%      \framebox[.65\columnwidth]{Last line}
%    \end{multlined}
%  \]
%
%  You can also choose the width yourself by specifying it as an
%  optional argument:
%  \begin{verbatim}
%    \[
%      \begin{multlined}[b][7cm]
%        \framebox[4cm]{first} \\
%        \framebox[4cm]{last}
%      \end{multlined} = B
%    \]
%  \end{verbatim}
%    \[
%       \begin{multlined}[b][7cm]
%            \framebox[4cm]{first} \\
%            \framebox[4cm]{last}
%          \end{multlined} = B
%    \]
%  There can be two optional arguments (position and width) and
%  they're interchangeable.
%
%  \medskip\noindent
%  \textbf{Bug 1:}
%  \CommentAdded{2014/05/11}  
%  If used inside an \env{array} or
%  a derivative (say, a \env{matrix} variant), \env{multlined} does
%  not work as expected. The implementation contains an `invisible'
%  line after the first \env{multline} row, inside an \env{array} this
%  line is no longer `invisible' because \env{array} sets
%  \cs{baselineskip} to zero. Currently we have no general workaround
%  for this. 
%
%  \medskip\noindent
%  \textbf{Bug 2:}
%  \CommentAdded{2015/11/12}  
%  Due to the way \env{multlined} is implemented, certain
%  constructions does not work inside \env{multlined}. We have added a
%  hook (\cs{MultlinedHook}) that can be added to. The default value
%  is a fix for \env{subarray} and \env{crampedsubarray} and thus for
%  \cs{substack} and \cs{crampedsubstack} (and a few others, thus add
%  to the hook, don't replace it).
%
%  This can actually be used to fix Bug~1:
% \begin{verbatim}
% \usepackage{mathtools,etoolbox}
% \newlength\Normalbaselineskip
% \setlength\Normalbaselineskip{\baselineskip} 
% \appto\MultlinedHook{
%   \setlength\baselineskip{\Normalbaselineskip}
% }
% \end{verbatim}
%  
%  \medskip\noindent
%  \textbf{Caveat:}
%  \CommentAdded{2017/05/22}  
%  \cs{shoveleft} and \cs{shoveright} does not work as expected if
%  used as the last line in \env{multlined}.
%
%
%  \subsubsection{More \env{cases}-like environments}
%
%  \begin{codesyntax}
%    \SpecialEnvIndex{dcases}
%    \cs{begin}\arg{dcases}\texttt{~}  \meta{math_column} |&| \meta{math_column}
%    \cs{end}\arg{dcases}\\
%    \SpecialEnvIndex{dcases*}
%    \cs{begin}\arg{dcases*}  \meta{math_column} |&| \makebox[\widthof{\meta{math\_column}}][l]{\meta{text\_column}}
%    \cs{end}\arg{dcases*}\\
%    \SpecialEnvIndex{rcases}
%    \cs{begin}\arg{rcases}\texttt{~~}  \meta{math_column} |&| \makebox[\widthof{\meta{math\_column}}][l]{\meta{math\_column}}
%    \cs{end}\arg{rcases}\\
%    \SpecialEnvIndex{rcases*}
%    \cs{begin}\arg{rcases*}\texttt{~}  \meta{math_column} |&| \makebox[\widthof{\meta{math\_column}}][l]{\meta{text\_column}}
%    \cs{end}\arg{rcases*}\\
%    \SpecialEnvIndex{drcases}
%    \cs{begin}\arg{drcases}\texttt{~}  \meta{math_column} |&| \makebox[\widthof{\meta{math\_column}}][l]{\meta{math\_column}}
%    \cs{end}\arg{drcases}\\
%    \SpecialEnvIndex{drcases*}
%    \cs{begin}\arg{drcases*}  \meta{math_column} |&| \makebox[\widthof{\meta{math\_column}}][l]{\meta{text\_column}}
%    \cs{end}\arg{drcases*}\\
%    \SpecialEnvIndex{cases*}
%    \cs{begin}\arg{cases*}\texttt{~}  \meta{math_column} |&| \makebox[\widthof{\meta{math\_column}}][l]{\meta{text\_column}}
%    \cs{end}\arg{cases*}
%  \end{codesyntax}
%  \FeatureRequest{Lars Madsen}{2004/07/01}
%  Anyone who have tried to use an integral in the regular
%  \env{cases} environment from \pkg{amsmath} will have noticed that
%  it is set as
%  \[
%    a=\begin{cases}
%      E = m c^2     & \text{Nothing to see here} \\
%      \int x-3\, dx & \text{Integral is text style}
%    \end{cases}
%  \]
%  \pkg{mathtools} provides two environments similar to \env{cases}.
%  Using the \env{dcases} environment you get the same output as with
%  \env{cases} except that the rows are set in display style.
%  \begin{verbatim}
%  \[
%    \begin{dcases}
%      E = m c^2     & c \approx 3.00\times 10^{8}\,\mathrm{m}/\mathrm{s} \\
%      \int x-3\, dx & \text{Integral is display style}
%    \end{dcases}
%  \]
%  \end{verbatim}
%  \[
%    \begin{dcases}
%      E = m c^2     & c \approx 3.00\times 10^{8}\,\mathrm{m}/\mathrm{s} \\
%      \int x-3\, dx & \text{Integral is display style}
%    \end{dcases}
%  \]
%  Additionally the environment \env{dcases*} acts just the same, but
%  the second column is set in the normal roman font of the
%  document.\footnote{Or rather: it inherits the font characteristics
%  active just before the \env{dcases*} environment.}
%  \begin{verbatim}
%  \[
%    a= \begin{dcases*}
%      E = m c^2     & Nothing to see here \\
%      \int x-3\, dx & Integral is display style
%    \end{dcases*}
%  \]
%  \end{verbatim}
%  \[
%    a= \begin{dcases*}
%      E = m c^2     & Nothing to see here \\
%      \int x-3\, dx & Integral is display style
%    \end{dcases*}
%  \]
%  The environments \env{rcases}, \env{rcases*}, \env{drcases} and
%  \env{drcases*} are equivalent to \env{cases} and \env{dcases}, but
%  here the brace is placed on the right instead of on the left.
% \begin{verbatim}
% \[
% \begin{rcases*}
%   x^2 & for $x>0$\\ 
%   x^3 & else
% \end{rcases*} \quad \Rightarrow \cdots
% \]
% \end{verbatim}
% \[
% \begin{rcases*}
%   x^2 & for $x>0$\\ 
%   x^3 & else
% \end{rcases*} \quad \Rightarrow\cdots
% \]
%
%
%  \subsubsection{Emulating indented lines in alignments}
%  \begin{codesyntax}
%    \SpecialEnvIndex{\MoveEqLeft}\cs{MoveEqLeft}\oarg{number}
%  \end{codesyntax}
%  \ProvidedBy{Lars Madsen}{2008/06/05} In \cite{Swanson}, Ellen
%  Swanson recommends that when ever one has a long displayed formula,
%  spanning several lines, and it is unfeasible to align against a
%  relation within the first line, then all lines in the display
%  should be aligned at the left most edge of the first line, and all
%  subsequent lines should be indented by 2\,em (or if needed by a
%  smaller amount). That is we are talking about displays that end up
%  looking like this
%  \begin{align*}
%    \MoveEqLeft \framebox[10cm][c]{Long first line}\\
%    & = \framebox[6cm][c]{ \hphantom{g} 2nd line}\\ 
%    & \leq \dots
%  \end{align*}
%  Traditionally one could do this by starting subsequent lines by
%  \verb+&\qquad ...+, but that is tedious. Instead the example above
%  was made using \cs{MoveEqLeft}:
%  \begin{verbatim}
%  \begin{align*}
%    \MoveEqLeft \framebox[10cm][c]{Long first line}\\
%    & = \framebox[6cm][c]{ \hphantom{g} 2nd line}\\ 
%    & \leq \dots
%  \end{align*}
%  \end{verbatim}
%  \cs{MoveEqLeft} is placed instead of the \verb+&+ on the first
%  line, and will effectively \emph{move} the entire first line
%  \oarg{number} of ems to the left (default is 2). If you choose to
%  align to the right of the relation, use \cs{MoveEqLeft}\verb+[3]+
%  to accommodate the extra distance to the alignment point:
%  \begin{verbatim}
%  \begin{align*}
%    \MoveEqLeft[3] \framebox[10cm][c]{Part 1}\\
%     = {} & \framebox[8cm][c]{2nd line}\\ 
%          & + \framebox[4cm][c]{ last part}
%  \end{align*}
%  \end{verbatim}
%  \begin{align*}
%    \MoveEqLeft[3] \framebox[10cm][c]{Long first line}\\
%     = {} & \framebox[6cm][c]{  2nd line}\\ 
%          & + \framebox[4cm][c]{ last part}
%  \end{align*}
%
%
% \noindent\textbf{Caveat regarding \cs{MoveEqLeft}}: If the first
% part of the equation starts with say \verb|[a]|, \cs{MoveEqLeft} may
% attempt to eat it! You can prevent this by specifying the optional
% argument (remember the default is the same as \cs{MoveEqLeft[2]} or
% by using \cs{MoveEqLeft\{\}}.
%
%  \subsubsection{Boxing a single line in an alignment}
%  
%  The \texttt{amsmath} package provides the \cs{boxed} macro to box
%  material in math mode. But this of course will not work if the box
%  should cross an alignment point. We provide a macro that
%  can.\footnote{Note that internally \cs{Aboxed} does use
%  \cs{boxed}.}  \hskip1sp \marginpar{%
%    \parbox[b]{\marginparwidth}{\small\sffamily\raggedright
%      \strut Evolved from a request by\\Merciadri Luca\\
%       2010/06/28\\on comp.text.tex%
%    }\strut
%  }%
%   \marginpar{\strut\\%
%    \parbox[b]{\marginparwidth}{\small\sffamily\raggedright
%      \strut Reimplemented by\\Florent Chervet (GL) \\
%       2011/06/11\\on comp.text.tex%
%    }\strut
%  }%
%  \begin{codesyntax}
%    \SpecialEnvIndex{\Aboxed}\cs{Aboxed}\marg{left hand side 
%     \quad\texttt{\textnormal{\&}}\quad right hand side}
%  \end{codesyntax}
%  Example
% \begin{verbatim}
% \begin{align*}
%   \Aboxed{ f(x) & = \int h(x)\, dx} \\
%                 & = g(x)
% \end{align*}
% \end{verbatim}
% Resulting in:
% \begin{align*}
%   \Aboxed{ f(x) & = \int h(x)\, dx} \\
%                 & = g(x)
% \end{align*}
% One can have multiple boxes on each line, and the
% >>\texttt{\textnormal{\&}}\quad right hand side<< can even be
% missing. Here is an example of how the padding in the box can be changed
% \begin{verbatim}
% \begin{align*}
%   \setlength\fboxsep{1em}
%   \Aboxed{ f(x) &= 0 } & \Aboxed{ g(x) &= b} \\
%   \Aboxed{ h(x) }      & \Aboxed{ i(x) }   
% \end{align*}
% \end{verbatim}
% \begin{align*}
%   \setlength\fboxsep{1em}
%   \Aboxed{ f(x) &= 0 } & \Aboxed{ g(x) &= b} \\
%   \Aboxed{ h(x) }      & \Aboxed{ i(x) }   
% \end{align*}
% Note how the \cs{fboxsep} change only affect the box coming
% immediately after it.  
%
%  \marginpar{\strut\\%
%    \parbox[b]{\marginparwidth}{\small\sffamily\raggedright
%      \strut Comment by Qrrbrbirlbel on \url{tex.stackexchange.com}, 2013/05/20.
%    }\strut
%  }%
% As \cs{Aboxed} looks for the alignment \texttt{\&} it may be
% necessary to \emph{hide} constructions like matrices that also make
% use of \texttt{\&}. Just add a set of braces around the construction
% you want to hide. 
%
%  \subsubsection{Adding arrows between lines in an alignment}
%
%  This first macro is a bit misleading, it is only intended to be
%  used in combination with the \env{alignat(*)} environment.
%  \begin{codesyntax}
%    \SpecialEnvIndex{\ArrowBetweenLines}\cs{ArrowBetweenLines}\oarg{symbol}\\
%    \SpecialEnvIndex{\ArrowBetweenLines*}\cs{ArrowBetweenLines*}\oarg{symbol}
%  \end{codesyntax}
%    \hskip1sp
%   \marginpar{%
%    \parbox[b]{\marginparwidth}{\small\sffamily\raggedright
%      \strut Evolved from a request by\\Christian
%      Bohr-Halling\\2004/03/31\\on dk.edb.tekst%
%    }
%  }%
%  To add, say $\Updownarrow$ between two lines in an alignment use
%  \cs{ArrowBetweenLines} and the \env{alignat} environment (note the
%  extra pair of  \texttt{\&}'s in front):
%  \begin{verbatim}
%  \begin{alignat}{2}
%    && \framebox[1.5cm]{} &= \framebox[3cm]{}\\
%    \ArrowBetweenLines % \Updownarrow is the default
%    && \framebox[1.5cm]{} &= \framebox[2cm]{}
%  \end{alignat}
%  \end{verbatim}
%  resulting in
%  \begin{alignat}{2}
%    && \framebox[1.5cm]{} &= \framebox[3cm]{}\\
%    \ArrowBetweenLines 
%    && \framebox[1.5cm]{} &= \framebox[2cm]{}
%  \end{alignat}
%  Note the use of \verb+&&+ starting each \emph{regular} line of
%  math. For adding the arrow on the right, use
%  \cs{ArrowBetweenLines*}\oarg{symbol}, and end each line of math
%  with \verb+&&+.
%  \begin{verbatim}
%  \begin{alignat*}{2}
%    \framebox[1.5cm]{} &= \framebox[3cm]{}  &&\\
%    \ArrowBetweenLines*[\Downarrow] 
%    \framebox[1.5cm]{} &= \framebox[2cm]{}  &&
%  \end{alignat*}
%  \end{verbatim}
%  resulting in
%  \begin{alignat*}{2}
%    \framebox[1.5cm]{} &= \framebox[3cm]{}  &&\\
%    \ArrowBetweenLines*[\Downarrow] 
%    \framebox[1.5cm]{} &= \framebox[2cm]{}  &&
%  \end{alignat*}
%
%
% \subsubsection{Centered \texorpdfstring{\cs{vdots}}{\textbackslash vdots}}
%
%  If one want to mark a vertical continuation, there is
%  the \verb?\vdots? command, but combine this with an alignment and
%  we get something rather suboptimal
% \FeatureRequest{Bruno Le Floch \\(and many others)}{2011/01/25}
%  \begin{align*}
%    \framebox[1.5cm]{} &= \framebox[3cm]{}\\
%                       & \vdots\\
%                       &= \framebox[3cm]{} 
%  \end{align*}
%  It would be nice to have (1) a \verb?\vdots? centered within the
%  width of another symbol, and (2) a construction similar to
%  \verb?\ArrowBetweenLines? that does not take up so much space. 
%  We provide both.
%  \begin{codesyntax}
%    \SpecialUsageIndex{\vdotswithin}\cs{vdotswithin}\marg{symbol}\\
%    \SpecialUsageIndex{\shortvdotswithin}\cs{shortvdotswithin}\marg{symbol}\\
%    \SpecialUsageIndex{\shortvdotswithin*}\cs{shortvdotswithin*}\marg{symbol}\\
%    \SpecialUsageIndex{\MTFlushSpaceAbove}\cs{MTFlushSpaceAbove}\\
%    \SpecialUsageIndex{\MTFlushSpaceBelow}\cs{MTFlushSpaceBelow}\\
%    \SpecialKeyIndex{shortvdotsadjustabove}\makebox{$\key{shortvdotsadjustabove}=\meta{length}$}\\
%    \SpecialKeyIndex{shortvdotsadjustbelow}\makebox{$\key{shortvdotsadjustbelow}=\meta{length}$}
%  \end{codesyntax}
%  Two examples in one
% \begin{verbatim}
% \begin{align*}
%   a &= b              \\
%     & \vdotswithin{=} \\
%     & = c             \\
%     \shortvdotswithin{=}
%     & = d
% \end{align*}
% \end{verbatim}
% yielding
% \begin{align*}
%   a &= b              \\
%     & \vdotswithin{=} \\
%     & = c             \\
%     \shortvdotswithin{=}
%     & = d
% \end{align*}
% Thus \verb?\vdotswithin{=}? creates a box corresponding to
% \verb?{}={}? and typeset a >>$\vdots$<< centered inside it. When
% doing this as a normal line in an alignment leaves us with excessive
% space which \verb?\shortvdotswithin{=}? takes care with for us.
%
% The macro call \verb?\shortvdotswithin{=}? corresponds to
% \begin{verbatim}
% \MTFlushSpaceAbove
% & \vdotswithin{=} 
% \MTFlushSpaceBelow
% \end{verbatim}
% whereas \verb?\shortvdotswithin*{=}? is the case with
% \begin{verbatim}
% \MTFlushSpaceAbove
% \vdotswithin{=} &
% \MTFlushSpaceBelow
% \end{verbatim}
% Note how \verb?\MTFlushSpaceBelow?  implicitly adds a \verb?\\? at
% the end of the line. Thus one cannot have more on the line when
% using \verb?\shortvdotswithin? or the starred version. But, if
% needed, one can de-construct the macro and arrive at
% \begin{verbatim}
% \begin{alignat*}{3}
%   A&+ B &&= C &&+ D \\
%   \MTFlushSpaceAbove
%   &\vdotswithin{+} &&&& \vdotswithin{+}
%   \MTFlushSpaceBelow
%   C &+ D &&= Y &&+K
% \end{alignat*}
% \end{verbatim}
% yielding
% \begin{alignat*}{3}
%   A&+ B &&= C &&+ D \\
%   \MTFlushSpaceAbove
%   &\vdotswithin{+} &&&& \vdotswithin{+}
%   \MTFlushSpaceBelow
%   C &+ D &&= Y &&+K
% \end{alignat*}
% If one has the need for such a construction.
%
% The de-spaced version does support the \env{spreadlines}
% environment. The actual amount of space being \emph{flushed} above
% and below can be controlled by the user using the two options
% indicated. Their original values are \verb?2.15\origjot? and
% \verb?\origjot? respectively (\verb?\origjot? is usually 3pt). 
%
%  \subsection{Intertext and short intertext}
%
%
%  \begin{codesyntax}
%    \SpecialUsageIndex{\shortintertext}\cs{shortintertext}\marg{text}
%  \end{codesyntax}
%  \cttPosting{Gabriel Zachmann and Donald Arseneau}{2000/05/12--13}
%  \pkg{amsmath} provides the command \cs{intertext} for interrupting
%  a multiline display while still maintaining the alignment points.
%  However the spacing often seems quite excessive as seen below.
%  \begin{verbatim}
%    \begin{align}
%      a&=b \intertext{Some text}
%      c&=d
%    \end{align}
%  \end{verbatim}
%    \begin{align}
%      a&=b \intertext{Some text}
%      c&=d
%    \end{align}
%
%  Using the command \cs{shortintertext} alleviates this situation
%  somewhat:
%  \begin{verbatim}
%    \begin{align}
%      a&=b \shortintertext{Some text}
%      c&=d
%    \end{align}
%  \end{verbatim}
%  \begin{align}
%    a&=b \shortintertext{Some text}
%    c&=d
%  \end{align}
%
%  \noindent
%  It turns out that both \cs{shortintertext} and the original
%  \cs{intertext} from \pkg{amsmath} has a slight problem. If we use
%  the \env{spreadlines} (see section~\ref{sec:spread}) to open up
%  the equations in a multiline calculation, then this opening up
%  value also applies to the spacing above and below the original
%  \cs{shortintertext} and \cs{intertext}.  \tsxPosting{Tobias Weh
%    \\(referring to a suggestion by Chung-chieh Shan)}{2011/05/29}
% It can be illustrated using the following example, an interested
% reader, can apply it with and without the original \cs{intertext}
% and \cs{shortintertext}.
% \begin{verbatim}
% % the original \intertext and \shortintertext
% \mathtoolsset{original-intertext,original-shortintertext}
% \newcommand\myline{\par\noindent\rule{\textwidth}{1mm}} 
% \myline
% \begin{spreadlines}{1em}
%   \begin{align*}
%     AA\\  BB\\  \intertext{\myline}
%     AA\\  BB\\  \shortintertext{\myline}
%     AA\\  BB
%   \end{align*}
% \end{spreadlines}
% \myline
% \end{verbatim}
%
%  We now fix this internaly for both \cs{intertext} and
%  \cs{shortintertext}, plus we add the possibility to fine tune
%  spacing around these constructions. The original versions can be
%  brought back using the \texttt{original-x} keys below.
%  \begin{codesyntax}
%    \SpecialUsageIndex{\intertext}\cs{intertext}\marg{text}\\
%    \SpecialUsageIndex{\shortintertext}\cs{shortintertext}\marg{text}\\
%    \SpecialKeyIndex{original-intertext}$\key{original-intertext}=\texttt{true}\vert\texttt{false}$ \quad(default: \texttt{false})\\
%    \SpecialKeyIndex{original-shortintertext}$\key{original-shortintertext}=\texttt{true}\vert\texttt{false}$ 
%    \quad(default: \texttt{false})\\ 
%    \SpecialKeyIndex{above-intertext-sep}$\key{above-intertext-sep}=\meta{dimen}$ \quad(default: 0pt)\\
%    \SpecialKeyIndex{below-intertext-sep}$\key{below-intertext-sep}=\meta{dimen}$ \quad(default: 0pt)\\
%    \SpecialKeyIndex{above-shortintertext-sep}$\key{above-shortintertext-sep}=\meta{dimen}$ \quad(default: 3pt)\\
%    \SpecialKeyIndex{below-shortintertext-sep}$\key{below-shortintertext-sep}=\meta{dimen}$ \quad(default: 3pt)
%  \end{codesyntax}
%  The updated \cs{shortintertext} will look like the original version
%  unless for areas with an enlarged \cs{jot} value (see for example
%  the \env{spreadlines}, section~\ref{sec:spread}). Whereas \cs{intertext}
%  will have a slightly smaller value above and below (corresponding
%  to about 3pt less space above and below), the spacing around
%  \cs{intertext} should now match the normal spacing going into and
%  out of an \env{align}.
%
% \textbf{Tip:} \cs{intertext} and \cs{shortintertext} also works
% within \env{gather}.
%
%
%  \subsection{Paired delimiters}
%
%
%  \begin{codesyntax}
%    \SpecialUsageIndex{\DeclarePairedDelimiter}
%    \cs{DeclarePairedDelimiter}\marg{cmd}\marg{left_delim}\marg{right_delim}
%  \end{codesyntax}
%  \FeatureRequest{Lars Madsen}{2004/06/25}
%  In the \pkg{amsmath} documentation it is shown how to define a few
%  commands for typesetting the absolute value and norm. These
%  definitions are:
%  \begin{verbatim}
%    \newcommand*\abs[1]{\lvert#1\rvert}
%    \newcommand*\norm[1]{\lVert#1\rVert}
%  \end{verbatim}
%  \DeclarePairedDelimiter\abs\lvert\rvert
%  While they produce correct horizontal spacing you have to be
%  careful about the vertical spacing if the argument is just a
%  little taller than usual as in
%  \[
%    \abs{\frac{a}{b}}
%  \]
%  Here it won't give a nice result, so you have to manually put in
%  either \cs{left}--\cs{right} pair or a \cs{bigl}--\cs{bigr} pair.
%  Both methods mean that you have to delete your \cs{abs} command,
%  which may not sound like an ideal solution.
%
%  With the command \cs{DeclarePairedDelimiter} you can combine all
%  these features in one easy to use command. Let's show an example:
%  \begin{verbatim}
%    \DeclarePairedDelimiter\abs{\lvert}{\rvert}
%  \end{verbatim}
%  This defines the command \cs{abs} just like in the
%  \pkg{amsmath} documentation but with a few additions:
%  \begin{itemize}
%    \item A starred variant: \cs{abs*} produces delimiters that are preceded
%  by \cs{left} and \cs{right} resp.:
%  \begin{verbatim}
%  \[
%    \abs*{\frac{a}{b}}
%  \]
%  \end{verbatim}
%      \[
%        \abs*{\frac{a}{b}}
%      \]
%  \item A variant with an optional argument:
%  \cs{abs}\oarg{size_cmd}, where
%  \meta{size_cmd} is either \cs{big}, \cs{Big}, \cs{bigg}, or
%  \cs{Bigg} (if you have any bigggger versions you can use them
%  too).
%  \begin{verbatim}
%  \[
%    \abs[\Bigg]{\frac{a}{b}}
%  \]
%  \end{verbatim}
%      \[
%        \abs[\Bigg]{\frac{a}{b}}
%      \]
%  \end{itemize}
%
%  \begin{codesyntax}
%    \SpecialUsageIndex{\DeclarePairedDelimiterX}
%    \cs{DeclarePairedDelimiterX}\marg{cmd}\oarg{num args}\marg{left_delim}\marg{right_delim}\marg{body}\\
%     \cs{delimsize}
%  \end{codesyntax}
%  Sometimes one might want to have the capabilities of
%  \cs{DeclarePairedDelimiter}, but also want a macro the takes more
%  than one argument and specify plus being able to specify the body
%  of the generated macro. 
%
%  \cs{DeclarePairedDelimiterX} extends the features of
%  \cs{DeclarePairedDelimiter} such that the user will get a macro
%  which is fenced off at either end, plus the capability to provide
%  the \meta{body} for what ever the macro should do within these
%  fences.
%
%  Inside the \meta{body} part, the macro \cs{delimsize} refer to the
%  size of the outer fences. It can then be used inside \meta{body} to
%  scale any inner fences.
%
%  Thus
%  \begin{quote}
%   \cs{DeclarePairedDelimiter}\marg{cmd}\marg{left_delim}\marg{right_delim}
%  \end{quote}
%  is the same thing as
%  \begin{quote}
%   \cs{DeclarePairedDelimiterX}\marg{cmd}\texttt{[1]}\marg{left_delim}\marg{right_delim}\rlap{\texttt{\{\#1\}}}
%  \end{quote}
%  %
%  Let us do some examples. First we want to prepare a macro for inner
%  products, with two arguments such that we can hide the character
%  separating the arguments (a~journal style might require a
%  semi-colon, so we will save a lot of hand editing). This can be
%  done via
% \begin{verbatim}
% \DeclarePairedDelimiterX\innerp[2]{\langle}{\rangle}{#1,#2}
% \end{verbatim}
% More interestingly we can refer to the size inside the
% \meta{code}. Here we do a weird three argument `braket'
% \begin{verbatim}
% \DeclarePairedDelimiterX\braket[3]{\langle}{\rangle}%
% {#1\,\delimsize\vert\,\mathopen{}#2\,\delimsize\vert\,\mathopen{}#3}
% \end{verbatim}
% \begin{itshape}
%   Note the use of >>\verb|\mathopen{}|<< in the \meta{body} of
%   \cs{braket}, this is very important when a scalable delimiter is
%   being used and it does not present itself as a left or right
%   delimiter. You will see why it is needed if you use
%   \verb|\braket{A}{-B}| in a version without
%   \verb|\mathopen{}|.\footnote{Basically, the problem is that
%   \cs{vert} is a `symbol', thus $\vert-B$ is interpreted
%   \emph{subtraction}, not a symbol followed by negative $B$. When
%   \cs{mathopen\{\}} is added, \TeX\ is told that an opening
%   delimiter just occurred, and it will adjust the minus
%   accordingly.}
% \end{itshape}
%
% Then we get
% \DeclarePairedDelimiterX\innerp[2]{\langle}{\rangle}{#1,#2}
% \DeclarePairedDelimiterX\braket[3]{\langle}{\rangle}%
% {#1\,\delimsize\vert\,\mathopen{}#2\,\delimsize\vert\,\mathopen{}#3}
% \begin{verbatim}
% \[
% \innerp*{A}{ \frac{1}{2} } \quad
% \braket[\Big]{B}{\sum_{k} f_k}{C}
% \]
% \end{verbatim}
% \[
% \innerp*{A}{ \frac{1}{2} } \quad
% \braket[\Big]{B}{\sum_{k}}{C}
% \]
% 
%
% \iffalse
% \bigskip
% 
% \noindent
% \textbf{\textit{Side note:}} We have changed the internal code of
% \cs{DeclarePairedDelimiter} and \cs{DeclarePairedDelimiterX} such
% that the starred version does no longer give the odd spacings that
% \cs{left}\dots\cs{right} sometimes give. Compare this
% \begin{verbatim}
% \[
% 2\innerp{A}{B} \quad  2\innerp*{A}{B} \quad 2\left\langle A,B \right\rangle
% \]
% \end{verbatim}
% \[
% \sin\innerp{A}{B} \quad
% \sin\innerp*{A}{B} \quad
% \sin\left\langle A,B \right\rangle
% \]
% The spacing in the last one does not behave like other fences (this
% is a feature of the \cs{left}\dots\cs{right} construction).
% \fi
%
% With the inner scaling, we can provide macros whos syntax closely
% follow the mathematical meaning. Fx for building sets, try
% this\footnote{The reason for using a separate \cs{SetSymbol} macro
% has to do with complicated set definitions, where the condition
% spans several lines. In this case \cs{Set} cannot be used. Thus it
% is nice to be able to refer to the specific set building symbol we
% have decided to use in this document. Also remember to add
% \cs{allowbreak} \emph{before} the proceeding inserted space. Then
% that space will disappear when a line break occurs.}
% \begin{verbatim}
%   % just to make sure it exists
%   \providecommand\given{} 
%   % can be useful to refer to this outside \Set
%   \newcommand\SetSymbol[1][]{%
%      \nonscript\:#1\vert
%      \allowbreak
%      \nonscript\:
%      \mathopen{}}
%   \DeclarePairedDelimiterX\Set[1]\{\}{%
%      \renewcommand\given{\SetSymbol[\delimsize]}
%      #1
%   }
% \end{verbatim}
%   \providecommand\given{} 
%   \newcommand\SetSymbol[1][]{\nonscript\:#1\vert\allowbreak\nonscript\:\mathopen{}}
%   \DeclarePairedDelimiterX\Set[1]\{\}{%
%      \renewcommand\given{\SetSymbol[\delimsize]}
%      #1
%   }
% \begin{verbatim}
% \[ \Set*{ x \in X \given \frac{\sqrt{x}}{x^2+1} > 1 } \]
% \end{verbatim}
% \[ \Set*{ x \in X \given \frac{\sqrt{x}}{x^2+1} > 1 } \]
% Thus we end up with a syntax much closer to how we read this
% aloud. Also we hide the `given' symbol for easy
% replacement.\footnote{The \cs{nonscript} construction removes the
% \cs{:} in sub- and superscript, this might not always be
% preferable. You can use
% \cs{mathchoice\{}\cs{:\}\{}\cs{:\}\{}\cs{,\}\{}\cs{,\}} instead of \cs{nonscript}\cs{:}.}
%
% Combining with \pkg{etoolbox} it becomes easy to make a function
% that automatically provide a marker for a blank argument:
% \begin{verbatim}
% \usepackage{etoolbox}
% \DeclarePairedDelimiterX\norm[1]\lVert\rVert{
%   \ifblank{#1}{\:\cdot\:}{#1}
% }  
% \end{verbatim}
% Then \verb|\norm{}| will give you $\lVert\:\cdot\:\rVert$ while
% \verb|\norm{a}| gives the expected $\lVert a \rVert$.
%
% \begin{codesyntax}
% \SpecialUsageIndex{\DeclarePairedDelimiterXPP}
% \cs{DeclarePairedDelimiterXPP}\marg{cmd}\oarg{num args}\marg{pre
%  code}\marg{left_delim}\\%
% \marg{right_delim}\marg{post code}\marg{body}
% \end{codesyntax}
% \FeatureRequest{Barbara Beeton (on TSE)}{2013/10/07}
% \cs{DeclarePairedDelimiterX} has an annoying caveat: it is very hard
% to make a macro \verb|\lnorm{a}| that should result in $\lVert
% a\rVert_2$.\footnote{The added >>\texttt{\_2}<< is hard to get around
% the argument parsing.} 
%
% As a consequence we provide
% \cs{DeclarePairedDelimiterXPP}.\footnote{Extended
% DeclarePairedDelimiter with Pre and Post code.} With the addition of
% the \marg{pre code} and \marg{post code} it is identical to
% \cs{DeclarePairedDelimiterX}. It should be interpreted as
% \begin{center}
% \marg{pre code} \marg{left_delim} \marg{body}
% \marg{right_delim} \marg{post code}  
% \end{center}
% 
%
% An $\mathcal{L}^2$ norm can now be defined as
% \begin{verbatim}
% \DeclarePairedDelimiterXPP\lnorm[1]{}\lVert\rVert{_2}{#1}
% \end{verbatim}
% A probability macro with build in support for conditionals
% (\cs{given} initialized as above)
% \begin{verbatim}
% \DeclarePairedDelimiterXPP\Prob[1]{\mathbb{P}}(){}{
%   \renewcommand\given{\nonscript\:\delimsize\vert\nonscript\:\mathopen{}}
%   #1}
% \end{verbatim}
% Thus giving support for \verb|\Prob{A \given B}|.
% 
% \medskip\noindent\textbf{Note 1:} As the number of arguments increase
% the \cs{DeclarePairedDelimiter...} macros become hard for users to
% understand. A key-value interface would be better. This is planed
% for a future release. In
% \url{http://tex.stackexchange.com/a/136767/3929} there is a
% suggested replacement for \cs{DeclarePairedDelimiter}, that greatly
% reduces the number of macros and provides a key-val
% interface. However, the code use \pkg{xparse}, and if we want to use
% \pkg{xparse} for some of our macros, we might just as well rewrite
% the entire \pkg{mathtools} package in \pkg{expl3}. Also it is not
% obvious how to get \pkg{xparse} to support \verb|[3]| for the number
% of arguments. We will consider this for a future release.
%
% \medskip\noindent\textbf{Note 2:} If you want to define your own
% manual scaler macros, it is important that you besides \cs{foo} also
% defines \cs{fool} and \cs{foor}. When a scaler is specified, in say
% \cs{abs[\cs{big}]}\marg{arg}, we actually use \cs{bigl} and
% \cs{bigr}.
%
%
%  \subsubsection{Expert use}
%
%  Within the starred version of \cs{DeclarePairedDelimiter} and
%  \cs{DeclarePairedDelimiterX} we make a few changes such that the
%  auto scaled \cs{left} and \cs{right} fences behave as opening and
%  closing fences, i.e.\ $\sin(x)$ vs. $\sin\left(x\right)$ (the later
%  made via \verb|$\sin\left(x\right)$|), notice the gap between
%  '$\sin$' and '('.  In some special cases it may be useful to be
%  able to tinker with the behavior.
%  \begin{codesyntax}
%    \SpecialUsageIndex{\reDeclarePairedDelimiterInnerWrapper}
%    \cs{reDeclarePairedDelimiterInnerWrapper}\marg{macro name}\%\\
%    \qquad \marg{\textnormal{\texttt{star}} or \textnormal{\texttt{nostarnonscaled}} or \textnormal{\texttt{nostarscaled}}}\marg{code}
%  \end{codesyntax}
%  Internally several macros are created, including three call backs
%  that take care of the formatting. There is one internal macro for
%  the starred version, labeled \texttt{star}, the other two are
%  labeled \texttt{nostarnonscaled} and \texttt{nostarscaled}. Within
%  \meta{code}, \texttt{\#1} will be replaced by the (scaled) left
%  fence, \texttt{\#3} the corresponding (scaled) right fence, and
%  \texttt{\#2} the stuff in between. For example, here is how one
%  might turn the content into \cs{mathinner}:
% \begin{verbatim}
% \DeclarePairedDelimiter\abs\lvert\rvert
% \reDeclarePairedDelimiterInnerWrapper\abs{star}{#1#2#3}
% \reDeclarePairedDelimiterInnerWrapper\abs{nostarnonscaled}{\mathinner{#1#2#3}}
% \reDeclarePairedDelimiterInnerWrapper\abs{nostarscaled}{\mathinner{#1#2#3}}
% \end{verbatim}
%  The default values for the call backs corresponds to
% \begin{verbatim}
% star:            \mathopen{}\mathclose\bgroup #1#2\aftergroup\egroup #3
% nostarnonscaled: \mathopen#1#2\mathclose#3
% nostarscaled:    \mathopen{#1}#2\mathclose{#3}
% \end{verbatim}
%  The two \texttt{nostar...} versions look the same, but they are
%  not. In most (math) fonts, the first item in this list will be
%  different from the rest (the superscript sits
%  higher).\footnote{Interestingly it did not show up in the font of
%  this manual, which uses the \pkg{fourier} font set.}
% \begin{verbatim}
% \mathclose{\rvert}^2\mathclose\rvert^2\rvert^2
% \end{verbatim}
%
%
% \bigskip\noindent \textbf{Breaking change:} Prior to May 2017, we
% used two wrappers, the other named \texttt{nostar}. As of May 2017
% \texttt{nostar} has been split into \texttt{nostarnonscaled} and
% \texttt{nostarscaled} and \texttt{nostar} alone is no longer
% supported (will give an error).
%
%  \bigskip\noindent \textbf{Note:} Since we are using macros to add
%  the \verb|\left...\right| constructions around some body, it
%  \emph{is} (in principle) possible to make such a construction
%  breakable across lines, even breakable within an \env{align}
%  construction. Currently, this can only be applied to macros made
%  using \cs{DeclarePairedDelimiter} and \emph{not} macros made using
%  \cs{DeclarePairedDelimiterX} or \cs{DeclarePairedDelimiterXPP} as
%  the contents is typeset inside a group (to limit \cs{delimsize})
%  and thus hide any \verb|&| or \verb|\\| from \env{align} and
%  \env{align} breaks down.
%
%  When \cs{DeclarePairedDelimiter} is used, Sebastien Gouezel has
%  provided the following example \ProvidedBy{Sebastien
%  Gouezel}{2014/05/14}
% \begin{verbatim}
% \newcommand\MTkillspecial[1]{% helper macro
%  \bgroup
%  \catcode`\&=9
%  \let\\\relax%
%  \scantokens{#1}%
%  \egroup
% }
% \DeclarePairedDelimiter\abs\lvert\rvert
% \reDeclarePairedDelimiterInnerWrapper\abs{star}{
%   \mathopen{#1\vphantom{\MTkillspecial{#2}}\kern-\nulldelimiterspace\right.}
%   #2
%   \mathclose{\left.\kern-\nulldelimiterspace\vphantom{\MTkillspecial{#2}}#3}}
% \end{verbatim}
%  Then this example works just fine:
% \begin{verbatim}
% \begin{align*}
%   f(a) &=
%   \!\begin{aligned}[t]
%    \abs*{ & \frac{1}{2}+\cdots \\
%           & \dots+\frac{1}{2} }
%  \end{aligned}
% \end{align*}
% \end{verbatim}
% \newcommand\MTkillspecial[1]{% helper macro
%  \bgroup
%  \catcode`\&=9
%  \let\\\relax%
%  \scantokens{#1}%
%  \egroup
% }
% \let\abs\relax
% \MHInternalSyntaxOn
% \MH_let:NwN\MT_delim_abs_nostar:\relax 
% \MHInternalSyntaxOff
% \DeclarePairedDelimiter\abs\lvert\rvert
% \reDeclarePairedDelimiterInnerWrapper\abs{star}{
%   \mathopen{#1\vphantom{\MTkillspecial{#2}}\kern-\nulldelimiterspace\right.}
%   #2
%   \mathclose{\left.\kern-\nulldelimiterspace\vphantom{\MTkillspecial{#2}}#3}}
% \begin{align*}
%   f(a) &=
%   \!\begin{aligned}[t]
%    \abs*{&\frac{1}{2}+\cdots \\
%      &\dots+\frac{1}{2}}
%  \end{aligned}
% \end{align*}
%  
%
%  \subsection{Special symbols}
%
%  This part of the manual is about special symbols. So far only one
%  technique is covered, but more will come.
%
%  \subsubsection{Left and right parentheses}
%
%  \begin{codesyntax}
%    \SpecialUsageIndex{\lparen}\cs{lparen}\texttt{~~}
%    \SpecialUsageIndex{\rparen}\cs{rparen}
%  \end{codesyntax}
%  When you want a big parenthesis or bracket in a math display you
%  usually just type
%  \begin{quote}
%    |\left( ... \right)|\quad  or\quad |\left[ ... \right]|
%  \end{quote}
%  \LaTeX{} also defines the macro names \cs{lbrack} and \cs{rbrack}
%  to be shorthands for the left and right square bracket resp., but
%  doesn't provide similar definitions for the parentheses. Some
%  packages need command names to work with\footnote{The \pkg{empheq}
%  package needs command names for delimiters in order to make
%  auto-scaling versions.} so \pkg{mathtools} defines the commands
%  \cs{lparen} and \cs{rparen} to represent the left and right
%  parenthesis resp.
%
%
%  \subsubsection{Vertically centered colon}
%
%  \begin{codesyntax}
%    \SpecialKeyIndex{centercolon}$\key{centercolon}=\texttt{true}\vert\texttt{false}$\\
%    \SpecialUsageIndex{\vcentcolon}\cs{vcentcolon}\texttt{~~}
%    \SpecialUsageIndex{\ordinarycolon}\cs{ordinarycolon}
%  \end{codesyntax}
%  \cttPosting{Donald Arseneau}{2000/12/07}
%  When trying to show assignment operations as in $ a := b $, one
%  quickly notices that the colon is not centered on the math axis as
%  the equal sign, leading to an odd-looking output. The command
%  \cs{vcentcolon} is a shorthand for such a vertically centered
%  colon, and can be used as in |$a \vcentcolon= b$| and results in
%  the desired output:  $a \vcentcolon= b$. % for now
%  See also the \pkg{colonequals} package.
%
%  Typing \cs{vcentcolon} every time is quite tedious, so one can use
%  the key \key{centercolon} to make the colon active instead.
%  \begin{verbatim}
%  \mathtoolsset{centercolon}
%  \[
%    a := b
%  \]
%  \mathtoolsset{centercolon=false}
%  \end{verbatim}
%  \[\mathtoolsset{centercolon}
%    a := b
%  \]
%  In this case the command \cs{ordinarycolon} typesets an~\ldots\
%  ordinary colon (what a surprise).
%
%  \medskip
%  \noindent\textbf{Warning:} \texttt{centercolon} \emph{does not}
%  work with languages that make use of an active colon, most notably
%  \emph{French}. Sadly the \texttt{babel} package does not distinguish
%  between text and math when it comes to active characters. Nor does
%  it provide any hooks to deal with math. So currently no general
%  solution exists for this problem.
%
%  \begin{codesyntax}
%    \SpecialUsageIndex{\coloneqq}\cs{coloneqq}\texttt{~~~~~}
%    \SpecialUsageIndex{\Coloneqq}\cs{Coloneqq}\texttt{~~~~~}
%    \SpecialUsageIndex{\coloneq}\cs{coloneq}\texttt{~~~}
%    \SpecialUsageIndex{\Coloneq}\cs{Coloneq}\\
%    \SpecialUsageIndex{\eqqcolon}\cs{eqqcolon}\texttt{~~~~~}
%    \SpecialUsageIndex{\Eqqcolon}\cs{Eqqcolon}\texttt{~~~~~}
%    \SpecialUsageIndex{\eqcolon}\cs{eqcolon}\texttt{~~~}
%    \SpecialUsageIndex{\Eqcolon}\cs{Eqcolon}\\
%    \SpecialUsageIndex{\colonapprox}\cs{colonapprox}\texttt{~~}
%    \SpecialUsageIndex{\Colonapprox}\cs{Colonapprox}\texttt{~~}
%    \SpecialUsageIndex{\colonsim}\cs{colonsim}\texttt{~~}
%    \SpecialUsageIndex{\Colonsim}\cs{Colonsim}\\
%    \SpecialUsageIndex{\dblcolon}\cs{dblcolon}
%  \end{codesyntax}
%  The font packages \pkg{txfonts} and \pkg{pxfonts} provides various
%  symbols that include a vertically centered colon but with tighter
%  spacing. For example, the combination |:=| exists as the symbol
%  \cs{coloneqq} which typesets as $\coloneqq$ instead of
%  $\vcentcolon=$. The primary disadvantage of using these fonts are
%  the support packages' lack of support for \pkg{amsmath} (and thus
%  \pkg{mathtools}) and worse yet, the side-bearings are way too
%  tight; see~\cite{A-W:MG04} for examples. If you're not using these
%  fonts, \pkg{mathtools} provides the symbols for you. Here are a few
%  examples:
%  \begin{verbatim}
%  \[
%    a \coloneqq b \quad c \Colonapprox d \quad e \dblcolon f
%  \]
%  \end{verbatim}
%  \[
%    a \coloneqq b \quad c \Colonapprox d \quad e \dblcolon f
%  \]
%
%
%
%  \subsubsection{A few missing symbols}
%
%  Most provided math font sets are missing the symbols \cs{nuparrow}
%  and \cs{ndownarrow} (i.e.\ negated up- and downarrow) plus a `big'
%  version of \cs{times}. Therefore we will provide constructed
%  versions of these whenever they are not already available.
%  \begin{codesyntax}
%    \SpecialUsageIndex{\nuparrow}\cs{nuparrow}\\
%    \SpecialUsageIndex{\ndownarrow}\cs{ndownarrow}\\
%    \SpecialUsageIndex{\bigtimes}\cs{bigtimes}
%  \end{codesyntax}
%
%  \noindent
%  \textbf{Note:} that these symbols are constructed via
%  features from the \pkg{graphicx} package, and thus may not display
%  correctly in most DVI previewers. Also note that \cs{nuparrow} and
%  \cs{ndownarrow} are constructed via \cs{nrightarrow} and
%  \cs{nleftarrrow} respectively, so these needs to be
%  present. Usually this is done via \pkg{amssymb}, but some packages
%  may be incompatible with \pkg{amssymb} so the user will have to
%  load \pkg{amssymb} or a similar package, that provides
%  \cs{nrightarrow} and \cs{nleftarrow}, themselves. 
%
%  With those requirements in place, we have
%  \begin{verbatim}
%    \[
%       \lim_{a\ndownarrow 0} f(a) \neq \bigtimes_n X_n \qquad
%       \frac{ \bigtimes_{k=1}^7 B_k \nuparrow \Omega }{2}     
%    \]
%  \end{verbatim}
%    \[
%       \lim_{a\ndownarrow 0} f(a) \neq \bigtimes_n X_n \qquad
%       \frac{ \bigtimes_{k=1}^7 B_k \nuparrow \Omega }{2}     
%    \]
%
%
%
%
%  \section{A tribute to Michael J.~Downes}
%
%  Michael J.~Downes (1958--2003) was one of the major architects
%  behind \pkg{amsmath} and member of the \LaTeX{} Team. He made many
%  great contributions to the \TeX{} community; not only by the means
%  of widely spread macro packages such as \pkg{amsmath} but also in
%  the form of actively giving advice on newsgroups. Some of
%  Michael's macro solutions on the newsgroups never made it into
%  publicly available macro packages although they certainly deserved
%  it, so \pkg{mathtools} tries to rectify this matter. The macros
%  described in this section are either straight copies or heavily
%  inspired by his original postings.
%
%  \subsection{Mathematics within italic text}
%
%  \begin{codesyntax}
%    \SpecialKeyIndex{mathic}$\key{mathic}=\texttt{true}\vert\texttt{false}$
%  \end{codesyntax}
%  \cttPosting{Michael J.~Downes}{1998/05/14}
%  \TeX{} usually takes care of italic corrections in text, but fails
%  when it comes to math. If you use the \LaTeX{} inline math
%  commands \cs{(} and \cs{)} you can however work around it by
%  setting the key \key{mathic} to true as shown below.
%  \begin{verbatim}
%    \begin{quote}\itshape
%    Compare these lines: \par
%    \mathtoolsset{mathic} % or \mathtoolsset{mathic=true}
%    Subset of \(V\) and subset of \(A\). \par
%    \mathtoolsset{mathic=false}
%    Subset of \(V\) and subset of \(A\).
%    \par
%    \end{quote}
%  \end{verbatim}
%  \begin{quote}\itshape
%  Compare these lines: \par
%  \mathtoolsset{mathic}
%  Subset of \(V\) and subset of \(A\). \par
%  \mathtoolsset{mathic=false}
%  Subset of \(V\) and subset of \(A\).
%  \par
%  \end{quote}
%  \noindent
%  As of 2013, \cs(\cs) are robust, as is the italic corrected versions.
%
%  \medskip\noindent \textbf{Caveat:} Italic correction is a
%  treacherous area. For example any penalties will cancel the italic
%  correction inserted by
%  \verb|\(| (for an explanation see \cite{TBT}, section 4.3.3). We
%  have changed Michaels original to accommodate one specific penalty
%  construction: the \emph{tie}, i.e.,
%  >>\verb|text~\(|<< will work as expected (as~of July, 2014).
%
%  \subsection{Left sub/superscripts}
%
%  \begin{codesyntax}
%    \SpecialUsageIndex{\prescript}
%        \cs{prescript}\marg{sup}\marg{sub}\marg{arg}\texttt{~~}
%    \SpecialKeyIndex{prescript-sup-format}
%        $\key{prescript-sup-format}=\meta{cmd}$\\
%    \SpecialKeyIndex{prescript-sub-format}
%        $\key{prescript-sub-format}=\meta{cmd}$\hfill
%    \SpecialKeyIndex{prescript-arg-format}
%        \rlap{$\key{prescript-arg-format}=\meta{cmd}$}^^A
%        \phantom{$\key{prescript-sup-format}=\meta{cmd}$}
%  \end{codesyntax}
%  \cttPosting{Michael J.~Downes}{2000/12/20}
%  Sometimes one wants to put a sub- or superscript on the left of
%  the argument. The \cs{prescript} command does just that:
%  \begin{verbatim}
%    \[
%      {}^{4}_{12}\mathbf{C}^{5+}_{2}          \quad
%      \prescript{14}{2}{\mathbf{C}}^{5+}_{2}  \quad
%      \prescript{4}{12}{\mathbf{C}}^{5+}_{2}  \quad
%      \prescript{14}{}{\mathbf{C}}^{5+}_{2}   \quad
%      \prescript{}{2}{\mathbf{C}}^{5+}_{2}
%    \]
%  \end{verbatim}
%  \[
%    {}^{4}_{12}\mathbf{C}^{5+}_{2}          \quad
%    \prescript{14}{2}{\mathbf{C}}^{5+}_{2}  \quad
%    \prescript{4}{12}{\mathbf{C}}^{5+}_{2}  \quad
%    \prescript{14}{}{\mathbf{C}}^{5+}_{2}   \quad
%    \prescript{}{2}{\mathbf{C}}^{5+}_{2}
%  \]
%
%  The formatting of the arguments is controlled by three keys. This
%  silly example shows you how to use them:
%  \begin{verbatim}
%  \newcommand*\myisotope[3]{%
%    \begingroup % to keep changes local. We cannot use a brace group
%                % as it affects spacing!
%      \mathtoolsset{
%        prescript-sup-format=\mathit,
%        prescript-sub-format=\mathbf,
%        prescript-arg-format=\mathrm,
%      }%
%    \prescript{#1}{#2}{#3}%
%    \endgroup
%  }
%  \[
%    \myisotope{A}{Z}{X}\to \myisotope{A-4}{Z-2}{Y}+
%    \myisotope{4}{2}{\alpha}
%  \]
%  \end{verbatim}
%  \newcommand*\myisotope[3]{%
%    \begingroup
%      \mathtoolsset{
%        prescript-sup-format=\mathit,
%        prescript-sub-format=\mathbf,
%        prescript-arg-format=\mathrm,
%      }%
%    \prescript{#1}{#2}{#3}%
%    \endgroup
%  }
%  \[
%    \myisotope{A}{Z}{X}\to \myisotope{A-4}{Z-2}{Y}+
%    \myisotope{4}{2}{\alpha}
%  \]
% (Though a package like \pkg{mhchem} might be more suitable for this
% type of material.)
%
%
%  \subsection{Declaring math sizes}
%
%  \begin{codesyntax}
%    \SpecialUsageIndex{\DeclareMathSizes}
%    \cs{DeclareMathSizes}\marg{dimen}\marg{dimen}\marg{dimen}\marg{dimen}
%  \end{codesyntax}
%  \cttPosting{Michael J.~Downes}{2002/10/17}
%  If you don't know about \cs{DeclareMathSizes}, then skip the rest
%  of this text. If you do know, then all that is needed to say is
%  that with \pkg{mathtools} it is patched so that all regular
%  dimension suffixes are now valid in the last three arguments. Thus
%  a declaration such as
%  \begin{verbatim}
%    \DeclareMathSize{9.5dd}{9.5dd}{7.5dd}{6.5dd}
%  \end{verbatim}
%  will now work (it doesn't in standard \LaTeX). When this bug has
%  been fixed in \LaTeX, this fix will be removed from
%  \pkg{mathtools}.
%
%  The \CommentAdded{2015/11/12} fix was added to the \LaTeX{} kernel
%  in 2015. We will continue to provide it on older kernels.
%
%  \subsection{Spreading equations}\label{sec:spread}
%
%  \begin{codesyntax}
%    \SpecialEnvIndex{spreadlines}
%    \cs{begin}\arg{spreadlines}\marg{dimen} \meta{contents}
%    \cs{end}\arg{spreadlines}
%  \end{codesyntax}
%  \cttPosting{Michael J.~Downes}{2002/10/17}
%  The spacing between lines in a multiline math environment such as
%  \env{gather} is governed by the dimension \cs{jot}. The
%  \env{spreadlines} environment takes one argument denoting the
%  value of \cs{jot} inside the environment:
%  \begin{verbatim}
%    \begin{spreadlines}{20pt}
%    Large spaces between the lines.
%    \begin{gather}
%      a=b\\
%      c=d
%    \end{gather}
%    \end{spreadlines}
%    Back to normal spacing.
%    \begin{gather}
%      a=b\\
%      c=d
%    \end{gather}
%  \end{verbatim}
%    \begin{spreadlines}{20pt}
%    Large spaces between the lines.
%    \begin{gather}
%      a=b\\
%      c=d
%    \end{gather}
%    \end{spreadlines}
%    Back to normal spacing.
%    \begin{gather}
%      a=b\\
%      c=d
%    \end{gather}
%
%
%  \subsection{Gathered environments}\label{subsec:gathered}
%
%  \begin{codesyntax}
%    \SpecialEnvIndex{lgathered}\cs{begin}\arg{lgathered}\oarg{pos}
%    \meta{contents}  \cs{end}\arg{lgathered} \\
%    \SpecialEnvIndex{rgathered}\cs{begin}\arg{rgathered}\oarg{pos}
%    \meta{contents}  \cs{end}\arg{rgathered} \\
%    \SpecialUsageIndex{\newgathered}\cs{newgathered}\marg{name}\marg{pre_line}\marg{post_line}\marg{after}\\
%    \SpecialUsageIndex{\renewgathered}\cs{renewgathered}\marg{name}\marg{pre_line}\marg{post_line}\marg{after}
%  \end{codesyntax}
%  \cttPosting{Michael J.~Downes}{2001/01/17}
%  In a document set in \opt{fleqn}, you might sometimes want an
%  inner \env{gathered} environment that doesn't center its lines but
%  puts them flush left. The \env{lgathered} environment works just
%  like the standard \env{gathered} except that it flushes its
%  contents left:
%  \begin{verbatim}
%    \begin{equation}
%      \begin{lgathered}
%        x=1,\quad x+1=2 \\
%        y=2
%      \end{lgathered}
%    \end{equation}
%  \end{verbatim}
%  \begin{equation}
%    \begin{lgathered}
%        x=1,\quad x+1=2 \\
%        y=2
%    \end{lgathered}
%  \end{equation}
%  Similarly the \env{rgathered} puts it contents flush right.
%
%  More interesting is probably the command \cs{newgathered}. In this
%  example we define a gathered version that centers the lines and
%  also prints a star and a number at the left of each line.
%  \begin{verbatim}
%    \newcounter{steplinecnt}
%    \newcommand\stepline{\stepcounter{steplinecnt}\thesteplinecnt}
%    \newgathered{stargathered}
%                {\llap{\stepline}$*$\quad\hfil}% \hfil for centering
%                {\hfil}%                         \hfil for centering
%                {\setcounter{steplinecnt}{0}}%   reset counter
%  \end{verbatim}
%  \newcounter{steplinecnt}
%  \newcommand\stepline{\stepcounter{steplinecnt}\thesteplinecnt}
%  \newgathered{stargathered}{\llap{\stepline}$*$\quad\hfil}{\hfil}{\setcounter{steplinecnt}{0}}
%  With these definitions we can get something like this:
%  \begin{verbatim}
%    \begin{gather}
%      \begin{stargathered}
%        x=1,\quad x+1=2 \\
%        y=2
%      \end{stargathered}
%    \end{gather}
%  \end{verbatim}
%  \begin{gather}
%    \begin{stargathered}
%      x=1,\quad x+1=2 \\
%      y=2
%    \end{stargathered}
%  \end{gather}
%  \cs{renewgathered} renews a gathered environment of course.
%
%  In all fairness it should be stated that the original concept by
%  Michael has been extended quite a bit in \pkg{mathtools}. Only the
%  end product of \env{lgathered} is the same.
%
%  \subsection{Split fractions}
%
%  \begin{codesyntax}
%    \SpecialUsageIndex{\splitfrac}\cs{splitfrac}\marg{start line}\marg{continuation}\\
%    \SpecialUsageIndex{\splitdfrac}\cs{splitdfrac}\marg{start line}\marg{continuation}
%  \end{codesyntax}
%  \cttPosting{Michael J.~Downes}{2001/12/06}
%  These commands provide split fractions e.g., multiline fractions:
%  \begin{verbatim}
%    \[
%      a=\frac{
%          \splitfrac{xy + xy + xy + xy + xy}
%                    {+ xy + xy + xy + xy}
%        }
%        {z}
%      =\frac{
%          \splitdfrac{xy + xy + xy + xy + xy}
%                    {+ xy + xy + xy + xy}
%        }
%        {z}
%  \]
%  \end{verbatim}
%  \[
%  a=\frac{
%      \splitfrac{xy + xy + xy + xy + xy}
%                {+ xy + xy + xy + xy}
%    }
%    {z}
%  =\frac{
%      \splitdfrac{xy + xy + xy + xy + xy}
%                {+ xy + xy + xy + xy}
%    }
%    {z}
%  \]
% The difference between \cs{splitfrac} and \cs{splitdfrac} is that
% the former forces its arguments to be typeset in text-mode math, the
% latter does not.
%
% \medskip\noindent \textbf{Note:} If you try to nest \cs{splitfrac}
% inside each other you may need to add \cs{mathstrut} to the first
% argument of the nested fraction to get the spacing look even. It is
% not added by default as often a more cramped looked is desired:
% \begin{verbatim}
%  \[
%  \frac{
%      \splitfrac{xy + xy + xy + xy + xy}
%      {
%      \splitfrac{xy + xy + xy + xy + xy}
%                {+ xy + xy + xy + xy}
%      }
%    }
%    {z}
%  =\frac{
%      \splitfrac{xy + xy + xy + xy + xy}
%      {
%      \splitfrac{\mathstrut xy + xy + xy + xy + xy}
%                {+ xy + xy + xy + xy}
%      }
%    }
%    {z}
%  \]
% \end{verbatim}
%  \[
%  \frac{
%      \splitfrac{xy + xy + xy + xy + xy}
%      {
%      \splitfrac{xy + xy + xy + xy + xy}
%                {+ xy + xy + xy + xy}
%      }
%    }
%    {z}
%  =\frac{
%      \splitfrac{xy + xy + xy + xy + xy}
%      {
%      \splitfrac{\mathstrut  xy + xy + xy + xy + xy}
%                {+ xy + xy + xy + xy}
%      }
%    }
%    {z}
%  \]
% Note how the line spaces aren't even on the left hand side.
%
% For even more control, one can use \env{aligned} or \env{gathered}
% instead of \cs{splitfrac}.
%
%
%  \newpage
%
% \section{New additions}
%
% \subsection{Variable math strut}
%  \begin{codesyntax}
%    \SpecialUsageIndex{\xmathstrut}\cs{xmathstrut}\oarg{depth increase}\marg{increase}
%  \end{codesyntax}
%  \FeatureRequest{Frank Mittelbach}{2020} In typography we use
%  \emph{struts} to ensure specific line spacing. In text we have the
%  \cs{strut} and in math \cs{mathstrut}. Both have no width, but
%  equals the height and depth of an »(« from the current text/math
%  font and size.  In math we often need to make minute adjustments in
%  macro definitiones etc. The \emph{extended} math strut
%  \cs{xmathstrut} allows to \emph{increase} (decrease if negative)
%  the math strut in two ways.
%
%  For an \meta{increase} (a decimal number), say~\verb|0.1|,
%  \begin{center}
%      \verb|\xmathstrut{0.1}| 
%  \end{center}
%  will give you a new strut where 10\% of the \emph{total height of
%  the normal math stut} (\verb|\mathstrut=\xmathstrut{0}|) will be
%  added to both \emph{the height} and \emph{the depth} of the
%  original strut (thus 20\% gets added in total). On the other hand
%  \begin{center}
%      \verb|\xmathstrut[0.2]{0.1}| 
%  \end{center}
%  will result in a strut where 20\% is added to the depth and 10\% is
%  added to the height, resulting in a strut that is 30\% larger than
%  normal. 
%
%  The following example is inspired (with permission) from an example
%  showcasing \cs{xmathstrut} in the upcoming third edition of
%  \emph{The LaTeX Companion}. The example is very relevant as the
%  entries of the \env{cases(*)} environment are typeset in \emph{text
%  style} math and thus may end up looking quite
%  squished.\footnote{Which is why Frank suggested the macro
%  (including its implementation) in the first place.}
%  \[
%    \begin{cases*}
%      \frac{\frac{ x-1 }{ x-\sin x} }{ \sqrt{ 1 -x }} & $x >0$ \\
%      0 & otherwise
%    \end{cases*}
%  \qquad\text{vs.}\qquad
%    \begin{cases*}
%      \frac{\frac{ \xmathstrut{0.1} x-1 }{ \xmathstrut{0.25} x-\sin x} }{\xmathstrut{0.4} \sqrt{ 1 -x }} & $x >0$ \\
%      0 & otherwise
%    \end{cases*}
%  \]
% \begin{verbatim}
%  \[ \begin{cases*}
%      \frac{ \frac{ x-1 }{ x-\sin x } }{ \sqrt{ 1-x } } & $x>0$ \\
%      0 & otherwise
%    \end{cases*}
%  \qquad\text{vs.}\qquad
%    \begin{cases*}
%      \frac{ \frac{ \xmathstrut{0.1} x-1 }
%                  { \xmathstrut{0.25} x-\sin x } }
%           {\xmathstrut{0.4} \sqrt{ 1-x } }             & $x>0$ \\
%      0 & otherwise
%    \end{cases*} \]
% \end{verbatim}
% 
% To showcase the optional argument, it is
% probably easiest to make the strut visible. Here you'll see that
% \verb|\mathstrut| and \verb|\xmathstrut{0}| is the same:
%  \newcommand\vfb[1]{\begingroup\fboxsep=0pt\boxed{\,#1\,}\endgroup}
% \[
%   \llap{\rlap{\rule{18mm}{0.1pt}}\quad}a \vfb{\mathstrut} \ \vfb{\xmathstrut{0}}\
%   \vfb{\xmathstrut{0.5} } \  \vfb{\xmathstrut{-0.1} }\  \vfb{\xmathstrut[0.5]{0}} a
% \]
% \begin{verbatim}
% \newcommand\vfb[1]{\begingroup\fboxsep=0pt\boxed{\,#1\,}\endgroup}
% \[
% a
%    \vfb{ \mathstrut }          \   % normal strut
%    \vfb{ \xmathstrut{0} }      \   % just 0 => normal strut
%    \vfb{ \xmathstrut{0.5} }    \   % twice as large 50% + 50%
%    \vfb{ \xmathstrut{-0.1} }   \   % negative gives something smaller
%    \vfb{ \xmathstrut[0.5]{0} }     % change only the depth
% a
% \]
% \end{verbatim}
% -- the last box showcases a strut where we have only changed the
% depth of the strut. One can see \verb|\xmathstrut[0.5]{0}| kind of the
% opposite of \verb|\smash[b]{...}|, the former makes the depth larger
% and the latter ignores the depth.
%
%
%
%
%
%
%
%
%
%
%
%
%
%
%
%
%
%
%
%
%
%
%
%
%
%
% 
%
%
%
%
%
%
%
%
%
%
%
%
%
%
%
%
%
%
%
%
%
%
%
%
%
%
%
%
%
%
%
%
%
%
%
%
%
%
%
%
%
%
%
%
%
%
%
%
%
%
%
%
%
%
%
%
%
%
%
%
%
%
%
%
%
%
%
%
%
%
%
%
%
%
%
%
%
%
% \begin{thebibliography}{9}
% \bibitem{Perlis01} Alexander R. Perlis, \emph{A complement to
%   \cs{smash}, \cs{llap}, and \cs{rlap}}, TUGboat 22(4)
%   (2001). Available at
%   \url{https://www.tug.org/TUGboat/tb22-4/tb72perlS.pdf}.
% \bibitem{Ams99} American Mathematical Society and Michael Downes,
%   \emph{Technical notes on the \pkg{amsmath} package}. Version 2.0,
%   1999/10/29. Available via
%   \url{http://mirrors.ctan.org/macros/latex/required/amsmath/technote.pdf}.
% \bibitem{Ams00} Frank Mittelbach, Rainer Sch\"opf, Michael Downes,
%   David M.~Jones and David Carlisle, \emph{The \pkg{amsmath}
%   package}. Version 2.17e, 2020/01/20. Maintained by the \LaTeX3
%   project.  Available as file
%   \url{http://mirrors.ctan.org/macros/latex/required/amsmath/amsmath.dtx}.
%  \bibitem{A-W:MG04}
%    Frank Mittelbach and Michel Goossens.
%     \emph{The {\LaTeX} Companion}.
%    Tools and Techniques for Computer Typesetting. Addison-Wesley,
%    Boston, Massachusetts, 2 edition, 2004.
%    With Johannes Braams, David Carlisle, and Chris Rowley.
%
%  \bibitem{Carl99} David Carlisle, \emph{The \pkg{keyval} Package}.
%    Version 1.15, 2014/10/28.  Available via
%    \url{https://ctan.org/pkg/keyval}.
%
%  \bibitem{Voss:2004} Herbert Vo\ss, \emph{Math mode}. Version 2.47,
%    2014/01/30.  Available as
%    \url{http://mirrors.ctan.org/obsolete/info/math/voss/mathmode/Mathmode.pdf}. Please
%    note that the author has marked the document as \emph{obsolete}.
%  
%   \bibitem{Swanson} 
%     Ellen Swanson,
%     \emph{Mathematics into type}.
%     American Mathematical Society, updated edition, 1999.
%     Updated by Arlene O'Sean and Antoinette Schleyer. Available from
%     the AMS at \url{https://www.ams.org/arc/styleguide/mit-2.pdf}.
%
%   \bibitem{TBT} Victor Eijkhout, \emph{\TeX\ by Topic, A Texnician's
%     Reference}, 2007.  The book is available at
%     \url{https://ctan.org/pkg/texbytopic}.
%  \end{thebibliography}
%
%
%  \StopEventually{}  
%
%
%  \section{Options and package loading}
%
%
%  Lets start the package.
%    \begin{macrocode}
%<*package>
\ProvidesPackage{mathtools}%
  [2021/03/18 v1.25 mathematical typesetting tools]
%    \end{macrocode}
% \changes{v1.10}{2011/02/12}{Might as well make sure that we need the
% latest version of \texttt{mhsetup}}
%    \begin{macrocode}
\RequirePackage{keyval,calc}
\RequirePackage{mhsetup}[2021/03/18]
\MHInternalSyntaxOn
%    \end{macrocode}
% \changes{v1.13}{2013/02/11}{Robustifying \cs{(}\cs{)}\cs{[}\cs{]}}
% We'd like to make \cs{(}\cs{)} and \cs{[}\cs{]} robust. This can
% easily be done via the \pkg{fixltx2e} package, but auto loading that
% package may cause problems for some. We make copies instead.
% \begin{macro}{\EQ_MakeRobust}
% \begin{macro}{\forced_EQ_MakeRobust}
%    \begin{macrocode}
 % borrowed from fixltx2e
\def\EQ_MakeRobust#1{%
  \@ifundefined{\expandafter\@gobble\string#1}{%
    \@latex@error{The control sequence `\string#1' is undefined!%
      \MessageBreak There is nothing here to make robust}%
    \@eha
  }%
  {%
    \@ifundefined{\expandafter\@gobble\string#1\space}%
    {%
      \expandafter\let\csname
      \expandafter\@gobble\string#1\space\endcsname=#1%
      \edef\reserved@a{\string#1}%
      \def\reserved@b{#1}%
      \edef\reserved@b{\expandafter\strip@prefix\meaning\reserved@b}%
      \edef#1{%
        \ifx\reserved@a\reserved@b
          \noexpand\x@protect\noexpand#1%
        \fi
        \noexpand\protect\expandafter\noexpand
        \csname\expandafter\@gobble\string#1\space\endcsname}%
    }%
    {\@latex@info{The control sequence `\string#1' is already robust}}%
   }%
}
%    \end{macrocode}
% We also provide a handy alternativ version, that will robustify no
% matter what. Useful for the \texttt{mathic} option.
%    \begin{macrocode}
\def\forced_EQ_MakeRobust#1{%
  \@ifundefined{\expandafter\@gobble\string#1}{%
    \@latex@error{The control sequence `\string#1' is undefined!%
      \MessageBreak There is nothing here to make robust}%
    \@eha
  }%
  {%
    % \@ifundefined{\expandafter\@gobble\string#1\space}%
    % {%
      \expandafter\let\csname
      \expandafter\@gobble\string#1\space\endcsname=#1%
      \edef\reserved@a{\string#1}%
      \def\reserved@b{#1}%
      \edef\reserved@b{\expandafter\strip@prefix\meaning\reserved@b}%
      \edef#1{%
        \ifx\reserved@a\reserved@b
          \noexpand\x@protect\noexpand#1%
        \fi
        \noexpand\protect\expandafter\noexpand
        \csname\expandafter\@gobble\string#1\space\endcsname}%
 %    }%
 %    {\@latex@info{The control sequence `\string#1' is already robust}}%
   }%
}
%    \end{macrocode}
% \end{macro}
% \end{macro}
% 
% \begin{macro}{\MT_options_name:}
%  \begin{macro}{\mathtoolsset}
%  The name for the options and a user interface for setting keys.
%    \begin{macrocode}
\def\MT_options_name:{mathtools}
\newcommand*\mathtoolsset[1]{\setkeys{\MT_options_name:}{#1}}
%    \end{macrocode}
%  \end{macro}
%  \end{macro}
%
%  Fix \pkg{amsmath} bugs (strongly recommended!). It requires a
%  great deal of typing to avoid fixing the bugs. He he.
%    \begin{macrocode}
\MH_new_boolean:n {fixamsmath}
\DeclareOption{fixamsmath}{
  \MH_set_boolean_T:n {fixamsmath}
}
\DeclareOption{donotfixamsmathbugs}{
  \MH_set_boolean_F:n {fixamsmath}
}
%    \end{macrocode}
%  Disallow spaces before optional arguments in certain \pkg{amsmath}
%  building blocks.
%    \begin{macrocode}
\DeclareOption{allowspaces}{
  \MH_let:NwN \MaybeMHPrecedingSpacesOff
              \relax
    \MH_let:NwN \MH_maybe_nospace_ifnextchar:Nnn \kernel@ifnextchar
}
\DeclareOption{disallowspaces}{
  \MH_let:NwN \MaybeMHPrecedingSpacesOff
              \MHPrecedingSpacesOff
  \MH_let:NwN \MH_maybe_nospace_ifnextchar:Nnn \MH_nospace_ifnextchar:Nnn
}
%    \end{macrocode}
% \changes{v1.13}{2013/02/11}{Option to disable robustifying
% \cs{(}\cs{)} and \cs{[}\cs{]}}
% Option to disallow robustifying \cs{(}\cs{)} and \cs{[}\cs{]}.
%    \begin{macrocode}
\MH_new_boolean:n {robustify}
\MH_set_boolean_T:n {robustify} 
\DeclareOption{nonrobust}{
  \MH_set_boolean_F:n {robustify} 
}
%    \end{macrocode}
%  Pass all other options directly to \pkg{amsmath}.
%    \begin{macrocode}
\DeclareOption*{
  \PassOptionsToPackage{\CurrentOption}{amsmath}
}
%    \end{macrocode}
% Executing options
%    \begin{macrocode}
\ExecuteOptions{fixamsmath,disallowspaces}
\ProcessOptions\relax
%    \end{macrocode}
%  We have to turn off the new syntax when \pkg{amstext} is loaded.
%    \begin{macrocode}
\MHInternalSyntaxOff
\RequirePackage{amsmath}[2016/11/05]
\MHInternalSyntaxOn
\AtEndOfPackage{\MHInternalSyntaxOff}
%    \end{macrocode}
%  \begin{macro}{\MT_true_false_error:}
%  Make sure the user selects either `true' or `false' when asked too.
%    \begin{macrocode}
\def\MT_true_false_error:{
  \PackageError{mathtools}
    {You~ have~ to~ select~ either~ `true'~ or~ `false'}
    {I'll~ assume~ you~ chose~ `false'~ for~ now.}
}
%    \end{macrocode}
%  \end{macro}
% \changes{v1.13}{2013/02/11}{Make \cs{(}\cs{)} and \cs{[}\cs{]}
% robust}
% We start of with making \cs{(}\cs{)} and \cs{[}\cs{]} robust,
% unless the user explicitly asked us not to.
%    \begin{macrocode}
\MH_if_boolean:nT {robustify}{
  \EQ_MakeRobust\(
  \EQ_MakeRobust\)
  \EQ_MakeRobust\[
  \EQ_MakeRobust\]
}
%    \end{macrocode}
%
%
%  \section{Macros I got ideas for myself}
%
%
%
%  \subsection{Tag forms}
%  This is quite simple, but why isn't it then a part of some widely
%  distributed package? Beats me.
%
%  \begin{macro}{\MT_define_tagform:nwnn}
%  We start out by defining a command that will allow us to define
%  commands similar to \cs{tagform@} only this will give us tag form
%  \emph{types}. The actual code is very similar to the one in
%  \pkg{amsmath}.
%    \begin{macrocode}
\def\MT_define_tagform:nwnn #1[#2]#3#4{
  \@namedef{MT_tagform_#1:n}##1
    {\maketag@@@{#3\ignorespaces#2{##1}\unskip\@@italiccorr#4}}
}
%    \end{macrocode}
%  \end{macro}
%
%  \begin{macro}{\newtagform}
%  Similar to \cs{newcommand}. Check if defined and scan for presence
%  of optional argument. Then call generic command.
%    \begin{macrocode}
\providecommand*\newtagform[1]{%
  \@ifundefined{MT_tagform_#1:n}
  {\@ifnextchar[%
    {\MT_define_tagform:nwnn #1}%
    {\MT_define_tagform:nwnn #1[]}%
  }{\PackageError{mathtools}
  {The~ tag~ form~ `#1'~ is~ already~ defined\MessageBreak
  You~ probably~ want~ to~ look~ up~ \@backslashchar renewtagform~
  instead}
  {I~ will~ just~ ignore~ your~ wish~ for~ now.}}
}
%    \end{macrocode}
%  Provide a default tag form which---surprise, surprise---is
%  identical to the standard definition.
%    \begin{macrocode}
\newtagform{default}{(}{)}
%    \end{macrocode}
%  \end{macro}
%  \begin{macro}{\renewtagform}
%  Similar to \cs{renewcommand}.
%    \begin{macrocode}
\providecommand*\renewtagform[1]{%
  \@ifundefined{MT_tagform_#1:n}
  {\PackageError{mathtools}
  {The~ tag~ form~ `#1'~ is~ not~ defined\MessageBreak
  You~ probably~ want~ to~ look~ up~ \@backslashchar newtagform~ instead}
  {I~ will~ just~ ignore~ your~ wish~ for~ now.}}
  {\@ifnextchar[%
    {\MT_define_tagform:nwnn #1}%
    {\MT_define_tagform:nwnn #1[]}%
  }
}
%    \end{macrocode}
%  \end{macro}
%  \begin{macro}{\usetagform}
%  Then the activator. Test if the tag form is defined and then
%  activate it by redefining \cs{tagform@}.
%    \begin{macrocode}
\providecommand*\usetagform[1]{%
  \@ifundefined{MT_tagform_#1:n}
    {
      \PackageError{mathtools}{%
        You~ have~ chosen~ the~ tag~ form~ `#1'\MessageBreak
        but~ it~ appears~ to~ be~ undefined}
        {I~ will~ use~ the~ default~ tag~ form~ instead.}%
        \@namedef{tagform@}{\@nameuse{MT_tagform_default:n}}
      }
  { \@namedef{tagform@}{\@nameuse{MT_tagform_#1:n}} }
%    \end{macrocode}
%  Here we patch if we're using the special ``show only referenced
%  equations'' feature.
%    \begin{macrocode}
  \MH_if_boolean:nT {show_only_refs}{
    \MH_let:NwN \MT_prev_tagform:n \tagform@
    \def\tagform@##1{\MT_extended_tagform:n {##1}}
  }
}
%    \end{macrocode}
%  \end{macro}
%
%  \subsubsection{Showing only referenced tags}
%  A little more interesting is the way to print only the equation
%  numbers that are actually referenced.
%
%  A few booleans to help determine which situations we're in.
%    \begin{macrocode}
\MH_new_boolean:n {manual_tag}
\MH_new_boolean:n {raw_maketag}
%    \end{macrocode}
%  \begin{macro}{\MT_AmS_tag_in_align:}
%  \begin{macro}{\tag@in@align}
%  \begin{macro}{\tag@in@display}
%  We'll need to know when the user has put in a manual tag, and since
%  \cs{tag} is \cs{let} to all sorts of things inside the \pkg{amsmath}
%  code it is safer to provide a small hack to the functions it is copied
%  from. Note that we can't use \cs{iftag@}.
%    \begin{macrocode}
\MH_let:NwN \MT_AmS_tag_in_align: \tag@in@align
\def\tag@in@align{
  \global\MH_set_boolean_T:n {manual_tag}
  \MT_AmS_tag_in_align:
}
\def\tag@in@display#1#{
  \relax
  \global\MH_set_boolean_T:n {manual_tag}
  \tag@in@display@a{#1}
}
%    \end{macrocode}
%  \end{macro}
%  \end{macro}
%  \end{macro}
%
%  \begin{macro}{\MT_extended_tagform:n}
%  \changes{v1.01}{2004/08/03}{Simplified quite a bit}
%  The extended version of \cs{tagform@}.
%    \begin{macrocode}
\def\MT_extended_tagform:n #1{
  \MH_set_boolean_F:n {raw_maketag}
%    \end{macrocode}
% We test if the equation was labelled. We already know if it was
% tagged manually. Have to watch out for \TeX\ inserting a blank line
% so do not let the tag have width zero. Rememeber
% \cs{@safe@activestrue/false} in order to handle active chars in labels.
% \changes{v1.12}{2012/04/24}{Added \cs{@safe@activestrue/false}}
%  \changes{v1.19}{2017/03/31}{Added MH\_ prefix}
%    \begin{macrocode}
  \MH_if_meaning:NN \df@label\@empty
    \MH_if_boolean:nTF {manual_tag}% this was \MH_if_boolean:nT before
    { \MH_if_boolean:nTF {show_manual_tags}
      { \MT_prev_tagform:n {#1} }
      { \stepcounter{equation}  }
    }{\kern1sp}% this last {\kern1sp} is new.
  \MH_else:
    \MH_if_boolean:nTF {manual_tag}
      { \MH_if_boolean:nTF {show_manual_tags}
          { \MT_prev_tagform:n {#1} }
          { \@safe@activestrue
            \@ifundefined{MT_r_\df@label}
%    \end{macrocode}
% Next we need to remember to deactivate the manual tags switch. This
% is usually done using \verb|\MT_extended_maketag:n|, but this is not
% the case if the show manual tags is false and the manual tag is not
% referred to. 
% \changes{v1.12}{2011/06/08}{Added the falsification of manual tag
% when show manual tags is off and maual tag is not referred to}
%  \changes{v1.19}{2017/03/31}{Added MH\_ prefix}
% \changes{v1.23}{2020/01/17}{Added \cs{kern1sp} in a few places to
% make sure the tag has a non zero width.}
%    \begin{macrocode}
              { \global\MH_set_boolean_F:n {manual_tag} \kern1sp } % kern added 2020
              { \MT_prev_tagform:n {#1} }
              \@safe@activesfalse
          }
      }
      { 
        \@safe@activestrue
        \@ifundefined{MT_r_\df@label}
          { \kern1sp }% kern added 2020
          { \refstepcounter{equation}\MT_prev_tagform:n {#1} }
        \@safe@activesfalse
      }
  \MH_fi:
  \global\MH_set_boolean_T:n {raw_maketag}
}
%    \end{macrocode}
%  \end{macro}
%  \begin{macro}{\MT_extended_maketag:n}
%  The extended version of \cs{maketag@@@}.
% \changes{v1.12}{2012/04/24}{Added \cs{@safe@activestrue/false}}
%  \changes{v1.19}{2017/03/31}{Added MH\_ prefix}
%    \begin{macrocode}
\def\MT_extended_maketag:n #1{
  \ifx\df@label\@empty
    \MT_maketag:n {#1}
  \MH_else:
    \MH_if_boolean:nTF {raw_maketag}
      {
        \MH_if_boolean:nTF {show_manual_tags}
          { \MT_maketag:n {#1} }
          { \@safe@activestrue
            \@ifundefined{MT_r_\df@label}
              { }
              { \MT_maketag:n {#1}     }
            \@safe@activesfalse
          }
      }
      { \MT_maketag:n {#1} }
  \MH_fi:
%    \end{macrocode}
%  As this function is always called we let it set the marker for a manual
%  tag false when exiting (well actually not true, see above).
%    \begin{macrocode}
  \global\MH_set_boolean_F:n {manual_tag}
}
%    \end{macrocode}
%  \end{macro}
%  \begin{macro}{\MT_extended_eqref:n}
%  \changes{v1.01}{2004/08/03}{Make it robust}
%  We let \cs{eqref} write the label to the \file{aux} file, which is
%  read at the beginning of the next run. Then we print the equation
%  number as usual.
% \changes{v1.14}{2013/03/07}{It was not really robust}
%    \begin{macrocode}
\def\MT_extended_eqref:n #1{
  \protected@write\@auxout{}
    {\string\MT@newlabel{#1}}
%    \end{macrocode}
% \changes{v1.16}{2015/05/11}{Fix bug in \cs{eqref} in \cs{text} vs. \textsf{showonlyrefs}}
% The manner in which \key{showonlyrefs} is implemented causes
% problems when \cs{eqref} is used within \cs{text}. The problem is
% that \cs{MT_prev_tagform:n} is used both in \cs{eqref} and within,
% say, \env{align}. Thus a test within \cs{MT_prev_tagform:n} fails
% for \cs{eqref}. We fix this by making sure we end up in the proper
% branch of the test, when \cs{MT_prev_tagform:n} is used to typeset \cs{eqref}.
%    \begin{macrocode}
  \textup{\let\df@label\@empty\MT_prev_tagform:n {\ref{#1}}}
}
\EQ_MakeRobust\MT_extended_eqref:n
%    \end{macrocode}
%  \end{macro}
%
%  \begin{macro}{\refeq}
%  \begin{macro}{\MT_extended_refeq:n}
%  Similar to \cs{eqref} and \cs{MT_extended_eqref:n}.
%    \begin{macrocode}
\newcommand*\refeq[1]{
  \textup{\ref{#1}}
}
\def\MT_extended_refeq:n #1{
  \protected@write\@auxout{}
    {\string\MT@newlabel{#1}}
  \textup{\ref{#1}}
}
%    \end{macrocode}
%  \end{macro}
%  \end{macro}
%
%  \begin{macro}{\MT@newlabel}
%  We can't use |:| or |_| in the command name (yet). We define the
%  special labels for the equations that have been referenced in the
%  previous run.
%    \begin{macrocode}
\newcommand*\MT@newlabel[1]{  \global\@namedef{MT_r_#1}{}  }
%    \end{macrocode}
%  \end{macro}
%  \changes{v1.19}{2017/03/31}{Added MH\_ prefix}
%    \begin{macrocode}
\MH_new_boolean:n {show_only_refs}
\MH_new_boolean:n {show_manual_tags}
\define@key{\MT_options_name:}{showmanualtags}[true]{
  \@ifundefined{MH_boolean_show_manual_tags_#1:}
    { \MT_true_false_error:
      \@nameuse{MH_boolean_show_manual_tags_false:}
    }
    { \@nameuse{MH_boolean_show_manual_tags_#1:} }
}
%    \end{macrocode}
%  \begin{macro}{\MT_showonlyrefs_true:}
%  The implementation is based on the idea that \cs{tagform@} can be
%  called in two circumstances: when the tag is being printed in the
%  equation and when it is being printed during a reference.
%    \begin{macrocode}
\newcommand*\MT_showonlyrefs_true:{
  \MH_if_boolean:nF {show_only_refs}{
    \MH_set_boolean_T:n {show_only_refs}
%    \end{macrocode}
%  Save the definitions of the original commands.
%    \begin{macrocode}
    \MH_let:NwN \MT_incr_eqnum: \incr@eqnum
    \MH_let:NwN \incr@eqnum \@empty
    \MH_let:NwN \MT_array_parbox_restore: \@arrayparboxrestore
    \@xp\def\@xp\@arrayparboxrestore\@xp{\@arrayparboxrestore
      \MH_let:NwN \incr@eqnum \@empty
    }
    \MH_let:NwN \MT_prev_tagform:n \tagform@
    \MH_let:NwN \MT_eqref:n \eqref
    \MH_let:NwN \MT_refeq:n \refeq
    \MH_let:NwN \MT_maketag:n \maketag@@@
    \MH_let:NwN \maketag@@@ \MT_extended_maketag:n
%    \end{macrocode}
%  We redefine \cs{tagform@}.
%    \begin{macrocode}
    \def\tagform@##1{\MT_extended_tagform:n {##1}}
%    \end{macrocode}
%  Then \cs{eqref}:
%    \begin{macrocode}
    \MH_let:NwN \eqref \MT_extended_eqref:n
    \MH_let:NwN \refeq \MT_extended_refeq:n
  }
}
%    \end{macrocode}
%  \end{macro}
%  \begin{macro}{\MT_showonlyrefs_false:}
%  This macro reverts the settings.
%    \begin{macrocode}
\def\MT_showonlyrefs_false: {
  \MH_if_boolean:nT {show_only_refs}{
    \MH_set_boolean_F:n {show_only_refs}
    \MH_let:NwN \tagform@  \MT_prev_tagform:n
    \MH_let:NwN \eqref \MT_eqref:n
    \MH_let:NwN \refeq \MT_refeq:n
    \MH_let:NwN \maketag@@@ \MT_maketag:n
    \MH_let:NwN \incr@eqnum \MT_incr_eqnum:
    \MH_let:NwN \@arrayparboxrestore \MT_array_parbox_restore:
  }
}
\define@key{\MT_options_name:}{showonlyrefs}[true]{
  \@nameuse{MT_showonlyrefs_#1:}
}
%    \end{macrocode}
%  \end{macro}
%
%
%  \begin{macro}{\nonumber}
%  \changes{v1.01}{2004/08/03}{Fixed using \cs{notag} or \cs{nonumber}
%  with the \key{showonlyrefs} feature}
%  We have to redefine \cs{nonumber} else it will subtract one from the
%  equation number where we don't want it. This is probably not needed
%  since \cs{nonumber} is unnecessary when \key{showonlyrefs} is in
%  effect, but now you can use it with old documents as well.
%  \changes{v1.19}{2017/03/31}{Added MH\_ prefix}
%    \begin{macrocode}
\renewcommand\nonumber{
  \if@eqnsw
    \MH_if_meaning:NN \incr@eqnum\@empty
%    \end{macrocode}
%  Only subtract the number if |show_only_refs| is false.
%  \changes{v1.19}{2017/03/31}{Added MH\_ prefix}
%    \begin{macrocode}
      \MH_if_boolean:nF {show_only_refs}
        {\addtocounter{equation}\m@ne}
    \MH_fi:
  \MH_fi:
  \MH_let:NwN \print@eqnum\@empty \MH_let:NwN \incr@eqnum\@empty
  \global\@eqnswfalse
}
%    \end{macrocode}
%  \end{macro}
%
%  \begin{macro}{\noeqref}
%   \changes{v1.04}{2008/03/26}{Added \cs{noeqref} (daleif)}
%   \changes{v1.12}{2012/04/20}{Labels containing active chars (babel)
%   are now allowed}
%   \changes{v1.12}{2012/04/24}{\cs{noeqref} will now make a reference
%   warning if users use undefined labels in \cs{noeqref}, requested
%   by Tue Christensen}
%   Macro for adding numbers to non-referred equations. Syntax similar
%   to \cs{nocite}.
%    \begin{macrocode}
\MHInternalSyntaxOff
\newcommand\noeqref[1]{\@bsphack%
  \@for\@tempa:=#1\do{%
    \@safe@activestrue%
    \edef\@tempa{\expandafter\@firstofone\@tempa}%
    \@ifundefined{r@\@tempa}{%
      \protect\G@refundefinedtrue%
      \@latex@warning{Reference `\@tempa' on page \thepage \space
        undefined (\string\noeqref)}%
    }{}%
    \if@filesw\protected@write\@auxout{}%
    {\string\MT@newlabel{\@tempa}}\fi%
  \@safe@activesfalse}%
  \@esphack}

%    \end{macrocode}
%  \end{macro}
%    
% \begin{macro}{\@safe@activestrue}
% \begin{macro}{\@safe@activesfalse}
%   These macros are provided by babel. We \emph{provide} them here,
%   just to make sure they exist.
%    \begin{macrocode}
\providecommand\@safe@activestrue{}%
\providecommand\@safe@activesfalse{}%

\MHInternalSyntaxOn
%    \end{macrocode}
%   
% \end{macro}
% \end{macro}
%
%  \subsection{Extensible arrows etc.}
%
%  \begin{macro}{\xleftrightarrow}
%  \begin{macro}{\MT_leftrightarrow_fill:}
%  \begin{macro}{\xLeftarrow}
%  \begin{macro}{\xRightarrow}
%  \begin{macro}{\xLeftrightarrow}
%
%  These are straight adaptions from \pkg{amsmath}.
% \changes{v1.24}{2020/03/13}{Added spaces to move the argument away
% from the arrow head in \cs{xLeftarrow}, \cs{xRightarrow} and
% \cs{xLeftrightarrow}. Suggested by FM}
%    \begin{macrocode}
\providecommand*\xleftrightarrow[2][]{%
  \ext@arrow 3095\MT_leftrightarrow_fill:{#1}{#2}}
\def\MT_leftrightarrow_fill:{%
  \arrowfill@\leftarrow\relbar\rightarrow}
\providecommand*\xLeftarrow[2][]{%
  \ext@arrow 0055{\Leftarrowfill@}{\ #1}{\ #2}}
\providecommand*\xRightarrow[2][]{%
  \ext@arrow 0055{\Rightarrowfill@}{#1\ }{#2\ }}
\providecommand*\xLeftrightarrow[2][]{%
  \ext@arrow 0055{\Leftrightarrowfill@}{\ #1\ }{\ #2\ }}
%    \end{macrocode}
%  \end{macro}
%  \end{macro}
%  \end{macro}
%  \end{macro}
%  \end{macro}
%  \begin{macro}{\MT_rightharpoondown_fill:}
%  \begin{macro}{\MT_rightharpoonup_fill:}
%  \begin{macro}{\MT_leftharpoondown_fill:}
%  \begin{macro}{\MT_leftharpoonup_fill:}
%  \begin{macro}{\xrightharpoondown}
%  \begin{macro}{\xrightharpoonup}
%  \begin{macro}{\xleftharpoondown}
%  \begin{macro}{\xleftharpoonup}
%  \begin{macro}{\xleftrightharpoons}
%  \begin{macro}{\xrightleftharpoons}
%  The harpoons.
%    \begin{macrocode}
\def\MT_rightharpoondown_fill:{%
  \arrowfill@\relbar\relbar\rightharpoondown}
\def\MT_rightharpoonup_fill:{%
  \arrowfill@\relbar\relbar\rightharpoonup}
\def\MT_leftharpoondown_fill:{%
  \arrowfill@\leftharpoondown\relbar\relbar}
\def\MT_leftharpoonup_fill:{%
  \arrowfill@\leftharpoonup\relbar\relbar}
\providecommand*\xrightharpoondown[2][]{%
  \ext@arrow 0359\MT_rightharpoondown_fill:{#1}{#2}}
\providecommand*\xrightharpoonup[2][]{%
  \ext@arrow 0359\MT_rightharpoonup_fill:{#1}{#2}}
\providecommand*\xleftharpoondown[2][]{%
  \ext@arrow 3095\MT_leftharpoondown_fill:{#1}{#2}}
\providecommand*\xleftharpoonup[2][]{%
  \ext@arrow 3095\MT_leftharpoonup_fill:{#1}{#2}}
\providecommand*\xleftrightharpoons[2][]{\mathrel{%
  \raise.22ex\hbox{%
    $\ext@arrow 3095\MT_leftharpoonup_fill:{\phantom{#1}}{#2}$}%
  \setbox0=\hbox{%
    $\ext@arrow 0359\MT_rightharpoondown_fill:{#1}{\phantom{#2}}$}%
  \kern-\wd0 \lower.22ex\box0}}
\providecommand*\xrightleftharpoons[2][]{\mathrel{%
  \raise.22ex\hbox{%
    $\ext@arrow 0359\MT_rightharpoonup_fill:{\phantom{#1}}{#2}$}%
  \setbox0=\hbox{%
    $\ext@arrow 3095\MT_leftharpoondown_fill:{#1}{\phantom{#2}}$}%
  \kern-\wd0 \lower.22ex\box0}}
%    \end{macrocode}
%  \end{macro}
%  \end{macro}
%  \end{macro}
%  \end{macro}
%  \end{macro}
%  \end{macro}
%  \end{macro}
%  \end{macro}
%  \end{macro}
%  \end{macro}
%  \begin{macro}{\xhookleftarrow}
%  \begin{macro}{\xhookrightarrow}
%  \begin{macro}{\MT_hookright_fill:}
%  The hooks.
%    \begin{macrocode}
\providecommand*\xhookleftarrow[2][]{%
  \ext@arrow 3095\MT_hookleft_fill:{#1}{#2}}
\def\MT_hookleft_fill:{%
  \arrowfill@\leftarrow\relbar{\relbar\joinrel\rhook}}
\providecommand*\xhookrightarrow[2][]{%
  \ext@arrow 3095\MT_hookright_fill:{#1}{#2}}
\def\MT_hookright_fill:{%
  \arrowfill@{\lhook\joinrel\relbar}\relbar\rightarrow}
%    \end{macrocode}
%  \end{macro}
%  \end{macro}
%  \end{macro}
%  \begin{macro}{\xmapsto}
%  \begin{macro}{\MT_mapsto_fill:}
%  The maps-to arrow.
%    \begin{macrocode}
\providecommand*\xmapsto[2][]{%
  \ext@arrow 0395\MT_mapsto_fill:{#1}{#2}}
\def\MT_mapsto_fill:{%
  \arrowfill@{\mapstochar\relbar}\relbar\rightarrow}
%    \end{macrocode}
%  \end{macro}
%  \end{macro}
%  \subsection{Underbrackets etc.}
%  \begin{macro}{\underbracket}
%  \begin{macro}{\MT_underbracket_I:w}
%  \begin{macro}{\MT_underbracket_II:w}
%  \begin{macro}{\upbracketfill}
%  \begin{macro}{\upbracketend}
%  The \cs{underbracket} macro. Scan for two optional arguments. When
%  \pkg{xparse} becomes the standard this will be so much easier.
%    \begin{macrocode}
\providecommand*\underbracket{
  \@ifnextchar[
    {\MT_underbracket_I:w}
    {\MT_underbracket_I:w[\l_MT_bracketheight_fdim]}}
\def\MT_underbracket_I:w[#1]{
  \@ifnextchar[
    {\MT_underbracket_II:w[#1]}
    {\MT_underbracket_II:w[#1][.7\fontdimen5\textfont2]}}
\def\MT_underbracket_II:w[#1][#2]#3{%
  \mathop{\vtop{\m@th\ialign{##
    \crcr
      $\hfil\displaystyle{#3}\hfil$%
    \crcr
      \noalign{\kern .2\fontdimen5\textfont2 \nointerlineskip}%
      \upbracketfill {#1}{#2}%
    \crcr}}}
  \limits}
\def\upbracketfill#1#2{%
  \sbox\z@{$\braceld$}
  \edef\l_MT_bracketheight_fdim{\the\ht\z@}%
  \upbracketend{#1}{#2}
  \leaders \vrule \@height \z@ \@depth #1 \hfill
  \upbracketend{#1}{#2}%
}
\def\upbracketend#1#2{\vrule \@height #2 \@width #1\relax}
%    \end{macrocode}
%  \end{macro}
%  \end{macro}
%  \end{macro}
%  \end{macro}
%  \end{macro}
%  \begin{macro}{\overbracket}
%  \begin{macro}{\MT_overbracket_I:w}
%  \begin{macro}{\MT_overbracket_II:w}
%  \begin{macro}{\downbracketfill}
%  \begin{macro}{\downbracketend}
%  The overbracket is quite similar.
%    \begin{macrocode}
\providecommand*\overbracket{
  \@ifnextchar[
    {\MT_overbracket_I:w}
    {\MT_overbracket_I:w[\l_MT_bracketheight_fdim]}}
\def\MT_overbracket_I:w[#1]{
  \@ifnextchar[
    {\MT_overbracket_II:w[#1]}
    {\MT_overbracket_II:w[#1][.7\fontdimen5\textfont2]}}
\def\MT_overbracket_II:w[#1][#2]#3{%
  \mathop{\vbox{\m@th\ialign{##
        \crcr
          \downbracketfill{#1}{#2}%
        \crcr
          \noalign{\kern .2\fontdimen5\textfont2 \nointerlineskip}%
          $\hfil\displaystyle{#3}\hfil$
        \crcr}}}%
  \limits}
\def\downbracketfill#1#2{%
  \sbox\z@{$\braceld$}\edef\l_MT_bracketheight_fdim{\the\ht\z@}
  \downbracketend{#1}{#2}
  \leaders \vrule \@height #1 \@depth \z@ \hfill
  \downbracketend{#1}{#2}%
}
\def\downbracketend#1#2{\vrule \@width #1\@depth #2\relax}
%    \end{macrocode}
%  \end{macro}
%  \end{macro}
%  \end{macro}
%  \end{macro}
%  \end{macro}
%  \begin{macro}{\LaTeXunderbrace}
%  \begin{macro}{\underbrace}
%  Redefinition of \cs{underbrace} and \cs{overbrace}.
%    \begin{macrocode}
\MH_let:NwN \LaTeXunderbrace \underbrace
\def\underbrace#1{\mathop{\vtop{\m@th\ialign{##\crcr
   $\hfil\displaystyle{#1}\hfil$\crcr
   \noalign{\kern.7\fontdimen5\textfont2\nointerlineskip}%
%    \end{macrocode}
%  |.5\fontdimen5\textfont2| is the height of the tip of the brace.
%  the remaining |.2\fontdimen5\textfont2| is for space between
%    \begin{macrocode}
   \upbracefill\crcr\noalign{\kern.5\fontdimen5\textfont2}}}}\limits}
%    \end{macrocode}
%  \end{macro}
%  \end{macro}
%  \begin{macro}{\LaTeXoverbrace}
%  \begin{macro}{\overbrace}
%  Same technique for \cs{overbrace}.
%    \begin{macrocode}
\MH_let:NwN \LaTeXoverbrace \overbrace
\def\overbrace#1{\mathop{\vbox{\m@th\ialign{##\crcr
  \noalign{\kern.5\fontdimen5\textfont2}%
%    \end{macrocode}
%  Adjust for tip height
%    \begin{macrocode}
  \downbracefill\crcr
  \noalign{\kern.7\fontdimen5\textfont2\nointerlineskip}%
%    \end{macrocode}
%  |.5\fontdimen5\textfont2| is the height of the tip of the brace.
%  The remaining |.2\fontdimen5\textfont2| is for space between
%    \begin{macrocode}
  $\hfil\displaystyle{#1}\hfil$\crcr}}}\limits}
%    \end{macrocode}
%  \end{macro}
%  \end{macro}
%
%
%
%
%
%  \subsection{Special symbols}
%
%  \subsubsection{Command names for parentheses}
%  \begin{macro}{\lparen}
%  \begin{macro}{\rparen}
%  Just an addition to the \LaTeXe\ kernel.
%    \begin{macrocode}
\providecommand*\lparen{(}
\providecommand*\rparen{)}
%    \end{macrocode}
%  \end{macro}
%  \end{macro}

%  \subsubsection{Vertically centered colon}
%
%  \begin{macro}{\vcentcolon}
%  \begin{macro}{\ordinarycolon}
%  \begin{macro}{\MT_active_colon_true:}
%  \begin{macro}{\MT_active_colon_false:}
%  This is from the hands of Donald Arseneau. Somehow it is not
%  distributed, so I include it here. Here's the original text by
%  Donald:
%  \begin{verbatim}
%  centercolon.sty                 Dec 7, 2000
%  Donald Arseneau                 asnd@triumf.ca
%  Public domain.
%  Vertically center colon characters (:) in math mode.
%  Particularly useful for $ a:=b$, and still correct for
%  $f : x\to y$.  May be used in any TeX.
%  \end{verbatim}
%
% Slight change: the colon meaning is given only if \verb|centercolon|
% is explicitly requested (before it was always assigned even if : remained
% catcode 12). This allows better interaction with packages like babel
% that also make colon active.
%    \begin{macrocode}
\def\vcentcolon{\mathrel{\mathop\ordinarycolon}}
\providecommand\ordinarycolon{:}
\begingroup
  \catcode`\:=\active
  \lowercase{\endgroup
\def\MT_activate_colon{%
    \ifnum\mathcode`\:=32768\relax
      \let\ordinarycolon= :%
    \else
      \mathchardef\ordinarycolon\mathcode`\: %
    \fi 
    \let :\vcentcolon
  }
}
%    \end{macrocode}
% Option processing.
% The `false' branch can only be requested if the option has previously been set `true'.
% (By default neither are set.)
%    \begin{macrocode}
\MH_new_boolean:n {center_colon}
\define@key{\MT_options_name:}{centercolon}[true]{
  \@ifundefined{MT_active_colon_#1:}
    { \MT_true_false_error:n
      \@nameuse{MT_active_colon_false:}
    }
    { \@nameuse{MT_active_colon_#1:} }
}
\def\MT_active_colon_true: {
  \MT_activate_colon
  \MH_if_boolean:nF {center_colon}{
    \MH_set_boolean_T:n {center_colon}
    \edef\MT_active_colon_false:
      {\mathcode`\noexpand\:=\the\mathcode`\:\relax}
    \mathcode`\:=32768
  }
}
%    \end{macrocode}
%  \end{macro}
%  \end{macro}
%  \end{macro}
%  \end{macro}
%  \begin{macro}{\dblcolon}
%  \begin{macro}{\coloneqq}
%  \begin{macro}{\Coloneqq}
%  \begin{macro}{\coloneq}
%  \begin{macro}{\Coloneq}
%  \begin{macro}{\eqqcolon}
%  \begin{macro}{\Eqqcolon}
%  \begin{macro}{\eqcolon}
%  \begin{macro}{\Eqcolon}
%  \begin{macro}{\colonapprox}
%  \begin{macro}{\Colonapprox}
%  \begin{macro}{\colonsim}
%  \begin{macro}{\Colonsim}
%  This is just to simulate all the \cs{..colon..} symbols from
%  \pkg{txfonts} and \pkg{pxfonts}.
% \changes{v1.08c}{2010/11/17}{Enclosed all in \cs{mathrel}}
% \changes{v1.18}{2015/11/12}{Moved the enclosing \cs{mathrel} to
% \cs{mkern}. This is a result of
% \url{http://chat.stackexchange.com/transcript/message/23630342#23630342} and \url{http://chat.stackexchange.com/transcript/message/25348032#25348032}}
%    \begin{macrocode}
\AtBeginDocument{
  \providecommand*\dblcolon{\vcentcolon\mathrel{\mkern-.9mu}\vcentcolon}
  \providecommand*\coloneqq{\vcentcolon\mathrel{\mkern-1.2mu}=}
  \providecommand*\Coloneqq{\dblcolon\mathrel{\mkern-1.2mu}=}
  \providecommand*\coloneq{\vcentcolon\mathrel{\mkern-1.2mu}\mathrel{-}}
  \providecommand*\Coloneq{\dblcolon\mathrel{\mkern-1.2mu}\mathrel{-}}
  \providecommand*\eqqcolon{=\mathrel{\mkern-1.2mu}\vcentcolon}
  \providecommand*\Eqqcolon{=\mathrel{\mkern-1.2mu}\dblcolon}
  \providecommand*\eqcolon{\mathrel{-}\mathrel{\mkern-1.2mu}\vcentcolon}
  \providecommand*\Eqcolon{\mathrel{-}\mathrel{\mkern-1.2mu}\dblcolon}
  \providecommand*\colonapprox{\vcentcolon\mathrel{\mkern-1.2mu}\approx}
  \providecommand*\Colonapprox{\dblcolon\mathrel{\mkern-1.2mu}\approx}
  \providecommand*\colonsim{\vcentcolon\mathrel{\mkern-1.2mu}\sim}
  \providecommand*\Colonsim{\dblcolon\mathrel{\mkern-1.2mu}\sim}
}
%    \end{macrocode}
%  \end{macro}
%  \end{macro}
%  \end{macro}
%  \end{macro}
%  \end{macro}
%  \end{macro}
%  \end{macro}
%  \end{macro}
%  \end{macro}
%  \end{macro}
%  \end{macro}
%  \end{macro}
%  \end{macro}
%
%
%  \subsection{Multlined}
%
%  \begin{macro}{\g_MT_multlinerow_int}
%  \begin{macro}{\l_MT_multwidth_dim}
%  Helpers.
%    \begin{macrocode}
\let \AMS@math@cr@@ \math@cr@@
\MH_new_boolean:n {mult_firstline}
\MH_new_boolean:n {outer_mult}
\newcount\g_MT_multlinerow_int
\newdimen\l_MT_multwidth_dim
%    \end{macrocode}
%  \end{macro}
%  \end{macro}
%  \begin{macro}{\MT_test_for_tcb_other:nnnnn}
%  This tests if the token(s) is/are equal to either t, c, or~b, or
%  something entirely different.
%  \changes{v1.19}{2017/03/31}{Added MH\_ prefix}
%    \begin{macrocode}
\newcommand*\MT_test_for_tcb_other:nnnnn [1]{
  \MH_if:w t#1\relax
    \expandafter\MH_use_choice_i:nnnn
  \MH_else:
    \MH_if:w c#1\relax
      \expandafter\expandafter\expandafter\MH_use_choice_ii:nnnn
    \MH_else:
      \MH_if:w b#1\relax
        \expandafter\expandafter\expandafter
        \expandafter\expandafter\expandafter\expandafter
        \MH_use_choice_iii:nnnn
      \MH_else:
        \expandafter\expandafter\expandafter
        \expandafter\expandafter\expandafter\expandafter
        \MH_use_choice_iv:nnnn
      \MH_fi:
    \MH_fi:
  \MH_fi:
}
%    \end{macrocode}
%  \end{macro}
%  \begin{macro}{\MT_mult_invisible_line:}
%  An invisible line.
%  \textbf{BUG:} \cs{baselineskip} is zero inside an array or matrix,
%  thus in those cases the line is \emph{not} invisible.
%  \changes{v1.14}{2014/05/21}{Added -\cs{jot}}
%  \changes{v1.19}{2017/05/24}{Removed -\cs{jot} again, do not
%  remember why we added it in the first place}
%    \begin{macrocode}
\def\MT_mult_invisible_line: {
  \crcr
  \global\MH_set_boolean_F:n {mult_firstline}
  \hbox to \l_MT_multwidth_dim{}\crcr
  \noalign{\vskip-\baselineskip \vskip-\normallineskip}
}
%    \end{macrocode}
%  \end{macro}
%  \begin{macro}{\MT_mult_mathcr_atat:w}
%  The normal \cs{math@cr@@} with our hooks.
%  \changes{v1.19}{2017/03/31}{Added MH\_ prefix}
%    \begin{macrocode}
\def\MT_mult_mathcr_atat:w [#1]{%
  \MH_if_num:w 0=`{\MH_fi: \iffalse}\MH_fi:
  \MH_if_boolean:nT {mult_firstline}{
    \kern\l_MT_mult_left_fdim
    \MT_mult_invisible_line:
  }
  \crcr
  \noalign{\vskip#1\relax}
  \global\advance\g_MT_multlinerow_int\@ne
  \MH_if_num:w \g_MT_multlinerow_int=\l_MT_multline_lastline_fint
    \MH_let:NwN \math@cr@@\MT_mult_last_mathcr:w
  \MH_fi:
}
%    \end{macrocode}
%  \end{macro}
%  \begin{macro}{\MT_mult_firstandlast_mathcr:w}
%  The special case where there is a two-line \env{multlined}. We
%  insert the first kern, then the invisible line of the desired
%  width, the optional vertical space and then the last kern.
%  \changes{v1.19}{2017/03/31}{Added MH\_ prefix}
%    \begin{macrocode}
\def\MT_mult_firstandlast_mathcr:w [#1]{%
  \MH_if_num:w 0=`{\MH_fi: \iffalse}\MH_fi:
  \kern\l_MT_mult_left_fdim
  \MT_mult_invisible_line:
  \noalign{\vskip#1\relax}
  \kern\l_MT_mult_right_fdim
}
%    \end{macrocode}
%  \end{macro}
%  \begin{macro}{\MT_mult_last_mathcr:w}
%  The normal last \cs{math@cr@@} which inserts the last kern.
%  \changes{v1.19}{2017/03/31}{Added MH\_ prefix}
%    \begin{macrocode}
\def\MT_mult_last_mathcr:w [#1]{
  \MH_if_num:w 0=`{\MH_fi: \iffalse}\MH_fi:\math@cr@@@
  \noalign{\vskip#1\relax}
  \kern\l_MT_mult_right_fdim}
%    \end{macrocode}
%  \end{macro}
%  \begin{macro}{\MT_start_mult:N}
%  Setup for \env{multlined}. Finds the position.
%    \begin{macrocode}
\newcommand\MT_start_mult:N [1]{
  \MT_test_for_tcb_other:nnnnn {#1}
    { \MH_let:NwN \MT_next:\vtop }
    { \MH_let:NwN \MT_next:\vcenter }
    { \MH_let:NwN \MT_next:\vbox }
    {
      \PackageError{mathtools}
        {Invalid~ position~ specifier.~ I'll~ try~ to~ recover~ with~
        `c'}\@ehc
    }
  \collect@body\MT_mult_internal:n
}
%    \end{macrocode}
%  \end{macro}
%  \begin{macro}{\MT_shoveright:wn}
%  \begin{macro}{\MT_shoveleft:wn}
%  Extended versions of \cs{shoveleft} and \cs{shoveright}.
%    \begin{macrocode}
\newcommand*\MT_shoveright:wn [2][0pt]{%
  #2\hfilneg
  \setlength\@tempdima{#1}
  \kern\@tempdima
}
\newcommand*\MT_shoveleft:wn [2][0pt]{%
  \hfilneg
  \setlength\@tempdima{#1}
  \kern\@tempdima
  #2
}
%    \end{macrocode}
%  \end{macro}
%  \end{macro}
%  \begin{macro}{\MT_mult_internal:n}
%  \changes{v1.01a}{2004/10/10}{Added Ord atom to beginning of each line}
%  The real internal \env{multlined}.
%  \changes{v1.19}{2017/05/22}{Added \cs{alignedspace@left} instead of
%  \cs{null}\cs{,}}
%    \begin{macrocode}
\newcommand*\MT_mult_internal:n [1]{
 \MH_if_boolean:nF {outer_mult}{\alignedspace@left} %<-- requires amsmath 2016/11/05
  \MT_next:
  \bgroup
%    \end{macrocode}
%  Restore the meaning of \cmd{\\} inside \env{multlined}, else it
%  wouldn't work in the \env{equation} environment. Set the fake row
%  counter to zero.
%    \begin{macrocode}
    \Let@
    \def\l_MT_multline_lastline_fint{0 }
    \chardef\dspbrk@context\@ne \restore@math@cr
%    \end{macrocode}
%  Use private versions.
%    \begin{macrocode}
    \MH_let:NwN \math@cr@@\MT_mult_mathcr_atat:w
    \MH_let:NwN \shoveleft\MT_shoveleft:wn
    \MH_let:NwN \shoveright\MT_shoveright:wn
    \spread@equation
    \MH_set_boolean_F:n {mult_firstline}
%    \end{macrocode}
%  Do some measuring.
%    \begin{macrocode}
    \MT_measure_mult:n {#1}
%    \end{macrocode}
%  Make sure the box is wide enough.
%  \changes{v1.19}{2017/03/31}{Added MH\_ prefix}
%    \begin{macrocode}
    \MH_if_dim:w \l_MT_multwidth_dim<\l_MT_multline_measure_fdim
      \MH_setlength:dn \l_MT_multwidth_dim{\l_MT_multline_measure_fdim}
    \fi
    \MH_set_boolean_T:n {mult_firstline}
%    \end{macrocode}
%  Tricky bit: If we only encountered one \cmd{\\} then use a very
%  special \cs{math@cr@@} that inserts everything needed.
%  \changes{v1.19}{2017/03/31}{Added MH\_ prefix}
%    \begin{macrocode}
    \MH_if_num:w \l_MT_multline_lastline_fint=\@ne
      \MH_let:NwN \math@cr@@ \MT_mult_firstandlast_mathcr:w
    \MH_fi:
%    \end{macrocode}
%  Do the typesetting.
%    \begin{macrocode}
    \ialign\bgroup
      \hfil\strut@$\m@th\displaystyle{}##$\hfil
      \crcr
      \hfilneg
      #1
}
%    \end{macrocode}
%  \end{macro}
%  \begin{macro}{\MT_measure_mult:n}
%  \changes{v1.01a}{2004/10/10}{Added Ord atom to beginning of each line}
%  Measuring. Disable all labelling and check the number of lines.
% \changes{v1.18}{2015/11/12}{Added \cs{measuring@true} inside the
% \env{multlined}}
%    \begin{macrocode}
\newcommand\MT_measure_mult:n [1]{
  \begingroup
    \measuring@true
    \g_MT_multlinerow_int\@ne
    \MH_let:NwN \label\MT_gobblelabel:w
    \MH_let:NwN \tag\gobble@tag
    \setbox\z@\vbox{
      \ialign{\strut@$\m@th\displaystyle{}##$
        \crcr
        #1
        \crcr
      }
    }
    \xdef\l_MT_multline_measure_fdim{\the\wdz@}
    \advance\g_MT_multlinerow_int\m@ne
    \xdef\l_MT_multline_lastline_fint{\number\g_MT_multlinerow_int}
  \endgroup
  \g_MT_multlinerow_int\@ne
}
%    \end{macrocode}
%  \end{macro}
%  \begin{macro}{\MT_multlined_second_arg:w}
%  Scan for a second optional argument.
%  \changes{v1.19}{2017/03/31}{Added MH\_ prefix}
%    \begin{macrocode}
\MaybeMHPrecedingSpacesOff
\newcommand*\MT_multlined_second_arg:w [1][\@empty]{
  \MT_test_for_tcb_other:nnnnn {#1}
    {\def\MT_mult_default_pos:{#1}}
    {\def\MT_mult_default_pos:{#1}}
    {\def\MT_mult_default_pos:{#1}}
    {
      \MH_if_meaning:NN \@empty#1\@empty
      \MH_else:
        \setlength \l_MT_multwidth_dim{#1}
      \MH_fi:
    }
  \MT_start_mult:N \MT_mult_default_pos:
}
%    \end{macrocode}
%  \end{macro}
%
%  \begin{macro}{\MultlinedHook}
% \changes{v1.18}{2015/11/12}{Added this hook}
% Due to the methods used by \env{multlined}, some constructions may
% fail. We already know that \env{multlined} does not work well inside
% arrays or matrices. Be things used inside \cs{multlined} may also
% fail. Thus we provide a hook that can be added to in order to make
% redefinitions of these constructions when used inside
% \env{multlined}. The default value include a fix to \env{subarray}
% and thus \cs{substack}. The fix to this environment was suggested by
% Ulrike Fisher, \url{http://chat.stackexchange.com/transcript/message/25105970#25105970}
% \changes{v1.22}{2019/07/22}{Also added crampedsubarray}
% \changes{v1.22}{2019/07/22}{Also added fixes for smallmatrix and the
% MT versions of these}
% \changes{v1.26}{2021/03/27}{We need a lualatex branch of this as
% well. We need to find a better way to manage this. Patching?}
%    \begin{macrocode}
\ifx\directlua\@undefined
  % THIS IS NORMAL
\newcommand\MultlinedHook{
  \renewenvironment{subarray}[1]{%
    \vcenter\bgroup
    \Let@ \restore@math@cr \default@tag
    \let\math@cr@@\AMS@math@cr@@  % <--- the fix
    \baselineskip\fontdimen10 \scriptfont\tw@
    \advance\baselineskip\fontdimen12 \scriptfont\tw@
    \lineskip\thr@@\fontdimen8 \scriptfont\thr@@
    \lineskiplimit\lineskip
    \ialign\bgroup\ifx c##1\hfil\fi
    $\m@th\scriptstyle####$\hfil\crcr
  }{%
    \crcr\egroup\egroup
  }
  \renewenvironment{crampedsubarray}[1]{%
    \vcenter\bgroup
    \Let@ \restore@math@cr \default@tag
    \let\math@cr@@\AMS@math@cr@@  % <--- the fix
    \baselineskip\fontdimen10 \scriptfont\tw@
    \advance\baselineskip\fontdimen12 \scriptfont\tw@
    \lineskip\thr@@\fontdimen8 \scriptfont\thr@@
    \lineskiplimit\lineskip
    \ialign\bgroup\ifx c##1\hfil\fi
%    \end{macrocode}
% \changes{v1.25}{2021/03/18}{use \cs{MT_cramped_internal:Nn} instead,
% see issue \#17}
% Here we should use the proper cramped internal macro
%    \begin{macrocode}
      \span\MT_cramped_internal:Nn \scriptstyle {####}%
    \hfil\crcr
  }{%
    \crcr\egroup\egroup
  }
   % from mathtools
  \def\MT_smallmatrix_begin:N ##1{%
    \Let@\restore@math@cr\default@tag
    \let\math@cr@@\AMS@math@cr@@  % <--- the fix
    \baselineskip6\ex@ \lineskip1.5\ex@ \lineskiplimit\lineskip
    \csname MT_smallmatrix_##1_begin:\endcsname
  }
  % from amsmath
  \renewenvironment{smallmatrix}{\null\,\vcenter\bgroup
    \Let@\restore@math@cr\default@tag
    \let\math@cr@@\AMS@math@cr@@  % <--- the fix
    \baselineskip6\ex@ \lineskip1.5\ex@ \lineskiplimit\lineskip
    \ialign\bgroup\hfil$\m@th\scriptstyle####$\hfil&&\thickspace\hfil
    $\m@th\scriptstyle####$\hfil\crcr
  }{%
    \crcr\egroup\egroup\,%
  }
}
\else
%    \end{macrocode}
% The LuaLaTeX version, only difference is the definition of crampedsubarray
%    \begin{macrocode}
  % THIS IS LUALATEX
\newcommand\MultlinedHook{
  \renewenvironment{subarray}[1]{%
    \vcenter\bgroup
    \Let@ \restore@math@cr \default@tag
    \let\math@cr@@\AMS@math@cr@@  % <--- the fix
    \baselineskip\fontdimen10 \scriptfont\tw@
    \advance\baselineskip\fontdimen12 \scriptfont\tw@
    \lineskip\thr@@\fontdimen8 \scriptfont\thr@@
    \lineskiplimit\lineskip
    \ialign\bgroup\ifx c##1\hfil\fi
    $\m@th\scriptstyle####$\hfil\crcr
  }{%
    \crcr\egroup\egroup
  }
  % from mathtools
  \newenvironment{crampedsubarray}[1]{%
    \vcenter\bgroup
    \Let@ \restore@math@cr \default@tag
    \let\math@cr@@\AMS@math@cr@@  % <--- the fix
    \baselineskip \Umathstacknumup \scriptstyle
    \advance\baselineskip \Umathstackdenomdown \scriptstyle
    \lineskip \Umathstackvgap \scriptstyle
    \lineskiplimit \lineskip
    \ialign\bgroup\ifx c#1\hfil\fi
    \Ustartmath
      \crampedscriptstyle{####}
    \Ustopmath
    \hfil\crcr
  }{%
    \crcr\egroup\egroup
  }
   % from mathtools
  \def\MT_smallmatrix_begin:N ##1{%
    \Let@\restore@math@cr\default@tag
    \let\math@cr@@\AMS@math@cr@@  % <--- the fix
    \baselineskip6\ex@ \lineskip1.5\ex@ \lineskiplimit\lineskip
    \csname MT_smallmatrix_##1_begin:\endcsname
  }
  % from amsmath
  \renewenvironment{smallmatrix}{\null\,\vcenter\bgroup
    \Let@\restore@math@cr\default@tag
    \let\math@cr@@\AMS@math@cr@@  % <--- the fix
    \baselineskip6\ex@ \lineskip1.5\ex@ \lineskiplimit\lineskip
    \ialign\bgroup\hfil$\m@th\scriptstyle####$\hfil&&\thickspace\hfil
    $\m@th\scriptstyle####$\hfil\crcr
  }{%
    \crcr\egroup\egroup\,%
  }
}
\fi


%    \end{macrocode}
%  \end{macro}
%
%  \begin{environment}{multlined}
%  The user environment. Scan for an optional argument.
% \changes{v1.18}{2015/11/12}{Added \cs{MultlinedHook}}
%  \changes{v1.19}{2017/03/31}{Added MH\_ prefix}
%    \begin{macrocode}
\newenvironment{multlined}[1][]
  {\MH_group_align_safe_begin:
    \MultlinedHook
  \MT_test_for_tcb_other:nnnnn {#1}
    {\def\MT_mult_default_pos:{#1}}
    {\def\MT_mult_default_pos:{#1}}
    {\def\MT_mult_default_pos:{#1}}
    {
      \MH_if_meaning:NN \@empty#1\@empty
      \MH_else:
        \setlength \l_MT_multwidth_dim{#1}
      \MH_fi:
    }
    \MT_multlined_second_arg:w
  }
  {
    \hfilneg  \endaligned \MH_group_align_safe_end:
  }
\MHPrecedingSpacesOn
%    \end{macrocode}
%  \end{environment}
%  The keys needed.
%    \begin{macrocode}
\define@key{\MT_options_name:}
  {firstline-afterskip}{\def\l_MT_mult_left_fdim{#1}}
\define@key{\MT_options_name:}
  {lastline-preskip}{\def\l_MT_mult_right_fdim{#1}}
\define@key{\MT_options_name:}
  {multlined-width}{\setlength \l_MT_multwidth_dim{#1}}
\define@key{\MT_options_name:}
  {multlined-pos}{\def\MT_mult_default_pos:{#1}}
\setkeys{\MT_options_name:}{
  firstline-afterskip=\multlinegap,
  lastline-preskip=\multlinegap,
  multlined-width=0pt,
  multlined-pos=c,
}
%    \end{macrocode}
%  \begin{macro}{\MT_gobblelabel:w}
%  Better than to assume that \cs{label} has exactly one mandatory
%  argument, hence the \texttt{w} specifier.
%    \begin{macrocode}
\def\MT_gobblelabel:w #1{}
%    \end{macrocode}
%  \end{macro}
%
%
%
%
%  \section{Macros suggested/requested by Lars Madsen}
%
%  The macros in this section are all requests made by Lars Madsen.
%
%  \subsection{Paired delimiters}
%
%  \changes{v1.13}{2012/05/10}{Extended \cs{DeclarePairedDelimiter(X)}}
%  \begin{macro}{\MT_delim_default_inner_wrappers:n}
%  In some cases users may want to control the internals a bit more. We
%  therefore create two call back macros each time the
%  \cs{DeclarePaired...} macro is issued. The default value of these
%  call backs are provided by |\MT_delim_default_inner_wrappers:n|
%    \begin{macrocode}
\newcommand\MT_delim_default_inner_wrappers:n [1]{
   \@namedef{MT_delim_\MH_cs_to_str:N #1 _star_wrapper:nnn}##1##2##3{
      \mathopen{}\mathclose\bgroup ##1 ##2 \aftergroup\egroup ##3
    }
%    \end{macrocode}
% Turns out that there is a difference between \verb|\mathclose{)}|
% and \verb|\mathclose)| (placement of for example superscript), so instead of two wrappers, we need three. 
% \changes{v1.19}{2017/05/22}{added \cs{...nostarnonscaled..} wrapper}
% \changes{v1.19}{2017/05/22}{renamed \cs{...nostar..} to \cs{...nostarscaled..}  }
%    \begin{macrocode}
    \@namedef{MT_delim_\MH_cs_to_str:N #1 _nostarscaled_wrapper:nnn}##1##2##3{
      \mathopen{##1}##2\mathclose{##3}
    }
    \@namedef{MT_delim_\MH_cs_to_str:N #1 _nostarnonscaled_wrapper:nnn}##1##2##3{
      \mathopen##1##2\mathclose##3
    }
  }

%    \end{macrocode}
% \end{macro}
%    
% \begin{macro}{\reDeclarePairedDelimiterInnerWrapper}
%   Macro enabling the user to alter an existing call back.  First
%   argument is the name of the macro we are altering (as defined via
%   \cs{DeclarePaired...}), the second is \texttt{star},
%   \texttt{nostarnonscaled} or \texttt{nostarscaled}. In the last
%   argument \texttt{\#1} and \texttt{\#3} respectively refer to the
%   scaled fences and \texttt{\#2} refer to whatever come between.
%   \changes{v1.19}{2017/05/23}{Added a test and error}
%    \begin{macrocode}
\newcommand\reDeclarePairedDelimiterInnerWrapper[3]{
  \@ifundefined{MT_delim_\MH_cs_to_str:N #1 _ #2 _wrapper:nnn}{
    \PackageError{mathtools}{
      Wrapper~not~found~for~\string#1~and~option~'#2'.\MessageBreak 
      Either~\string#1~is~ not~ defined,~or~ you~ are~using~ the~
      \MessageBreak
      'nostar'~ option,~which~ is~ no~ longer~ supported.~ 
      \MessageBreak
      Please~ use~ 'nostarnonscaled'~ or~ 'nostarscaled~
      \MessageBreak instead.~ 
    }{See the manual}
  }{
    \@namedef{MT_delim_\MH_cs_to_str:N #1 _ #2 _wrapper:nnn}##1##2##3{
      #3
    }
  }
}

%    \end{macrocode}
%   
% \begin{macro}{\MT_etb_ifdefempty_x:nnn}
% \begin{macro}{\MT_etb_ifblank:nnn}
%   \changes{v1.19}{2017/05/23}{Added so we can test for empty args}
%   This is a copy of \cs{etb@ifdefempty} (it is \emph{not} the same
%   as \cs{ifdefempty}) and \cs{ifblank} from \pkg{etoolbox},
%   currently we do not want to make \pkg{etoolbox} a requirement,
%   that may change in the future. We need a reliable method to check
%   whether a macro is blank or not (including spaces).
%    \begin{macrocode}
\def\MT_etb_ifdefempty_x:nnn #1{ 
  \expandafter\expandafter\expandafter
  \MT_etb_ifblank:nnn 
  \expandafter\expandafter\expandafter{%
    \expandafter\strip@prefix\meaning#1} 
}
\def\MT_etb_ifblank:nnn #1{
  \expandafter\ifx\expandafter\relax\detokenize\expandafter{\@gobble#1?}\relax
    \expandafter\@firstoftwo
  \else
    \expandafter\@secondoftwo
  \fi
}


%    \end{macrocode}
% \end{macro}
% \end{macro}
%  %
%
%  \end{macro}
%  \begin{macro}{\DeclarePairedDelimiter}
%  \changes{v1.06}{2008/08/01}{Made user command robust}
%  This macro defines |#1| to be a control sequence that takes either
%  a star or an optional argument and one mandatory argument.
%    \begin{macrocode}
\newcommand*\DeclarePairedDelimiter[3]{%
  \@ifdefinable{#1}{
%    \end{macrocode}
%  Define the starred command to just put \cs{left} and \cs{right}
%  before the delimiters, wrapped in a callback function.
%  \changes{1.08e}{2010/09/02}{`Fixed' \cs{left}\dots\cs{right} bad spacing}
%  \changes{1.08e}{2010/09/14}{redid the \cs{left}\dots\cs{right} fix,
%  see \cs{DeclarePairedDelimiterXPP} for details.}
%  \changes{v1.13}{2012/05/10}{Using call back instead}
%    \begin{macrocode}
    \MT_delim_default_inner_wrappers:n{#1} % define the wrappers
    \@namedef{MT_delim_\MH_cs_to_str:N #1 _star:}##1
      { \@nameuse{MT_delim_\MH_cs_to_str:N #1 _star_wrapper:nnn}%
           {\left#2}{##1}{\right#3} }%
%    \end{macrocode}
%  The command with optional argument. It should be \cs{bigg} or
%  alike.
%    \begin{macrocode}
    \@xp\@xp\@xp
      \newcommand
        \@xp\csname MT_delim_\MH_cs_to_str:N #1 _nostar:\endcsname
        [2][\\@gobble]
        { 
%    \end{macrocode}
%  With the default optional argument we wind up with \cs{relax},
%  else we get \cs{biggr} and \cs{biggl} etc.
%  \changes{v1.13}{2012/05/10}{Using call back instead}
%  \changes{v1.19}{2017/05/23}{Split the wrappers in two, so know
%  \cs{}\cs{@gobble} is now only used to check if the optional
%  argument is empty.}
%    \begin{macrocode}
          \def\@tempa{\\@gobble}
          \def\@tempb{##1}
          \ifx\@tempa\@tempb
%    \end{macrocode}
% As of May 2017, we now run explicit wrapper in the nonscaled
% version, so we can remove the feature to to add l and r version of
% the scaler (as there is none).
%    \begin{macrocode}
            \@nameuse{MT_delim_\MH_cs_to_str:N #1 _nostarnonscaled_wrapper:nnn}%
               {#2}
               {##2}
               {#3}
          \else
%    \end{macrocode}
% Next we need to check whether \texttt{\#\#1} is blank as that
% should not be scaled
%    \begin{macrocode}
            \MT_etb_ifblank:nnn {##1}{
              \@nameuse{MT_delim_\MH_cs_to_str:N #1 _nostarnonscaled_wrapper:nnn}%
                {#2}
                {##2}
                {#3}
            }{
              \@nameuse{MT_delim_\MH_cs_to_str:N #1 _nostarscaled_wrapper:nnn}%
                {\@nameuse {\MH_cs_to_str:N ##1 l} #2}
                {##2}
                {\@nameuse {\MH_cs_to_str:N ##1 r} #3}
            }
          \fi
        }
%    \end{macrocode}
%  The user command comes here. Just check for the star and choose
%  the right internal command.
%    \begin{macrocode}
    \DeclareRobustCommand{#1}{
      \@ifstar
        {\@nameuse{MT_delim_\MH_cs_to_str:N #1 _star:}}
        {\@nameuse{MT_delim_\MH_cs_to_str:N #1 _nostar:}}
    }
  }
}
%    \end{macrocode}
%  \end{macro}
%
% \begin{macro}{\MT_paired_delimx_arg_test:n}
%  This tests the \oarg{num args} part of
%  \cs{DeclarePairedDelimiterX} and \cs{DeclarePairedDelimiterXPP}
%  (code reuse), and complains if it is not 1,\dots,9.
%  \changes{v1.14}{2014/05/20}{Added}
%    \begin{macrocode}
\def\MT_paired_delimx_arg_test:n #1{
  \ifnum#1>9\relax
    \PackageError{mathtools}{No~ more~ than~ 9~ arguments}{}
  \else
    \ifnum#1<1\relax
      \PackageError{mathtools}{Macro~ need~ 1~ or~ more~
        arguments.\MessageBreak Please~ change~ [#1]~ to~ [1]~ ... [9]}{}
    \fi
  \fi
 }

%    \end{macrocode}
% \end{macro}
% \begin{macro}{\MT_delim_inner_generator:n..n}
%   \changes{v1.19}{2017/05/14}{Macro added, changes notes in the
%   implementation are taken from when this code was not factored out}
%   This generates the internal macro from
%   \cs{DeclarePairedDelimiterX} and \cs{DeclarePairedDelimiterXPP}
%   that handles the non-starred part of the generated macro. The
%   macro takes 7 arguments mimicing the arguments of
%   \cs{DeclarePairedDelimiterXPP}: 
% \begin{enumerate}\setlength\itemsep{0pt}
% \item external macro name
% \item number of arguments, 1-9, already checked at this stage
% \item predata
% \item left scaler
% \item right scaler
% \item postdata
% \item body
% \end{enumerate}
%
%    \begin{macrocode}
\def\@MHempty{}
\newcommand\MT_delim_inner_generator:nnnnnnn [7]{
%    \end{macrocode}
% In order for the starred and non-starred version of
% \cs{DeclarePairedDelimiterX(PP)} to have the same arguments, we need
% to introduce an extra macro to catch the optional argument (this
% means that the non-starred version can actually support ten
% arguments!).  Here we do things a little differently than with
% \cs{DeclarePairedDelimiter}. The optional argument have
% \cs{@MHempty} as the default. This was earlier locally redefined to
% eat l/r additions to the scaler when there were no scaler. In this
% implementation, we just look for that macro to see if no scaler has
% been given at all.
%    \begin{macrocode}
    \@xp\@xp\@xp
      \newcommand
        \@xp\csname MT_delim_\MH_cs_to_str:N #1 _nostar_inner:\endcsname
        [#2]
        {
%    \end{macrocode}
% Add the (possibly empty) precode:
%    \begin{macrocode}
          #3
%    \end{macrocode}
% Next we provide the inner workhorse. We need a bit of expansion
% magic to get \cs{delimsize} to work.
% \changes{v1.13}{2012/05/10}{Using call backs for \cs{mathopen} and
% \cs{mathclose} additions}
% \changes{v1.19}{2017/04/27}{Added resetting code for \cs{@MHempty},
% cannot rely on groups here. 2017/05/23 removed because of reimplementation}
%
% May 2017: Due to \verb+\mathclose{|}^2+ generally not being equal to
% \verb+\mathclose|^2+ we needed to split the wrapper macro into
% separate nonscaled and scaled versions. This also have the benefit
% of completely eliminating the need to gobble any chars. There is now
% the nonscaled version, which we do not add any scalers to and the
% manually scaled version where we \emph{know} there is a scaler. The
% complication is now that we need to be able to check for blank
% input, thus the including of a few macros from \pkg{etoolbox}.
% \changes{v1.19}{2017/05/23}{Split into two wrappers and some tests}
% First we need to look at \cs{delimsize} to see if it is equal to
% \cs{@MHempty}. If it is run the nonscaled version. 
%    \begin{macrocode}
          \def\@tempa{\@MHempty}
          \@xp\def\@xp\@tempb\@xp{\delimsize}
          \ifx\@tempa\@tempb
            \@nameuse{MT_delim_\MH_cs_to_str:N #1 _nostarnonscaled_wrapper:nnn}
              {#4}{#7}{#5}
          \else
%    \end{macrocode}
% Next we use a copy of a macro from \pkg{etoolbox} to determine
% whether \cs{delimsize} is empty or not. This is to be sure that
% \verb|\abs[]| and similar still end up in the unscaled branch.
%    \begin{macrocode}
            \MT_etb_ifdefempty_x:nnn {\delimsize}{
              \@nameuse{MT_delim_\MH_cs_to_str:N #1 _nostarnonscaled_wrapper:nnn}
                {#4}{#7}{#5}
            }{
%    \end{macrocode}
% In the scaled version we add features to convert til scaler to its
% l/r versions.
%    \begin{macrocode}
              \@nameuse{MT_delim_\MH_cs_to_str:N #1 _nostarscaled_wrapper:nnn}
              {
                \@xp\@xp\@xp\csname\@xp\MH_cs_to_str:N \delimsize l\endcsname #4
              }
              {#7}
              {
                \@xp\@xp\@xp\csname\@xp\MH_cs_to_str:N \delimsize r\endcsname #5
              }
            }
          \fi
%    \end{macrocode}
% Add the (possibly empty) post code
%    \begin{macrocode}
          #6
          \endgroup
        }
}

%    \end{macrocode}
%   
% \end{macro}
% \begin{macro}{\DeclarePairedDelimiterXPP}
%   \changes{v1.19}{2017/05/24}{Rewritten as we might as well
%   implement ...X via ...XPP} It turns out it is useful to have a
%   more general version of \cs{DeclarePairedDelimiter} where we can
%   access the body and perhaps even add stuff on the outside of the
%   fences. Historically, we added \cs{DeclarePairedDelimiterX} first,
%   giving access to the body and the most general
%   \cs{DeclarePairedDelimiterXPP} was requested after that, thus
%   ending up with two macros so not to break existing document. As of
%   May 2017, since \cs{DeclarePairedDelimiterX} is just
%   \cs{DeclarePairedDelimiterXPP} with no pre- or postcode, we will
%   use this update to just define \cs{DeclarePairedDelimiterXPP} and
%   define \cs{DeclarePairedDelimiterX} from that.
% 
%   The arguments for \cs{DeclarePairedDelimiterXPP} are interpreted
%   as
% \begin{center}
%   \marg{macroname}\oarg{num~args}\allowbreak\marg{pre~code}\allowbreak\marg{left
%   delim}\goodbreak \marg{right delim}\marg{post code}\marg{body}
% \end{center}
%  Inside \marg{body} \cs{delimsize}
% refers to the current size of the scaler (\cs{middle} in the starred version).
%    \begin{macrocode}
\def\DeclarePairedDelimiterXPP#1[#2]#3#4#5#6#7{%
  \@ifdefinable{#1}{
%    \end{macrocode}
% The constructor takes five arguments, the name of the macro, the
% number of arguments (1-9), the left and right delimiter, the inner
% code for the two macros. First we verify that the number of arguments fit.
% \changes{v1.14}{2014/05/20}{Factored out in separate macro
% (\cs{MT_paired_delimx_arg_test:n}) for reuse. Comment refers back to
% old \cs{DeclarePairedDelimiterX} implementation.} 
%    \begin{macrocode}
    \MT_paired_delimx_arg_test:n{#2}
%    \end{macrocode}
%  \changes{v1.13}{2012/05/10}{Using call back instead}
%  Define the default call backs.
%    \begin{macrocode}
    \MT_delim_default_inner_wrappers:n{#1}
%    \end{macrocode}
% We make sure to store the delimiter size in the local variable
% \cs{delimsize}. Then users can refer to the size in the \marg{body}
% argument. In the starred version it will refer to \cs{middle} and in
% the normal version it will hold the provided optional argument.
%    \begin{macrocode}
    \@xp\@xp\@xp
      \newcommand
        \@xp\csname MT_delim_\MH_cs_to_str:N #1 _star:\endcsname
        [#2]
        {
          \begingroup
            \def\delimsize{\middle}
%    \end{macrocode}
% Adding the \marg{precode} (possibly empty)
%    \begin{macrocode}
            #3
%    \end{macrocode}
% This is slightly controversial, \cs{left}\dots\cs{right} are known
% to produce an inner atom, thus may cause different spacing than
% normal delimiters. We `fix' this by introducing \cs{mathopen} and
% \cs{mathclose}. This change is now factored out into call backs. 
% \changes{v1.08e}{2010/09/14}{redid the left/right fix, inspired by
% ctt thread named `spacing after \cs{right}) and before \cs{left})'
% started 2010-08-12.}
% \changes{v1.13}{2012/05/10}{Using call back instead}
%    \begin{macrocode}
            \@nameuse{MT_delim_\MH_cs_to_str:N #1 _star_wrapper:nnn}
              {\left#4}{#7}{\right#5}
            #6  
          \endgroup
        }
%    \end{macrocode}
% In order for the starred and non-starred version to have the same
% arguments, we need to introduce an extra macro to catch the optional
% argument (this means that the non-starred version can actually
% support ten arguments!).  
%    \begin{macrocode}
    \@xp\@xp\@xp
      \newcommand
        \@xp\csname MT_delim_\MH_cs_to_str:N #1 _nostar:\endcsname
        [1][\@MHempty]
      {
%    \end{macrocode}
% We need to introduce a local group in order to support nesting. It
% is ended inside \verb|\MT_delim_\MH_cs_to_str:N #1 _nostar_inner:|,
% here we also stores the scaler in \cs{delimsize}.
%    \begin{macrocode}
        \begingroup
        \def\delimsize{##1}
        \@nameuse{MT_delim_\MH_cs_to_str:N #1 _nostar_inner:}
      }
%    \end{macrocode}
% To make the code easier to work with we factored the next macro
% out. Since \cs{DeclarePairedDelimiterX} is now implemented via
% \cs{DeclarePairedDelimiterXPP} we did not need to. But it makes the
% code a bit simpler. \cs{MT_delim_inner_generator:nnnnnnn} just
% takes care of generating the appropriate \cs{..._nostar_inner:} macro.
%    \begin{macrocode}
      \MT_delim_inner_generator:nnnnnnn {#1}{#2}{#3}{#4}{#5}{#6}{#7}
%    \end{macrocode}
% At the end, generate the actual user command.
%    \begin{macrocode}
    \DeclareRobustCommand{#1}{
      \@ifstar
        {\@nameuse{MT_delim_\MH_cs_to_str:N #1 _star:}}
        {\@nameuse{MT_delim_\MH_cs_to_str:N #1 _nostar:}}
    }
  }
}

%    \end{macrocode}
% \end{macro}
% \begin{macro}{\DeclarePairedDelimiterX}
%  \changes{v1.08}{2010/06/10}{Added
%  \cs{DeclarePairedDelimiterX}. 2017/05/24: old comment, kept for
%  historic reasons.}
%  \changes{v1.08e}{2010/09/02}{Provided better implementation of 
%  \cs{DeclarePairedDelimiterX}. 2017/05/24: old comment, kept for
%  historic reasons.}
% As of May 2017, this is now just a call to \cs{DeclarePairedDelimiterXPP}.
%    \begin{macrocode}
\def\DeclarePairedDelimiterX#1[#2]#3#4#5{
  \DeclarePairedDelimiterXPP{#1}[#2]{}{#3}{#4}{}{#5}
}
%    \end{macrocode}
% \end{macro}
%


%
%  \subsection{A \texttt{\textbackslash displaystyle} \env{cases} environment}
%
%  \begin{macro}{\MT_start_cases:nnn}
%  We define a single command that does all the hard work.
%  \changes{v1.08}{2010/06/10}{made \cs{MT_start_cases:nnnn} more general}
%    \begin{macrocode}
\def\MT_start_cases:nnnn #1#2#3#4{ % #1=sep,#2=lpreamble,#3=rpreamble,#4=delim
 \RIfM@\else
   \nonmatherr@{\begin{\@currenvir}}
 \fi
 \MH_group_align_safe_begin:
 \left#4
 \vcenter \bgroup
     \Let@ \chardef\dspbrk@context\@ne \restore@math@cr
     \let  \math@cr@@\AMS@math@cr@@
     \spread@equation
     \ialign\bgroup
%    \end{macrocode}
%  Set the first column flush left in \cs{displaystyle} math and the
%  second as specified by the second argument. The first argument is
%  the separation between the columns. It could be a \cs{quad} or
%  something entirely different.
%    \begin{macrocode}
       \strut@#2 &#1\strut@
       #3
       \crcr
}
%    \end{macrocode}
%  \end{macro}
% \begin{macro}{\MH_end_cases:}
%    \begin{macrocode}
\def\MH_end_cases:{\crcr\egroup
 \restorecolumn@
 \egroup
 \MH_group_align_safe_end:
}
%    \end{macrocode}
% \end{macro}
%  \begin{macro}{\newcases}
%  \begin{macro}{\renewcases}
%  Easy creation of new \env{cases}-like environments.
%  \changes{v1.08}{2010/06/10}{changed to match the change in \cs{MT_start_cases:nnnn}}
%    \begin{macrocode}
\newcommand*\newcases[6]{% #1=name, #2=sep, #3=preamble, #4=left, #5=right
 \newenvironment{#1}
   {\MT_start_cases:nnnn {#2}{#3}{#4}{#5}}
   {\MH_end_cases:\right#6}
}
\newcommand*\renewcases[6]{
 \renewenvironment{#1}
   {\MT_start_cases:nnnn {#2}{#3}{#4}{#5}}
   {\MH_end_cases:\right#6}
}
%    \end{macrocode}
%  \begin{environment}{dcases}
%  \begin{environment}{dcases*}
%  \begin{environment}{rcases}
%  \begin{environment}{rcases*}
%  \begin{environment}{drcases}
%  \begin{environment}{drcases*}
%  \begin{environment}{cases*}
%  \env{dcases} is a traditional cases with display style math in
%  both columns, while \env{dcases*} has text in the second column.
%  \changes{v1.08}{2010/06/10}{changed to match the change in
%  \cs{newcases} plus added rcases and drcases}
% \changes{v1.25}{2021/01/14}{Replaced \texttt{\{\#\#\}} by
% \texttt{\#\#}, otherwise one cannot counteract the \cs{hfil} in the
% definition with a \cs{hfil(l)} in a cell. Related to issue \#20 and https://tex.stackexchange.com/a/252412/3929}
%    \begin{macrocode}
\newcases{dcases}{\quad}{%
  $\m@th\displaystyle##$\hfil}{$\m@th\displaystyle##$\hfil}{\lbrace}{.}
\newcases{dcases*}{\quad}{%
  $\m@th\displaystyle##$\hfil}{##\hfil}{\lbrace}{.}
\newcases{rcases}{\quad}{%
  $\m@th##$\hfil}{$\m@th##$\hfil}{.}{\rbrace}
\newcases{rcases*}{\quad}{%
  $\m@th##$\hfil}{##\hfil}{.}{\rbrace}
\newcases{drcases}{\quad}{%
  $\m@th\displaystyle##$\hfil}{$\m@th\displaystyle##$\hfil}{.}{\rbrace}
\newcases{drcases*}{\quad}{%
  $\m@th\displaystyle##$\hfil}{##\hfil}{.}{\rbrace}
\newcases{cases*}{\quad}{%
  $\m@th##$\hfil}{##\hfil}{\lbrace}{.}
%    \end{macrocode}
%  \end{environment}
%  \end{environment}
%  \end{environment}
%  \end{environment}
%  \end{environment}
%  \end{environment}
%  \end{environment}
%  \end{macro}
%  \end{macro}
%
%  \subsection{New matrix environments}
%  \begin{macro}{\MT_matrix_begin:N}
%  \begin{macro}{\MT_matrix_end:}
%  Here are a few helpers for the matrices. \cs{MT_matrix_begin:N}
%  takes one argument specifying the column type for the array inside
%  the matrix. and \cs{MT_matrix_end:} inserts the correct ending.
%    \begin{macrocode}
\def\MT_matrix_begin:N #1{%
  \hskip -\arraycolsep
  \MH_let:NwN \@ifnextchar \MH_nospace_ifnextchar:Nnn
  \array{*\c@MaxMatrixCols #1}}
\def\MT_matrix_end:{\endarray \hskip -\arraycolsep}
%    \end{macrocode}
%  \end{macro}
%  \end{macro}
%  Before we define the environments we better make sure that spaces
%  before the optional argument is disallowed. Else a user who types
%  \begin{verbatim}
%  \[
%    \begin{pmatrix*}
%      [c] & a \\
%       b  & d
%    \end{pmatrix*}
%  \]
%  \end{verbatim}
%  will lose the \texttt{[c]}!
%    \begin{macrocode}
\MaybeMHPrecedingSpacesOff
%    \end{macrocode}
%  \begin{environment}{matrix*}
%  This environment is just like \env{matrix} only it takes an
%  optional argument specifying the column type.
%    \begin{macrocode}
\newenvironment{matrix*}[1][c]
  {\MT_matrix_begin:N #1}
  {\MT_matrix_end:}
%    \end{macrocode}
%  \end{environment}
%  \begin{environment}{pmatrix*}
%  \begin{environment}{bmatrix*}
%  \begin{environment}{Bmatrix*}
%  \begin{environment}{vmatrix*}
%  \begin{environment}{Vmatrix*}
%  Then starred versions of the other \AmS{} matrices.
%    \begin{macrocode}
\newenvironment{pmatrix*}[1][c]
  {\left(\MT_matrix_begin:N #1}
  {\MT_matrix_end:\right)}
\newenvironment{bmatrix*}[1][c]
  {\left[\MT_matrix_begin:N #1}
  {\MT_matrix_end:\right]}
\newenvironment{Bmatrix*}[1][c]
  {\left\lbrace\MT_matrix_begin:N #1}
  {\MT_matrix_end:\right\rbrace}
\newenvironment{vmatrix*}[1][c]
  {\left\lvert\MT_matrix_begin:N #1}
  {\MT_matrix_end:\right\rvert}
\newenvironment{Vmatrix*}[1][c]
  {\left\lVert\MT_matrix_begin:N #1}
  {\MT_matrix_end:\right\lVert}
%    \end{macrocode}
%  \end{environment}
%  \end{environment}
%  \end{environment}
%  \end{environment}
%  \end{environment}
%
% \changes{v1.10}{2011/02/12}{Added the code below, courtesy of Rasmus Villemoes}
% Now we are at it why not provide fenced versions of the
% \env{smallmatrix} construction as well. We will only provide a
% version that can be adjusted as \env{matrix*} above, thus we keep
% the * in the name. The implementation is courtesy of Rasmus
% Villemoes. Rasmus also suggested making the default alignment in
% these environments globally adjustable, so we did
% (\texttt{smallmatrix-align=c} by default). It \emph{is} possible to
% do something similar with the large matrix environments, but that
% might cause problems with the \texttt{array} package, thus for now
% we lease that feature alone.
%
% The base code is a variation over the original \env{smallmatrix}
% environmetn fround in \texttt{amsmath}, thus we will not comment it further.
% 
% TODO: make the code check that the optional argument is either
% \texttt{c}, \texttt{l} or \texttt{r}.
%    \begin{macrocode}
\def\MT_smallmatrix_begin:N #1{%
  \Let@\restore@math@cr\default@tag
  \baselineskip6\ex@ \lineskip1.5\ex@ \lineskiplimit\lineskip
  \csname MT_smallmatrix_#1_begin:\endcsname
}
\def\MT_smallmatrix_end:{\crcr\egroup\egroup\MT_smallmatrix_inner_space:}
\def\MT_smallmatrix_l_begin:{\null\MT_smallmatrix_inner_space:\vcenter\bgroup
  \ialign\bgroup$\m@th\scriptstyle##$\hfil&&\thickspace
  $\m@th\scriptstyle##$\hfil\crcr
}
\def\MT_smallmatrix_c_begin:{\null\MT_smallmatrix_inner_space:\vcenter\bgroup
  \ialign\bgroup\hfil$\m@th\scriptstyle##$\hfil&&\thickspace\hfil
  $\m@th\scriptstyle##$\hfil\crcr
}
\def\MT_smallmatrix_r_begin:{\null\MT_smallmatrix_inner_space:\vcenter\bgroup
  \ialign\bgroup\hfil$\m@th\scriptstyle##$&&\thickspace\hfil
  $\m@th\scriptstyle##$\crcr
}
\newenvironment{smallmatrix*}[1][\MT_smallmatrix_default_align:]
  {\MT_smallmatrix_begin:N #1}
  {\MT_smallmatrix_end:}
%    \end{macrocode}
% We would like to keep to the tradition of the \verb?Xmatrix? and
% \verb?Xmatrix*? macros we added earlier, since most code is similar
% we define them using a constructor macro. We also apply the trick
% used within \verb?\DeclarePairedDelimiter(X)? such that \verb?\left?
% \verb?\right?  constructions produce spacings corresponding to
% \verb?\mathopen? and \verb?\mathclose?.
%    \begin{macrocode}
\def\MT_fenced_sm_generator:nnn #1#2#3{%
  \@ifundefined{#1}{%
    \newenvironment{#1}
    {\@nameuse{#1hook}\mathopen{}\mathclose\bgroup\left#2\MT_smallmatrix_begin:N c}%
      {\MT_smallmatrix_end:\aftergroup\egroup\right#3}%
  }{}%
  \@ifundefined{#1*}{%
    \newenvironment{#1*}[1][\MT_smallmatrix_default_align:]%
    {\@nameuse{#1hook}\mathopen{}\mathclose\bgroup\left#2\MT_smallmatrix_begin:N ##1}%
      {\MT_smallmatrix_end:\aftergroup\egroup\right#3}%
  }{}%
}
\MT_fenced_sm_generator:nnn{psmallmatrix}()
\MT_fenced_sm_generator:nnn{bsmallmatrix}[]
\MT_fenced_sm_generator:nnn{Bsmallmatrix}\lbrace\rbrace
\MT_fenced_sm_generator:nnn{vsmallmatrix}\lvert\rvert
\MT_fenced_sm_generator:nnn{Vsmallmatrix}\lVert\rVert
%    \end{macrocode}
% 
% The options associated with this.
%    \begin{macrocode}
\define@key{\MT_options_name:}
  {smallmatrix-align}{\def\MT_smallmatrix_default_align:{#1}}
\define@key{\MT_options_name:}
  {smallmatrix-inner-space}{\def\MT_smallmatrix_inner_space:{#1}}
\setkeys{\MT_options_name:}{
  smallmatrix-align=c,
  smallmatrix-inner-space=\,
}

%    \end{macrocode}
%  Restore the usual spacing behavior.
%    \begin{macrocode}
\MHPrecedingSpacesOn
%    \end{macrocode}
%
%
%
%  \subsection{Smashing an operator with limits}
%
%  \begin{macro}{\smashoperator}
%  The user command. Define \cs{MT_smop_use:NNNNN} to be one of the
%  specialized commands \cs{MT_smop_smash_l:NNNNN},
%  \cs{MT_smop_smash_r:NNNNN}, or the default
%  \cs{MT_smop_smash_lr:NNNNN}.
%  \changes{v1.25}{2021/03/04}{Verify that opt arg corresponds to
%  something predefined}
%    \begin{macrocode}
\newcommand*\smashoperator[2][lr]{
  \@ifundefined{MT_smop_smash_#1:NNNNN}{
    \PackageError{mathtools}{Opt~ arg~ '#1'~ for~ \@backslashchar smashoperator~
      is~ not~ supported.\MessageBreak Use~ l,~r~or~lr~ (default)}{}
  }{
    \def\MT_smop_use:NNNNN {\@nameuse{MT_smop_smash_#1:NNNNN}}
    \toks@{#2}
    \expandafter\MT_smop_get_args:wwwNnNn
    \the\toks@\@nil\@nil\@nil\@nil\@nil\@nil\@@nil
  }
}
%    \end{macrocode}
%  \end{macro}
%  \begin{macro}{\MT_smop_remove_nil_vi:N}
%  \begin{macro}{\MT_smop_mathop:n}
%  \begin{macro}{\MT_smop_limits:}
%  Some helper functions.
%    \begin{macrocode}
\def\MT_smop_remove_nil_vi:N #1\@nil\@nil\@nil\@nil\@nil\@nil{#1}
\def\MT_smop_mathop:n {\mathop}
\def\MT_smop_limits: {\limits}
%    \end{macrocode}
%  \end{macro}
%  \end{macro}
%  \end{macro}
%  Some conditionals.
%    \begin{macrocode}
\MH_new_boolean:n {smop_one}
\MH_new_boolean:n {smop_two}
%    \end{macrocode}
%  \begin{macro}{\MT_smop_get_args:wwwNnNn}
%  The argument stripping. There are three different valid types of
%  input:
%  \begin{enumerate}
%    \item An operator with neither subscript nor superscript.
%    \item An operator with one subscript or superscript.
%    \item An operator with both subscript and superscript.
%  \end{enumerate}
%  Additionally an operator can be either a single macro as in
%  \cs{sum} or in \cs{mathop}\arg{A} and people might be tempted to
%  put a \cs{limits} after the operator, even though it's not
%  necessary. Thus the input with most tokens would be something like
%  \begin{verbatim}
%    \mathop{TTT}\limits_{sub}^{sup}
%  \end{verbatim}
%  Therefore we have to scan for seven arguments, but there might
%  only be one actually. So let's list the possible situations:
%  \begin{enumerate}
%    \item \verb|\mathop{TTT}\limits_{subsub}^{supsup}|
%    \item \verb|\mathop{TTT}_{subsub}^{supsup}|
%    \item \verb|\sum\limits_{subsub}^{supsup}|
%    \item \verb|\sum_{subsub}^{supsup}|
%  \end{enumerate}
%  Furthermore the |_{subsub}^{supsup}| part can also just be
%  |_{subsub}| or empty.
%    \begin{macrocode}
\def\MT_smop_get_args:wwwNnNn #1#2#3#4#5#6#7\@@nil{%
  \begingroup
    \def\MT_smop_arg_A: {#1} \def\MT_smop_arg_B: {#2}
    \def\MT_smop_arg_C: {#3} \def\MT_smop_arg_D: {#4}
    \def\MT_smop_arg_E: {#5} \def\MT_smop_arg_F: {#6}
    \def\MT_smop_arg_G: {#7}
%    \end{macrocode}
%  Check if A is \cs{mathop}. If it is, we know that B is the argument
%  of the \cs{mathop}.
%  \changes{v1.19}{2017/03/31}{Added MH\_ prefix}
%    \begin{macrocode}
    \MH_if_meaning:NN \MT_smop_arg_A: \MT_smop_mathop:n
%    \end{macrocode}
%  If A was \cs{mathop} we check if C is \cs{limits}
%  \changes{v1.19}{2017/03/31}{Added MH\_ prefix}
%    \begin{macrocode}
      \MH_if_meaning:NN \MT_smop_arg_C:\MT_smop_limits:
        \def\MT_smop_final_arg_A:{#1{#2}}%
%    \end{macrocode}
%  Now we have something like \verb|\mathop{TTT}\limits|. Then check
%  if D is \cs{@nil}.
%  \changes{v1.19}{2017/03/31}{Added MH\_ prefix}
%    \begin{macrocode}
        \MH_if_meaning:NN \MT_smop_arg_D: \@nnil
        \MH_else:
          \MH_set_boolean_T:n {smop_one}
          \MH_let:NwN \MT_smop_final_arg_B: \MT_smop_arg_D:
          \MH_let:NwN \MT_smop_final_arg_C: \MT_smop_arg_E:
          \MH_if_meaning:NN \MT_smop_arg_F: \@nnil
          \MH_else:
            \MH_set_boolean_T:n {smop_two}
            \MH_let:NwN \MT_smop_final_arg_D: \MT_smop_arg_F:
            \edef\MT_smop_final_arg_E:
              {\expandafter\MT_smop_remove_nil_vi:N \MT_smop_arg_G: }
          \MH_fi:
        \MH_fi:
      \MH_else:
%    \end{macrocode}
%  Here we have something like \verb|\mathop{TTT}|. Still check
%  if D is \cs{@nil}.
%  \changes{v1.19}{2017/03/31}{Added MH\_ prefix}
%    \begin{macrocode}
        \def\MT_smop_final_arg_A:{#1{#2}}%
        \MH_if_meaning:NN \MT_smop_arg_D: \@nnil
        \MH_else:
          \MH_set_boolean_T:n {smop_one}
          \MH_let:NwN \MT_smop_final_arg_B: \MT_smop_arg_C:
          \MH_let:NwN \MT_smop_final_arg_C: \MT_smop_arg_D:
          \MH_if_meaning:NN \MT_smop_arg_F: \@nnil
          \MH_else:
            \MH_set_boolean_T:n {smop_two}
            \MH_let:NwN \MT_smop_final_arg_D: \MT_smop_arg_E:
            \MH_let:NwN \MT_smop_final_arg_E: \MT_smop_arg_F:
          \MH_fi:
        \MH_fi:
      \MH_fi:
%    \end{macrocode}
%  If A was not \cs{mathop}, it is an operator in itself, so we check
%  if B is \cs{limits}
%  \changes{v1.19}{2017/03/31}{Added MH\_ prefix}
%    \begin{macrocode}
    \MH_else:
      \MH_if_meaning:NN \MT_smop_arg_B:\MT_smop_limits:
        \def\MT_smop_final_arg_A:{#1}%
        \MH_if_meaning:NN \MT_smop_arg_D: \@nnil
        \MH_else:
          \MH_set_boolean_T:n {smop_one}
          \MH_let:NwN \MT_smop_final_arg_B: \MT_smop_arg_C:
          \MH_let:NwN \MT_smop_final_arg_C: \MT_smop_arg_D:
          \MH_if_meaning:NN \MT_smop_arg_F: \@nnil
          \MH_else:
            \MH_set_boolean_T:n {smop_two}
            \MH_let:NwN \MT_smop_final_arg_D: \MT_smop_arg_E:
            \MH_let:NwN \MT_smop_final_arg_E: \MT_smop_arg_F:
          \MH_fi:
        \MH_fi:
      \MH_else:
%    \end{macrocode}
%  No \cs{limits} was found, so we already have the right input. Just
%  forget about the last two arguments.
%  \changes{v1.19}{2017/03/31}{Added MH\_ prefix}
%    \begin{macrocode}
        \def\MT_smop_final_arg_A:{#1}%
        \MH_if_meaning:NN \MT_smop_arg_C: \@nnil
        \MH_else:
          \MH_set_boolean_T:n {smop_one}
          \MH_let:NwN \MT_smop_final_arg_B: \MT_smop_arg_B:
          \MH_let:NwN \MT_smop_final_arg_C: \MT_smop_arg_C:
          \MH_if_meaning:NN \MT_smop_arg_D: \@nnil
          \MH_else:
            \MH_set_boolean_T:n {smop_two}
            \MH_let:NwN \MT_smop_final_arg_D: \MT_smop_arg_D:
            \MH_let:NwN \MT_smop_final_arg_E: \MT_smop_arg_E:
          \MH_fi:
        \MH_fi:
      \MH_fi:
    \MH_fi:
%    \end{macrocode}
%  No reason to measure if there's no sub or sup.
%    \begin{macrocode}
    \MH_if_boolean:nT {smop_one}{
      \MT_smop_measure:NNNNN
      \MT_smop_final_arg_A: \MT_smop_final_arg_B: \MT_smop_final_arg_C:
      \MT_smop_final_arg_D: \MT_smop_final_arg_E:
    }
    \MT_smop_use:NNNNN
      \MT_smop_final_arg_A: \MT_smop_final_arg_B: \MT_smop_final_arg_C:
      \MT_smop_final_arg_D: \MT_smop_final_arg_E:
  \endgroup
}
%    \end{macrocode}
%  \end{macro}
%  Typeset what is necessary and ignore width of sub and sup:
%    \begin{macrocode}
\def\MT_smop_needed_args:NNNNN #1#2#3#4#5{%
  \displaystyle #1
  \MH_if_boolean:nT {smop_one}{
%    \end{macrocode}
%  Let's use the internal versions of \cs{crampedclap} now that we now
%  it is set in \cs{scriptstyle}.
%    \begin{macrocode}
    \limits#2{\MT_cramped_clap_internal:Nn \scriptstyle{#3}}
    \MH_if_boolean:nT {smop_two}{
      #4{\MT_cramped_clap_internal:Nn \scriptstyle{#5}}
    }
  }
}
%    \end{macrocode}
%  Measure the natural width. \cs{@tempdima} holds the dimen we need to
%  adjust it all with.
%    \begin{macrocode}
\def\MT_smop_measure:NNNNN #1#2#3#4#5{%
  \MH_let:NwN \MT_saved_mathclap:Nn \MT_cramped_clap_internal:Nn
  \MH_let:NwN \MT_cramped_clap_internal:Nn \@secondoftwo
  \sbox\z@{$\m@th\MT_smop_needed_args:NNNNN #1#2#3#4#5$}
  \MH_let:NwN \MT_cramped_clap_internal:Nn \MT_saved_mathclap:Nn
  \sbox\tw@{$\m@th\displaystyle#1$}
  \@tempdima=.5\wd0
  \advance\@tempdima-.5\wd2
}
%    \end{macrocode}
%  The `l' variant
%    \begin{macrocode}
\def\MT_smop_smash_l:NNNNN #1#2#3#4#5{
  \MT_smop_needed_args:NNNNN #1#2#3#4#5\kern\@tempdima
}
%    \end{macrocode}
%  The `r' variant
%    \begin{macrocode}
\def\MT_smop_smash_r:NNNNN #1#2#3#4#5{
  \kern\@tempdima\MT_smop_needed_args:NNNNN #1#2#3#4#5
}
%    \end{macrocode}
%  The `lr' variant
% \changes{v1.25}{2021/03/04}{add `rl' as an alias for `lr'}
%    \begin{macrocode}
\def\MT_smop_smash_lr:NNNNN #1#2#3#4#5{
  \MT_smop_needed_args:NNNNN #1#2#3#4#5
}
%    \end{macrocode}
% Add `rl' as an alias for `lr'
%    \begin{macrocode}
\MH_let:NwN \MT_smop_smash_rl:NNNNN \MT_smop_smash_lr:NNNNN
%    \end{macrocode}
% 
%
%  \subsection{Adjusting limits}
%
%
%  \begin{macro}{\MT_vphantom:Nn}
%  \begin{macro}{\MT_hphantom:Nn}
%  \begin{macro}{\MT_phantom:Nn}
%  \begin{macro}{\MT_internal_phantom:N}
%  The main advantage of \cs{phantom} et al., is the ability to
%  choose the right size automatically, but it requires the input to
%  be typeset four times. Since we will need to have a \cs{cramped}
%  inside a \cs{vphantom} it is much, much faster to choose the style
%  ourselves (we already know it). These macros make it possible.
%    \begin{macrocode}
\def\MT_vphantom:Nn {\v@true\h@false\MT_internal_phantom:N}
\def\MT_hphantom:Nn {\v@false\h@true\MT_internal_phantom:N}
\def\MT_phantom:Nn {\v@true\h@true\MT_internal_phantom:N}
\def\MT_internal_phantom:N #1{
  \ifmmode
    \expandafter\mathph@nt\expandafter#1
  \else
    \expandafter\makeph@nt
  \fi
}
%    \end{macrocode}
%  \end{macro}
%  \end{macro}
%  \end{macro}
%  \end{macro}
%
%  \begin{macro}{\adjustlimits}
%  This is for making sure limits line up on two consecutive
%  operators.
%    \begin{macrocode}
\newcommand*\adjustlimits[6]{
%    \end{macrocode}
%  We measure the two operators and save the difference of their
%  depths.
%    \begin{macrocode}
  \sbox\z@{$\m@th \displaystyle #1$}
  \sbox\tw@{$\m@th \displaystyle #4$}
  \@tempdima=\dp\z@ \advance\@tempdima-\dp\tw@
%    \end{macrocode}
%  We force \cs{displaystyle} for the operator and \cs{scripstyle}
%  for the limit. If we make use of the regular \cs{smash},
%  \cs{vphantom}, and \cs{cramped} macros, and let \TeX{} choose the
%  right style for each one of them, we get a lot of redundant code
%  as we have no need for the combination
%  $(\cs{displaystyle},\cs{textstyle})$ etc. Only
%  $(\cs{scriptstyle},\cs{scriptstyle})$ is useful.
%  \changes{v1.19}{2017/03/31}{Added MH\_ prefix}
%    \begin{macrocode}
  \MH_if_dim:w \@tempdima>\z@
    \mathop{#1}\limits#2{#3}
  \MH_else:
    \mathop{#1\MT_vphantom:Nn \displaystyle{#4}}\limits
    #2{
        \def\finsm@sh{\ht\z@\z@ \box\z@}
        \mathsm@sh\scriptstyle{\MT_cramped_internal:Nn \scriptstyle{#3}}
        \MT_vphantom:Nn \scriptstyle
          {\MT_cramped_internal:Nn \scriptstyle{#6}}
    }
  \MH_fi:
  \MH_if_dim:w \@tempdima>\z@
    \mathop{#4\MT_vphantom:Nn \displaystyle{#1}}\limits
    #5
    {
      \MT_vphantom:Nn \scriptstyle
        {\MT_cramped_internal:Nn \scriptstyle{#3}}
      \def\finsm@sh{\ht\z@\z@ \box\z@}
      \mathsm@sh\scriptstyle{\MT_cramped_internal:Nn \scriptstyle{#6}}
    }
  \MH_else:
    \mathop{#4}\limits#5{#6}
  \MH_fi:
}
%    \end{macrocode}
%  \end{macro}
% 
%  \subsection{Swapping above display skip}
%
% \begin{macro}{\SwapAboveDisplaySkip}
%   This macro is intended to be used at the start of \AmS\
%   environments, in order to force it to use
%   \cs{abovedisplayshortskip} instead of \cs{abovedisplayskip} above
%   the displayed math. Because of the use of \cs{noalign} it will not
%   work inside \env{equation} or \env{multline}.
%    \begin{macrocode}
\newcommand\SwapAboveDisplaySkip{%
  \noalign{\vskip-\abovedisplayskip\vskip\abovedisplayshortskip}
}

%    \end{macrocode}
%   
% \end{macro}
%
%  \subsection{An aid to alignment}
%
% \begin{macro}{\MoveEqLeft}
%   \changes{v1.05}{2008/06/05}{Added \cs{MoveEqLeft} (daleif)}
%   \changes{v1.05b}{2008/06/18}{We don't need \cs{setlength} here
%     (daleif), after discussion about \cs{global} and \cs{setlength}
%     on ctt} 
%   \changes{v1.12}{2011/06/12}{We don't even need lengths. GL
%   suggested on ctt to just apply them directly.}
%   This is a very simple macro, we `move' a line in an
%   alignment backwards in order to simulate that all subsequent lines
%   have been indented. Note that simply using \verb+\kern-2m+ after
%   the \verb+&+ is not enough, then the alignemnt environment never
%   detects that there is anything (though simulated) in the cell
%   before the \verb+&+.
%    \begin{macrocode}
\newcommand\MoveEqLeft[1][2]{\kern #1em  &   \kern -#1em}
%    \end{macrocode}
% \end{macro}
%
% \begin{macro}{\Aboxed}
% \changes{v1.08}{2010/06/29}{Added \cs{Aboxed}}
% The idea from \cs{MoveEqLeft} can be used for other things. Here we
% create a macro that will allow a user to box an equation inside an
% alignment.
% \changes{v1.12}{2011/08/17}{\cs{Aboxed} reimplemented, cudos to GL}
%    \begin{macrocode}
\newcommand\Aboxed[1]{\let\bgroup{\romannumeral-`}\@Aboxed#1&&\ENDDNE}
%    \end{macrocode}
% \sloppy
% The macro has been reimplemented courtesy of Florent Chervet out of
% a posting on ctt,  \url{https://groups.google.com/group/comp.text.tex/browse_thread/thread/5d66395f2a1b5134/93fd9661484bd8d8?#93fd9661484bd8d8}
%    \begin{macrocode}
\def\@Aboxed#1&#2&#3\ENDDNE{%
  \ifnum0=`{}\fi \setbox \z@
    \hbox{$\displaystyle#1{}\m@th$\kern\fboxsep \kern\fboxrule }%
    \edef\@tempa {\kern  \wd\z@ &\kern -\the\wd\z@ \fboxsep
        \the\fboxsep \fboxrule \the\fboxrule }\@tempa \boxed {#1#2}%
} 
%    \end{macrocode}
% \end{macro}
%
% \begin{macro}{\ArrowBetweenLines}
%   \changes{v1.05}{2008/06/05}{Added \cs{ArrowBetweenLines} as it
%   belongs here and not just in my \LaTeX book (daleif)}
%   ????Implementation notes are needed????
%    \begin{macrocode}
\MHInternalSyntaxOff
\def\ArrowBetweenLines{\relax
  \iffalse{\fi\ifnum0=`}\fi
  \@ifstar{\ArrowBetweenLines@auxI{00}}{\ArrowBetweenLines@auxI{01}}}
\def\ArrowBetweenLines@auxI#1{%
  \@ifnextchar[%
  {\ArrowBetweenLines@auxII{#1}}%
  {\ArrowBetweenLines@auxII{#1}[\Updownarrow]}}
\def\ArrowBetweenLines@auxII#1[#2]{%
  \ifnum0=`{\fi \iffalse}\fi
%    \end{macrocode}
% It turns out that for some reason the \cs{crcr} (next) removes the
% automatic equation number replacement. The replacement hack seems to
% the trick, though I have no idea why \cs{crcr} broke things (/daleif).
% \changes{v1.08}{2010/06/15}{fixed eq num replacement bug}
%    \begin{macrocode}
%  \crcr
    \expandafter\in@\expandafter{\@currenvir}%
      {alignedat,aligned,gathered}%
      \ifin@ \else
      \notag
      \fi%
   \\
  \noalign{\nobreak\vskip-\baselineskip\vskip-\lineskip}%
  \noalign{\expandafter\in@\expandafter{\@currenvir}%
      {alignedat,aligned,gathered}%
      \ifin@ \else\notag\fi%
  }%
  \if#1 &&\quad #2\else #2\quad\fi
  \\\noalign{\nobreak\vskip-\lineskip}}

\MHInternalSyntaxOn
%    \end{macrocode}
% \end{macro}
%
%
% \subsection{Centered vertical dots}
% 
% Doing a \verb?\vdots? centered within a different sized box, is
% rather easy with the tools available. Note that it does \emph{not}
% check for the style we are running in, thus do not expect this to
% work well within \verb?\scriptstyle? and smaller. Basically we
% create a box of a width corresponding to \verb?{}#1{}? and center
% the \verb?\vdots? within it.
%    \begin{macrocode}
\newcommand\vdotswithin[1]{%
  {\mathmakebox[\widthof{\ensuremath{{}#1{}}}][c]{{\vdots}}}}
%    \end{macrocode}
% Next we are inspired by \verb?\ArrowBetweenLines? and provide a
% costruction to be used within alignments with much less vertical
% space above and below. 
%
% First in order to support \env{spreadlines} we need to store the
% original value of \verb?\jot? (and hope the user does not mess with it).
%    \begin{macrocode}
\newlength\origjot
\setlength\origjot{\jot}
%    \end{macrocode}
% Next define how much we spacing we flush out, and make this user adjustable.
%    \begin{macrocode}
\newdimen\l_MT_shortvdotswithinadjustabove_dim
\newdimen\l_MT_shortvdotswithinadjustbelow_dim
\define@key{\MT_options_name:}
  {shortvdotsadjustabove}{\setlength\l_MT_shortvdotswithinadjustabove_dim{#1}}
\define@key{\MT_options_name:}
  {shortvdotsadjustbelow}{\setlength\l_MT_shortvdotswithinadjustbelow_dim{#1}}
%    \end{macrocode}
% The actual defaults we found by trail and error.
%    \begin{macrocode}
\setkeys{\MT_options_name:}{
  shortvdotsadjustabove=2.15\origjot,
  shortvdotsadjustbelow=\origjot
}
%    \end{macrocode}
% The user macro comes in two versions, starred version corresponding
% to alignment \emph{before} and \verb?&? and a non-starred version
% with alignment \emph{after} \verb?&?.
%    \begin{macrocode}
\def\shortvdotswithin{\relax
  \@ifstar{\MT_svwi_aux:nn{00}}{\MT_svwi_aux:nn{01}}}
\def\MT_svwi_aux:nn #1#2{
  \MTFlushSpaceAbove
  \if#1 \vdotswithin{#2}& \else &\vdotswithin{#2}  \fi
  \MTFlushSpaceBelow
}
%    \end{macrocode}
% We will need a way to remove any tags (eq. numbers) on the
% \verb?\vdots? line. We cannot use the method used by
% \verb?\ArrowBetweenLines? so we use inspiration from
% \texttt{etoolbox}.  
%    \begin{macrocode}
\def\MT_remove_tag_unless_inner:n #1{%
  \begingroup
  \def\etb@tempa##1|#1|##2\MT@END{\endgroup
    \ifx\@empty##2\@empty\notag\fi}%
  \expandafter\etb@tempa\expandafter|alignedat|aligned|split|#1|\MT@END}
  %| emacs
%    \end{macrocode}
% These macros take care of removing the space above or below. Since
% these may be useful for the user in very special cases, we provide
% them as separate macros.
%    \begin{macrocode}
\newcommand\MTFlushSpaceAbove{  
  \expandafter\MT_remove_tag_unless_inner:n\expandafter{\@currenvir}
  \\
  \noalign{%
    \nobreak\vskip-\baselineskip\vskip-\lineskip%
      \vskip-\l_MT_shortvdotswithinadjustabove_dim 
      \vskip-\origjot
      \vskip\jot
  }%
  \noalign{
    \expandafter\MT_remove_tag_unless_inner:n\expandafter{\@currenvir}
  }
}
\newcommand\MTFlushSpaceBelow{
  \\\noalign{%
    \nobreak\vskip-\lineskip
    \vskip-\l_MT_shortvdotswithinadjustbelow_dim
    \vskip-\origjot
    \vskip\jot
  }
}

%    \end{macrocode}
%
%
%  \section{A few extra symbols}
%
%  Most math font sets are missing three symbols: \cs{nuparrow},
%  \cs{ndownarrow} and \cs{bigtimes}. We provide \emph{simulated}
%  versions of these symbols in case they are missing. 
%
%  \subsection{Negated up- and down arrows} 
%  
%  Note that the \cs{nuparrow} and the \cs{ndownarrow} are made from
%  \cs{nrightarrow} and \cs{nleftarrow}, so these have to be
%  present. If they are not, we throw an error at use. The
%  implementation details are due to Enrico Gregorio
%  (\url{http://groups.google.com/group/comp.text.tex/msg/689cc8bd604fdb51}),
%  the basic idea is to reflect and rotate existing negated
%  arrows. Note that the reflection and rotation will not show up i
%  most DVI previewers.
% \begin{macro}{\MH_nrotarrow:NN}
% \changes{v1.07}{2008/08/11}{Added support for \cs{nuparrow} and \cs{ndownaddow}}
%  First a common construction macro.
%    \begin{macrocode}
\def\MH_nrotarrow:NN #1#2{%
  \setbox0=\hbox{$\m@th#1\uparrow$}\dimen0=\dp0
  \setbox0=\hbox{%
    \reflectbox{\rotatebox[origin=c]{90}{$\m@th#1\mkern2.22mu #2$}}}%
  \dp0=\dimen0 \box0 \mkern2.3965mu
}
%    \end{macrocode}
% \end{macro}
% The negated arrows are then made using this macro on respectively
% \cs{nrightarrow} and \cs{nleftarrow}
% \begin{macro}{\MH_nuparrow:}
% \begin{macro}{\MH_ndownarrow:}
%    \begin{macrocode}
\def\MH_nuparrow: {%
  \mathrel{\mathpalette\MH_nrotarrow:NN\nrightarrow} }
\def\MH_ndownarrow: {%
  \mathrel{\mathpalette\MH_nrotarrow:NN\nleftarrow} }
%    \end{macrocode}
% \end{macro}
% \end{macro}
% \begin{macro}{\nuparrow}
% \begin{macro}{\ndownarrow}
%   Next we provide \cs{nuparrow} and \cs{ndownarrow} at begin
%   document. Since they depend on \cs{nrightarrow} and
%   \cs{nleftarrow} we test for these and let the macros throw an
%   error if they are missing.
% \changes{v1.08b}{2010/07/21}{Moved graphicx loading down here such
% that we do not get into option clash problems}
%    \begin{macrocode}
\AtBeginDocument{%
  \RequirePackage{graphicx}%
  \@ifundefined{nrightarrow}{%
    \providecommand\nuparrow{%
      \PackageError{mathtools}{\string\nuparrow\space~ is~
        constructed~ from~ \string\nrightarrow,~ which~ is~ not~
        provided.~ Please~ load~ the~ amssymb~ package~ or~ similar}{}
    }}{ \providecommand\nuparrow{\MH_nuparrow:}}
  \@ifundefined{nleftarrow}{%
    \providecommand\ndownarrow{%
      \PackageError{mathtools}{\string\ndownarrow\space~ is~
        constructed~ from~ \string\nleftarrow,~ which~ is~ not~
        provided.~ Please~ load~ the~ amssymb~ package~ or~ similar}{}
    }}{ \providecommand\ndownarrow{\MH_ndownarrow:}} }
%    \end{macrocode}
% \end{macro}
% \end{macro}
% 
%
%  \subsection{Providing bigtimes}
%
%  The idea is to use the original \cs{times} and then scale it
%  accordingly. Again the implementation details have been improved by
%  Enrico Gregorio
%  (\url{http://groups.google.com/group/comp.text.tex/msg/9685c9405df2ff94}). 
%
% \begin{macro}{\MH_bigtimes_scaler:N}
% \begin{macro}{\MH_bigtimes_inner:}
% \begin{macro}{\MH_csym_bigtimes:}
% \changes{v1.07}{2008/08/11}{Added support for \cs{bigtimes}}
%    \begin{macrocode}
\def\MH_bigtimes_scaler:N #1{%
  \vcenter{\hbox{#1$\m@th\mkern-2mu\times\mkern-2mu$}}}
%    \end{macrocode}
%  This is then combined with \cs{mathchoice} to form the inner parts
%  of the macro
%    \begin{macrocode}
\def\MH_bigtimes_inner: {
  \mathchoice{\MH_bigtimes_scaler:N \huge}         % display style
             {\MH_bigtimes_scaler:N \LARGE}        % text style
             {\MH_bigtimes_scaler:N {}}            % script style
             {\MH_bigtimes_scaler:N \footnotesize} % script script style
}
%    \end{macrocode}
%  And thus the internal prepresentaion of the \cs{bigtimes} macro.
%    \begin{macrocode}
\def\MH_csym_bigtimes: {\mathop{\MH_bigtimes_inner:}\displaylimits}
%    \end{macrocode}
% \end{macro}
% \end{macro}
% \end{macro}
% \begin{macro}{\bigtimes}
%   In the end we provide \cs{bigtimes} if otherwise not defined.
%    \begin{macrocode}
\AtBeginDocument{
  \providecommand\bigtimes{\MH_csym_bigtimes:}
}
%    \end{macrocode}
% \end{macro}
%
%  \section{Macros by other people}
%
%  \subsection{Intertext and short intertext}
%
%  It turns out that \cs{intertext} use a bit too much
%  space. Especially noticable if combined with the \env{spreadlines}
%  environment, the extra space is also applied above and below
%  \cs{intertext}, which ends up looking unproffesional.  Chung-chieh
%  Shan
%  (\url{http://conway.rutgers.edu/~ccshan/wiki/blog/posts/Beyond_amsmath/})
%  via Tobias Weh suggested a fix. We apply it here, but also keep the
%  original \cs{intertext} in case a user would rather want it.
% \begin{macro}{\MT_orig_intertext:}
% \begin{macro}{\MT_intertext:}
% \changes{v1.13}{2012/08/19}{\cs{l_MT_X_intertext_dim} renamed to 
% \cs{l_MT_X_intertext_sep}}
% \begin{macro}{\l_MT_above_intertext_sep}
% \begin{macro}{\l_MT_below_intertext_sep}
%  First store the originam Ams version.
%    \begin{macrocode}
\MH_let:NwN \MT_orig_intertext: \intertext@
%    \end{macrocode}
% And then for some reconfiguration. First a few lengths
% \changes{v1.13}{2012/08/19}{Fixed typos and changed the names}
%    \begin{macrocode}
\newdimen\l_MT_above_intertext_sep
\newdimen\l_MT_below_intertext_sep
\define@key{\MT_options_name:}
  {aboveintertextdim}{\setlength\l_MT_above_intertext_sep{#1}}
\define@key{\MT_options_name:}
  {belowintertextdim}{\setlength\l_MT_below_intertext_sep{#1}}
\define@key{\MT_options_name:}
  {above-intertext-dim}{\setlength\l_MT_above_intertext_sep{#1}}
\define@key{\MT_options_name:}
  {below-intertext-dim}{\setlength\l_MT_below_intertext_sep{#1}}
\define@key{\MT_options_name:}
  {above-intertext-sep}{\setlength\l_MT_above_intertext_sep{#1}}
\define@key{\MT_options_name:}
  {below-intertext-sep}{\setlength\l_MT_below_intertext_sep{#1}}
%    \end{macrocode}
% Their default values are zero. Now for our extended version of
% CCShan's solution.  
%    \begin{macrocode}
\def\MT_intertext: {%
  \def\intertext##1{%
    \ifvmode\else\\\@empty\fi
    \noalign{%
      \penalty\postdisplaypenalty\vskip\belowdisplayskip
      \vskip-\lineskiplimit      % CCS
      \vskip\normallineskiplimit % CCS
      \vskip\l_MT_above_intertext_sep
       \vbox{\normalbaselines
%    \end{macrocode}
% \changes{v1.17}{2015/06/17}{Added extra `in list' check}
%  Johannes Böttcher has reported a problem in \cs{intertext} in
%  relation to usage within lists or constructs that locally changes
%  the margin (often implemented as a list). Markus Kohm suggested to
%  test not only \cs{linewidth} against \cs{columnwidth} but also if
%  \cs{@totalleftmargin} is not zero. The implementation used is due to
%  David Carlisle.
% \changes{v1.19}{2016/05/26}{Added \cs{ignorespaces} after
% \cs{noindent}, bug inherited from \pkg{amsmath}}
%    \begin{macrocode}
         \ifdim
           \ifdim\@totalleftmargin=\z@
             \linewidth
           \else
             -\maxdimen
           \fi
         =\columnwidth
        \else \parshape\@ne \@totalleftmargin \linewidth
        \fi
        \noindent\ignorespaces##1\par}%
      \penalty\predisplaypenalty\vskip\abovedisplayskip%
      \vskip-\lineskiplimit      % CCS
      \vskip\normallineskiplimit % CCS
%    \end{macrocode}
% \changes{v1.21}{2018/01/08}{Typo, it should use \texttt{\_below\_}
% not \texttt{\_above\_}}
%    \begin{macrocode}
      \vskip\l_MT_below_intertext_sep
   }%
 }%
%    \end{macrocode}
%  We might as well hook on to this and only activate
%  \cs{shortintertext} whenever \cs{intertext} is active. This allows
%  us to get the same error messages as \cs{intertext}.
%  \changes{v1.19}{2017/03/31}{Added}
%    \begin{macrocode}
 % also activate \shortintertext, such that is is only available when
 % \intertext is 
 \MH_let:NwN \shortintertext \shortintertext@
}
%    \end{macrocode}
% And provide a key to switch
% \changes{v1.19}{2017/03/31}{Changed \cs{MT_orig_intertext_true:}} 
% Changed \cs{MT_orig_intertext_true:} to also activate
% \cs{shortintertext}, because when this is active, our change to
% \cs{intertext@} is not active. 
%    \begin{macrocode}
\def\MT_orig_intertext_true:  { 
  \MH_let:NwN \intertext@ \MT_orig_intertext: 
  \MH_let:NwN \shortintertext \shortintertext@
}
\def\MT_orig_intertext_false: { \MH_let:NwN \intertext@ \MT_intertext: }
\define@key{\MT_options_name:}{original-intertext}[true]{
  \@nameuse{MT_orig_intertext_#1:}
}
%    \end{macrocode}
% And use the new version as default.
%    \begin{macrocode}
\setkeys{\MT_options_name:}{
  original-intertext=false
}
%    \end{macrocode}
% 
% \end{macro}
% \end{macro}
% \end{macro}
% \end{macro}
%
%  Gabriel Zachmann, Donald Arseneau on comp.text.tex 2000/05/12-13
%  \begin{macro}{\shortintertext}
%  \begin{macro}{\MT_orig_shortintertext}
%  \begin{macro}{\MT_shortintertext}
%  \begin{macro}{\l_above_shortintertext_sep}
%  \begin{macro}{\l_below_shortintertext_sep}
%  This is like \cs{intertext} but uses shorter skips between the
%  math. Again this turned out to have the same problem as
%  \cs{intertext}, so we provide two versions.
%    \begin{macrocode}
\def\MT_orig_shortintertext:n #1{%
  \ifvmode\else\\\@empty\fi
  \noalign{%
    \penalty\postdisplaypenalty\vskip\abovedisplayshortskip
    \vbox{\normalbaselines
%    \end{macrocode}
% \changes{v1.17}{2015/06/17}{Added extra `in list' check}
% Same comment as for \cs{intertext} above.
% \changes{v1.19}{2016/05/26}{Added \cs{ignorespaces} after
% \cs{noindent}, bug inherited from \pkg{amsmath}}
%  \changes{v1.19}{2017/03/31}{Added MH\_ prefix}
%    \begin{macrocode}
      \MH_if_dim:w
        \MH_if_dim:w \@totalleftmargin=\z@
          \linewidth
        \MH_else:
          -\maxdimen
        \MH_fi:
        =\columnwidth
      \MH_else:
        \parshape\@ne \@totalleftmargin \linewidth
      \MH_fi:
      \noindent\ignorespaces#1\par}%
    \penalty\predisplaypenalty\vskip\abovedisplayshortskip%
  }%
}
%    \end{macrocode}
% Lengths like above
% \changes{v1.13}{2012/08/19}{The option was named differently in the
% manual. Also renamed to use the postfix \emph{sep} instead. Though
% the old names remain for compatibility.}
% \changes{v1.14}{2014/01/20}{Apparently did not check it good enough,
% the option should have shortintertext not short-intertext}
%    \begin{macrocode}
\newdimen\l_MT_above_shortintertext_sep
\newdimen\l_MT_below_shortintertext_sep
\define@key{\MT_options_name:}
  {aboveshortintertextdim}{\setlength \l_MT_above_shortintertext_sep{#1}}
\define@key{\MT_options_name:}
  {belowshortintertextdim}{\setlength \l_MT_below_shortintertext_sep{#1}}
\define@key{\MT_options_name:}
  {above-short-intertext-dim}{\setlength \l_MT_above_shortintertext_sep{#1}}
% old typo in manual
\define@key{\MT_options_name:}
  {below-short-intertext-dim}{\setlength \l_MT_below_shortintertext_sep{#1}}
\define@key{\MT_options_name:}
  {above-short-intertext-sep}{\setlength \l_MT_above_shortintertext_sep{#1}}
% old typo in manual
\define@key{\MT_options_name:}
  {below-short-intertext-sep}{\setlength \l_MT_below_shortintertext_sep{#1}}
\define@key{\MT_options_name:}
  {above-shortintertext-sep}{\setlength \l_MT_above_shortintertext_sep{#1}}
\define@key{\MT_options_name:}
  {below-shortintertext-sep}{\setlength \l_MT_below_shortintertext_sep{#1}}
%    \end{macrocode}
% Looks best with the `old' values of the original \cs{jot}
% setting. So we set them to 3pt each.
%    \begin{macrocode}
\setkeys{\MT_options_name:}{
  aboveshortintertextdim=3pt,
  belowshortintertextdim=3pt
}
%    \end{macrocode}
%  Next, just add the same as we did for \cs{intertext}
%    \begin{macrocode}
\def\MT_shortintertext:n #1{%
  \ifvmode\else\\\@empty\fi
  \noalign{%
    \penalty\postdisplaypenalty\vskip\abovedisplayshortskip
    \vskip-\lineskiplimit      
    \vskip\normallineskiplimit 
    \vskip\l_MT_above_shortintertext_sep
    \vbox{\normalbaselines
%    \end{macrocode}
% \changes{v1.17}{2015/06/17}{Added extra `in list' check}
% Same comment as for \cs{intertext} above.
% \changes{v1.19}{2016/05/26}{Added \cs{ignorespaces} after
% \cs{noindent}, bug inherited from \pkg{amsmath}}
%  \changes{v1.19}{2017/03/31}{Added MH\_ prefix}
%    \begin{macrocode}
      \MH_if_dim:w
        \MH_if_dim:w \@totalleftmargin=\z@
          \linewidth
        \MH_else:
          -\maxdimen
        \MH_fi:
        =\columnwidth
      \MH_else:
        \parshape\@ne \@totalleftmargin \linewidth
      \MH_fi:
      \noindent\ignorespaces#1\par}%
    \penalty\predisplaypenalty\vskip\abovedisplayshortskip%
    \vskip-\lineskiplimit      
    \vskip\normallineskiplimit 
    \vskip\l_MT_below_shortintertext_sep
  }%
}  
%    \end{macrocode}
% Next we need to be able to switch.
% \changes{v1.19}{2017/03/31}{Changed \cs{shortintertext} to be
% internal here.}
%    \begin{macrocode}
\def\MT_orig_shortintertext_true:  { \MH_let:NwN \shortintertext@ \MT_orig_shortintertext:n }
\def\MT_orig_shortintertext_false: { \MH_let:NwN \shortintertext@ \MT_shortintertext:n }
%    \end{macrocode}
% Next make \cs{shortintertext} a no-go everywhere. Thus \cs{shortintertext} is only
% available when \cs{intertext} is, or when
% \verb+original-intertext=true+ is used.
% \changes{v1.19}{2017/03/31}{Added}
%    \begin{macrocode}
\newcommand{\shortintertext}{\@amsmath@err{\Invalid@@\shortintertext}\@eha}
%    \end{macrocode}
%    \begin{macrocode}
\define@key{\MT_options_name:}{original-shortintertext}[true]{
  \@nameuse{MT_orig_shortintertext_#1:}
}
%    \end{macrocode}
% With the updated one as the default.
%    \begin{macrocode}
\setkeys{\MT_options_name:}{
  original-shortintertext=false
}
%    \end{macrocode}
%  \end{macro}
%  \end{macro}
%  \end{macro}
%  \end{macro}
%  \end{macro}
%
%  \subsection{Fine-tuning mathematical layout}
%
%  \subsubsection{A complement to \texttt{\textbackslash smash},
%  \texttt{\textbackslash llap}, and \texttt{\textbackslash rlap}}
%  \begin{macro}{\clap}
%  \begin{macro}{\mathllap}
%  \begin{macro}{\mathrlap}
%  \begin{macro}{\mathclap}
%  \begin{macro}{\MT_mathllap:Nn}
%  \begin{macro}{\MT_mathrlap:Nn}
%  \begin{macro}{\MT_mathclap:Nn}
%  First we'll \cs{provide} those macros (they are so simple that I
%  think other packages might define them as well).
% \changes{v1.15}{2021/03/04}{Add \cs{relax} before the \cs{ifx}'s to
% prevent corner case premature expansions}
%    \begin{macrocode}
\providecommand*\clap[1]{\hb@xt@\z@{\hss#1\hss}}
\providecommand*\mathllap[1][\@empty]{
  \relax\ifx\@empty#1\@empty
    \expandafter \mathpalette \expandafter \MT_mathllap:Nn
  \else
    \expandafter \MT_mathllap:Nn \expandafter #1
  \fi
}
\providecommand*\mathrlap[1][\@empty]{
  \relax\ifx\@empty#1\@empty
    \expandafter \mathpalette \expandafter \MT_mathrlap:Nn
  \else
    \expandafter \MT_mathrlap:Nn \expandafter #1
  \fi
}
\providecommand*\mathclap[1][\@empty]{
  \relax\ifx\@empty#1\@empty
    \expandafter \mathpalette \expandafter \MT_mathclap:Nn
  \else
    \expandafter \MT_mathclap:Nn \expandafter #1
  \fi
}
%    \end{macrocode}
%  We have to insert |{}| because we otherwise risk triggering a
%  ``feature'' in \TeX.
%    \begin{macrocode}
\def\MT_mathllap:Nn #1#2{{}\llap{$\m@th#1{#2}$}}
\def\MT_mathrlap:Nn #1#2{{}\rlap{$\m@th#1{#2}$}}
\def\MT_mathclap:Nn #1#2{{}\clap{$\m@th#1{#2}$}}
%    \end{macrocode}
%  \end{macro}
%  \end{macro}
%  \end{macro}
%  \end{macro}
%  \end{macro}
%  \end{macro}
%  \end{macro}
%  \begin{macro}{\mathmbox}
%  \begin{macro}{\MT_mathmbox:nn}
%  \begin{macro}{\mathmakebox}
%  \begin{macro}{\MT_mathmakebox_I:w}
%  \begin{macro}{\MT_mathmakebox_II:w}
%  \begin{macro}{\MT_mathmakebox_III:w}
%  Then the \cs{mathmbox}\marg{arg} and
%  \cs{mathmakebox}\oarg{width}\oarg{pos}\marg{arg} macros which are
%  very similar to \cs{mbox} and \cs{makebox}. The differences are:
%  \begin{itemize}
%    \item \meta{arg} is set in math mode of course.
%    \item No need for \cs{leavevmode} as we're in math mode.
%    \item No need to make them \cs{long} (we're still in math mode).
%    \item No need to support a picture version.
%  \end{itemize}
%  The first is easy.
%    \begin{macrocode}
\providecommand*\mathmbox{\mathpalette\MT_mathmbox:nn}
\def\MT_mathmbox:nn #1#2{\mbox{$\m@th#1#2$}}
%    \end{macrocode}
%  We scan for the optional arguments first.
%    \begin{macrocode}
\providecommand*\mathmakebox{
  \@ifnextchar[  \MT_mathmakebox_I:w
                 \mathmbox}
\def\MT_mathmakebox_I:w[#1]{%
  \@ifnextchar[  {\MT_mathmakebox_II:w[#1]}
                 {\MT_mathmakebox_II:w[#1][c]}}
%    \end{macrocode}
%  We had to get the optional arguments out of the way before calling
%  upon the powers of \cs{mathpalette}.
%    \begin{macrocode}
\def\MT_mathmakebox_II:w[#1][#2]{
  \mathpalette{\MT_mathmakebox_III:w[#1][#2]}}
\def\MT_mathmakebox_III:w[#1][#2]#3#4{%
  \@begin@tempboxa\hbox{$\m@th#3#4$}%
    \setlength\@tempdima{#1}%
    \hbox{\hb@xt@\@tempdima{\csname bm@#2\endcsname}}%
  \@end@tempboxa}
%    \end{macrocode}
%  \end{macro}
%  \end{macro}
%  \end{macro}
%  \end{macro}
%  \end{macro}
%  \end{macro}
%  \begin{macro}{\mathsm@sh}
%  Fix \cs{smash}.
%    \begin{macrocode}
\def\mathsm@sh#1#2{%
  \setbox\z@\hbox{$\m@th#1{#2}$}{}\finsm@sh}
%    \end{macrocode}
%  \end{macro}
%
%
%  \subsubsection{A cramped style}
%
%  comp.text.tex on 1992/07/21 by Michael Herschorn.
%  With speed-ups by the Grand Wizard himself as shown on
%  \begin{quote}\rightskip-\leftmargini
%  \url{http://mirrors.ctan.org/info/digests/tex-implementors/042}
%  \end{quote}
%  The (better) user interface by the author.
%
%  \begin{macro}{\cramped}
%  Make sure the expansion is timed correctly.
% \changes{v1.15}{2021/03/04}{Add \cs{relax} before the \cs{ifx}'s to
% prevent corner case premature expansions}
%    \begin{macrocode}
\providecommand*\cramped[1][\@empty]{
  \relax\ifx\@empty#1\@empty
    \expandafter \mathpalette \expandafter \MT_cramped_internal:Nn
  \else
    \expandafter \MT_cramped_internal:Nn \expandafter #1
  \fi
}
%    \end{macrocode}
%  \end{macro}
%  \begin{macro}{\MT_cramped_internal:Nn}
%  The internal command.
%    \begin{macrocode}
\def\MT_cramped_internal:Nn #1#2{
%    \end{macrocode}
%  Create a box containing the math and force a cramped style by
%  issuing a non-existing radical.
% \changes{v1.25}{2020/12/16}{Don't set \cs{nulldelimiterspace} to
% zero, back kern intead, solves issue \#18}
% \changes{v1.25}{2021/03/18}{placed \cs{sbox}\cs{z@} by
% \cs{setbox}\cs{z@}\cs{hbox} to better handle expansion in
% \env{crampedsubarray}, see issue \#17}
%    \begin{macrocode}
  %% \sbox\z@{$\m@th#1\kern-\nulldelimiterspace\radical\z@{#2}$}
  \setbox\z@\hbox{$\m@th#1\kern-\nulldelimiterspace\radical\z@{#2}$}
%    \end{macrocode}
%  Then make sure the height is correct.
%    \begin{macrocode}
    \ifx#1\displaystyle
      \dimen@=\fontdimen8\textfont3
      \advance\dimen@ .25\fontdimen5\textfont2
    \else
      \dimen@=1.25\fontdimen8
      \ifx#1\textstyle\textfont
      \else
        \ifx#1\scriptstyle
          \scriptfont
        \else
          \scriptscriptfont
        \fi
      \fi
      3
    \fi
    \advance\dimen@-\ht\z@ \ht\z@=-\dimen@
%    \end{macrocode}
% Leave vertical mode before typesetting the box and add a
% \texttt{\{\}} in front to gard against unnecessary reboxing  
% \changes{v1.25}{2021/03/04}{Added vmode protection and reboxing
% gard, see issue \#19}
%    \begin{macrocode}
    \ifvmode\leavevmode\fi
    {}\box\z@
}
%    \end{macrocode}
%  \end{macro}
%
%  \subsubsection{Cramped versions of \texttt{\textbackslash
%  mathllap}, \texttt{\textbackslash mathclap}, and
%  \texttt{\textbackslash mathrlap}}
%  \begin{macro}{\crampedllap}
%  \begin{macro}{\MT_cramped_llap_internal:Nn}
%  \begin{macro}{\crampedclap}
%  \begin{macro}{\MT_cramped_clap_internal:Nn}
%  \begin{macro}{\crampedrlap}
%  \begin{macro}{\MT_cramped_rlap_internal:Nn}
%  Cramped versions of \cs{mathXlap} (for speed). Made by the author.
% \changes{v1.15}{2021/03/04}{Add \cs{relax} before the \cs{ifx}'s to
% prevent corner case premature expansions}
%    \begin{macrocode}
\providecommand*\crampedllap[1][\@empty]{
  \relax\ifx\@empty#1\@empty
    \expandafter \mathpalette \expandafter \MT_cramped_llap_internal:Nn
  \else
    \expandafter \MT_cramped_llap_internal:Nn \expandafter #1
  \fi
}
\def\MT_cramped_llap_internal:Nn #1#2{
  {}\llap{\MT_cramped_internal:Nn #1{#2}}
}
\providecommand*\crampedclap[1][\@empty]{
  \relax\ifx\@empty#1\@empty
    \expandafter \mathpalette \expandafter \MT_cramped_clap_internal:Nn
  \else
    \expandafter \MT_cramped_clap_internal:Nn \expandafter #1
  \fi
}
\def\MT_cramped_clap_internal:Nn #1#2{
  {}\clap{\MT_cramped_internal:Nn #1{#2}}
}
\providecommand*\crampedrlap[1][\@empty]{
  \relax\ifx\@empty#1\@empty
    \expandafter \mathpalette \expandafter \MT_cramped_rlap_internal:Nn
  \else
    \expandafter \MT_cramped_rlap_internal:Nn \expandafter #1
  \fi
}
\def\MT_cramped_rlap_internal:Nn #1#2{
  {}\rlap{\MT_cramped_internal:Nn #1{#2}}
}
%    \end{macrocode}
%  \end{macro}
%  \end{macro}
%  \end{macro}
%  \end{macro}
%  \end{macro}
%  \end{macro}
%
%
%
% \subsubsection{Cramped versions of subarray and \cs{substack}}
%
% 
% \changes{v1.22}{2019/07/22}{Added cramped versions of subarray and
% \cs{substack}} This was suggested by Henri Menke in
% \url{https://github.com/latex3/latex2e/issues/149} and
% \url{https://chat.stackexchange.com/transcript/message/50943985#50943985}.
% As we are slowly moving away from the expl3 like syntax, we'll
% implement this
% with an adjusted copy of \env{subarray} from \pkg{amsmath}.
% \changes{v1.26}{2021/03/27}{Fresh copy of subarray from \pkg{amsmath}
% including the lualatex branch}
% \begin{environment}{crampedsubarray}
%    \begin{macrocode}
\ifx\directlua\@undefined
\newenvironment{crampedsubarray}[1]{%
  \vcenter\bgroup
  \Let@ \restore@math@cr \default@tag
  \baselineskip\fontdimen10 \scriptfont\tw@
  \advance\baselineskip\fontdimen12 \scriptfont\tw@
  \lineskip\thr@@\fontdimen8 \scriptfont\thr@@
  \lineskiplimit\lineskip
  \ialign\bgroup\ifx c#1\hfil\fi
%    \end{macrocode}
% Though here we ought to use the proper cramped internal macro, as
% otherwise there will be a slight difference in vertical spacing
% \changes{v1.25}{2021/03/18}{Added \cs{MT_cramped_internal:Nn}, see
% issue \#17}
%    \begin{macrocode}
  %%$\m@th\scriptstyle\kern-\nulldelimiterspace\radical\z@{##}$% <-- changed line
  \span\MT_cramped_internal:Nn \scriptstyle {##}%
}{%
  \crcr\egroup\egroup
}
\else
\newenvironment{crampedsubarray}[1]{%
  \vcenter\bgroup
  \Let@ \restore@math@cr \default@tag
  \baselineskip \Umathstacknumup \scriptstyle
  \advance\baselineskip \Umathstackdenomdown \scriptstyle
  \lineskip \Umathstackvgap \scriptstyle
  \lineskiplimit \lineskip
  \ialign\bgroup\ifx c#1\hfil\fi
  \Ustartmath
%    \end{macrocode}
% Here we simply use the build in cramped command from LuaLaTeX
%    \begin{macrocode}
    \crampedscriptstyle{##}
  \Ustopmath
  \hfil\crcr
}{%
  \crcr\egroup\egroup
}
\fi
%    \end{macrocode}
% \end{environment}
% \begin{macro}{\crampedsubstack}
% And the identical implementation for \cs{crampedsubstack}:
%    \begin{macrocode}
\newcommand{\crampedsubstack}[1]{\crampedsubarray{c}#1\endcrampedsubarray}
%    \end{macrocode}
% \end{macro}
%
%
%
%  \section{Macros by Michael J.~Downes}
%
%  The macros in this section are all by Michael J.~Downes. Either
%  they are straight copies of his original macros or inspired and
%  extended here.
%
%
%  \subsection{Prescript}
%  \begin{macro}{\prescript}
%  This command is taken from a posting to comp.text.tex on
%  December~20th 2000 by Michael J.~Downes. The comments are his. I
%  have added some formatting options to the arguments so that a user
%  can emulate the \pkg{isotope} package.
%
% \changes{v.1.12}{2012/04/19}{Extended \cs{prescript} to change style
% if used in say S context. Requestd by Oliver Buerschaper.}
%  Update 2012: One drawback from MJD's original implementation, is
%  that the math style is hardwired, such that if used in say
%  \cs{scriptstyle} context, then the style/size of the prescript
%  remain the same size. A slightly expensive fix, is to use the
%  \cs{mathchoice} construction. First exdent MJD's code a little
%  (keeping his comments)
% \begin{macro}{\MT_prescript_inner:}
% We make the style an extra forth argument
%    \begin{macrocode}
\newcommand{\MT_prescript_inner:}[4]{
%    \end{macrocode}
%  Put the sup in box 0 and the sub in box 2.
%  \changes{v1.19}{2017/03/31}{Added MH\_ prefix}
%    \begin{macrocode}
  \@mathmeasure\z@#4{\MT_prescript_sup:{#1}}
  \@mathmeasure\tw@#4{\MT_prescript_sub:{#2}}
  \MH_if_dim:w \wd\tw@>\wd\z@
    \setbox\z@\hbox to\wd\tw@{\hfil\unhbox\z@}
  \MH_else:
    \setbox\tw@\hbox to\wd\z@{\hfil\unhbox\tw@}
  \MH_fi:
%    \end{macrocode}
%  Do not let a preceding mathord symbol approach without any
%  intervening space.
%    \begin{macrocode}
  \mathop{}
%    \end{macrocode}
%  Use \cs{mathopen} to suppress space between the prescripts and the
%  base object even when the latter is not of type ord.
%    \begin{macrocode}
  \mathopen{\vphantom{\MT_prescript_arg:{#3}}}^{\box\z@}\sb{\box\tw@}
  \MT_prescript_arg:{#3}
}
%    \end{macrocode}
% \end{macro}
% Next create \cs{prescript} using \cs{mathchoice} 
%    \begin{macrocode}
\DeclareRobustCommand{\prescript}[3]{
  \mathchoice
%    \end{macrocode}
%  In D and T style, we use MJD's default:
%    \begin{macrocode}
    {\MT_prescript_inner:{#1}{#2}{#3}{\scriptstyle}}
    {\MT_prescript_inner:{#1}{#2}{#3}{\scriptstyle}}
%    \end{macrocode}
%  In the others we step one style down. Of couse in SS style, using
%  \cs{scriptscript} may seem wrong, but there is no lower style.
%    \begin{macrocode}
    {\MT_prescript_inner:{#1}{#2}{#3}{\scriptscriptstyle}}
    {\MT_prescript_inner:{#1}{#2}{#3}{\scriptscriptstyle}}
}
%    \end{macrocode}
%  \end{macro}
%  Then the named arguments. Can you see I'm preparing for templates?
%    \begin{macrocode}
\define@key{\MT_options_name:}
  {prescript-sup-format}{\def\MT_prescript_sup:{#1}}
\define@key{\MT_options_name:}
  {prescript-sub-format}{\def\MT_prescript_sub:{#1}}
\define@key{\MT_options_name:}
  {prescript-arg-format}{\def\MT_prescript_arg:{#1}}
\setkeys{\MT_options_name:}{
  prescript-sup-format={},
  prescript-sub-format={},
  prescript-arg-format={},
}
%    \end{macrocode}
%
%  \subsection{Math sizes}
%  \begin{macro}{\@DeclareMathSizes}
%  This command is taken from a posting to comp.text.tex on
%  October~17th 2002 by Michael J.~Downes. The purpose is to be able
%  to put dimensions on the last three arguments of
%  \cs{DeclareMathSizes}.
%
%  As of 2015, this fix has made it into the \LaTeX{} kernel. Thus we
%  only provide the fix for older kernels.
% \changes{v1.18}{2015/11/12}{This was now moved into the kernel, thus
% we only provide it for older kernels}
%  \changes{v1.19}{2017/03/31}{Added MH\_ prefix}
%    \begin{macrocode}
\ifx\e@alloc\@undefined% kernel thus older than 2015
  \def\@DeclareMathSizes #1#2#3#4#5{%
    \@defaultunits\dimen@ #2pt\relax\@nnil
    \MH_if:w $#3$%
      \MH_let:cN {S@\strip@pt\dimen@}\math@fontsfalse
    \MH_else:
      \@defaultunits\dimen@ii #3pt\relax\@nnil
      \@defaultunits\@tempdima #4pt\relax\@nnil
      \@defaultunits\@tempdimb #5pt\relax\@nnil
      \toks@{#1}%
      \expandafter\xdef\csname S@\strip@pt\dimen@\endcsname{%
        \gdef\noexpand\tf@size{\strip@pt\dimen@ii}%
        \gdef\noexpand\sf@size{\strip@pt\@tempdima}%
        \gdef\noexpand\ssf@size{\strip@pt\@tempdimb}%
        \the\toks@
      }%
    \MH_fi:
  }
\fi
%    \end{macrocode}
%  \end{macro}
%
%  \subsection{Mathematics within italic text}
%  mathic: Michael J.~Downes on comp.text.tex, 1998/05/14.
%  \begin{macro}{\MT_mathic_true:}
%  \begin{macro}{\MT_mathic_false:}
%  Renew \cs{(} so that it detects the slant of the font and inserts
%  an italic correction. In January 2013 Andrew Swann suggested that
%  this should be made robust. It is a little tricky, but seems to
%  work.
% \changes{v1.13}{2013/02/11}{Added robustness code}
%    \begin{macrocode}
\def\MT_mathic_true: {
  \MH_if_boolean:nF {math_italic_corr}{
    \MH_set_boolean_T:n {math_italic_corr}
%    \end{macrocode}
%  Save the original meaning if you need to go back (note that this
%  does not save the robust part of the macro, we fix this later on by
%  rerobustifying).
%  \changes{v1.19}{2017/03/31}{Added MH\_ prefix}
%    \begin{macrocode}
    \MH_if_boolean:nTF {robustify}{
      \MH_let:NwN \MT_mathic_redeffer: \DeclareRobustCommand
    }{
      \MH_let:NwN \MT_mathic_redeffer: \renewcommand
    }
    \MH_let:NwN \MT_begin_inlinemath: \(
    %\renewcommand*\({
    \MT_mathic_redeffer:*\({
      \relax\ifmmode\@badmath\else
      \ifhmode
        \MH_if_dim:w \fontdimen\@ne\font>\z@
%    \end{macrocode}
%  We have a small problem here, if the user use >>\verb|text~\(|<< then
%  the italic correction is lost due to the penalty (see \cite{TBT},
%  section 4.3.3). However, we \emph{know} what \verb|~| does, a
%  (positive) penalty and a skip.
%  \changes{v1.19}{2017/03/31}{Added MH\_ prefix}
%    \begin{macrocode}
          \MH_if_dim:w \lastskip>\z@
            \skip@\lastskip\unskip
%    \end{macrocode}
%  \changes{v1.15}{2014/07/16}{Added penalty workaround}
%  Here is the fix.
%  \changes{v1.19}{2017/03/31}{Added MH\_ prefix}
%    \begin{macrocode}
            \MH_if_num:w \lastpenalty>\z@
              \count@\lastpenalty\unpenalty
            \MH_fi:
            \@@italiccorr
%    \end{macrocode}
%  \changes{v1.15}{2014/07/16}{Added penalty workaround}
%  And here it is inserted again.
%  \changes{v1.19}{2017/03/31}{Added MH\_ prefix}
%    \begin{macrocode}
            \MH_if_num:w \count@>\z@
              \penalty\count@
            \MH_fi:
            \hskip\skip@
          \MH_else:
            \@@italiccorr
          \MH_fi:
        \MH_fi:
      \MH_fi:
      $\MH_fi:
    }
  }
}
%$ for emacs coloring ;-)
%    \end{macrocode}
%  Just for restoring the old behavior.
%    \begin{macrocode}
\def\MT_mathic_false: {
  \MH_if_boolean:nT {math_italic_corr}{
    \MH_set_boolean_F:n {math_italic_corr}
%    \end{macrocode}
% When \cs{(} is robust, we cannot simply relet it to the old value,
% as all the robust macro is not saved. Instead we redefined partly
% and force robustness.
%    \begin{macrocode}
    \MH_if_boolean:nTF {robustify}{
      \edef\({\MT_begin_inlinemath:}%
      \forced_EQ_MakeRobust\(%
    }{
      \MH_let:NwN \( \MT_begin_inlinemath:
    }
  }
}
\MH_new_boolean:n {math_italic_corr}
\define@key{\MT_options_name:}{mathic}[true]{
  \@ifundefined{MT_mathic_#1:}
    { \MT_true_false_error:
      \@nameuse{MT_mathic_false:}
    }
    { \@nameuse{MT_mathic_#1:} }
}
%    \end{macrocode}
%  \end{macro}
%  \end{macro}
%
%  \subsection{Spreading equations}
%
%  Michael J.~Downes on comp.text.tex 1999/08/25
%  \begin{environment}{spreadlines}
%  This is meant to be used outside math, just like
%  \env{subequations}.
%    \begin{macrocode}
\newenvironment{spreadlines}[1]{
  \setlength{\jot}{#1}
  \ignorespaces
}{ \ignorespacesafterend }
%    \end{macrocode}
%  \end{environment}
%
%  \subsection{Gathered}
%
%  Inspired by Michael J.~Downes on comp.text.tex 2002/01/17.
%  \begin{environment}{MT_gathered_env}
%  Just like the normal \env{gathered}, only here we're allowed to
%  specify actions before and after each line.
% \changes{2017/10/30}{v1.20}{added \cs{alignedspace@left} instead of
% \cs{null}\cs{,} as in the other amsmath adjustments}
%    \begin{macrocode}
\MaybeMHPrecedingSpacesOff
\newenvironment{MT_gathered_env}[1][c]{%
    \RIfM@\else
        \nonmatherr@{\begin{\@currenvir}}%
    \fi
    %\null\,%
    \alignedspace@left%
    \if #1t\vtop \else \if#1b\vbox \else \vcenter \fi\fi \bgroup
        \Let@ \chardef\dspbrk@context\@ne \restore@math@cr
        \spread@equation
        \ialign\bgroup
            \MT_gathered_pre:
            \strut@$\m@th\displaystyle##$
            \MT_gathered_post:
            \crcr
}{%
  \endaligned
  \MT_gathered_env_end:
}
\MHPrecedingSpacesOn
%    \end{macrocode}
%  \end{environment}
%  \begin{macro}{\newgathered}
%  \begin{macro}{\renewgathered}
%  \begin{environment}{lgathered}
%  \begin{environment}{rgathered}
%  \begin{environment}{gathered}
%  An easier interface.
%    \begin{macrocode}
\newcommand*\newgathered[4]{
  \newenvironment{#1}
    { \def\MT_gathered_pre:{#2}
      \def\MT_gathered_post:{#3}
      \def\MT_gathered_env_end:{#4}
      \MT_gathered_env
    }{\endMT_gathered_env}
}
\newcommand*\renewgathered[4]{
  \renewenvironment{#1}
    { \def\MT_gathered_pre:{#2}
      \def\MT_gathered_post:{#3}
      \def\MT_gathered_env_end:{#4}
      \MT_gathered_env
    }{\endMT_gathered_env}
}
\newgathered{lgathered}{}{\hfil}{}
\newgathered{rgathered}{\hfil}{}{}
\renewgathered{gathered}{\hfil}{\hfil}{}
%    \end{macrocode}
%  \end{environment}
%  \end{environment}
%  \end{environment}
%  \end{macro}
%  \end{macro}
%
%  \subsection{Split fractions}
%
%  Michael J.~Downes on comp.text.tex 2001/12/06.
%  \begin{macro}{\splitfrac}
%  \begin{macro}{\splitdfrac}
%  These commands use \cs{genfrac} to typeset a split fraction. The
%  thickness of the fraction rule is simply set to zero.
%    \begin{macrocode}
\newcommand*\splitfrac[2]{%
  \genfrac{}{}{0pt}{1}%
    {\textstyle#1\quad\hfill}%
    {\textstyle\hfill\quad\mathstrut#2}%
}
\newcommand*\splitdfrac[2]{% 
  \genfrac{}{}{0pt}{0}{#1\quad\hfill}{\hfill\quad\mathstrut #2}%
}
%    \end{macrocode}
%  \end{macro}
%  \end{macro}
%
%
%  \section{Bug fixes for \pkg{amsmath}}
%  The following fixes some bugs in \pkg{amsmath}, but only if the
%  switch is true.
%    \begin{macrocode}
\MH_if_boolean:nT {fixamsmath}{
%    \end{macrocode}
%  \begin{macro}{\place@tag}
%  This corrects a bug in \pkg{amsmath} affecting tag placement in
%  \env{flalign}.\footnote{See
%  \url{http://www.latex-project.org/cgi-bin/ltxbugs2html?pr=amslatex/3591}}
%    \begin{macrocode}
\def\place@tag{%
  \iftagsleft@
    \kern-\tagshift@
%    \end{macrocode}
%  The addition. If we're in \env{flalign} (meaning
%  $\cs{xatlevel@}=\cs{tw@}$) we skip back by an amount of
%  \cs{@mathmargin}. This test is also true for the \env{xxalignat}
%  environment, but it doesn't matter because a)~it's not
%  supported/described in the documentation anymore so new users
%  won't know about it and b)~it forbids the use of \cs{tag}
%  anyway.
%  \changes{v1.19}{2017/03/31}{Added MH\_ prefix}
%    \begin{macrocode}
    \if@fleqn
      \MH_if_num:w \xatlevel@=\tw@
        \kern-\@mathmargin
      \MH_fi:
    \MH_fi:
%    \end{macrocode}
%  End of additions.
%  \changes{v1.19}{2017/03/31}{Added MH\_ prefix}
%    \begin{macrocode}
    \MH_if:w 1\shift@tag\row@\relax
      \rlap{\vbox{%
        \normalbaselines
        \boxz@
        \vbox to\lineht@{}%
        \raise@tag
      }}%
    \MH_else:
      \rlap{\boxz@}%
    \MH_fi:
    \kern\displaywidth@
  \MH_else:
    \kern-\tagshift@
    \MH_if:w 1\shift@tag\row@\relax
      \llap{\vtop{%
        \raise@tag
        \normalbaselines
        \setbox\@ne\null
        \dp\@ne\lineht@
        \box\@ne
        \boxz@
      }}%
    \MH_else:
      \llap{\boxz@}%
    \MH_fi:
  \MH_fi:
}
%    \end{macrocode}
%  \end{macro}
%
%  \begin{macro}{\x@calc@shift@lf}
%  This corrects a bug\footnote{See
%  \url{http://www.latex-project.org/cgi-bin/ltxbugs2html?pr=amslatex/3614}}
%  in \pkg{amsmath} that could cause a non-positive value of the dimension
%  \cs{@mathmargin} to cause an
%  \begin{verbatim}
%  ! Arithmetic overflow.
%  <recently read> \@tempcntb
%  \end{verbatim}
%  when in \mode{fleqn,leqno} mode. Not very comprehensible for the user.
%  \changes{v1.19}{2017/03/31}{Added MH\_ prefix}
%    \begin{macrocode}
\def\x@calc@shift@lf{%
  \MH_if_dim:w \eqnshift@=\z@
    \global\eqnshift@\@mathmargin\relax
      \alignsep@\displaywidth
      \advance\alignsep@-\totwidth@
%    \end{macrocode}
%  The addition: If \cs{@tempcntb} is zero we avoid division.
%  \changes{v1.19}{2017/03/31}{Added MH\_ prefix}
%    \begin{macrocode}
      \MH_if_num:w \@tempcntb=0
      \MH_else:
        \global\divide\alignsep@\@tempcntb % original line
      \MH_fi:
%    \end{macrocode}
%  Addition end.
%  \changes{v1.19}{2017/03/31}{Added MH\_ prefix}
%    \begin{macrocode}
      \MH_if_dim:w \alignsep@<\minalignsep\relax
        \global\alignsep@\minalignsep\relax
      \MH_fi:
  \MH_fi:
  \MH_if_dim:w \tag@width\row@>\@tempdima
    \saveshift@1%
  \MH_else:
    \saveshift@0%
  \MH_fi:}%
%    \end{macrocode}
%  \end{macro}
%    \begin{macrocode}
}
%    \end{macrocode}
%  End of bug fixing.
%
%  \subsection{Making environments safer}
%
%  \begin{macro}{\aligned@a}
%  Here we make the \pkg{amsmath} inner environments disallow spaces
%  before their optional positioning specifier.
%    \begin{macrocode}
\MaybeMHPrecedingSpacesOff
\renewcommand\aligned@a[1][c]{\start@aligned{#1}\m@ne}
\MHPrecedingSpacesOn
%    \end{macrocode}
%  \end{macro}
%
%  \section{Additional features 2020–}
%
%  The selection below is meant for additions addeed 2020 onwards that
%  are not integrated into existing macros.
%
%  \subsection{\cs{xmathstrut}}
% 
% \begin{macro}{\xmathstrut@box}
% \begin{macro}{\xmathstrut@dim}
% \begin{macro}{\xmathstrut}
% \begin{macro}{\xmathstrut@}
% \begin{macro}{\xmathstrut@do}
% Suggested and implemented by Frank Mittelbach. The basic idea is to
% add (or subtract) a fraction of the total height of the normal math
% strut to both the height and depth of the new strut. The input from
% the user is suppose to be a decimal number. The value zero results
% in no change compared to the normal math strut. 
%  \changes{v1.24}{2020/03/06}{Added \cs{xmathstrut}}
%    \begin{macrocode}
\newbox\xmathstrut@box
\newdimen\xmathstrut@dim
\def\xmathstrut{\@dblarg\xmathstrut@}
\def\xmathstrut@[#1]#2{%
  \def\xmathstrut@dp{#1}% 
  \vphantom{\mathpalette\xmathstrut@do{#2}}%
}
%    \end{macrocode}
% The algorithm is simple. Start by recording the size (a box) of \verb|$($|
% in the current style (via \verb|\mathpalette|)
%    \begin{macrocode}
\def\xmathstrut@do#1#2{%
  \setbox\xmathstrut@box\hbox{$#1($}%)%emacs
   \xmathstrut@dim\dimexpr\ht\xmathstrut@box+\dp\xmathstrut@box\relax
%    \end{macrocode}
% Then manipulate the size of the box by adding the user input times
% the total height of the original box to the height and depth
% respectfully. 
%    \begin{macrocode}
   \ht\xmathstrut@box\dimexpr\ht\xmathstrut@box
           +#2\xmathstrut@dim
           \relax
   \dp\xmathstrut@box\dimexpr\dp\xmathstrut@box
         +\xmathstrut@dp\xmathstrut@dim
         \relax
%    \end{macrocode}
% In the end ship the modified box (it is typeset inside a \verb|\vphantom|).
%    \begin{macrocode}
   \box\xmathstrut@box}

%    \end{macrocode}
% Do note that this means that for \verb|\xmathstrut{0.1}| 10\% is
% added twice. But it make explaining \verb|\xmathstrut[0.2]{0.1}| a
% lot easier as it is just 10\% added to the height and 20\% added to
% the depth.
% \end{macro}
% \end{macro}
% \end{macro}
% \end{macro}
% \end{macro}
%
%  This is the end of the \pkg{mathtools} package.
%    \begin{macrocode}
%</package>
%    \end{macrocode}
%
%  \Finale
\endinput
